\chapter{Отдых, досуг, восстановление и развлечения}
Здесь вы узнаете, как герои отдыхают и чем занимаются в свободное от приключений время. 

\section{Отдых}
Если герой пережил Боевую сцену, его ЕЗ могут быть восстановлены до максимума посредством Отдыха, врачебной помощи или уникальных свойств и способностей.
\newline Герои отдыхают в Антракте, а так же в Интерлюдиях с пометкой «Отдых», и делают это очень по-разному. Но, что бы они ни учудили, это помогает залечить раны, хлебнуть чутка оптимизма и исполниться решимостью в достижении целей.
\paragraph{Во время Отдыха} герой:
\begin{itemize}
  \item Восстанавливает число Единиц Здоровья и/или Единиц Характеристик, равное \textbf{|МВн|} (минимум 1). ЕЗ и ЕХ восстанавливаются в любых комбинациях в пределах восстанавливаемого значения. Герой Восстанавливает в 2 раза больше ЕЗ и/или ЕХ (минимум 2), если получает врачебный уход;
  \item Восстанавливает число Эн, равное \textbf{|МОб|} (минимум 1). - Понижает Интоксикацию на число, равное \textbf{|числу восстановленных ЕЗ|}.
\end{itemize}
\paragraph{Отдых и переломы:} сломанные конечности требуют особого внимания. Чтобы герой восстановил ЕЗ, потерянные при Переломе, врач должен успешно проверить Медицину против \textbf{|15|}. 
\paragraph{Отдых и доспехи:} герой может полноценно отдыхать (и восстанавливать ЕЗ и/или ЕХ), пока одет в доспех с БД +4 или меньше. Герой, отдыхавший в доспехе с большим БД, не получает эффектов Отдыха. Обратите внимание, что ему все же удается худо-бедно поспать, а потому шанс задремать на посту или за столом в таверне существенно ниже, чем без сна вообще.

\section{Восстановление ЕЗ при помощи Медицины}
Хороший врач поставит на ноги почти любого. Остальных он профессионально успокоит перед неминуемой смертью.
\newline Один раз за Сцену герой может совершить проверку Медицины против \textbf{|10|}. Преуспев, герой восстанавливает цели 1 ЕЗ и дополнительно 1 ЕЗ за каждые 5 единиц успеха.
\begin{tcolorbox}
  Т.е. получив при проверке Медицины 15, герой восстановит цели 2 ЕЗ, получив 20 – восстановит 3 ЕЗ, и т.д.
\end{tcolorbox}
Проверив Медицину против \textbf{|15|}, герой может привести существо в сознание, не восстанавливая его ЕЗ. 
\newline Применение навыка в бою занимает \textbf{|10 – Медицина(Ин, Мд)|} Очередей (минимум 1 Очередь). В это время герой не может совершать маневры и Перемещаться. Если в процессе медик одномоментно теряет ЕЗ, превышающие \textbf{|Вл|}, проверка считается проваленной. 

\section{Досуг, Услуги и Развлечения}
Чем занимаются герои, когда выдается свободная минутка? Ответ на этот, казалось бы, незначительный вопрос во многом определяет развитие истории. 
\newline Эти правила призваны наполнить историю событиями, как случайными, так и логически вытекающими из сюжетной канвы. Игроки и мастер могут использовать их, чтобы дать героям игромеханические преимущества или узнать что-то важное, а также черпать идеи для развития сюжета. 
\newline Если у игрока нет конкретных идей, он может решить, как герой проведет свободное время, выбрав вариант из перечня Досуга, Развлечений и Услуг.

\paragraph{Услуга} включает все то, в чем герой нуждается, но не может получить/сделать сам. Подразумевает наличие в поселении специалиста, обладающего необходимым снаряжением и желающего продать свои навыки. Иногда герои могут вынудить статиста оказать им Услугу, надавив на него или заплатив побольше.
\paragraph{Развлечение} представляет занятие, способное на время развеять скуку. Требует минимум специального оборудования, но обычно нуждается в организаторе и предварительных затратах для участника. 
\paragraph{Досуг} обозначает виды занятий, для которых достаточно самого героя и желания чем-нибудь себя занять. Досуг доступен, когда герой может выполнить условия и имеет все необходимое для занятия.
\newline Все виды Досуга и Развлечений и многие Услуги являются Интерлюдиями.

\paragraph{Эффекты:} входят в игру в следующей Сцене и длятся до ее окончания. Мастер может сохранять их на большее время, если это соответствует контексту. Разумеется, все материальные ценности, которые приобрел герой, останутся при нем.
\paragraph{Отдых в Интерлюдиях:} эффекты Интерлюдий с пометкой «Отдых» применяются в дополнение к эффектам Отдыха.
\paragraph{Досуг и Антракт:} при уходе в Антракт герой может выбрать \textbf{|1+ММд|} (минимум 2) вида Досуга с Риском 0 и не предполагающих проверок Скрытой угрозы, которыми он займется в свободное время.
\paragraph{Проблемы:} возможные негативные последствия Досуга и Развлечений. Так же, как и эффекты, входят в игру в следующей Сцене и длятся до ее окончания. Мастер может сохранять их на большее время, если это соответствует контексту.
\paragraph{Риск:} вероятность того, что Досуг или Развлечение обернутся тоской, унынием или чем похуже. Риск вычитается из проверок Скрытой угрозы Досуга и Развлечений, если в описании не указано обратного.
\paragraph{Сложность приобретения:} за веселье приходится платить, да и самые обычные с виду занятия могут потребовать некоторых трат. Если герой совмещает несколько видов Досуга и/ или Развлечений, сложите СП. 
\newline Если рядом со значением СП стоит «+», это значит, что контекст, экспозиция или даже герой лично могут повлиять на окончательную стоимость.
\paragraph{Скрытая угроза:} иногда самые невинные занятия заканчиваются сущим кошмаром! Некоторые виды Досуга и Развлечений предполагают проверки Скрытой угрозы. Мастер может инициировать проверку Скрытой угрозы, даже если она не указана в столбце «Проблемы», но подразумевается контекстом. Если ситуация не располагает к проверке Скрытой угрозы, мастер вправе игнорировать ее.
\paragraph{Всего да побольше:} виды Досуга и Развлечений могут совмещаться друг с другом. Герой может одновременно совмещать не более \textbf{|1+ММд|} (минимум 2) видов Досуга и Развлечений. При необходимости используйте больший показатель Риска.
\newline Когда совмещенные виды Досуга и Развлечений следуют друг за другом, герой может тратить приобретенные эффекты на необходимые проверки Досуга и Развлечений. Если он не сделает этого, эффекты не считаются потерянными.
\paragraph{Сложность проверок:} сложность всех проверок в Сцене Досуга или Развлечения равна \textbf{|10+Риск|}. Проверки могут усложняться или заменяться в соответствии с контекстом и логикой ситуации.

\subsection{Подработка и шарлатанство}
Герои могут организовывать Развлечения и оказывать Услуги. Тогда они выступают продавцами, и повышают Богатство в соответствии с правилами продажи. Не забудьте, как обычно, определить СП Услуги или Развлечения, прежде чем продавать их – не исключено, что у героев есть конкуренты и возможности для демпинга.
\newline При этом любой профессиональный навык может быть заменен Выступлением. В этом случае шарлатану лучше исчезнуть вовремя или серьезно озаботиться личной безопасностью.
\subsection{Доступность Развлечений}
Герои проверяют Доступность Развлечений, что бы узнать, есть ли в наличии желаемое. Разумеется, только случае, если у мастера и игроков есть какие-то сомнения на этот счет.
\trouble
{Для плебса}{Развлечение доступно и дешево. Используйте указанную СП.}
{Для ценителей}{Развлечение распространено, но не так уж доступно. СП возрастает вдвое.}
{Для элиты}{Развлечение доступно, но довольно дорого. СП возрастает втрое.}
{Для маргиналов}{Развлечение недоступно, хотя Плут и герои с некоторыми другими Атрибутами могут найти возможности. СП возрастает втрое даже для них.}

\subsection{Перечень услуг}
\genAndGet{leisure}{leisure}{Услуга}

\subsection{Перечень развлечений}
\genAndGet{leisure}{leisure}{Развлечение}

\subsection{Перечень досуга}
\genAndGet{leisure}{leisure}{Досуг}
