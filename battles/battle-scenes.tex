\section{Боевые сцены}
Боевая сцена начинается, если один или несколько героев подверглись нападению, либо напали на кого-то сами.

\subsection{Детали Боевой сцены}
\paragraph{Болевой шок:} если герой одномоментно теряет число ЕЗ, превышающее его \textbf{|Вл|}, то его следующая проверка получает Помеху. 
Разумеется, эффект относится к существам, которые способны испытывать боль. 
\paragraph{Опасная рана:} если герой одномоментно теряет 1/5 и более от максимальных ЕЗ, он проверяет Вн против \textbf{|15|}. При успехе герой продолжает сражаться, при провале теряет сознание.
\paragraph{Потеря сознания:} герой, потерявший сознание, очнется в начале следующей Сцены. Если не замерзнет, не истечет кровью, не будет добит или съеден. Герой, пришедший в сознание в 0 ЕЗ, находится в Агонии!
\newline Герой немедленно приходит в себя, если восстанавливает 1+ ЕЗ.
\paragraph{Смерть:} потеряв последнюю ЕЗ, герой проверяет Вн против \textbf{|15|}. При успехе он остается жив, теряет сознание и обзаводится шрамом на память. При провале герой или персона находятся При смерти.
\newline Некоторые эффекты убивают вне зависимости от числа ЕЗ. Возможность проверки Вн при такой смерти определяется мастером и контекстом Сцены.
\paragraph{Внезапная Смерть.} Порой урон очевидно смертелен, и никакая Защита и Выносливость здесь не спасут. Это включает попадания из танковых орудий, воздействие фугасных бомб, массированные артобстрелы, горные обвалы, извержения вулканов, метеоритные дожди и старые недобрые ядерные взрывы.
\newline Если герой или Персона теоретически имеют шанс уцелеть, проверяется  Внезапная смерть. Иммунитет к Пв атаки, вызывающей Внезапную смерть, дает 1 Преимущество на эту проверку, Родная Стихия - 2 Преимущества:
\trouble
{Ни царапины}{Герой невредим. Как ему это удалось - загадка.}
{Задело}{Герой получает Опасную рану и хороший повод для отборной брани - если успешно проверит Вн.}
{Потрепало}{Герой получает Опасную рану и тяжелую травму. Определите по таблице Случайного поражения, какая часть тела героя выведена из строя до конца следующей Сцены с его участием. Насколько серьезна травма в принципе, подскажут контекст и опытный врач.}
{Смерть}{Герой теряет все свои ЕЗ, а затем следует обычным правилам Смерти. Он получает тяжелую травму. Определите по таблице Случайного поражения, какая часть тела героя выведена из строя до конца следующей Сцены с его участием. Насколько серьезна травма в принципе, подскажут контекст и опытный врач.}
\paragraph{Смерть и статисты:} мастер вправе считать проверки Вн статистов (но не Персон) при Опасных ранах и Смерти автоматически проваленными. Также статисты (но не Персоны) всегда получают "Смерть" при Внезапной смерти. Попав в Состояние "При смерти", они сразу же погибают, если только Состояние не было вызвано Ходом "Повезло".
\paragraph{Все на одного:} сражение с толпой требует высочайшего мастерства. Если герой находится в Боевом контакте у нескольких противников, в начале Очереди он выбирает \textbf{|1 + ММд героя|} (минимум 1) противников, на которых сконцентрирован. Они атакуют героя, как обычно. Остальные проверяют Дб с Преимуществом, выбрав его целью.
\paragraph{Внезапное нападение.} Если герой не заметил врага, провалив Наблюдательность против его Скрытности, либо вовсе не ждет нападения из-за Эмоционального фона Сцены, он захвачен врасплох. Это значит, что:
\begin{itemize}
  \item Герой не может Перемещаться и Действовать независимо от значения Рц;
  \item МЛв и БЩ вычитаются из Зщ героя;
  \item Атаки по герою получают Преимущество;
  \item Если герой дожил до своей Очереди, он страдает от Ошеломления;
  \item Герой не имеет Боевого контакта.
\end{itemize}
\paragraph{Лежащие и сидящие герои:} лечь на землю или опуститься на корточки может быть хорошей идеей в одной ситуации и губительной - в другой.
\begin{itemize}
  \item Мт по лежащему или сидящему герою проверяется с Помехой;
  \item Дб по лежащему или сидящему герою проверяется с Преимуществом;
  \item Лежащий или сидящий на корточках герой может Перемещаться на \textbf{|МЛв|} метров;
  \item Лежащий или сидящий на корточках герой проверяет Дб с Помехой;
  \item Боевой контакт лежащего героя всегда равен 1;
  \item В свою Очередь герой может упасть, израсходовав Быстрое действие.
\end{itemize}
\paragraph{Невидимки и бой с ними:} тем, кто невидим, в бою попроще. Но и им не всегда удается избежать нападения. Особенно, когда противник точно знает, что они где-то здесь.
\begin{itemize}
  \item Герой может определить примерное местонахождение Невидимого противника по косвенным признакам в \textbf{|Наблюдательности(Мд)|} метров от себя.
  \item Если герой атакует Невидимого противника, Дб или Мт проверяются с 2 Помехами;
  \item Герой может израсходовать Действие и обнаружить Невидимую цель, успешно проверив Наблюдательность(Ин). Затем он может израсходовать Быстрое действие и указать местоположение цели союзникам. В этом случае они проверяют Дб и Мт с 1 Помехой, пока цель не сменит местоположение;
  \item Если нападающий невидим благодаря Скрытности, то, атаковав, он обнаружит себя;
  \item Для проверки Скрытности герою требуется разорвать зрительный контакт со всеми, от которых он желает спрятаться. Затем герой расходует Действие и при успехе проверки становится Невидимым.
\end{itemize}
Перемещение через занятые области: если герой преодолевает область, занятую противником, он должен проверить Атлетику против \textbf{|10 + [Дб противника] + [МРз]|}. Преуспев, герой Перемещается через область, при провале падает на границе и завершает Перемещение.
\newline Герой без проверок преодолевает области, занятые союзниками.
\paragraph{Трудный ландшафт:} иногда Перемещение героя затруднено густым подлеском, глубоким снегом, скользкой грязью или тряской автомобиля на полном ходу. В такой ситуации Ск героя ополовинивается.
\paragraph{Укрытия:} мешают стрельбе и помогают прятаться. Делятся на: 
\begin{itemize}
  \item Мягкие укрытия (кусты, высокая трава, куча хвороста) дают Помеху на Мт;
  \item Твердые укрытия  (камни, стены, башенный щит) дают 2 Помехи на Мт. 
\end{itemize}
В случае промаха атакующего по цели, Пв получает укрытие. Оно может быть уничтожено.
\newline Помехи за Укрытие подразумевают, что герой виден в какой-то момент Круга. Если герой весь Круг лежит ничком за толстой бетонной плитой и не отсвечивает, его не вправе заявлять целью. Зато вправе накрывать зоной поражения эффектов, если всем известно, где залег герой.

\subsection{В начале Боевой сцены}
\begin{enumerate}
  \item \textbf{Определите, подвергся ли кто-то из участников Внезапному нападению.} Обычно это сопряжено с провалом Наблюдательности против Скрытности нападающих. Если никто из участников Боевой сцены не пытался передвигаться скрытно до ее начала и ожидал нападения, фактор внезапности вряд ли стоит учитывать. С другой стороны, герой с трюком "Гопля!" или высокой Реакцией может застать противников врасплох, молниеносно выхватив оружие и атаковав.
  \item \textbf{Определите позиции участников Сцены относительно друг друга и расположение окружающих предметов и/или элементов ландшафта.} На обсуждение стоит потратить достаточно времени, чтобы дать игрокам представление о возможностях героев и их противников. 
    \begin{tcolorbox}
      Для предельной ясности вы можете использовать тактическую карту, расчерченную на клетки. Авторский коллектив рекомендует установить размер клетки, как 1 кв.м. Обозначьте фишками героев и их противников, разместите или нарисуйте на карте предметы и ландшафт. Тактическая карта - превосходный вариант, если ваша игровая команда предпочитает детализировнные Боевые сцены.
    \end{tcolorbox}
  \item \textbf{Определите Очередность участников Сцены.} Обычно первым действует участник боя с наибольшей Реакцией. Если у двух участников одинаковая Рц, первым действует счастливчик с большей Ловкостью. Если Лв равна, первым действует выигравший Состязание в Рц. 
    \newline Когда мастер желает привнести в Сцену немного интриги, очередность действий может определяться проверкой Реакции. Первым действует участник, получивший большую величину успеха, и так далее.
  \item \textbf{Начало Круга.} Все участники Сцены действуют согласно определенной Очередности. Возможно, для завершения битвы понадобится больше одного Круга.
\end{enumerate}

\subsection{Структура Очереди}
В течение своей Очереди герой может совершить Перемещение, Действие и Быстрое действие. 
\paragraph{Во время Действия} герой использует предметы, совершает маневры, творит феномены, убеждает окружающих прекратить кровопролитие или делает что-то еще, требующее концентрации внимания. 
\paragraph{Для Перемещения} используется значение Скорости - столько метров/клеток может преодолеть герой.
\paragraph{Быстрое действие} включает односложные фразы, отрывистые жесты, Шаги в пределах 1 метра и броски предметов без проверок Мт. Хрупкие предметы могут сломаться или разбиться. 
\newline Герой может иметь несколько Быстрых действий. Любое применение способности, использующее Быстрое действие, расходует одно из них. Резерв Быстрых действий героя обновляется в начале каждой его Очереди. Если герой не выполнил в течение своей Очереди Действие или Перемещение, его резерв Быстрых действий возрастает на один за каждое из них.
\newline Иногда выполнение Быстрого действия возможно и в чужую Очередь. Такие случаи указаны в описании соответствующих Трюков, Атрибутов и заклинаний, либо продиктованы контекстом Сцены.

\paragraph{Отказ от Действия или Перемещения} позволяет сделать герою следующее: 
\begin{itemize}
  \item Совершить дополнительное Перемещение;
  \item Совершить дополнительное Быстрое действие;
  \item Подняться с земли;
  \item Взять предмет;
  \item Аккуратно положить предмет;
  \item Привести оружие в боевую готовность;
  \item Убрать оружие в ножны или кобуру;
  \item Презарядить оружие;
  \item Вскочить в седло скакуна, за руль тачки или в ее кузов.
\end{itemize}

\paragraph{Последовательность} Действия, Быстрого действия и Перемещения может быть произвольной в пределах Очереди. Например, совершая Быструю атаку, герой со Скоростью 6 может за одну свою Очередь пройти на 1, атаковать, потом пройти на 2, атаковать еще раз, затем пройти на 3 и уронить предмет.
\begin{tcolorbox}
  Быстрые действия могут показаться чем-то малозначительным, но на деле с их использованием сногое связано. Уникальные ходы и мощные Трюки часто расходуют Быстрое действие, и ориентированному на бой герою стоит иметь пару-тройку в запасе. Не забывайте, отказавшийся от Перемещения получает дополнительное Быстрое действие. Это значит, что герой, уже попавший в гущу событий, располагает минимум двумя Быстрыми действиями.
\end{tcolorbox}

\subsection{Атаки ближнего боя}
Насилие на расстоянии вытянутой руки, или чуточку дальше. Загодя обзавестись хоть каким-то оружием ближнего боя - надежный фундамент победы. 
\newline Чтобы поразить цель в ближнем бою, герой должен располагать такой целью (или целями) в своем Боевом контакте.
\begin{itemize}
  \item Боевой контакт героя Крошечного, Маленького или Среднего размера, не снаряженного Длинным оружием, составляет 1.
  \item Совершая Атаку ближнего боя, герой проверяет Доблесть.
\end{itemize}

\subsection{Дистанционные атаки}
Представляют всю широту эффективных методов убийства издалека. Гибель цивилизации -  не повод отказываться от ее плодов, даже если они слегка подгнили.
\newline Дистанционные атаки имеют 2 типа дистанций: Ближняя и Дальняя. Они указаны в описании оружия и феноменов.
\begin{itemize}
  \item Если цель находится за пределами Дальней дистанции, герой не может поразить ее.
  \item Атаки на Дальней дистанции совершаются с Помехой.
  \item Совершая Дистанционную атаку, герой проверяет Меткость.
\end{itemize}
\paragraph{Дистанционные атаки и Боевой контакт:} ближний бой - не лучшее место для демонстрации навыков стрельбы. Но выбор не всегда есть даже у самых лучших стрелков.
\begin{itemize}
  \item Герой совершает Дистанционные атаки с Помехой, если находится в Боевом контакте у одного или нескольких противников.
  \item Герой совершает Дистанционные атаки с Помехой, если цель находится в чьем-то Боевом контакте, кроме его собственного.
\end{itemize}
\paragraph{Перемещение и Дистанционные атаки:} если в свою Очередь герой перемещается на расстояние, превышающее 1 метр, он проверяет Меткость с Помехой. Конечно, герой может выстрелить без Помехи до Перемещения.
\subsection{Зоны поражения}
Зоны поражения позволяют закончить бой быстро и наверняка, особенно, если герой нападает Внезапно. Также они помогут вывести противника из строя, не убивая его.
\newline Герой вправе выбрать любую из Зон. Если при атаке не заявлена специфическая Зона, удар наносится в торс.
\begin{center} \begin{tabular}{|c|c|} \hline
  \textbf{Зона поражения} & \textbf{Штраф к Дб/Мт} \\ \hline
  Торс & 0 \\ \hline
  Конечность & -2 \\ \hline
  Пах & -3 \\ \hline
  Шея & -5 \\ \hline
  Голова & -5 \\ \hline
  Глаз & -7 \\ \hline
\end{tabular} \end{center}
\paragraph{Торс:} попадание не имеет никаких эффектов, кроме синяков, шрамов, шишек и потери уверенности в себе вкупе с некоторым числом ЕЗ. 
\paragraph{Конечность:} попадание по руке или ноге может вызвать Перелом или Потерю конечности. Когда
\begin{itemize}
  \item Если конечность одномоментно теряет 1/5 от максимальных ЕЗ, она сломана. Перелом является Опасной раной;
  \item Если конечность одномоментно теряет  1/4 от максимальных ЕЗ, она отрубается или разлетается в клочки. Пв, превышающие требуемое для уничтожения конечности число ЕЗ, теряются;
  \item Если конечность потеряна, герой страдает от Кровотечения (на внушительные 1/4 ЕЗ). Само собой, потеря конечности - Опасная рана.
\end{itemize}
\paragraph{Пах:} существование этой Зоны - веский повод носить бронегульфик.
\begin{itemize}
  \item Мужчина, отхвативший в пах, проверяет Вн против \textbf{|10 + [потерянные ЕЗ]|}. При провале он не может Действовать и Перемещаться число Очередей, на которое провалил проверку, хотя может орать, сквернословить и совершать Быстрые действия. Все атаки по нему получают Преимущество.
  \item Женщины тоже не в восторге от ударов в пах, но столь разрушительных игромеханических последствий не испытывают. 
\end{itemize}
\paragraph{Шея:} зомби, грибы, трехголовые мутанты и тому подобные твари могут какое-то время обходиться без шеи и головы.
\newline Остальным сложнее, так как если шея одномоментно теряет 1/4 и более от максимальных ЕЗ, герой эффектно расстается с головой и умирает.
\paragraph{Голова:} если и есть что-то, более хрупкое, чем пах, так это она. К тому же
\begin{itemize}
  \item Получивший удар проверяет Вн против \textbf{|10 + [потерянные ЕЗ]|}. При провале он теряет сознание.
  \item Если в голову нанесена Опасная рана, жертва, как обычно, проверяет Вн против \textbf{|15|}, но умирает при провале.
\end{itemize}
\paragraph{Глаз:} самое уязвимое место на теле огромных монстров. А еще
\begin{itemize}
  \item Глаз не может быть поражен Громоздким оружием;
  \item Одномоментная потеря \textbf{|2 + МРз|} ЕЗ и более приводит к потере глаза;
  \item Маленькие и Крошечные существа лишаются глаза при потере им 1 ЕЗ;
  \item Жертва, потерявшая все глаза, Ослеплена;
  \item Все лишние Пв получает голова;
  \item Если атака имеет Колющие или Проникающие Пв, потерянные жертвой ЕЗ удваиваются - атака поражает не только глаз, но и мозг. Жертва проверяет Вн с Помехой против \textbf{|10 + потерянные ЕЗ|} и теряет сознание при провале;
  \item Если Колющая или Проникающая атака наносит Опасную рану в глаз, жертва проверяет Вн против \textbf{|15|} с Помехой, и умирает при провале.
\end{itemize}
\subsection{Таблица случайного поражения}
Если в контексте ситуации важно, куда именно пришелся удар или подействовал эффект, а специфические зоны не были заявлены, герой проверяет Случайное поражение, когда является нападающим или целью. Во всех прочих случаях проверку совершает мастер.
\begin{center} \begin{tabular}{|c|c|} \hline
  \textbf{Результат на К20} & \textbf{Зона поражения} \\ \hline
  1-6 & Торс \\ \hline
  7-8 & Левая нога \\ \hline
  9-10 & Правая нога \\ \hline
  11 & Пах \\ \hline
  12-13 & Левая рука \\ \hline
  14-15 & Правая рука \\ \hline
  16 & Шея \\ \hline
  17 & Голова \\ \hline
  18 & Левый глаз \\ \hline
  19 & Правый глаз \\ \hline
  20 & Проверяющий сам выбирает Зону поражения \\ \hline
\end{tabular} \end{center}
