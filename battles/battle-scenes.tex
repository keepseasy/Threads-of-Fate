\section{Боевые сцены}
Боевая сцена начинается, если один или несколько героев подверглись атаке, либо напали на кого-то сами. В начале боевой сцены:
\begin{enumerate}
\item Определите, подвергся ли кто-то из участников Внезапному нападению. Обычно это сопряжено с проваленными проверками Наблюдательности против Скрытности противника. Если никто из участников сражения не пытался передвигаться скрытно и ожидал нападения, фактор внезапности вряд ли стоит учитывать. С другой стороны, Стремительный герой, герой с трюком "Гопля!" или герой с высокой Реакцией вполне может застать противников врасплох, молниеносно выхватив оружие и атаковав!
\item Определите позиции участников сцены относительно друг друга и окружающих предметов. На обсуждение этого стоит потратить несколько минут (или даже больше), чтобы игроки четко представляли возможности героев и их противников.
\item Начало Круга. Все участники сцены действуют в порядке, определяемом их Реакцией. Возможно, для завершения битвы понадобится больше одного Круга.
\end{enumerate}
\paragraph{Круг:} в бою герои и статисты действуют по Очереди. Один полный \textbf{Круг} (то есть Очереди всех героев и статистов, принимающих участие в сцене) занимает \textbf{5 секунд}. Подразумевается, что герои и статисты действуют в бою одновременно, однако для удобства игры круг разделен на Очереди. Статисты под управлением мастера действуют в одну общую Очередь, однако мастер может разделить их Очереди в соответствии с параметрами Реакции. Общая Очередь ориентируется на статиста с наименьшей Реакцией.
\paragraph{Очередность в бою:} первым действует герой или статист с наибольшей Реакцией. Если у двух героев или статистов одинаковая Реакция, первым действует тот, у кого больше Ловкость. Если и они равны, первым действует выигравший Состязание в Реакции.
\paragraph{Очередь:} Очередь героя состоит из Перемещения, Действия и Быстрого действия. В течение своей Очереди герой может:
\begin{itemize}
\item[--] \textbf{Совершить Перемещение.} Для Перемещения используется значение Ск - столько метров/клеток может преодолеть герой. Герой может отказаться от Перемещения, чтобы получить дополнительное Быстрое действие.
\item[--] \textbf{Совершить Действие.} Во время Действия герой использует предметы, атакует, творит чары, убеждает окружающих прекратить кровопролитие или делает что-то еще, требующее концентрации внимания.
\item[--] \textbf{Совершить Быстрое действие.} Быстрое действие включает односложные фразы, быстрые жесты, шаги в пределах 1 метра и броски предметов без намерения кому-то повредить или поразить конкретную цель. Хрупкие предметы могут сломаться или разбиться! У героя есть лишь одно Быстрое действие за круг, но мастер может отступать от правила, если считает ситуацию располагающей к этому. Иногда выполнение Быстрого действия возможно и в чужую Очередь. Такие случаи указаны в описании соответствующих Трюков или Атрибутов.
\end{itemize}
\begin{tcolorbox}
Герой может отказатьсяот Действия, чтобы совершить дополнительное Перемещение или дополнительное Быстрое действие.
\newline
Так же герой может отказаться от Перемещения, чтобы совершить дополнительное Быстрое действие.
\end{tcolorbox}
Герой может отказаться от Перемещения или Действия, чтобы подняться с земли, или взять предмет, или аккуратно положить предмет, или привести оружие в боевую готовность, или убрать оружие в ножны, или перезарядить оружие со свойством "Перезарядка", или вскочить в седло.
\newline
Перемещение, Действие и Быстрое действие используются в любых комбинациях. Например, совершая Быструю атаку, герой со Скоростью 6 может за одну Очередь пройти на 1, атаковать, потом пройти на 2, атаковать еще раз, затем пройти на 3 и уронить предмет.
\subsection{Детали боевой сцены}
\paragraph{Внезапное нападение.} Если герой не заметил врага или заметил в последний момент (то есть провалил проверку Наблюдательности против Скрытности врага), он захвачен врасплох. Герой не может Перемещаться и Действовать независимо от своей Реакции, а также вычитает МЛв и БЩЗщ из своей Зщ. Если герой дожил до своей следующей Очереди, он сражается по обычным правилам.
\paragraph{Все на одного:} сражение с несколькими противниками требует от воина высочайшего мастерства. Если герой сражается с несколькими противниками, в начале Очереди он должен выбрать, кому из них он уделяет больше внимания. Выберите число противников, равное \textbf{|1 + ММд героя|} (минимум 1). Они атакуют героя, как обычно. Все противники сверх этого числа атакуют героя с Преимуществом.
\paragraph{Ложись!:} герой может упасть, использовав Быстрое действие. Дистанционные атаки по лежащему герою совершаются с Помехой, атаки в ближнем бою по нему совершаются с Преимуществом. Лежащий герой может ползти с 1/2 Ск и атаковать в ближнем бою с Помехой. В остальном герой действует по обычным правилам.
\paragraph{Невидимки:} если герой атакует невидимого (из-за чар, укрытия или по иным причинам) противника, бросок совершается с Помехой. Если невидимого противника нет в атакуемой области, герой автоматически промахивается. Невидимые существа атакуют по правилам Внезапного нападения. Если существо невидимо благодаря Скрытности, а не волшебству, то, атаковав, оно обнаружит себя вне зависимости от успеха атаки.
\paragraph{Перемещение через занятые области:} если герой желает пройти через область, занятую враждебным существом, он должен пройти проверку Атлетики против \textbf{|БАЗщ + Дб противника|}. В случае успеха герой передвигается через занятую область, в случае провала падает рядом с противником. Перемещение героя на этом заканчивается.
\paragraph{Трудный ландшафт.} Иногда перемещение героя затруднено густым подлеском, глубоким снегом, скользким льдом, качающейся палубой. В такой ситуации Ск героя ополовинивается.
\subsection{Зоны поражения}
Герой может выбрать для атаки любую из перечисленных зон. Если при атаке не заявлена специфическая Зона поражения, удар наносится в торс.
\newline Сложность маневра Возрастает на указанную в таблице Зон поражения, но при успехе маневра возможен дополнительный эффект.
\begin{center}
\begin{tabular}{|c|c|}
\hline
Зона поражения & Сложность \\ \hline
Торс & 0 \\ \hline
Конечность & 2 \\ \hline
Пах & 3 \\ \hline
Шея & 5 \\ \hline
Голова & 5 \\ \hline
Глаз & 7 \\ \hline
\end{tabular}
\end{center}
\paragraph{Торс:} попадание по торсу не вызывает никаких эффектов, кроме синяков, шрамов, шишек и потери Единиц Здоровья.
\paragraph{Конечность:} попадание по руке или ноге может вызвать Перелом или Потерю конечности.
\paragraph{— Переломы:} попадание по руке или ноге может вызвать Перелом. Когда конечность одномоментно получает Пв, превышающие \textbf{|1/5 от максимальных ЕЗ|}, она считается сломанной. Переломы считаются Опасной раной.
\paragraph{— Потеря конечностей:} если конечность одномоментно получает Пв, превышающие \textbf{|2/5 от максимальных ЕЗ|}, то она отрублена, размолота в кашу или приведена в полнейшую негодность каким-то иным образом. Повреждения, превышающие требуемое для уничтожения конечности число, теряются. Герой, потерявший конечность, находится в состоянии Кровотечения. Само собой, Потеря конечности считается Опасной раной!
\paragraph{Пах:} мужчина, получивший удар в пах, должен пройти проверку Вн против \textbf{|10 + полученные Пв|}. При провале он не может действовать и перемещаться число Очередей, равное числу, на которое провалил проверку (хотя может орать, сквернословить и совершать Быстрые действия). Все атаки по нему совершаются с Преимуществом.
\paragraph{Шея:} если шея одномоментно получает Пв, превышающие \textbf{|1/4 от максимальных ЕЗ + МВн|}, то жертва умирает. Без проверок. В случае Колющего или Проникающего удара смерть наступает от обильного кровотечения, Дробящие атаки ломают позвоночник, Рубящие удары обезглавливают.
\newline
Разумеется, это относится к живым гуманоидным существам, у которых мозг находится в голове. Умертвия, разумные грибы, трехголовые драконы и тому подобные создания вполне могут существовать и без головы (или одной из них).
\paragraph{Голова:} получивший удар в голову должен пройти проверку Вн против \textbf{|10 + полученные Пв|}. При провале жертва Теряет сознание. Если в голову нанесена Опасная рана, жертва должна преуспеть в проверке Вн против 15 или умереть.
\paragraph{Глаз} не может быть выбран для поражения Громоздким оружием. 2 и более Пв приводят к потере глаза. Существо, потерявшее все свои глаза, Ослеплено. Крупных существ ослепить сложнее — Большое существо лишается глаза при получении глазом 3 Пв, огромное — 4 Пв, гигантское — 5 Пв. Маленькие и крошечные существа лишаются глаза при получении в глаз 1 Пв. Лишние Пв наносятся в голову.
\newline
Если атака имеет Колющие или Проникающие Пв, удвойте успешно нанесенные Пв — атака поражает не только глаз, но и мозг! В этом случае получивший удар должен пройти проверку Вн против \textbf{|10 + полученные Пв|} или Потерять сознание. Проверка совершается с Помехой.
\newline
Если Колющая или Проникающая атака наносит Опасную рану в глаз, жертва должна преуспеть в проверке Вн против 15 или умереть. Проверка совершается с Помехой.
\newline
Зачастую глаза — самое уязвимое место на теле огромных монстров!
