\subsection{Зоны поражения}
Герой может выбрать для атаки любую из перечисленных зон. Если при атаке не заявлена специфическая Зона поражения, удар наносится в торс.
\newline Сложность маневра Возрастает на указанную в таблице Зон поражения, но при успехе маневра возможен дополнительный эффект.
\begin{center}
\begin{tabular}{|c|c|}
\hline
Зона поражения & Сложность \\ \hline
Торс & 0 \\ \hline
Конечность & 2 \\ \hline
Пах & 3 \\ \hline
Шея & 5 \\ \hline
Голова & 5 \\ \hline
Глаз & 7 \\ \hline
\end{tabular}
\end{center}
\paragraph{Торс:} попадание по торсу не вызывает никаких эффектов, кроме синяков, шрамов, шишек и потери Единиц Здоровья.
\paragraph{Конечность:} попадание по руке или ноге может вызвать Перелом или Потерю конечности.
\paragraph{— Переломы:} попадание по руке или ноге может вызвать Перелом. Когда конечность одномоментно получает Пв, превышающие \textbf{|1/5 от максимальных ЕЗ|}, она считается сломанной. Переломы считаются Опасной раной.
\paragraph{— Потеря конечностей:} если конечность одномоментно получает Пв, превышающие \textbf{|1/4 от максимальных ЕЗ + МВн|}, то она отрублена, размолота в кашу или приведена в полнейшую негодность каким-то иным образом. Повреждения, превышающие требуемое для уничтожения конечности число, теряются. Герой, потерявший конечность, находится в состоянии Кровотечения. Само собой, Потеря конечности считается Опасной раной!
\paragraph{Пах:} мужчина, получивший удар в пах, должен пройти проверку Вн против \textbf{|10 + полученные Пв|}. При провале он не может действовать и перемещаться число Очередей, равное числу, на которое провалил проверку (хотя может орать, сквернословить и совершать Быстрые действия). Все атаки по нему совершаются с Преимуществом.
\paragraph{Шея:} если шея одномоментно получает Пв, превышающие \textbf{|1/4 от максимальных ЕЗ + МВн|}, то жертва умирает. Без проверок. В случае Колющего или Проникающего удара смерть наступает от обильного кровотечения, Дробящие атаки ломают позвоночник, Рубящие удары обезглавливают.
\newline
Разумеется, это относится к живым гуманоидным существам, у которых мозг находится в голове. Умертвия, разумные грибы, трехголовые драконы и тому подобные создания вполне могут существовать и без головы (или одной из них).
\paragraph{Голова:} получивший удар в голову должен пройти проверку Вн против \textbf{|10 + полученные Пв|}. При провале жертва Теряет сознание. Если в голову нанесена Опасная рана, жертва должна преуспеть в проверке Вн против 15 или умереть.
\paragraph{Глаз} не может быть выбран для поражения Громоздким оружием. 2 и более Пв приводят к потере глаза. Существо, потерявшее все свои глаза, Ослеплено. Крупных существ ослепить сложнее — Большое существо лишается глаза при получении глазом 3 Пв, огромное — 4 Пв, гигантское — 5 Пв. Маленькие и крошечные существа лишаются глаза при получении в глаз 1 Пв. Лишние Пв наносятся в голову.
\newline
Если атака имеет Колющие или Проникающие Пв, удвойте успешно нанесенные Пв — атака поражает не только глаз, но и мозг! В этом случае получивший удар должен пройти проверку Вн против \textbf{|10 + полученные Пв|} или Потерять сознание. Проверка совершается с Помехой.
\newline
Если Колющая или Проникающая атака наносит Опасную рану в глаз, жертва должна преуспеть в проверке Вн против 15 или умереть. Проверка совершается с Помехой.
\newline
Зачастую глаза — самое уязвимое место на теле огромных монстров!