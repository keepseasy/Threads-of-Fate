\subsection{Детали боевой сцены}
\paragraph{Внезапное нападение.} Если герой не заметил врага или заметил в последний момент (то есть провалил проверку Наблюдательности против Скрытности врага), он захвачен врасплох. Герой не может Перемещаться и Действовать независимо от своей Реакции, а также не добавляет МЛв и БЩ к своей Зщ. Если герой дожил до своей следующей Очереди, он сражается по обычным правилам.
\paragraph{Все на одного:} сражение с несколькими противниками требует от воина высочайшего мастерства. Если герой сражается с несколькими противниками, в начале Очереди он должен выбрать, кому из них он уделяет больше внимания. Выберите число противников, равное \textbf{|1 + ММд героя|} (минимум 1). Они атакуют героя, как обычно. Все противники сверх этого числа атакуют героя с Преимуществом.
\paragraph{Ложись!:} герой может упасть, использовав Быстрое действие. Дистанционные атаки по лежащему герою совершаются с Помехой, атаки в ближнем бою по нему совершаются с Преимуществом. Лежащий герой может ползти с 1/2 Ск и атаковать в ближнем бою с Помехой. В остальном герой действует по обычным правилам.
\paragraph{Невидимки:} если герой атакует невидимого (из-за чар, укрытия или по иным причинам) противника, бросок совершается с Помехой. Если невидимого противника нет в атакуемой области, герой автоматически промахивается. Невидимые существа атакуют по правилам Внезапного нападения. Если существо невидимо благодаря Скрытности, то, атаковав, оно обнаружит себя вне зависимости от успеха атаки.
\paragraph{Перемещение через занятые области:} если герой желает пройти через область, занятую враждебным существом, он должен пройти проверку Атлетики против \textbf{|БАЗщ + Дб противника|}. В случае успеха герой передвигается через занятую область, в случае провала падает рядом с противником. Перемещение героя на этом заканчивается.
\paragraph{Трудный ландшафт.} Иногда перемещение героя затруднено густым подлеском, глубоким снегом, скользким льдом, качающейся палубой. В такой ситуации Ск героя ополовинивается.