\section{Маневры}
Во время Действия герой может совершить маневр. Некоторые маневры могут составить цепочку-комбо, требовать особого снаряжения, а так же траты Перемещения и/или Быстрого действия.
\paragraph{Маневр расходует} Действие, если в его описании не указано обратного.
\paragraph{Комбо:} многие маневры сочетаются друг с другом, иногда в весьма неожиданных комбинациях. Комбо считается единым маневром.
Чтобы составить комбо:
\begin{enumerate}
  \item Выберите 1 Базовый маневр.
  \item Выберите 0-2 разных Модифицирующих маневра.
  \item Выберите 0-1 Специальный маневр.
\end{enumerate}
Итого, максимальное число маневров в комбо – 4.

\paragraph{Два оружия ближнего боя в двух руках} обычно не лучшая идея, если у героя нет Трюка «Амбидекстр». Оружие мешает друг другу и больше путает нападающего, чем вредит цели. Тем не менее, если оба оружия Легкие, герою будет проще управиться с ними при Быстрой атаке. 

\paragraph{Два дальнобойных оружия в двух руках:} способны чуток увеличить темп стрельбы. Да, герой с «Амбидекстром» стреляет с двух рук гораздо эффективнее. Но и герой без него
\begin{itemize}
  \item Выбирает основное оружие в одной из рук, и прибавляет к нему Скорострельность вспомогательного оружия в другой.
  \item Всегда выбирает основным оружие с меньшим БПв. Координировать такой поток стрельбы совсем не просто, потери в точности и ущербе неизбежны.
  \item Может повысить Скорострельность основного оружия на число, не превышающее своего \textbf{|МЛв|} и Скорострельности вспомогательного оружия (минимум на 1).
  \item Может повысить Скорострельность основного оружия на число, не превышающее своего \textbf{|МЛв*2|} и Скорострельности вспомогательного оружия (минимум на 2), если одно из оружий Легкое.
\end{itemize}

\subsection{Совершение маневра}
\begin{enumerate}
  \item Выберите нападающего.
  \item Выберите Базовый маневр или составьте комбо на его основе.
  \item Выберите цели нападающего.
  \item Определите штрафы и бонусы к проверке нападающего. Как правило, это:
    \begin{itemize}
      \item Штраф зоны поражения;
      \item Штраф за размер предмета;
      \item Штраф и Помеха при поражении на Дальней дистанции;
      \item Помехи и Преимущества за положение цели, в т.ч. за Укрытие;
      \item Помехи и Премущества, сопутствующие некоторым маневрам;
      \item Помехи и Преимущества, связанные со свойствами воздействия нападающего и защитных средств цели.
    \end{itemize}
  \item Совершите необходимые проверки и определите последствия. 
\end{enumerate}
\begin{tcolorbox}
  Внимательно изучите возможности, которые предлагает перечень маневров. Он включает абсолютное большинство действий, которые может совершить герой со своими противниками или окружением.
  \newline Маневры предполагают сопротивление врага. Захваченные врасплох противники также сопротивляются – просто недостаточно быстро. Если цель не способна или не желает сопротивляться, то используйте рекомендации из раздела «Когда бросать кубик».
\end{tcolorbox}

\subsection{Базовые маневры}
Та самая база. Выбрав маневр, решите, будет ли герой составлять комбо. Если нет, он ограничится простыми и действенными приемами. 
\paragraph{Атака.} Герой атакует в ближнем бою.
$\bullet$ Герой выбирает цель в своем Боевом контакте и проверяет Дб.

\paragraph{Дистанционная атака.} Герой стреляет или бросает предмет в цель. 
$\bullet$ Герой выбирает цель в досягаемости оружия и проверяет Мт. 

\paragraph{Комбинированная атака.} Герой использует оружие ближнего и дальнего боя одновремено. 
$$\bullet$$ Герой выбирает выбирает две цели (или одну и ту же цель дважды) в досягаемости оружия и в своем Боевом контакте. Он получает Помеху на проверки Дб и Мт до завершения Очереди, затем проверяет Дб или Мт в соответствии с выбранной целью. 

\subsection{Модифицирующие маневры ближнего боя}
Эффект Модифицирующих маневров в комбо замещает эффекты Базовых в случае противоречий. 

\subsubsection{Быстрая атака.}
Герой жертвует точностью ради быстроты.
$\bullet$ Герой выбирает две цели (или одну и ту же цель дважды). 
\newline $\bullet$ Он проверяет Дб с Помехой по каждой из целей. 
\newline $\bullet$ Если герой имеет оружие в каждой руке и оба оружия Легкие, только одна проверка совершается с Помехой. 

\subsubsection{Сокрушительная атака.}
Герой вкладывает всю злость в один мощный удар.
$\bullet$ Герой проверяет Дб с Преимуществом. До начала его следующей Очереди все атаки по нему совершаются с Преимуществом. 

\subsubsection{Точная атака.}
Герой атакует уязвимые места противника. По крайней мере, пытается. 
$\bullet$ Шанс выпадения КУ при проверке Дб возрастает на ММд или МЛв героя (минимум на 1).

\subsection{Модифицирующие маневры дистанционного боя}
Эффект Модифицирующих маневров в комбо замещает эффекты Базовых в случае противоречий.

\subsubsection{Беглый огонь.}
\tbd Автоматический режим стрельбы – вот истинное чудо цивилизации.
$\bullet$ Герой может выбрать несколько целей, стреляя из оружия (в том числе, в двух руках) с суммарной Скорострельностью 2 и выше. Число целей не должно превышать \textbf{|1 + ММд|} героя (минимум 2) и значения Скорострельности оружия. 
\newline $\bullet$ Маневр допускает выбор различных зон поражения у разных целей. Герой совершает одну проверку Мт с Помехой и наибольшим штрафом зон поражения. Эффекты КУ применяются только к одному из противников по выбору стрелка.

\subsubsection{Верный выстрел.}
Герой замирает на пару секунд, чтобы точно не промазать.
$\bullet$ Герой совершает маневр Дистанционной атаки с Преимуществом. До начала его следующей Очереди атаки по нему совершаются с Преимуществом. 

\subsubsection{Концентрированный огонь.}
Возможность расстрелять весь рожок в какого-нибудь засранца.
$\bullet$ Маневр применим, если герой использует оружие (в том числе, в двух руках) с суммарной Скорострельностью 2 и выше. Когда Скорострельное оружие используется против одной цели, БПв оружия возрастает на 1 за каждый заряд после первого, выпущенный по ней. 
\newline Например, пистолет имеет БПв +2 и Скорострельность 3. Если герой трижды стреляет из пистолета в одну цель, БПв пистолета возрастает до +4 (то есть на 2). Если герой применит пистолет-пулемет с БПв +2 и Скорострельностью 10, по одной цели, то его БПв возрастет до +11 (то есть на 9).
\newline $\bullet$ Маневр может совмещаться с Беглым огнем. Например, из пистолета-пулемета с БПв +2 герой может выпустить 4 пули в одну цель,  1 – в другую и 5 – в третью. Первый выстрел будет иметь БПв +5, второй – БПв +2, а третий – БПв +6.

\subsection{Специальные маневры}
Эффект Специальных маневров в комбо замещает эффекты Базовых и Модифицирующих в случае противоречий.

\subsubsection{Захват.}
Герой пытается схватить и удержать противника:
\begin{itemize}
  \item Герой проверяет РДб против \textbf{|БАЗщ + [РДб противника]|}. При успехе противник не получает Пв, но становится захваченным.
  \item Если герой использует для Захвата несколько конечностей, он получает к проверке бонус, равный числу конечностей. Например, Захватывая 2 руками, герой получит +2.
  \item Пока противник захвачен, все проверки по нему в Боевом контакте получают Преимущество.
\end{itemize}
Для Маневра нужна минимум 1 свободная рука. Некоторые Трюки и виды оружия позволяют проводить Захват. В этом случае замените для нападающего РДб на Дб во всех формулах.
\newline Если у героя есть Трюк «Знаток оружия», то он вправе использовать Дб, даже являясь захваченным.

\paragraph{Нападающий может} (в том числе в ту же Очередь, когда совершен Захват):
\begin{itemize}
  \item Атаковать захваченного одноручным оружием в свободной руке в соответствии с условиями Модифицирующих маневров. Захваченный вычитает МЛв и БЩ из своей Зщ;
  \item Атаковать другие цели одноручным оружием в свободной руке в соответствии с условиями Модифицирующих маневров;
  \item Перемещаться на половину Ск. Нападающий передвигается без ограничений, если его МРз на 2+ больше МРз Захваченного; Нападающий не может перемещаться, если его МРз на 2+ меньше МРз захваченного;
  \item Сменить Зону, за которую Захватил. Герой проверяет РДб против \textbf{|БАЗщ + РДб захваченного|}. Если герой использует для Захвата несколько конечностей, он получает бонус, равный числу конечностей. Например, Захватывая 2 руками, герой получит +2 к РДб;
  \item Блокировать захваченного. До начала своей следующей Очереди захваченный не в состоянии совершать любые действия (даже Быстрые). Все, что может блокированный захваченный – вырываться;
  \item Душить захваченного. Для этого жертва должна быть захвачена за шею. Захваченный получает Пв = |К20 + [РДб нападающего] – [РДб захваченного] – [МВн захваченного]|. Если в ходе удушения ЕЗ захваченного достигают 0, он теряет сознание. Отрицательный МВн прибавится к Пв от удушения;
  \item Метнуть захваченного. Герой не может метать цели, вес которых превышает его \textbf{|[Комфортную нагрузку]*2|}.
    \newline Нападающий считает захваченного Громоздким Импровизированным Метательным оружием с БПв = \textbf{|МРз|} захваченного, Ближней дистанцией = \textbf{|МСл + МЛв|} нападающего, КУ = \textbf{|20 – МРз|} захваченного и Дробящими Пв. Захваченного нельзя метнуть на Дальнюю дистанцию;
  \item Отпустить захваченного.
\end{itemize}
Смена Зоны захвата и метание захваченного требуют наличия в комбо атак, не израсходованных на непосредственно Захват, а удушение, блокирование и отпускание – нет.
\newline После того, как герой успешно совершил Захват, в свои последующие Очереди он вправе выбирать другие Специальные маневры.

\paragraph{Захваченный может}:
\begin{itemize}
  \item Атаковать нападающего с Помехой одноручным Легким оружием;
  \item Атаковать любые цели с Помехой одноручным Легким оружием, если МРз захваченного больше МРз нападающего;
  \item Творить феномены, если соблюдет необходимые условия. Проверки, необходимые для успеха феномена, совершаются с Помехой;
  \item Вырываться. Захваченный расходует Действие и проверяет РДб, Атлетику (Сл, Лв), Сл или Лв против \textbf{|БАЗщ + РДб + МРз|} нападающего. При успехе он освобождается;
  \item Достать Легкое оружие. Захваченный может отказаться от Перемещения, чтобы достать оружие, несмотря на то, что фактически обездвижен;
  \item Совершать Быстрые действия;
  \item Перемещаться, если его МРз на 2+ больше МРз нападающего. 
\end{itemize}
\begin{tcolorbox}
  Если герой планирует применять болевые приемы, обратите внимание на Трюк «Костолом».
\end{tcolorbox}

\subsubsection{Касание}
Легкое прикосновение, с трудом ощутимое сквозь одежду и совершенно незаметное для облаченных в броню.
\begin{itemize}
  \item Маневр не наносит Пв. Если герою требуется дотронуться до сопротивляющейся цели, он должен преуспеть в проверке РДб против \textbf{|БАЗщ + МЛв|} цели;
  \item \tbd Герой должен выбрать открытый участок тела в качестве зоны поражения, если касание предполагает контакт с телом. Чем выше БД, тем сложнее отыскать такой участок!
\end{itemize}

\subsubsection{Разоружение}
Герой пытается выбить предмет из рук противника или сбить предмет с его тела.
\begin{itemize}
  \item Маневр не наносит Пв. Герой должен преуспеть в проверке Дб против \textbf{|БАЗщ + Дб + БЩ|} цели;
  \item Герой получает штраф к Дб за размер предмета (сверьтесь с таблицей ниже);
  \item Если противник держит предмет в 2 руках, герой получает -2 к Дб;
  \item Маневр не может повредить предмету, однако хрупкие предметы могут разбиться при падении;
  \item При успехе предмет падает на расстоянии от цели, не превышающем МЛв нападающего. Нападающий выбирает точку, в которую упадет предмет;
  \item Если хотя бы одна рука героя свободна, он может выхватить предмет или оружие из рук противника, применив РДб. Если герой использует обе руки, он получает +2 к РДб. При успехе предмет оказывается в его руках;
  \item Когда герой выбирает целью маневра щит, то БЩ не учитывается при проверке. Щиты основательно закреплены на руке. Чтобы сорвать щит понадобится дополнительная проверка Сл, Лв или Атлетики против \textbf{|10 + БЩ|}.
\end{itemize}

\subsubsection{Сбить с ног}
Герой настоятельно предлагает противнику прилечь и отдохнуть.
\begin{itemize}
  \item Маневр не наносит Пв. Чтобы сбить противника с ног, герой должен успешно проверить Дб против \textbf{|10 + Дб + БЩ + МРз|} противника;
  \item Герой получает штраф -2 за зону поражения (ноги);
  \item Если существо четвероногое, герой получает дополнительные -2;
  \item Если существо передвигается на брюхе, как змеи или гигантские слизни, герой получает дополнительные -4;
  \item Маневр выполняется только Длинным оружием, или при помощи РДб.
\end{itemize}

\subsubsection{Сломать снаряжение}
Герой расточительно портит снарягу противника.
\begin{itemize}
  \item Чтобы нанести ущерб предмету в руках или на теле цели, герой должен проверить Дб или Мт против \textbf{|БАЗщ + Дб + БЩ|} противника.
  \item Предмет получает Пв, равные величине успеха героя. Если герой пытается разбить щит, БЩ не учитывается в сложности проверки.
  \item Герой получает штраф к Дб или Мт за размер предмета (сверьтесь с таблицей ниже).
\end{itemize}
\begin{tcolorbox}
  Большинство предметов, созданных для боевых действий, имеет высокую Прочность и весьма устойчиво к Повреждениям. Этого нельзя сказать обо всех остальных предметах – вряд ли их хватит хотя бы на пару ударов.
\end{tcolorbox}

\begin{center} \begin{tabular}{|p{7cm}|p{3cm}|} \hline
  \textbf{Оружие/предмет} & \textbf{Штраф к ДБ при разоружении/поломке} \\ \hline
  Доспехи, закрывающие большую часть тела (БД 7+), башенный щит. & 0 \\ \hline
  Громоздкое и Длинное двуручное оружие, доспехи, закрывающие значительную часть тела (БД 4-6). & -1 \\ \hline
  Громоздкое или Длинное двуручное оружие, большой щит, легкие доспехи (БД 1-3). & -2 \\ \hline
  Двуручное оружие, Громоздкое или Длинное Универсальное оружие, щит, большой рюкзак. & -3 \\ \hline
  Универсальное оружие, Громоздкое или Длинное одноручное оружие, рюкзак. & -4 \\ \hline
  Одноручное оружие, кулачный щит, широкий ремень. & -5 \\ \hline
  Легкое оружие, скрученная газета, шапка, барсетка. & -6 \\ \hline
  Кастет, перчатка, наруч, пузырек, узкий ремень. & -7 \\ \hline
  Перстень, браслет, ключ. & -8 \\ \hline
\end{tabular} \end{center}

\subsubsection{Толчок}
Иногда требуется лишь подтолкнуть, а гравитация сделает остальное.
\begin{itemize}
  \item Маневр не наносит Пв. Герой отталкивает цель на 1 метр прямо от себя, успешно проверив Дб, Сл или Атлетику (Сл) против \textbf{|БАЗщ + Дб - МРз|} цели;
  \item Если маневр совмещается с Разбегом, герой отталкивает цель на 1 метр прямо от себя за каждую единицу успеха;
  \item Герой не может толкать существ и предметы, вес которых превышают его максимальную нагрузку.
\end{itemize}

\subsubsection{Финт}
Герой отвлекает противника обманным выпадом или жестом.
\begin{itemize}
  \item Маневр не наносит Пв. Герой проверяет Общение (Мд, Об) или Владение оружием/Рукопашный бой против \textbf{|10 + [Владение оружием/Рукопашный бой(Ин, Мд)]|} цели;
  \item При успехе следующая атака (героя и любого его союзника) по цели Финта совершается с Преимуществом. Эффект длится до конца следующей Очереди героя, применившего Финт.
\end{itemize}

\subsection{Независимые маневры}
Условия выполнения Независимых маневров указаны в описании. Они дополняют эффекты прочих маневров.

\subsubsection{Выжидание}
Герой ожидает провоцирующего события или действия, прежде чем что-либо предпринимать.
\begin{itemize}
  \item Любое Действие может быть совмещено с этим маневром и выполнено, как Выжидающее. Например: Захватить, если враг потянется за оружием, Атаковать, если враг начнет активацию феномена, нажать кнопку, \textit{если} включится сирена воздушной тревоги и т.д.
  \item В этом случае герой не выполняет выбранные маневры, пока не произойдет провоцирующее событие или действие;
  \item Чтобы среагировать раньше, чем противник фактически совершит провоцирующее действие, герой должен успешно проверить Рц против \textbf{|10 + [Рц противника]|}. 
\end{itemize}

\subsubsection{Защитная стойка}
Герой полностью сосредочен на обороне.
\begin{itemize}
  \item Все атаки по герою до начала его следующей Очереди совершаются с Помехой;
  \item Лежащий или сидящий герой тможет выбрать этот маневр;
  \item Герой не может выбирать Модифицирующие и Специальные  маневры, если выбрал этот. 
\end{itemize}

\subsubsection{Огонь на подавление}
Герой упоенно поливает огнем пятачок земли, не заботясь о точности.
\begin{itemize}
  \item Он может использовать маневр, если снаряжен оружием (в том числе, оружием в двух руках) с суммарной Скорострельностью 5 и выше;
  \item Он не вправе совмещать маневр с каким-либо другим;
  \item Герой выбирает число смежных областей площадью 1х1 метр, не превышающее Скорострельности оружия. Он проверяет Мт с 2 Помехами и Осечкой 5. Если оружие уже имеет Осечку, используется большая;
  \item Он расходует Действие и Перемещение;
  \item Все существа, находящиеся в выбранных областях, получают Пв, если герой поразил их Зщ. Герой всегда расходует число зарядов, равное маскимальной Скорострельности своего вооружения. Он не может выбирать зоны поражения;
  \item Эффекты КУ не применяются при использовании этого маневра.
\end{itemize}

\subsubsection{Прицеливание}
Герой терпеливо ждет удачного момента для выстрела.
\begin{itemize}
  \item Маневр совмещается только с Дистанционной атакой.
  \item Герой может выбрать противника, Прицелится в него и сместить свою Очередь на 1-3 вниз, то есть действовать после менее быстрого героя или статиста. Если в текущем Круге герой действовал последним, то следующий пропуск Очереди смещает его ход на следующий Круг.
  \item Герой, сместивший свою Очередь на 1, проверяет Мт с Преимуществом. Враги атакуют его с Преимуществом, пока он Прицеливается.
  \item Герой, сместивший свою Очередь на 2, проверяет Мт с 2 Преимуществами. Враги атакуют его с 2 Преимуществами, пока он Прицеливается.
  \item Герой, сместивший свою Очередь на 3, проверяет Мт с 2 Преимуществами и игнорирует Помеху за Дальнюю дистанцию. Враги атакуют его с 2 Преимуществами, пока он Прицеливается.
  \item После совершения атаки герой снова может действовать в соответстви со значением своей Рц.
  \item Если во время Прицеливания герой одномоментно теряет ЕЗ, превышающие его \textit{|Вл|}, он  теряет все бонусы Прицеливания. 
\end{itemize}

\subsubsection{Провокация}
При помощи оскорбительных фразочек и еще более оскорбительных жестов герой привлекает внимание врага.
\begin{itemize}
  \item Цель должна видеть или слышать героя;
  \item Маневр может совмещаться с любым другим и расходует Быстрое действие;
  \item Успешно проверив Общение (Об) против \textbf{|10 + [Вл цели]|}, герой разъяряет противника и становится его следующей целью;
  \item Противник использует лучшие из своих способностей, чтобы напасть на героя. 
\end{itemize}
Если жертва не слышит героя, не видит его или не понимает его язык, проверка совершается с Помехой. Если героя и не видят, и не слышат, Провокация не работает.
\newline Если противник осознает, что неспособен навредить провокатору в течение своей Очереди, Провокация не работает.
\begin{tcolorbox}
  Провокация имеет смысл в случаях, когда враги могут атаковать или как-то еще наказать оскорбившего их героя без чрезмерных затруднений. Выманить противника из дота или танка Провокацией не получится, хотя отвлечь огонь пулеметчика или танкиста на себя – вполне.
\end{tcolorbox}

\subsubsection{Натиск}
Герой наседает на противника в ближнем бою и агрессивно теснит его, вынуждая отойти.
\begin{itemize}
  \item Маневр не наносит Пв, расходует Действие и Перемещение. Герой проверяет Дб против \textbf{|10 + [Дб цели]|}. При успехе цель отступает от героя на число метров, равное величине его успеха. Цель не может отступить на число метров, превышающее ее Ск;
  \item Герой двигается за целью, стремясь сохранить Боевой контакт, если это возможно и безопасно для него;
  \item Если Ск героя меньше, чем Ск цели, цель все равно должна отступить на число метров, равное величине успеха Дб героя. При этом цель может выйти из Боевого контакта героя;
  \item Герой выбирает, куда и по какой траектории отступит цель;
  \item Если цель не может отступить на требуемое число метров (ввиду недостаточно высокой Ск, либо из-за отсутствия места для отступления), она получает Помеху на Дб и Мт до конца своей следующей Очереди;
    \newline Цель не обязана отступать, если это приведет к заведомо гибельным последствиям – падению со скалы, погружению в зыбучий песок, глубокую воду и т.д. В этом случае она получает Помеху на Дб и Мт, как описано выше. Цель должна отступить, если это сделает ее положение потенциально опасным или просто менее выгодным. Она отойдет в неглубокий ручей с сильным течением, под медленно вращающиеся лопасти вентилятора, на крутую лестницу, гнилые помостки, подтаявший, но с виду достаточно прочный лед и т.д.
  \item Герой не может выбирать Базовые, Модифицирующие и Специальные  маневры в дополнение к этому.
\end{itemize}
\begin{tcolorbox}
  Цель может и должна отступать, даже если она двигалась в предшествующую Натиску очередь. 
\end{tcolorbox}

\subsubsection{Разбег}
Герой обрушивается на врага с воинственным криком и инерцией разгона.

\begin{itemize}
  \item Маневр расходует Перемещение;
  \item Герой должен преодолеть по прямой расстояние в интервале от своей \textbf{|Ск+1|} до своей \textbf{|Ск*2|}. Сделав это, он получает Преимущество на следующую проверку Дб, ММт, или проверку, использующую МСл;
  \item Все атаки по герою совершаются с Преимуществом до начала его следующей Очереди;
  \item Этот маневр полезен даже в небоевых ситуациях.
\end{itemize}
