\section{Маневры}
Во время Действия герой может совершить Маневр. Перечисленные ниже Маневры может выполнить любой герой, если он снаряжен соответствующим образом. Некоторые Маневры могут совмещаться друг с другом или дополнительно требовать траты Перемещения или Быстрого действия.
\paragraph{Совершение Маневра:}
\begin{enumerate}
\item Выберите цель. Герой не может поразить цель за пределами досягаемости оружия или заклинания.
\item Определите штрафы и бонусы к проверке. Как правило, это штраф зоны поражения или штрафы за размер предмета при Разоружении или Поломке оружия. Также перед совершением проверки определите, есть ли у героя Преимущество или Помеха на нее.
\item Совершите необходимую проверку и определите последствия. Получает цель Маневра Повреждения, или герой промахивается, нанесен ли Критический Удар и т. д. Например, успех при Разоружении или Поломке оружия лишит противника оружия или уничтожит (или повредит) ценный предмет на его теле, успех при Захвате даст герою возможность справиться с противником, не нанося Повреждений, или просто сломать ему хребет.
\end{enumerate}
\subsection{Атака}
Герой атакует в ближнем бою. Герой может атаковать противника в 1 метре от себя (или в 2 метрах, если его оружие Длинное) и должен совершить проверку Дб против \textbf{|Зщ цели|}. Количество нанесенных Пв равно величине успеха проверки. Неподвижные цели герой атакует с Преимуществом. Если в результате подсчета получается 0 или меньше, то удар соскользнул с доспеха или просто не достиг цели. Герой может понизить свою Дб на любое число (минимум до 0), если желает нанести удар не в полную силу.
\subsection{Атака с разбега}
Герой преодолевает по прямой расстояние в интервале от своей \textbf{|Ск +1|} до своей \textbf{|Ск × 2|} и атакует с Преимуществом. До начала его следующей Очереди все атаки по нему совершаются с Преимуществом. Атака с разбега не может совмещаться с Быстрой атакой, Выжиданием и Плетением чар, но может совмещаться с другими маневрами. Если герой совершает Сокрушительную атаку с разбега, он получает
2 Преимущества. Но так же и враги при атаках по нему до начала его следующей Очереди!
\newline
Перемещение героя уже учтено в этом Маневре, то есть при его использовании герой может преодолеть расстояние, не превышающее его \textbf{|Ск × 2|}. Герой не может совершить Перемещение и затем использовать Атаку с разбега!
\subsection{Быстрая атака}
Герой совершает 2 атаки с Помехой. Если герой имеет оружие в каждой руке и оба оружия Легкие, только одна атака совершается с Помехой. Каждой из этих атак герой может совершать Дистанционную атаку, Захватывать, Ломать снаряжение, Разоружать, Сбивать с ног, Толкать или выполнять Финт.
\subsection{Выжидание}
Герой ожидает совершения некоего действия кем-то из окружающих. Любой иной маневр может быть выполнен как Выжидающий.
\newline
Очередность событий при этом определяется Реакцией. Чтобы
опередить противников, герой должен преуспеть в проверке
Реакции против \textbf{|10 + Рц статиста|}.
\subsection{Захват}
Для маневра нужна как минимум 1 свободная рука. Герой проходит проверку Дб (Рукопашный бой) против \textbf{|БАЗщ + Дб(Рукопашный бой) противника|}. При успехе противник не получает Повреждений, но становится Захваченным. Маневр может сочетаться с Атакой с разбега, Быстрой атакой (это имеет смысл, если герой пытается захватить сразу двух
противников) и Сокрушительной атакой. Если Захват успешен, все атаки в ближнем бою против схваченного получают Преимущество, пока он не освободится. Некоторые Трюки и виды оружия позволяют проводить Захват при помощи Владения оружием. В этом случае замените во всех формулах Рукопашный бой на Владение оружием для инициатора Захвата. Если у героя есть Трюк «Знаток оружия», то он может использовать во всех формулах Владение оружием, даже являясь Захваченным!
\begin{tcolorbox}
Если герой использует для хватания или удерживания две руки, он получает +2 ко всем проверкам захвата. Если у героя больше двух рук, то он получает еще +2 к проверкам за каждую пару рук.
\end{tcolorbox}
\paragraph{Схвативший может} (в том числе в ту же Очередь, когда совершен Захват) cовершать любые действия со следующими дополнениями и исключениями:
\begin{itemize}
\item[--] Атаковать схваченного любым одноручным оружием (даже дистанционным). Схваченный не добавляет МЛв и БЩ к своей Зщ.
\item[--] Атаковать другие цели по обычным правилам, если у него есть свободная рука с одноручным оружием ближнего боя или одноручным дистанционным оружием.
\item[--] Перемещаться на половину своей Ск. Схвативший передвигается без ограничений, если на 2 категории и более превышает Размером схваченного.
\item[--] Сменить зону, за которую удерживает противника. Герой должен пройти проверку Захвата по новой выбранной Зоне. В случае провала, он продолжает держать цель в Захвате, но зона, за которую он держит цель не меняется.
\item[--] Блокировать противника. До начала своей следующей Очереди схвативший не дает схваченному выполнять любые действия (даже говорить). Все что может делать схваченный — это пытаться вырваться. Схвативший не может перемещаться и совершать никаких действий, кроме Быстрых.
\item[--] Душить схваченного. Для этого противник должен быть схвачен за шею. Схвативший совершает проверку Дб(Рукопашный бой) против \textbf{|Дб(Рукопашный бой) + МВн схваченного|}. Величина успеха равна нанесенным Пв. Если в ходе удушения ЕЗ схваченного достигает 0, он Теряет сознание. Обратите внимание, что отрицательный МВн прибавляется к Пв от удушения.
\item[--] Бросить схваченного. Герой не может бросать существ и предметы, вес которых превышают его \textbf{|комфортную нагрузку × 2|}. Дальность броска не может быть дальше, чем \textbf{|МСл+МЛв|} бросающего. Увеличьте максимальное расстояние броска в 2 раза за каждую категорию размера, на которую бросающий больше бросаемого, и уменьшите в 2 раза за каждую категорию, на которую бросающий меньше бросаемого.
\newline Для того, чтобы бросить схваченного в конкретную точку, бросающей должен совершить проверку Меткости для Метательного оружия с БПв 0. При падении бросаемый получает столько Пв, сколько метров пролетел. Если точке, куда совершен бросок, находится другое существо, то бросаемый падает ему под ноги.
\newline Для того, чтобы попасть по другому существу, нужно совершить проверку Меткости с Помехой для метательного оружия с БПв равным \textbf{|МРз + БД бросаемого|}.
\item[--] Отпустить схваченного.
\end{itemize}
\paragraph{Схваченный может} совершать перечисленные действия:
\begin{itemize}
\item[--] Проводить атакующие Маневры по схватившему с Помехой. Для атаки может использоваться только Легкое оружие, шипы на доспехе, а также кулаки (зубы, когти, клювы, щупальца и т. п.). В остальном атаки проводятся по обычным правилам.
\item[--] Творить Феномены, если может соблюсти необходимые для этого условия (что бывает затруднительно, если героя держат). Все проверки, необходимые для успеха Феномена, совершаются с Помехой.
\item[--] Пытаться вырваться. Чтобы вырваться, схваченный должен пройти проверку Дб (Рукопашный бой), Атлетики (Сл, Лв), Сл или Лв против \textbf{|БАЗщ + Дб (Рукопашный бой) схватившего|}. Попытка вырваться считается Действием.
\item[--] Достать Легкое оружие. Схваченный может отказаться от Перемещения, чтобы достать оружие, несмотря на то, что фактически обездвижен.
\item[--] Совершать Быстрые действия.
\end{itemize}
\paragraph{Схваченный не может} совершать следующие действия:
\begin{itemize}
\item[--] Передвигаться, если только не превышает схватившего размером на 2 или больше. Также существа, размером превышающие схватившего, могут атаковать и другие цели (без Помехи), кроме схватившего. 
\end{itemize}
Схвативший/схваченный автоматически получают Пв, если на доспехе схватившего/схваченного есть шипы. Пв равны разнице бонуса доспехов схватившего и схваченного. Например, если герой в кольчужной рубахе (БД +3) схватил противника в шипованном кольчато-пластинчатом доспехе (БД +6), то герой получит 3 Пв (противник, само собой, не получает Пв). Шипы наносят Пв в начале каждой Очереди схватившего/схваченного, пока схвативший не отпустит противника. Для того чтобы удерживать противника, покрытого шипами, схвативший должен пройти проверку Вл против \textbf{|10 + полученные Пв|}. При провале он отпускает его в начале своей следующей Очереди!
\subsection{Защитная стойка}
Герой сосредочен на обороне. Все атаки по нему до начала его следующей Очереди совершаются с Помехой. Лежащий на земле герой также может выбирать этот маневр!
\subsection{Провокация}
Этот Маневр может сочетаться с любым другим и требует Быстрого действия. При помощи оскорбительных фразочек и еще более оскорбительных жестов герой привлекает к себе внимание врага. Совершив проверку Общения (Об) против \textbf{|10 + Вл цели|}, герой провоцирует противника и становится целью его следующей атаки. В некоторых ситуациях герою точно не обойтись без Луженой глотки!
Если жертвы не слышат героя или не видят его, или не понимают язык, на котором он говорит, проверка совершается с Помехой. Если героя и не видят, и не слышат, Провокация не сработает! Маневр действует только на одну цель одновременно, но если у героя больше одного Быстрого действия, он может Провоцировать несколько раз за Очередь, привлекая внимание разных противников!
\begin{tcolorbox}
Провокация действует только в боевых сценах (то есть, когда бой уже начался) и имеет смысл в тех случаях, когда враги могут атаковать оскорбившего их героя. В противном случае они выберут другую цель. 
\end{tcolorbox}
\subsection{Разоружение}
Чтобы выбить предмет из рук противника, герой должен пройти проверку Дб против \textbf{|10 + БЩ + Дб противника|}. Герой получает штраф к Дб за размер предмета-цели (сверьтесь с таблицей ниже). Если противник держит предмет в 2 руках, герой получает дополнительные -2 к проверке. Маневр не может повредить предмету-цели, однако хрупкие предметы могут разбиться, упав на землю. В случае успеха маневра предмет падает на землю на расстоянии от разоруженного, не превышающем МСл или МЛв разоружающего. Разоружающий выбирает точку, в которую упадет предмет. Уменьшите расстояние в 2 раза, если выбитый из рук предмет Громоздкий или Длинный. Если в результате получается 0, предмет падает под ноги разоруженного.
\newline
Если хотя бы одна рука героя свободна, он может выхватить предметы и оружие из рук противника, используя при Разоружении Рукопашный бой! Если герой использует обе руки, он получает +2 к проверке. В остальном он действует по обычным правилам Разоружения. При успехе Маневра оружие или предмет оказывается в его руках. Если герой выбирает целью Маневра щит, то БЩ не учитывается при проверке. Однако зачастую щиты основательно закреплены на руке, и, чтобы сорвать с нее щит, понадобится дополнительная проверка Сл, Лв или Атлетики (Сл, Лв) против |10 + БЩ|.
\subsection{Сбить с ног}
Чтобы сбить противника с ног, герой должен пройти проверку Дб против \textbf{|БАЗщ + БЩ + Дб противника|}.
\newline
Герой получает штраф -2 за область поражения (ноги). Если существо стоит на 4 ногах, герой получает дополнительные -2. Если существо передвигается на брюхе, как люди-змеи или гигантские слизни, герой получает дополнительные -4 к проверке! Маневр может выполняться лишь Длинным оружием или при помощи Навыка «Рукопашный бой».
\subsection{Сломать снаряжение}
Чтобы нанести Повреждения предмету в руках или на теле противника, герой должен пройти проверку Дб или Мт против \textbf{|10 + БЩ + Дб противника|}.
\newline
Герой получает штраф к Дб или Мт за размер предмета-цели (сверьтесь с таблицей ниже). За каждую 1, на которую герой прошел проверку, он наносит 1 Пв предмету. Если герой пытается разбить щит, БЩ не учитывается в сложности проверки. 
\begin{center}
\begin{tabular}{|p{10cm}|p{4cm}|}
\hline
Оружие/предмет & Штраф к Доблести атакующего при разоружении/поломке \\ \hline
Доспехи, закрывающие большую часть тела (БД 7+), башенный щит & 0 \\ \hline
Громоздкое и Длинное двуручное оружие, доспехи, закрывающие значительную часть тела (БД 4..6) & -1 \\ \hline
Громоздкое или Длинное двуручное оружие, большой щит, легкие доспехи (БД 1..3) & -2 \\ \hline
Двуручное оружие, Громоздкое или Длинное Универсальное оружие, щит, большой рюкзак & -3 \\ \hline
Универсальное оружие, Громоздкое или Длинное одноручное оружие, рюкзак & -4 \\ \hline
Одноручное оружие, баклер, широкий ремень & -5 \\ \hline
Легкое оружие, скрученный свиток, шапка, кошель на поясе & -6 \\ \hline
Кастет, перчатка, наруч, фиал с зельем, узкий ремень & -7 \\ \hline
Перстень, браслет, ключ & -8 \\ \hline
\end{tabular}
\end{center}

\subsection{Сокрушительная атака}
Герой атакует с Преимуществом. До начала его следующей Очереди все атаки по нему совершаются с Преимуществом. Сокрушительная атака не может совмещаться с Быстрой атакой, но может совмещаться с Атакой с разбега, Захватом, Поломкой оружия, Разоружением, Сбиванием с ног, Толчком или Финтом.
\subsection{Толчок}
Герой отталкивает противника на 1 метр прямо от себя. Чтобы сделать это, герой должен пройти проверку Дб против \textbf{|БАЗщ + Дб противника|}.
\newline
Если Маневр применяется одновременно с Атакой с разбега, герой отталкивает противника на 1 метр за каждую единицу, на которую прошел проверку. Герой не может оттолкнуть противника на большее число метров, чем преодолел сам в эту Очередь, но всегда толкает минимум на 1 метр в случае успеха проверки.
Увеличьте максимальное расстояние толчка в 2 раза за каждую категорию размера, на которую толкающий больше толкаемого, и уменьшите в 2 раза за каждую категорию, на которую толкающий меньше толкаемого.
\newline
Герой не может толкать существ и предметы, вес которых превышают его максимальную нагрузку.
\subsection{Финт}
Герой отвлекает противника. Герой должен пройти проверку Общения (Мд, Об) или Владения оружием/Рукопашного боя против \textbf{|10 + Владение оружием/Рукопашный бой противника (Ин, Мд)|}. Если проверка успешна, следующая атака (героя и любого его союзника) по этому противнику совершается с Преимуществом. Эффект длится до конца следующей Очереди героя, применившего Финт.
\subsection{Дистанционная атака}
Герой стреляет или бросает предмет в цель. Если цель находится на Дальней дистанции, атака совершается с Помехой. В Неподвижные цели герой стреляет с Преимуществом.
\newline
Оружие, не предназначенное для метания, может быть брошено с Помехой. Максимальная дистанция для броска такого оружия — 5. Подразумевается бросок, наносящий цели Повреждения, а не его максимальная дальность в принципе.
\newline
Если для вашей истории важно, на какую предельную дистанцию герой может метнуть предмет (например, при метании гранаты), то она равна \newline{|Сл героя × 3 — вес предмета в килограммах|} метров. Разделите результат на вес предмета в килограммах (минимум 1), если он не предназначен для метания. Например, герой с 10 Силой сможет метнуть двуручный топор весом в 5 кг на (10 × 3 — 5) ÷ 5 = 5 метров. Используйте Мт для попадания в некую конкретную область. В подобных случаях, пятачок земли 1 × 1 метр имеет БАЗщ 10. Разумеется, при броске на предельную дистанцию атака совершается с Помехой.
\paragraph{Укрытия.} При использовании маневра Дистанционная атака следует учитывать Укрытие, за котором спряталась цель атаки. Они делятся на 2 типа: мягкое (кусты, высокая трава, куча хвороста) и твердое (камни, стены, башенные щиты). При стрельбе по противнику в мягком укрытии герой получает 1 Помеху, при стрельбе по противнику в твердом укрытии — 2 Помехи. В случае промаха герой поражает укрытие. Укрытия могут быть Повреждены и уничтожены.
\paragraph{Дистанционные атаки имеют 2 типа дистанций: Ближняя и Дальняя.} Эти дистанции указаны в статистиках дальнобойного оружия и описании феноменов. Если цель находится за пределами Дальней дистанции, герой не может поразить ее. Атаки на Дальней дистанции совершаются с Помехой. Совершая Дистанционную атаку, герой должен сделать проверку Меткости против \textbf{|Зщ цели|}. Герой может понизить свою Мт на любое число (вплоть до 0), если желает поразить цель вскользь.
\paragraph{Дистанционные атаки в ближнем бою:} герой совершает Дистанционные атаки с Помехой, если в 1 метре от него находится противник. Герой совершает Дистанционные атаки с Помехой, если его цель находится в ближнем бою, в котором он не принимает участия.
\paragraph{Перемещение и Дистанционные атаки:} если в свою Очередь герой перемещается на расстояние, превышающее 1 метр, и стреляет, проверка Меткости совершается с Помехой. Обратите внимание, что герой может выстрелить без Помехи до Перемещения.
\subsection{Прицеливание}
Выбрав Маневр Дистанционной атаки, герой может объявить, что он Прицеливается. В боевой сцене герой может сместить свою Очередь на 1—3 Очереди вниз, то есть действовать после менее быстрого героя или статиста. Если в текущем Круге герой действовал последним, то следующий прокуск Очереди перемещает его ход на следующий Круг в Боевой Сцене. Герой, сместивший свою Очередь на 1 вниз, получает Преимущество при проверках Меткости, а враги получают Преимущество, атакуя его. Герой, сместивший свою Очередь на 2 вниз, получает 2 Преимущества, а враги получают 2 Преимущества, атакуя его. Если герой сместил свою Очередь на 3 вниз, он получает 2 Преимущества и игнорирует Помеху при стрельбе на Дальнюю дистанцию. Враги получают 2 Преимущества, атакуя его.
\newline
Если во время Прицеливания герой одномоментно получает Пв, равные или превышающие его Вл, он теряет все бонусы Прицеливания и должен начинать заново. Прицеливание не может сочетаться с Быстрой атакой, Беглым Огнем и Огнем на подавление. Герой может прицеливаться только в цель, которую видит.
\subsection{Беглый огонь}
Герой может выбрать несколько целей для стрельбы из оружия со Скорострельностю выше 1. Количество целей не должно превышать \textbf{|ММд+1|(минимум 2)}. Герой выбирает точное количество выстрелов, произведенных оружием в каждую цель. Это число не должно превышать Ск оружия, но может быть меньше. Например, если герой с ММд 2 стреляет из оружия с Ск3, он может выбрать 4 цели, но Ск оружия не позволяет распределить пули по всем целям, поэтому он может поразить только троих.
\newline
Совершите одну проверку Мт для всех выстрелов (отдельные проверки для каждого выстрела возможны по предварительной договоренности игроков и мастера) с количеством помех равным \textbf{|количеству целей-1(минимум 0)|}. Когда герой атакует несколько целей и выбирает различные Зоны поражения, для всех бросков используется наибольший штраф, если только отдельные проверки Мт по каждой из целей не оговорены заранее.
\newline
Эффекты КУ применяются только к одному из противников по выбору стрелка.
\subsection{Огонь на подавление}
Этот маневр может использовать герой с любым оружием со Скорострельностью 5 и выше. Герой поливает свинцом небольшой пятачок земли, не заботясь о точности. Герой может выбрать число смежных областей площадью 1х1 метр, не превышающее Скорострельность оружия, и совершить атаку с БПв оружия, модифицированным за дальность в случае необходимости. Атака совершается с 2 Помехами и Осечкой 5. Если оружие уже имеет параметр Осечки, используйте наибольший. Все существа, находящиеся в выбранных областях, получают Повреждения, если герой поразил их Защиту. В этом режиме оружие расходует 10 зарядов (даже если его фактическая Скорострельность ниже). Ведущий Огонь на подавление герой не может выбирать Зоны поражения.
\newline
Эффекты КУ не применяются при использовании этого маневра.