\section{Кавалерия в бою}
\tbd литературная вставка.
\paragraph{Атаки наездника:} он получает Преимущество на Дб, если атакует цели меньшего размера, чем его скакун.
\paragraph{Атаки по наезднику:} атакуя наездника в Боевом контакте, нападающий, не снаряженный Длинным оружием, получает Помеху на Дб, если МРз скакуна превышает его МРз. Возможна ситуация, когда из-за размеров скакуна наездник окажется вне его Боевого контакта.
\paragraph{Атаки по скакуну:} совершаются по обычным правилам.
\paragraph{Атаки скакуна:} во время Действия скакун может совершать любые маневры, доступные благодаря его Навыкам и снаряжению. Скакун не обязан заявлять такой же маневр, как и наездник. Исключение составляет Разбег.
\paragraph{Действие и Перемещение скакуна:} скакун Действует и Перемещается в ту же Очередь, что и наездник.
\paragraph{Проверки скакуна:} Проверки Дб, Мт и Вл скакуна могут быть заменены Обращением с животными наездника. Скакун использует собственные параметры, если они лучше.
\paragraph{Реакция наездника} равна \textbf{|([Рц скакуна]+[Рц героя])/2|}.
\paragraph{Лечение скакуна:} скакун Отдыхает в Антракте и Интерлюдиях по тем же правилам, что и герои, до тех пор, пока это не противоречит контексту (вряд ли конь или ездовое кобо смогут отдохнуть в кабаке или бане, хотя бывает всякое). Обращение с животнымиЭ заменяет МедицинуЭ для скакуна во всех отношениях.

\section{Дорожные войны}
\tbd литературная вставка.
\paragraph{Не дрова везешь!:} из-за тряски пассажиры ТС совершают Активные проверки с Помехой. 
\paragraph{Отвлекать водителя воспрещается:} водителю проблематично заниматься чем-то, кроме управления ТС. Все проверки водителя, не связанные с управлением ТС, совершаются с 2 Помехами. Водитель не может использовать Двуручное оружие.
\paragraph{Под откос:} если машина переворачивается или врезается в препятствие, все пассажиры (включая водителя) получают Дробящие Пв, равные \textbf{|30 - [Эксплуатация водителя] - БД|}. Если ТС не оснащено ремнями безопасности, или все их проигнорировали, удвойте полученные пассажирами Пв. 
\newline Те из пассажиров, кто не был пристегнут, могут избежать Повреждений, совершив проверку Атлетики (Лв) против \textbf{|Пв, которые герой должен получить|}.
\paragraph{Борт к борту:} водитель может использовать транспортное средство, как оружие ближнего боя. Дб водителя равна \textbf{|Эксплуатация(МЛв) + Прч ТС|}. При провале маневра транспортное средство получает Пв, равные величине провала. Это не всегда означает тараны и удары бортами - в случае мопедов, мотоциклов и малолитражек водитель заманивает противника на обочину или в кювет.
\paragraph{Погоняем?:} герой может оторваться от преследования, или наоборот, догнать кого-то. Он проверяет Эксплуатацию против \textbf{|10 + [Эксплуатация(Лв) противника] + [Опасность местности]|}. В случае успеха герой добивается желаемого. При любом исходе ТС потребуется проверка Износа в конце Сцены.
