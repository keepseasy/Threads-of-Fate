\section{Наездники}
\paragraph{Атаки наездника:} герой получает Преимущество в ближнем бою, если атакует существ, меньших по размеру, чем его скакун.
\paragraph{Атаки по наезднику:} в ближнем бою при атаке по наезднику на скакуне, превышающем размер атакующего, атакующий получает Помеху, если не атакует Длинным оружием.
\paragraph{Атаки по скакуну:} совершаются по обычным правилам.
\paragraph{Атаки скакуна:} во время своего Действия скакун может совершать любые Маневры, доступные в соответствии его Навыкам и снаряжению. Скакун не обязан заявлять такой же Маневр, как и наездник. Исключение составляет Атака с разбега.
\paragraph{Действие и Перемещение скакуна:} скакун Действует и Перемещается в ту же Очередь, что и его наездник.
\paragraph{Проверки скакуна:} Проверки Дб, Мт и Вл скакуна совершаются с использованием Навыка "Обращение с животными" наездника.
\newline
Например, если скакун атакует, его Рукопашный бой заменяется Обращением с животными наездника. Скакун может использовать собственный Рукопашный бой, если он больше Обращения с животными наездника.
\paragraph{Реакция наездника} равна \textbf{|(Рц скакуна + Рц героя) ÷ 2|}.

\section{Дорожные войны}
\paragraph{Не дрова везешь!:} из-за тряскипассажиры транспортного средства совершают любые активные проверки с Помехой.
\paragraph{Отвлекать водителя воспрещается!:} водителю довольно проблематично заниматься чем-то еще, кроме управления транспортным средством. Все атаки водителя совершаются с 2 Помехами. Водитель не может использовать Двуручное оружие.
\paragraph{Под откос:} если машина переворачивается или врезается в препятствие, все пассажиры (включая водителя) получают Дробящие Повреждения, равные \textbf{|30 - Управление транспортом Водилы - Бонус доспеха|}. Если машина не была оснащена ремнями безопасности (или пассажиры не сочли нужным воспользоваться ими), удвойте успешно нанесенные Повреждения.
\newline
Те из пассажиров, кто не был пристегнут, могут избежать Повреждений, совершив проверку Атлетики (Лв) против \textbf{|10 + нанесенные Повреждения|}.
\paragraph{Борт к борту:} водитель может использовать транспортное средство, как оружие. Его Доблесть равна \textbf{|Эксплуатация + модификатор Ловкости + Прочность транспортного средства|}. При провале маневра транспортное средство получает Повреждения, равные величине провала. Обратите внимание, что это не всегда (хоть и зачастую) означает тараны и удары бортами — в случае гироскутеров, мотоциклов и небольших машин водитель заманивает соперника на обочину или в кювет!
\paragraph{Наперегонки!:} герой может попытаться оторваться от преследования, или наоборот, догнать кого-то. Если герой превышает скорость, указанную в разделе "по дороге всегда быстрее", то максимальная скорость в км/ч, которой он может достигнуть, равна \textbf{|Эксплуатация + Воля + Реакция — Опасность местности| х 10}. Разумеется, она все еще ограничена соответствующим параметром из таблицы. Совершите проверку Эксплуатации против \textbf{|10 + Эксплуатация(Лв) противника + Опасность местности|}. В случае успеха герой добивается желаемого, а машине потребуется проверка Износа в конце сцены. В случае провала смотрите раздел "Под откос".
\paragraph{Маневровая скорость:} в течение Очереди водителя транспортное средство может преодолеть расстояние в метрах, равное \textbf{|Эксплуатация(Лв) + Воля водителя + Реакция водителя — Опасность местности|}. Это расстояние включает в себя маневры любой сложности.