\section{Тип Повреждений}
в зависимости от типа, повреждения могут иметь разные эффекты КУ и могут быть увеличены или уменьшены в зависимости от того, какие есть сопротивления или уязвимости у цели. Атаки имеют один из следующих типов Повреждений:
\newline
\textbf{(Д)}робящие, \textbf{(Е)}дкие, \textbf{(К)}олющие, \textbf{(Л)}едяные, \textbf{(О)}гненные, \textbf{(П)}роникающие, \textbf{(Р)}убящие, \textbf{(Э)}лектрические, \textbf{(Я)}довитые.
\newline
Хотя задача большинства атак — понижение ЕЗ цели, достигается она по-разному. Пытаться повредить рапирой каменную стену — не самая здравая идея, здесь лучше сработают молот или хотя бы дубина. Зато меткий укол в глаз способен не только ослепить противника, но уложить его на месте!
\paragraph{Уязвимость, Сопротивление, Иммунитет и Родная стихия:} Уязвимость к определенному типу Пв увеличивает успешно нанесенный урон в 2 раза. Сопротивление к определенному типу Пв уменьшает успешно нанесенный урон в 2 раза. Иммунитет к определенному типу Пв делает существо абсолютно невосприимчивым к этому типу Пв. Тип Пв, являющийся для цели Родной стихией, не отнимает ее ЕЗ, а восстанавливает их на то число, которое цель потеряла бы, не имея Родной стихии. ЕЗ цели все еще не могут превышать максимальной величины.
\paragraph{Атаки с несколькими типами Повреждений:} некоторые виды оружия могут наносить Пв разных типов — например, меч может наносить как Рубящие, так и Колющие Пв. В этом случае выберите тип Пв перед атакой.
\newline
Также существуют виды оружия, которые имеют несколько типов Повреждений одновременно — например, удар горящим факелом нанесет Огненные и Дробящие Пв. Возможна ситуация, в которой цель будет иметь Уязвимость к нескольким типам Пв атаки одновременно или даже Уязвимость к одному из них и Сопротивление к другому.
\newline
Если цель имеет Уязвимость к нескольким типам Пв атаки сразу, удвойте успешно нанесенные Пв за каждую Уязвимость. Например, если гигантский слизень имеет Уязвимость к Огненным Пв и Дробящим Пв, то удар горящим факелом нанесет в 4 раза больше Пв, чем обычно.
\newline
Уязвимость и Сопротивление компенсируют друг друга. Например, если цель имеет Сопротивление к Огненным повреждениям и уязвимость к Дробящим, то атака факелом будет наносить обычное количество повреждений по цели.
\newline
Иммунитет к любому типу Пв, которые включает атака, не позволяет ей наносить Повреждения цели.
\paragraph{Эффекты КУ} от разных типов повреждений:
\begin{itemize}
\item[--] \textbf{Дробящие:} получивший КУ Оглушен.
\item[--] \textbf{Едкиe:} получивший КУ начинает Растворяться. Состояние наступает даже в том случае, если цель не получила Пв при атаке.
\item[--] \textbf{Колющие:} получивший КУ страдает от Внутреннего кровотечения.
\item[--] \textbf{Ледяные:} получивший КУ становится Неподвижным до конца своей следующей Очереди.
\item[--] \textbf{Огненные:} получивший КУ Загорается.
\item[--] \textbf{Проникающие:} снаряд засел внутри тела! Получивший КУ страдает от Внутреннего кровотечения. После завершения боя он страдает от Агонии до тех пор, пока снаряд не будет извлечен (Врачевание против 15).
\newline
Если герой подвергается магическому лечению до того, как снаряд извлечен, он больше не страдает от Агонии и Внутреннего кровотечения, но боль дает о себе знать. Все активные проверки героя совершаются с Помехой до тех пор, пока снаряд не извлечен. Последующая сложность проверки Врачевания для извлечения снаряда возрастает до 20.
\item[--] \textbf{Рубящие:} получивший КУ страдает от Кровотечения.
\item[--] \textbf{Электрические:} получивший КУ сбит с ног.
\item[--] \textbf{Ядовитые:} на получившего КУ накладывается эффект яда.
\end{itemize}