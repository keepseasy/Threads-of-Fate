\section{Состояния}
Физические и ментальные воздействия могут повлиять на состояние (и поведение) героя, как в бою, так и вне его. Эффекты двух одинаковых состояний, происходящих из различных источников, используют наихудший вариант.
\newline Исключениями являются Внутреннее Кровотечение и Растворение - их эффекты складываются.
\paragraph{Агония:} герой охвачен мучительной болью, и проверяет Вл против \textbf{|20|} в момент получения состояния, а затем - в начале каждой Сцены. 
\newline При провале все, на что способен герой - лежать, вопить и поносить Судьбу. При успехе он держит себя (и, возможно, части своего тела) в руках. Он может проверять Дб и Мт и передвигается с половинной Ск. 
\newline Состояние длится, пока герой не восстановит 1+ ЕЗ (если Агония вызвана потерей ЕЗ), или не получит врачебную помощь (если Агония вызвана иными причинами). 
\paragraph{Внутреннее кровотечение:} органы героя повреждены. В начале каждой своей Очереди он теряет \textbf{|5 - МВн|} ЕЗ (минимум 2 ЕЗ), пока не восстановит 1+ ЕЗ или не получит врачебную помощь (Медицина против \textbf{|15|}).
\paragraph{Возгорание:} в начале своей Очереди цель и ее снаряжение получают 5 Пв и теряют 1 ЕЗ. Горящий может израсходовать Действие и Перемещение, чтобы потушиться. В этом случае, до начала следующей Очереди цели, атаки по ней совершаются по правилам Внезапного нападения.
\newline Состояние длится \textbf{|5 - МЛв цели|} Кругов (минимум 1 Круг).
\paragraph{Кровотечение:} герой истекает кровью. В начале каждой своей Очереди он теряет число ЕЗ, равное \textbf{|ЕЗ, потерянным при КУ|}, пока не восстановит 1+ ЕЗ или будет перевязан (Медицина или другой уместный Навык против \textbf{|10|}). 
\paragraph{Невидимость:} герой невидим. Герой не может быть выбран целью, и нападает Внезапно.
\paragraph{Неподвижность:} герой не может двигаться и защищать себя. Возможно, он схвачен, заморожен, опутан сетью или просто мертвецки пьян. Зщ героя падает до \textbf{|БАЗщ + БД|}. Все атаки по герою в Боевом контакте совершаются с Преимуществом. 
\paragraph{Оглушение:} герой теряет связь с реальностью (обычно из-за доброго удара по голове или под дых). Герой не может Действовать в свою следующую Очередь. В дальнейшем герой совершает Активные проверки с Помехой число Очередей, равное \textbf{|5 - МВн|}.
\paragraph{Ослепление:} герой ничего не видит. Все атаки по нему совершаются по правилам Внезапного нападения. Герой проверяет Дб и Мт с 2 Помехами. Он не может нападать на цели, находящиеся дальше, чем его Наблюдательность(Мд) в метрах.
\paragraph{Отравление:} герой под действием Яда. В дополнение к Первичному эффекту Яда или Лекарства, герой совершает все проверки с Помехой.
\paragraph{Ошеломление:} герой в замешатешльстве. Как правило, из-за какой-то неожданности, вроде нападения врагов. Герой вычитает БЩ и МЛв из своей Зщ и ополовинивает Ск. Состояние длится число Очередей, равное \textbf{|5 - МИн|}.

\paragraph{При смерти:} герой находится между жизнью и смертью и не способен сколь-нибудь активно участвовать в происходящем до следующей Интерлюдии. Герой Неподвижен и без сознания. В Боевом контакте он может быть добит без каких-либо проверок, если никто из окружающих этому не помешает.
\newline Если герой дотянул до начала следующей Интерлюдии, в зависимости от жанра и настроения игры он:
\begin{itemize}
  \item Приходит в себя в 0 ЕЗ (и Агонии).
  \item Впадает в кому и покидает повествование до оказания квалифицированной медицинской помощи или выполнения иных обусловленных контекстом условий.
  \item Умирает, не приходя в сознание.
\end{itemize}
\begin{tcolorbox}
  Если герой впал в кому, авторский коллектив рекомендует передать игроку управление персоной или статистом, или даже ввести в повествование нового героя.
\end{tcolorbox}
\paragraph{Ранен:} если текущие ЕЗ не превышают 2/3 от максимальных, то герой Ранен. Его Ск ополовинивается. 
\newline Это состояние связано с потерей ЕЗ, но иногда наступает, если герой страдает от вывиха или болевого приема.
\paragraph{Растворение:} в начале своей Очереди цель и ее снаряжение теряют 1 ЕЗ. Жертва может избавиться от состояния, пропустив Очередь и сорвав одежду. Если одежды на жертве не было, ей понадобится помощь квалифицированного химика и проверка Науки против \textbf{|15|}. Провал проверки наделит жертву Недостатками "Урод" или "Старая рана" до завершения игровой встречи. Если жертва уже обладает этими Недостатками, мастер может немедленно ввести их в игру.
\newline Если меры не приняты, состояние завершится через \textbf{|10 + [Пв, полученные при вызвавшей состояние атаке]|} Кругов.
\newline Растворение вызывается множеством различных веществ. Способ нейтрализации, который сработал однажды, в следующий раз может навредить! Проверка Науки нужна для каждого нового Растворения.
\newline В конце Сцены проверьте Неприятности снаряжения. Не исключено, что также понадобится проверить Износ.
\trouble
{Отдушка}{Никаких последствий, кроме специфического запаха, который исчезнет после хорошей стирки.}
{Патина}{Незначительные повреждения. Герой может исправить их, проверив Ремонт против \textbf{|15|}, или возместив мастеру 1/4 СП предмета. До завершения ремонта Осечка предмета возрастает на 1.}
{Ржа}{Серьезные повреждения, все еще обратимые. Герой может исправить их, проверив Ремонт против \textbf{|20|}, или возместив мастеру 1/2 СП предмета. До завершения ремонта Осечка предмета возрастает на 5.}
{Труха}{Снаряжение приведено в полнейшую негодность. Возможно, удастся всучить его подвыпившему старьевщику.}
\paragraph{Разбит:} если текущие ЕЗ не превышают 1/3 от максимальных, то герой Разбит. Все Активные проверки совершаются с Помехой. 
\newline Это состояние связано с потерей ЕЗ, но иногда наступает, если герой страдает от вывиха или болевого приема. 
\paragraph{Сон:} герой спит, его Наблюдательность проверяется с Помехой. Во сне он Неподвижен. Если герой проснулся и вынужден действовать, он Ошеломлен до начала своей следующей Очереди. 
\newline Героя разбудят громкие звуки - выстрел из пистолета, звон будильника или вопли гибнущих товарищей. Тихие звуки потребуют проверки Наблюдательности (с Помехой).
\paragraph{Удушье:} герой задыхается, его проверки совершаются с Помехой. Герой потеряет сознание, если состояние продлится дольше, чем \textbf{|Вн*2|} Кругов в Боевой сцене, и умрет, если состояние продлится дольше, чем \textbf{|Вн*4|} Кругов.  
\paragraph{Ужас:} герой дрожит от ужаса. Все его Активные проверки совершаются с Помехой, пока источник Ужаса находится в зоне видимости или слышимости. Жертва получает Узы "Я бегу со всех ног от источника Ужаса".
\newline Когда источник Ужаса остался вне зоны видимости и слышимости, состояние длится число Очередей, равное \textbf{|10-Вл|}.
\paragraph{Усталость:} герой изнурен, все его Активные проверки совершаются с Помехой. Атаки по герою совершаются с Преимуществом.
\begin{tcolorbox}
\paragraph{} В некоторых кампаниях требуется ограничить возможность героев восполнять свои внутренние ресурсы - применения способностей, ЕЗ и Эн. В этих случаях можно добавить в игру два дополнительных состояния.
\paragraph{Истощение:} Герой долгое время функционировал на пределе, почти без сна и отдыха и его способности к восстановлению снижены. Все Заряды способностей, а так же восстановление ЕЗ и Эн во время Антрактов и Интерлюдий снижены ввдвое.
\paragraph{Сильное истощение:} Герой долгое время подвергался тяжелейшим испытаниям, при постоянной нехватки еды, воды и сна. Восстановиться совершенно не получается - ЕЗ, Эн и Заряды способностей возможны только с помощью зелий, медикаментов и механизмов.
\paragraph{} Для снятия состояний Истощение и Сильное истощение герою требуется провести 1 Антракт в полном покое при достаточном питании.
\end{tcolorbox}