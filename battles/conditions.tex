\section{Состояния}
Физические  воздействия могут повлиять на состояние (и поведение) героя, как в бою, так и вне его. Эффекты двух
одинаковых состояний, происходящих из различных источников, не складываются, за исключением Внутреннего Кровотечения, Возгорания, Кровотечения, Отравления и Растворения. Эффекты различных состояний воздействуют на героя одновременно.
\paragraph{Агония:} герой охвачен мучительной болью. Герой может действовать только в случае успешной проверки Вл против 20. В боевых сценах совершайте проверку каждую Очередь героя, в остальных — один раз за сцену. При провале все, на что способен герой — лежать, вопить и поносить свою злую Судьбу. При успехе проверки герой держит себя в руках, но не может совершать Маневры и передвигается с половинной Ск. Состояние длится, пока герой не будет исцелен чарами (если Агония вызвана потерей ЕЗ), не получит квалифицированную врачебную помощь (если Агония вызвана Переломом) или сильное обезболивающее.
\paragraph{Внутреннее кровотечение:} жизненно важные органы героя повреждены. Каждую свою Очередь герой теряет |5 — МВн| ЕЗ (минимум 2 ЕЗ) до тех пор, пока не будет исцелен чарами или не получит квалифицированную врачебную помощь (Врачевание против 15).
\paragraph{Возгорание:} в начале своей Очереди горящее существо или предмет теряет 5 ЕЗ. Горящее существо может пропустить свою Очередь, чтобы потушиться. В этом случае до начала следующей Очереди существа атаки по нему совершаются по правилам Внезапного нападения. Состояние длится \textbf{|5 — МЛв цели|} Кругов (минимум 1 Круг).
\paragraph{Кровотечение:} герой истекает кровью. Каждую свою Очередь герой теряет число ЕЗ, равное БПв оружия, вызвавшего КУ(мин 1). Состояние длится до тех пор, пока герой не будет исцелен или перевязан (Врачевание против 10).
\paragraph{Неподвижность:} по каким-то причинам герой не может двигаться и защищать себя. Он может быть схвачен, заморожен, опутан сетью или просто спит. Герой теряет бонус защиты щита и бонус ловкости к защите. Все атаки по герою совершаются с Преимуществом.
\paragraph{Оглушение:} герой ненадолго теряет связь с реальностью (обычно из-за доброго удара по голове или под дых). Герой не может совершать Действие в свою следующую Очередь. В дальнейшем герой совершает все свои активные проверки с Помехой число Очередей, равное \textbf{|5 — МВн|} (минимум 1 Очередь).
\paragraph{Ослепление:} герой ничего не видит. Все атаки по нему совершаются по правилам Внезапного нападения. Герой атакует с 2 Помехами. Он не может атаковать цели, находящиеся от него на расстоянии (в метрах) большем, чем его Наблюдательность.
\paragraph{Отравление:} герой находится под действием яда. Смотрите описание яда для определения эффекта.
\paragraph{Ошеломление:} герой приходит в замешательство и не может совершать Действие в свою следующую Очередь. В дальнейшем герой совершает все свои активные проверки с Помехой число Очередей, равное \paragraph{|5 — МИн|} (минимум 1 Очередь). Также это состояние может быть причиной недосыпа, злоупотребления алкоголем или наркотическими зельями. Тогда избавиться от него поможет только полноценный отдых.
\paragraph{Ранен:} если ЕЗ героя достигают 2/3 от максимальных, то он Ранен. Его Ск ополовинивается. Обычно это состояние связано с потерей ЕЗ героем, но иногда оно возможно и в иных случаях — например, если герой страдает от вывиха или последствий болевого приема.
\paragraph{Растворение:} в начале своей Очереди существо или предмет получает 1 Пв вне зависимости от своей Прч. Если жертва была одета, она может остановить действие эффекта, пропустив 1 Очередь и сорвав с себя одежду, хотя снятие доспеха может занять куда больше времени! Если же одежды на жертве не было… Ей понадобится помощь квалифицированного медика и проверка Медицины против 15. Растворение вызывается множеством самых различных субстанций и веществ, поэтому способ нейтрализации, который сработал в прошлый раз, в следующий может сделать только хуже! Проверка Медицины необходима для каждого нового источника Растворения.
\newline
Если меры не были приняты, состояние заканчивается через \textbf{|10 + Пв, нанесенные вызвавшей состояние атакой|} Кругов. В конце сцены совершите проверку Неприятностей, чтобы узнать, пришло ли в негодность снаряжение жертвы (при желании можно совершить отдельную проверку для каждого предмета).
\trouble
{Отдушка}%no sweat name
{Никаких последствий, кроме специфического запаха, который исчезнет после хорошей стирки}%no sweat description
{Патина}%tough day name
{Незначительные повреждения. Герой может исправить их самостоятельно, совершив проверку соответствующего НавыкаЭ против 15 или заплатив ремесленнику 1/4 СП предмета (минимум 1 СП). До завершения ремонта предмет получает Осечку 6.}%tough day description
{Ржа}%we have trouble name
{Серьезные повреждения, все еще поддающиеся ремонту.Герой может исправить их самостоятельно, совершив проверку Ремонта против 20 или заплатив ремесленнику 1/2 СП предмета (минимум 1 СП). До завершения ремонта предмет получает Осечку 10.}%we have trouble description
{Труха}%fiasco name
{Снаряжение приведено в полнейшую негодность. Возможно, удастся всучить его подвыпившему старьевщику и выручить пару медяков.}%fiasco description
\paragraph{Серьезно ранен:} если ЕЗ героя достигают 1/3 от максимальных, то он Серьезно ранен. Все его активные проверки совершаются с Помехой. Обычно это состояние связано с потерей ЕЗ героем, но иногда оно возможно и в иных случаях — например, если герой страдает от вывиха или последствий болевого приема.
\paragraph{Сон:} герой спит. Все проверки Наблюдательности героя совершаются с Помехой. Пока герой спит, он Неподвижен. Если герой проснулся и вынужден сразу же действовать, он Ошеломлен. Героя автоматически разбудит достаточно громкий звук — выстрел из пистолета или звон медного колокольчика. Более тихие звуки потребуют проверки Наблюдательности (с Помехой)!
\paragraph{Удушье:} герой задыхается. Все его проверки совершаются с Помехой. Герой Потеряет сознание, если состояние продлится дольше, чем \textbf{|Вн × 10|} секунд, и умрет вне зависимости от величины его ЕЗ, если состояние продлится дольше, чем \textbf{|Вн × 20|} секунд.
\paragraph{Ужас:} герой дрожит от ужаса. Он совершает все активные проверки с Помехой, пока источник его ужаса находится поблизости (в зоне видимости или слышимости). В дальнейшем состояние длится число Очередей, равное \textbf{|10 — Вл героя|}.
\paragraph{Усталость:} все активные проверки совершаются героем с Помехой. Атаки по герою совершаются с Преимуществом.