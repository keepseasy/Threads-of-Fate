\section{Раны и их последствия}
\paragraph{}
Единицы Характеристик (ЕХ) и их потеря: некоторые атаки — как правило, яды, отнимают не ЕЗ жертвы, а понижают ее Характеристики. Например, ядовитая паучья слюна понижает Ловкость жертвы на 2. Под потерей ЕХ понимаются все подобные случаи.
\newline
Если герой пережил Повреждения, Единицы Здоровья и Единицы Характеристик могут быть восстановлены до максимума посредством отдыха или медицинской помощи. О восстановлении ЕЗ и ЕХ подробнее читайте в разделе "Отдых".
\paragraph{Опасная рана:} если герой одномоментно получает Пв, превышающие \textbf{|1/5 от максимальных ЕЗ|}, он должен совершить проверку Вн против 15. Если герой проходит проверку, то он продолжает сражаться. В противном случае он Теряет сознание до конца Сцены.
\paragraph{Болевой шок:} если в течение Круга герой одномоментно теряет число ЕЗ, превышающее его \textbf{|Вл|}, то следующая активная проверка героя совершается с Помехой. Разумеется, это относится к живым существам, которые в принципе способны испытывать боль.
\paragraph{Потеря сознания:} герой, потерявший сознание, находится в этом состоянии до конца \textit{следующей} Сцены, после чего приходит в себя. Если не замерзнет, не истечет кровью, не будет добит победителями или съеден дикими тварями.
\newline
Герой немедленно приходит в сознание, если он восстанавливает хотя бы 1 ЕЗ.
\paragraph{Смерть:} как только герой теряет последнюю ЕЗ, он должен совершить проверку Вн против 15. Если герой проходит проверку, то Теряет сознание и остается жив (и обзаводится парой впечатляющих шрамов). В противном случае герой умирает. Он приходит в сознание в следующей Сцене в 0 ЕЗ, страдает от ран и находится в Агонии до следующей Интерлюдии!
\paragraph{Смерть и статисты:} для ускорения боевых сцен мастер может считать проверки Выносливости статистов при Потере конечностей, Опасных ранах и Смерти автоматически проваленными и совершать их только в тех сценах, которые он и игроки считают важными. Также мастер может делать проверки, если игроки желают захватить статиста живым.
\subsection{Тяжелые травмы}
\paragraph{Переломы:} попадание по руке или ноге может вызвать Перелом. Когда конечность одномоментно получает Пв, превышающие \textbf{|1/5 от максимальных ЕЗ|}, она считается сломанной. Переломы считаются Опасной раной.
\paragraph{Потеря конечностей:} если конечность одномоментно получает Пв, превышающие \textbf{|2/5 от максимальных ЕЗ|}, то она отрублена, размолота в кашу или приведена в полнейшую негодность каким-то иным образом. Повреждения, превышающие требуемое для уничтожения конечности число, теряются. Герой, потерявший конечность, находится в состоянии Кровотечения. Само собой, Потеря конечности считается Опасной раной!
\paragraph{Однорукие герои:} однорукие герои не могут использовать Двуручное оружие. Если рука обрублена ниже локтя, герой может закрепить на ней щит или одноручное оружие.
\paragraph{Одноногие герои:} одноногий герой с костылем или протезом считается перемещающимся по Трудному ландшафту. Да, это значит, что одноногий герой с костылем или протезом и Чувством равновесия не испытывает никаких неудобств при движении. Костыль может использоваться для атаки! Без костыля или протеза одноногий герой перемещается на 1/3 своей Ск. Одноногие герои не могут использовать Громоздкое оружие.
\paragraph{Безногие герои:} безногие герои понижают свой размер на 1 категорию. Ск безногих героев составляет 1. Если у безногого героя есть какое-то средство перемещения — например, тачка с колесиками — при помощи Перемещения он может двигаться на число метров, равное своему МЛв (минимум на 2 метра). Безногий герой атакует в ближнем бою с Помехой. В ближнем бою враги атакуют безногого героя с Преимуществом. Безногие герои не могут использовать Громоздкое и Длинное оружие.
\paragraph{Потеря глаз:} герой, лишившийся двух глаз, перманентно находится в состоянии Ослепления. Одноглазый герой совершает проверки Меткости с Помехой.