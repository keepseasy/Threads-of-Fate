\section{Нулевой уровень характеристик}
В приключениях героев подстерегает немало опасностей, и далеко не все из них явные. Яды, болезни, зараженная вода и пища могут понизить значение любой Характеристики до нуля. Понижение Основной Характеристики, даже временное, приводит к понижению зависящей от нее Вторичной.
\paragraph{}
Нулевой уровень Характеристики означает, что существо автоматически проваливает любые проверки, связанные с ней. Упавшие до нуля Сила, Ловкость и Скорость приводят к состоянию, близкому к параличу (хотя существо все слышит и может наблюдать за происходящим вокруг). Упавшие до нуля Интеллект, Мудрость и Обаяние приводят к коме. Если Реакция или Скорость существа падает до 0, оно впадает в ступор. Во всех этих случаях существо считается находящимся в состоянии Неподвижности. Существо с нулевой Волей покорно выполняет любые отданные ему приказы, даже очевидно самоубийственные. Если существо получит несколько приказов, противоречащих друг другу, то оно постарается выполнить их, удовлетворив максимальное число требований. Если Выносливость существа падает до нуля, то оно немедленно умирает или разваливается на части.
\paragraph{}
Исключение — боевые характеристики. Герой даже с отрицательными значениями боевых характеристик может пытаться атаковать противников, как бы безнадежно это не казалось.