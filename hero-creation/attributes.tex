\section{Атрибуты}
Атрибуты являются важнейшими деталями образа героя и во многом отвечают за то, как его воспринимают окружающие. Атрибуты состоят из набора достоинств, действующих постоянно, и Уникального хода, который вводится в игру при обрыве Нитей или при помощи проверки Неприятностей.
И герои, и статисты вполне могут обойтись (и зачастую обходятся) без Атрибутов. Далеко не каждая важная персона обладает неприкосновенностью, далеко не каждая красавица способна очаровывать окружающих, далеко не каждый офицер отдает толковые приказы. Герой вполне может важной персоной и занимать высокий пост, родиться миловидным, или получить офицерский чин, но при этом не иметь соответствующего Атрибута. Его приобретение будет означать, что на Элитного обратил внимание совет директоров (или герой нанял хорошего адвоката), Красавица расцвела (или поняла, как использовать свою красоту в достижении личных целей), а Офицер закалился в боях и заслужил уважение подчиненных (или заставил их себя бояться).
Атрибут дает герою следующие возможности:
\begin{itemize}
\item[--] Доступ к Экспертным навыкам. При получении Атрибута герой может получить возможность прокачивать Экспертные наыки, указанные в описании.
\item[--] Свойства - возможности, которые герой получает при приобритении Атрибута. Это могут быть разовые приобритения, постоянные улучшения или применяемые способности.
\item[--] Функции - способности, за применение которых герою потребуется использовать свою Энергию. В названии Функции в скобках указан ее стоимость в Энергии.
\item[--] Уникальный Ход Атрибута. Эта способность позволяет герою преуспевать в задачах, сложных или даже невозможных для героев с другими Атрибутами (или вовсе без них). Добиться успеха при совершении Хода может помочь как своенравная Судьба, так и способности самого героя. Некоторые из Ходов ориентированы на применение в бою. Это никоим образом не должно останавливать игроков и мастера от использования их в быту, если у кого-то возникла идея, соответствующая жанру и настроению игры. И напротив — мирные ходы наверняка найдут применение в боевых сценах. В названии Хода в скобках указана его стоимость в Нитях.
\newline \textbf{Уникальный ход без обрыва Нитей.} Герои достаточно компетентны, чтобы добиться эффекта Уникальных ходов и без вмешательства Судьбы, а у статистов и вовсе нет выбора. В этом случае реализация Хода требует проверки Неприятностей. Например, Неистовый герой может покалечить соратников, бездумно разрушая все вокруг, Крепыш — не болея сам, заразить друзей, а Бродяга — забрать то, что хотел съесть кто- то другой...
\newline
Обратите внимание, что если игрок решает использовать Ход героя без обрыва Нитей, надеясь выбросить на кубике хороший результат (например, если успех Хода требует обрыва нескольких Нитей), он может использовать Нити только для перебросов.
\newline
Если в описании Хода без обрыва Нитей не указана сложность проверки Неприятностей под Контролем, сложность задается Мастером.
\end{itemize}
\begin{tcolorbox}
Герой может взять несколько одинаковых Атрибутов, получая все их бонусы, согласно описанию, но при этом активация Уникального Хода не будет требовать меньшего количества Нитей.
\end{tcolorbox}
\paragraph{Темная сторона:} за все приходится платить, и возможности Атрибутов — не исключение. Фактически, Темная сторона — это Недостаток в Атрибуте! Например, Ветерана может попросить о помощи старый друг, которому он обязан жизнью, Богач почувствует себя очень неуютно во время бунта рабочих, а за Красавицу дадут хорошую цену на черном рынке!
\paragraph{Атрибуты, придуманные игроками и замена Уникальных ходов:} по договоренности с мастером игрок может начать игру с Атрибутом, который придумал сам. Возможности Атрибута и Уникального хода должны быть четко определены до начала игры. Также по договоренности с мастером игрок может заменить Уникальный ход одного Атрибута на Уникальный ход другого, если это соответствует жанру и настроению игры. Например, игрок может заменить Уникальный ход Ветерана на Уникальный ход Бродяги или Богача. Замена должна быть оговорена с мастером и произойти до начала игры или приобретения Атрибута.

\subsection{Атрибуты Наследия}
Некоторые атрибуты отображают не только способности героя, но и его происхождение. Их герой получил от своих предков и по праву наследует их, как представитель своего Рода. Герой может иметь только один Атрибут Наследия и не может приобрести его по ходу игры - это то, что нельзя приобрести, а только получить при рождении. Если герой полукровка и может претендовать на оба Наследия, ему прийдется выбрать только одно из них, или же отказаться от обоих, выбрав путь, совершенно отличный от обычаев предков.
\genAndGet{attributes}{attributes}{Наследие}

\subsection{Атрибуты Могущества}
Атрибуты, дающие герою доступ к использованию магии, мистическим способностям или псионическим силам являются Атрибутами Могущества. Герой может приобретать одновременно несколько Атрибутов Могущества, что будет отражать его знакомство с разными аспектами Феноменов.
\genAndGet{attributes}{attributes}{Могущество}

\subsection{Боевые Атрибуты}
Атрибуты, ориентированные на боевые столкновения. Они дают значительные преимущества в боевых ситуациях, однако в других сценах ими воспользоваться будет \textit{сложнее}.
\genAndGet{attributes}{attributes}{Боевой}

\subsection{Социальные Атрибуты}
Атрибуты, ориентированные на социальные взаимодействия. С их помощью можно заручиться поддержкой статистов, произвести хорошее Впечатление или даже избежать конфликта до его начала. Однако, когда присутствующие в сцене похватались за оружие, Социальные Атрибуты уже \textit{скорее всего} не помогут.
\genAndGet{attributes}{attributes}{Социальный}

\subsection{Вспомогательные Атрибуты}
Эти Атрибуты не имеют сильной специализации на тех или иных Сценах или же несут вспомогательную функцию, которая делает жизнь героя проще.
\genAndGet{attributes}{attributes}{Вспомогательный}

\printindex[attributes]
