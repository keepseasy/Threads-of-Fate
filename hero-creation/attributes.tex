\section{Атрибуты}
Атрибуты - важнейшие детали образа героя и во многом отвечают за то, как его воспринимают окружающие. 
\newline Впрочем, и герои, и статисты могут обойтись (и зачастую обходятся) без Атрибутов. Не каждый проповедник - пламенный оратор, не каждая красавица способна очаровывать окружающих, не каждый офицер отдает толковые приказы. Герой вполне может быть проповедником и занимать формальное место в иерархии культа, родиться миловидным, или получить офицерский чин, но при этом не иметь соответствующего Атрибута. Его приобретение будет означать, что на Проповедника снизошла благодать (или он наконец-то научился ладно излагать догматы и полоскать мозги), Красавица расцвела (или поняла, как использовать красоту в достижении целей), а Офицер закалился в боях и заслужил уважение подчиненных (или заставил себя бояться).
Атрибут дает герою следующие возможности:
\begin{itemize}
    \item[--] Набор Экспертных навыков, к которым герой получает доступ при приобритении Атрибута;
    \item[--] Набор свойств, действующих постоянно или вводимых в игру без участия Нитей или Энергии. Условия применения свойств указаны в их описании;
    \item[--] Функции, действующие за счет траты Энергии героя. Стоимость Функции в Эн и прочие условия применения указаны в скобках после названия;
    \item[--] Cнаряжениe с указанными свойствами и СП. Начальное снаряжение имеет полный набор расходников, т.е. транспортные средства заправлены, оружие заряжено и т.д. Начальное снаряжение может быть продано, потеряно, украдено, уничтожено;
    \item[--] Темную Сторону - ситуативный Недостатка, сопутствующий Атрибуту;
    \item[--] Уникального Хода, который входит в игру при обрыве Нитей или с помощью проверок Неприятностей. Стоимость Хода в Нитях и условия его применения, в том числе в Боевых сценах, указаны в скобках после названия.
\end{itemize}

\begin{tcolorbox}
    У Атрибута всегда есть Ход и хотя бы одно Свойство, но далеко не у каждого - набор Экспертных навыков, Функции или Начальное снаряжение. Выбирайте тот, число способностей которого позволит вам комфортно с ним взаимодействовать.
\end{tcolorbox}

\paragraph{Стоимость Начального снаряжения:} некоторое снаряжение имеет фиксированную сложность приобретения (СП), например, Офицерский значок (СП 30) и Энциклопедия (СП 20). Если в описании указано "СП Х и меньше", то герой получает любой предмет указанной категории с СП, равной или меньшей Х. Если в описании указано "суммарно Х СП", то сумма значений СП всех предметов, выбранных героем, не должна превышать Х. Прочее снаряжение герой выбирает и покупает самостоятельно, расходуя Богатство.
\newline Подробнее о СП читайте в главе "Богатство и снаряжение".
\paragraph{Свойства, Функции, Ходы и течение времени:} время, которое занимает применение свойства, Функции или Хода, определяется мастером. Иногда вполне допустимо позволить герою уладить свои дела в Интерлюдии или Антракте. Даже когда временные рамки обозначены, мастер вправе отступить от них, если это уместно в контексте ситуации.
\paragraph{Свойства и Ходы, зависящие от Модификаторов Характеристик} всегда могут быть использованы минимум 1 раз за указанный период или воздействовать минимум на 1 цель, даже если применяемый Модификатор нулевой или отрицательный. Например, Дипломат с 9 Об (МОб -1) 1 раз активировать свойство "Парламентер".
\paragraph{Восполнение способностей героев:} способности некоторых Атрибутов могут быть возобновлены без ухода в Антракт или использования Интерлюдий. Условия восполнения указаны в описании таких Атрибутов. Если условия требуют наличия Услуги, СП восполнения может возрасти.
%\paragraph{Антракт и способности героев:} пребывание в Антракте возобновляет запас всех способностей, число применений которых за игровую встречу ограничено - если контекст не противоречит этому. Это никогда не требует расхода каких-либо ресурсов. Способности, действие которых ограничено рамками игровой встречи, прекращают работать при уходе героя в Антракт.
\paragraph{Уникальный Ход} позволяет герою преуспевать в задачах, сложных или невозможных для героев с другими Атрибутами. Добиться успеха при совершении Хода помогут как своенравная Судьба, так и способности самого героя. 
\newline Некоторые из Ходов ориентированы на применение в бою. Это никоим образом не должно останавливать от использования их в быту, если у игроков и мастера возникла идея, соответствующая настроению игры. И напротив - мирные ходы наверняка найдут применение в Боевых сценах.
\paragraph{1 или более Нитей} Некоторые Уникальные ходы отдают стоимость в Нитях на откуп мастеру. Определяйте стоимость Хода в соответствии с контекстом и логикой ситуации. Помните, что Ходы из перечня "Повезло", позволяющие выкупить успех и Критический успех на любую проверку за 2 и 4 Нити, доступны абсолютно любому герою. Там, где герой без Атрибута может преуспеть при помощи покупки обычного успеха за 2 Нити (или при помощи обычной проверки), герой с применимым к ситуации Атрибутом обойдется 1 Нитью. Если герой без Атрибута может преуспеть, только купив Критический успех за 4 Нити, герой с применимым к ситуации Атрибутом уложится в 2-3 Нити. Успех, за который мастер потребует 5 Нитей, должен изображать нечто умопомрачительное - один из тех случаев, о котором и герой, и свидетели события будут вспоминать всю оставшуюся жизнь. Без сомнений, такой успех быстро обрастет слухами и домыслами, а со временем - легендами. Разумеется, без применимого к ситуации Атрибута такой успех попросту невозможен.
\paragraph{0 Нитей.} Условия некоторых Ходов могут снизить стоимость до 0 Нитей и меньше. В этом случае для успеха все еще требуется обрыв 1 Нити.
\paragraph{Уникальный ход без обрыва Нитей.} Герои и персоны достаточно компетентны, чтобы добиться эффекта Уникальных ходов без вмешательства Судьбы, а у статистов и вовсе нет выбора большую часть времени. В этом случае активация Хода требует проверки Неприятностей под контролем Навыков или Характеристик. 
\begin{tcolorbox}
    Обратите внимание, что если игрок использует Ход героя без обрыва Нитей, он не может использовать любые Ходы Судьбы для влияния на результат, хотя вправе пользоваться Функциями, Трюками, Успехами с Расплатой и т.д.
\end{tcolorbox}
\paragraph{Сложность Уникального Хода без обрыва Нитей.} Если сложность контрольной проверки не указана в описании Хода, ее задает мастер, ориентируясь на таблицу сложности задач. Не рекомендуется устанавливать ее выше |20| - обладающий атрибутом герой неплохо разбирается в том, что делает.
\paragraph{Уникальный Ход без обрыва Нитей и Расплата:} некоторые из Ходов требуют выбрать Расплату даже при провале. Это означает, что герой пожертвовал временем, силами и ресурсами, но в итоге его все равно постигла неудача. Да, случается и такое. Разумеется, Успех с Расплатой может использоваться при проверках, связанных с Уникальным Ходом.
\paragraph{Темная сторона:} за все приходится платить, и возможности Атрибутов - не исключение. Фактически, Темная сторона - это Недостаток в Атрибуте. 
\begin{tcolorbox}
Герой может взять несколько одинаковых Атрибутов, получая все их преимущества согласно описанию. При этом активация Уникального Хода не будет требовать меньшего числа Нитей.
\end{tcolorbox}
\paragraph{Атрибуты, придуманные игроками и замена Уникальных ходов}
По договоренности с мастером игрок может:
\begin{itemize}
    \item[--] Начать игру с Атрибутом, который придумал сам. Возможности Атрибута и Уникального Хода должны быть определены до начала игры.
    \item[--] Заменить Уникальный Ход одного Атрибута на Уникальный Ход другого. Например, игрок может заменить Уникальный Ход Технаря на Уникальный Ход Мусорщика или Гражданина убежища. Замена производится до приобретения Атрибута.
\end{itemize}

\subsection{Атрибуты Наследия}
Некоторые атрибуты отражают не только способности героя, но и его происхождение. Их герой получил от предков - так или иначе. Герой вправе иметь лишь один Атрибут Наследия и обычно не может приобрести его по ходу игры - это то, чему нельзя научиться, а только получить при рождении (или сборке). Если герой - полукровка, и может претендовать на несколько Наследий, ему придется выбрать одно из них, или же отказаться от всех, ступив на путь, совершенно отличный от его предназначения.
\begin{tcolorbox}
	Иногда, впрочем, допустимо приобретение Атрибута Наследия по ходу игры. Гены Биоконструкта просыпаются в прагматичном жителе убежища, путешественник Изменяется, укрывшись на ночь в таинственной пещере, а великого воина после ранения вживляют в корпус Боевого робота. Фантазируйте вместе - и вы придумаете интересное и правдоподобное объяснение случившемуся.
\end{tcolorbox}
\genAndGet{attributes}{attributes}{Наследие}

\subsection{Атрибуты Могущества}
Атрибуты, дающие герою доступ к мистическому дару, псионическим силам и прочим невероятным способностям, ошеломительными даже на фоне других невероятных способностей. Герой может приобретать несколько Атрибутов Могущества, что отражает знакомство с разными аспектами \textbf{Феноменов}.
\paragraph{Феноменальная характеристика (Фх).} Основная характеристика, с помощью которой герою активирует Феномены. Она указана в описании Атрибута, Трюка или Предмета, который используется для активации. Модификатор Феноменальной характеристики (МФх) используется в большинстве формул, описывающих Феномены. 
\newline Если у героя есть несколько источников Феноменальной характеристики, перед активацией Феномена игрок должен заявить, какой источник и соответствующая характеристика будет использоваться.
\begin{tcolorbox}
    Атрибуты Могущества не обязаны иметь мистическую подоплеку. Впрочем, способности проникать в древние информационные библиотеки, ощущать электромагнитные поля, считывать информацию с голографических иероглифов или подключаться к секретной спутниковой сети, иначе, как волшебством, не назовешь. Да и сами обладатели таких возможностей зачастую уверены в их сверхъестественном происхождении.
\end{tcolorbox}

\genAndGet{attributes}{attributes}{Могущество}

\subsection{Боевые Атрибуты}
Атрибуты, ориентированные на боевые столкновения. Они дают значительные преимущества в боевых ситуациях, однако в других сценах ими воспользоваться будет \textit{сложнее}.
\genAndGet{attributes}{attributes}{Боевой}

\subsection{Социальные Атрибуты}
Атрибуты, ориентированные на социальные взаимодействия. С их помощью можно заручиться поддержкой статистов, произвести хорошее Впечатление или даже избежать конфликта до его начала. Однако, когда присутствующие в сцене похватались за оружие, Социальные Атрибуты уже \textit{скорее всего} не помогут.
\genAndGet{attributes}{attributes}{Социальный}

\subsection{Вспомогательные Атрибуты}
Герои с этими Атрибутами прекрасно дополнят любую команду благодаря способностями, серьезно облегчающими жизнь - как им, так и их товарищам.
\begin{tcolorbox}
    В этой категории обретаются самые экзотические Атрибуты. Чем объяснить их способности, вам подскажут жанр и настроение истории, а еще, конечно же, ваши соигроки. Возможно, Двойник - разумная колония нанороботов, Перевертыш - невероятная мутация, а Паразит - биодрон-разведчик.  Хотя не исключено, что все совсем не так, верно? 
\end{tcolorbox}
\genAndGet{attributes}{attributes}{Вспомогательный}

\printindex[attributes]
