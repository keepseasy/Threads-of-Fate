\section{Недостатки}
Недостатки не обязательно являются отрицательным чертами (с точки зрения героя так уж точно). И все же они способны серьезно осложнить жизнь героя, а то и повернуть ее под совершенно непредсказуемым углом. Герой начинает игру с 0—2 Недостатками. Не рекомендуется одновременно брать герою больше 4 Недостатков и повторять Недостатки у героев в одной команде. 
\paragraph{Недостатки, придуманные игроками — отличная идея!} Недостатки из списка — всего лишь примеры. Следует, однако, помнить, что Недостаток должен осложнять герою жизнь, иначе в нем нет никакого смысла.
\paragraph{Замена и отказ от Недостатков:} мастер может позволить заменить один Недостаток на другой, если игрок того желает и это обусловлено логикой развития характера героя. В конце концов, лишь Судьбе известно, куда заведет кривая дорожка приключений!
Например, Вспыльчивый герой, нагрубив не тому человеку (и столкнувшись с последствиями), может стать Осторожным, а Любвеобильный, отыскавший ту самую, обзаведется Привязанностью! Игрок может вычеркнуть Недостаток героя, если это обусловлено логикой развития истории и характера героя.

\genAndGet{tricks}{tricks-flaws}
