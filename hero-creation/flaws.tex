\section{Недостатки}
Недостатки не обязательно являются отрицательным чертами (с точки зрения героя так уж точно). И все же они способны осложнить жизнь героя, а то и повернуть ее под совершенно непредсказуемым углом. Герой начинает игру с 0-1 Недостатком.
\newline Недостатки могут вводиться в игру по-разному - не стесняйтесь импровизировать. Неряха может выдать местоположение засады характерным запахом, Привязанность невовремя захочет поиграть в дамочку в беде, а Чужак случайно использует оскорбительный жест, заказывая выпивку. Способы испортить жизнь герою ограничены только контекстом и вашей фантазией.
\paragraph{Недостатки, придуманные игроками - отличная идея!} Главное, помните - Недостаток должен осложнять герою жизнь, и быть достаточно широко применимым, иначе в нем нет смысла.
\paragraph{Замена и отказ от Недостатков:} один Недостаток можно сменить на другой, если игрок того желает и это обусловлено развитием характера героя. Например, Вспыльчивый герой, нагрубив не тому человеку, может стать Осторожным, а Любвеобильный, отыскавший ту самую, обзаведется Привязанностью. Игрок вправе и просто вычеркнуть Недостаток, если контекст располагает к этому.
\begin{tcolorbox}
    Помимо прочего, Атрибуты, Трюки и Недостатки призваны создавать образ героя широкими мазками. «Гордая, но слегка Застенчивая Красавица-Прогрессор с Честным лицом и Чистым генофондом» – вполне завершенный образ. Или «Офицер-Ветеран, Мастер защиты, Знаток оружия, Пьяница и Грубиян» - звучит неплохо, верно? А главное, каждое слово имеет под собой игромеханическую основу и будет так или иначе работать на игру.
\end{tcolorbox}
\subsection{Перечень Недостатков}
\genAndGet{tricks}{tricks-flaws}{Недостаток}