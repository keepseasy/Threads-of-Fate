\section{Боевые характеристики}
\paragraph{Повреждения (Пв):} результатом успешных проверок Боевых характеристик (за исключением Защиты) являются Повреждения — потеря Единиц Здоровья от атак и вредоносных эффектов.
\paragraph{Бонус к Повреждениям (БПв)} обычно дается оружием и Феноменам в форме Снаряда. Чем он выше, тем больше шансов у героя нанести цели Повреждения. Бонус к Повреждениям является частью Боевых характеристик, Доблести и Меткости, и может изменяться в зависимости от того, какое оружие или способность использует герой.
\paragraph{Доблесть (Дб) = |Владение оружием + МРз + МСл + МЛв + БПв|.} Чем выше Доблесть героя, тем он опаснее в ближнем бою.
\newline \textbf{Рукопашная Доблесть} вычисляется по формуле \textbf{|Рукопашный бой + МРз + МСл + МЛв + БПв(Безоружный)|}. Безоружная Доблесть используется для нанесения Безоружных ударов руками, ногами, клыками, когтями и другими частями собсвтенного тела, а так же для совершения некоторых маневров.
\paragraph{Меткость (Мт) = |Стрельба - МРз + МЛв + БПв|.} Чем выше Меткость, тем опаснее герой в дистанционном бою.
\newline \textbf{Меткость Метания} вычисляется по формуле: \textbf{Мт = |Стрельба + МРз + МСл + МЛв + БПв|.} Она используется при метании предметов, гранат и использовании Метательного оружия(Луков в том числе). Обратите внимание, что для Меткости Метания Модификатор Размера добавляется так же, как и для Доблести, а не отнимается, как в обычной меткости.
\newline \textbf{Меткость Снарядов} вычисляется по формуле: \textbf{Мт = |Стрельба - МРз + МФх(модификатор Феноменальной характеристики) + БПв|.} Она используется при использовании Феноменов(см. соответствующую главу.)
\paragraph{Защита (Зщ) = |Базовая защита + МЛв + БД + БЩ|.} Чем выше Защита, тем сложнее поразить героя атаками.
\newline \textbf{Базовая защита(БАЗщ) = |10 - МРз|} определяет, насколько сложно попасть по цели в зависимости от ее размера.
\newline \textbf{БД(Бонус Доспеха)} равна сумме Бонусов к Защите, котрые дают надетые на героя доспехи и толстая шкура, если герой ей обзавелся.
\newline \textbf{БЩ(Бонус Щита)} равен сумме Бонусов к Защите, которые дают используемые героем щиты. Обычно герой может носить не более двух щитов одновременно, но если он обзаведется дополнительными руками или найдет летающий щит, он может использовать и больше.
\paragraph{Прочность (Прч):} камень, сталь, лед и многие другие материалы имеют показатель Прочности. Прочность вычитается из Повреждений, нанесенных цели. Некоторые существа — например, стальные големы и ожившие деревья — также обладают Прочностью! Прочность не является частью Боевых характеристик, но тесно связана с ними.
\newline Прочность, полученная героем из разных источников складывается.
\paragraph{Проверки Боевых характеристик:}
\begin{enumerate}
\item Бросьте К20 и прибавьте к нему значение Доблести или Меткости героя.
\item Сравните получившееся число с Защитой цели. Цель получает 1 Повреждение за каждую 1, на которую атакующий герой преодолел Защиту цели.
\end{enumerate}
Например, если герой с Доблестью 10 атаковал статиста с Защитой 18 и на К20 выпало 14, статист получит 10 +14 — 18 = 6 Повреждений.
\newline
Подробнее о проверках Боевых характеристик читайте в разделе <<Маневры>>.
\paragraph{Проверки Защиты:} в сценах, не подразумевающих
детализированных боевых действий — например, когда герой
прорывается сквозь разъяренную толпу, передвигается под беглым обстрелом или бежит по коридору, наполненному ловушками, мастер может определить полученные героем Повреждения при помощи проверки Защиты.
\begin{enumerate}
\item Бросьте К20 и прибавьте к нему значение \textbf{|Зщ — 10|}.
\item Сравните получившееся число со сложностью проверки. Герой получает 1 Повреждение за каждую 1, на которую провалил проверку.
\end{enumerate}
Например, герой Среднего размера с 16 Защитой передвигается
под беглым огнем вражеских стрелков. Мастер устанавливает
20 сложность проверки. На К20 выпадает 12. 12 +16 — 10 = 18.
Герой получает 2 Повреждения, так как 20 (Установленная
сложность) — 18 (Результат проверки) = 2.