%\documentclass[Нити судьбы.tex]{subfiles}
%\graphicspath{{\subfix{../images/}}}
%\begin{document}

\section{Вторичные характеристики}
\paragraph{Реакция (Рц) = | (Лв + Мд) ÷ 4|.} Реакция определяет порядок действия в боевых сценах и прочих ситуациях, в которых это важно. У Вторичных характеристик нет модификаторов. Во всех формулах используется их полное значение. Если результат деления меньше 1, то значение Вторичной характеристики все равно составит 1.
\paragraph{Воля (Вл) = | (Ин + Мд) ÷ 4|.} Отвечает за самоконтроль, сопротивление враждебным Могуществам и обычным соблазнам.
\paragraph{Скорость (Ск) = | (Лв + Вн) ÷ 4 + МРз|.} За 5 секунд времени герой может преодолеть число метров (или клеток, если при игре используется масштабная карта), равное своей Скорости.
\paragraph{Четвероногие существа:} четвероногие существа, такие как кони, кошки и гиганские ящеры, увеличивают свою Ск в 2 раза, перемещаясь по земле. Умножьте Ск после прибавления МРз. Тараканы, пауки и многоножки считаются четвероногими для определения Ск, а гигантские слизни, улитки и змеи — нет!
\paragraph{Полет:} если существо способно летать, его Ск полета в 3 раза выше, чем наземная Ск. Полет может использоваться до тех пор, пока нагрузка существа не превышает Комфортную. Если существо по каким-то причинам не может использовать полет, применяйте обычную Ск существа.
\paragraph{Единицы Здоровья (ЕЗ) = |Вн × (МРз+3)|.} Эта величина показывает, сколько ущерба герой способен вынести, прежде чем потеряет сознание или умрет.
\paragraph{Энергия (Эн) = | (Вн + Об) ÷ 4|.} Энергия определяет максимальный запас внутренних сил героя, которые он может использовать для творения заклинаний и активации некоторых предметов и атрибутов.
\newline
Текущий уровень энергии не может быть ниже нуля и выше максимальной Эн. Восстановить энергию герой может разными способами:
\begin{itemize}
\item[--] Во время Антракта герой полностью восстанавливает свою Энергию.
\item[--] Во время Интерлюдии герой восстанавливает Энергию в размере \textbf{|МОб|(мин. 1)}.
\item[--] Использовать зелья и стимуляторы для того чтобы восстановить Энергию.
\item[--] Иногда у героя есть возможность преобразовать топливо своего мира в собственную Энергию. Для того, чтобы восстановить 1 Эн нужно потратить 100 Зарядов.
\item[--] Трюки и Атрибуты тоже могут позволить герою восстанавливать Энергию. Способы могут быть как простыми, так и экзотическими. Смотрите описание соответствующих Трюков и Атрибутов для уточнения этих способов.
\end{itemize}
\paragraph{}Проверки Вторичных характеристик совершаются следующим образом:
\begin{enumerate}
\item Бросьте К20 и прибавьте к нему значение Вторичной характеристики.
\item Сравните получившееся число со сложностью проверки. Герой преуспевает, если число равно сложности или превышает ее.
\end{enumerate}
%\end{document}