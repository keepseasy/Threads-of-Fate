\section{Характеристики героя}
Очки характеристик: создавая героя, вы распределяете 70 Очков характеристик по Основным характеристикам. Очки распределяются в любых сочетаниях. Например, вы можете распределить 16 очков в любую одну Характеристику, получив значение 16. Если вы распределите 16 Очков по двум Характеристикам, то можете получить значения 9 и 7, 10 и 6 и т. д.

\subsection{Начальная и максимальная величина Характеристик}
При создании героя ни одна из Характеристик не может превышать 18. Игрок не вправе поднять Характеристики выше этого предела, даже если у героя есть свободные Очки Характеристик или Очки опыта. В дальнейшем, ни одна из Характеристик героя не может превысить 20 за счет траты Очков опыта, но может превысить 20 при помощи стимуляторов, аугментаций, специальных способностей и т.д.
\newline Базовые значения Основных Характеристик героев могут быть абсолютно любыми – в пределах от 1 до 20.
\newline Мастер может уменьшать или увеличивать количество Очков характеристик. Следует, однако, помнить, что герои, созданные меньше, чем на 60 очков, не вполне приспособлены к приключениям. 70 Очков характеристик – оптимальный вариант, так как в этом случае придется выбрать, в чем герой будет хорош, а в чем – не очень.
\newline Для антагонистов и прочих значимых фигур мастер вправе использовать любое количество Очков характеристик, даже превышающее то, на которое созданы герои. Для существ под управлением мастера не действует ограничение на максимальную величину Характеристик (хотя его стоит держать в голове).

\subsection{Основные характеристики}
\paragraph{Сила (Сл):} показывает, насколько герой развит физически. От этой Характеристики зависит, какой вес может нести герой, насколько далеко и высоко он прыгает, и ущерб, который он причиняет оружием, использующим мускульную силу. Также параметр Силы влияет на то, какое оружие герой сможет успешно применять.

\paragraph{Ловкость (Лв):} отвечает за быстроту и координацию движений. Эта Характеристика так же важна для воина, как и Сила. Еще она пригодится герою, который собирается стрелять, карабкаться по руинам и водить багги.

\paragraph{Выносливость (Вн):} пригодится любому герою. Высокая Выносливость означает, что герой крепок телом, редко болеет и легко переносит ранения. Также от этой Характеристики зависит, в каких доспехах герой сможет эффективно действовать.

\paragraph{Интеллект (Ин):} помогает запоминать информацию, делать выводы, а также учиться на своих и чужих ошибках. Интеллект необходим любому герою, который желает иметь высокие параметры Навыков – именно он устанавливает предел их роста.
\newline Герой с, например, 2 Интеллектом не сможет распределить в любой Навык больше 2 Очков опыта. Это не помешает ему относительно связно выражать мысли, найти работу, завести семью и вообще наслаждаться жизнью. Особенно, если удастся закорешиться с действительно умными ребятами, которые растолкуют, что к чему. 

\paragraph{Мудрость (Мд):} включает в себя находчивость, наблюдательность, бытовое здравомыслие и глубинные инстинкты. Именно Мудрость поможет герою вовремя заметить опасность… или просто избежать ее.
\newline Благодаря пассивным проверкам Наблюдательности, Мудрость – весьма важная Характеристика. Нужна она и героям, которые желают действовать раньше остальных, так как Мудрость входит в состав Вторичной Характеристики "Реакция".

\paragraph{Обаяние (Об):} позволит наладить контакт с окружающими и понравиться им, не особенно усердствуя. Этот параметр незаменим для того, кто предпочитает действовать исподволь и добиваться своего без применения физического насилия. Помимо этого, Обаяние определяет, в каком ключе герой воспринимает окружающий мир. Высокое Обаяние – залог оптимизма!
\newline Обаяние никак не связано с привлекательностью, хотя обаятельные герои часто кажутся окружающим симпатичными. 

\begin{tcolorbox}
    Игромеханически за внешнюю привлекательность отвечают Атрибут "Красивый", а так же Трюки "Соблазнительный" и "Стильный". 
    \newline Если ваш герой не планирует трепать языком, располагать к себе статистов, использовать Функции и феномены – Обаяние последняя в списке значимых для него Характеристик. Тем, кто не в восторге от героя из-за плохих проверок Впечатления, всегда можно отстрелить уши, отрубить тестикулы или просто от души наварить в торец.
\end{tcolorbox}

\subsection{Модификаторы Характеристик (М)} в большинстве формул используется не полное значение Основных характеристик, а \textbf{Модификатор = | (Основная характеристика — 10) ÷ 2|}.
\begin{center}
\begin{tabular}{ |c|c| }
\hline
\textbf{Основная характеристика} & \textbf{Модификатор}
\\ \hline
1 & -5
\\ \hline
2-3 & -4
\\ \hline
4-5 & -3
\\ \hline
6-7 & -2
\\ \hline
8-9 & -1
\\ \hline
10-11 & 0
\\ \hline
12-13 & 1
\\ \hline
14-15 & 2
\\ \hline
16-17 & 3
\\ \hline
18-19 & 4
\\ \hline
20 & 5
\\ \hline
\end{tabular}
\end{center}
\subsection{Проверки Основных характеристик}
Проверка Основной характеристики является Проверка с Бонусом, равным модификатору Основной характеристики.

\subsection{Размеры существ}
Правила берут за эталон существо ростом от 150 до 210 см. Оно имеет Средний размер. Но в сюжетах найдется немало отклонений от этого стандарта – мутанты, роботы, киборги и \tbd.
\newline Размер не влияет на Основные характеристики напрямую, однако большие существа – легкая мишень для атак, а маленькие существа вынуждены использовать маленькое оружие, наносящее меньше ущерба.
\newline Существа занимают область в зависимости от размеров. Это не означает, что существо занимает область целиком (хотя бывает и такое). Несомненно одно – в ней существо может и будет мешать передвижению недругов.
\newline Высота в холке четвероногих существ зачастую меньше, чем рост существ соответствующего размера, указанный в таблице.

\paragraph{Каков Размер моего героя?} Игрок вправе начать приключение Крошечным, Маленьким или Большим героем. 

\paragraph{Модификатор размера (МРз):} наравне с Выносливостью определяет, насколько существо или объект восприимчивы к урону. Чем больше цель атаки, тем сложнее нанести ей серьезный вред без специального снаряжения.
\newline Размер, отличный от Среднего, выгоден в одних ситуациях и мешает в других. Небольшому существу легче прятаться и избегать ударов, а крупному – атаковать противника, используя массу и габариты. 
\paragraph{На что влияет МРз?} МРз прибавляется к Скорости, результатам проверок Силы, Доблести, Меткости метания, и Навыков (Сл).
\newline МРз вычитается из БАЗщ, результатов проверок Меткости, Ловкости и Навыков (Лв).
\begin{tcolorbox}
    Не забывайте, минус на минус дает плюс. Это означает, что, например, Крошечное существо фактически повышает свою БАЗщ, Меткостьт, МЛв и навыки от Ловкости на 2.
    %Также обратите внимание, что Меткость Метания крупных существ не изменится, так как МРз одновременно и прибавляется к ней, и отнимается от нее.
\end{tcolorbox}

Некоторые способности учитывают размеры существ. В этом случае Большое существо считается как 2 Средних, Огромное – как 3 Средних, а Громадное – как 4!
\begin{center}
    \begin{tabular}{ |c|c|c|c|c| }
        \hline
        Размер & Модификатор размера & Возможный рост & Занимаемая область
        \\ \hline
        Крошечный(К) & -2 & 0.01-0.65 метра & 0.5 × 0.5 метра
        \\ \hline
        Маленький(М) & -1 & 0.66-1.49 метра & 1 × 1 метра
        \\ \hline
        Средний(С) & 0 & 1.5-2.1 метра & 2 × 2 метра
        \\ \hline
        Большой(Б) & +1 & 2.2-3 метра & 3 × 3 метра
        \\ \hline
        Огромный(О) & +2 & 3.01-9.99 метра & 5 × 5 метров
        \\ \hline
        Громадный(Г) & +3 & 10 метров и больше & 7 × 7 метров или даже больше
        \\ \hline
    \end{tabular}
\end{center}
\paragraph{Исполинские (И) существа и устройства.} Иногда существо или устройство немыслимо велики, а герои перед ними абсолютно незначительны. Исполинские существа не принимают участия в Сценах, они и есть Сцена, на которой разворачивается действие. Герои и существа, даже Громадные, могут взаимодействовать только с частью Исполина, а не с ним целиком.

\subsection{Вторичные характеристики}
\paragraph{Воля (Вл) = (Ин + Мд) ÷ 4.} Отвечает за самоконтроль, сопротивление всевозможным соблазнам и внешним влияниям.

\paragraph{Реакция (Рц) = (Лв + Мд) ÷ 4.} Определяет порядок действия в Боевых сценах и прочих ситуациях, в которых это важно.

\paragraph{Скорость (Ск) = (Лв + Вн) ÷ 4 + МРз.} За 5 секунд герой может преодолеть число метров (или клеток, если используется масштабная карта), равное своей Ск.
\paragraph{Ск и четвероногие существа:} четвероногие существа, такие, как кони, кошки и слоны, увеличивают свою Ск в 2 раза, перемещаясь по земле. Умножьте Ск после прибавления или вычитания МРз. Тараканы, пауки и многоножки считаются четвероногими для определения Ск, а гигантские слизни, улитки и змеи – нет!
\paragraph{Ск и [Полет]:} если существо способно летать, умножьте его Ск на 3 после прибавления или вычитания МРз. Если существо по каким-то причинам не может использовать [Полет], применяйте его обычную Ск.

\paragraph{Энергия (Эн) = (Вн + Об) ÷ 4.} Отражает абстрактный запас внутренних сил. Источником Энергии героя может служить оптимизм, ненависть, неукротимый дух, вездесущий эфир, миниатюрный ядерный реактор и множество других вещей и явлений, как философско-мистического, так и обыденного толка.
\newline Энергия расходуется, когда герой активирует феномен или Функцию Атрибута – невероятные способности, недоступные большинству окружающих. Текущее значение Энергии не может опуститься ниже 0, либо превысить ее максимальное значение.

\paragraph{Чтобы восстановить Энергию, герой должен:}
\begin{itemize}
    \item[--] Уйти в Антракт и восстановить Эн до максимального значения.
    \item[--] Воспользоваться Интерлюдией с пометкой "Отдых", чтобы восстановить \textbf{|МОб|} (минимум 1) Эн.
    \item[--] Употребить быстродействующие стимуляторы или зелья.
    \item[--] Преобразовать топливо в Энергию, если снаряжение или способности героя предоставляют такую возможность. Для того чтобы восстановить 1 Эн, следует потратить 100 Зарядов (Зр).
    \item[--] Применить Ход "Рука Судьбы", чтобы немедленно восстановить \textbf{|2 + МОб|} (минимум 1) Эн.
    \item[--] Трюки и Атрибуты также могут позволить герою восстанавливать Эн. 
\end{itemize}

\paragraph{Единицы Здоровья (ЕЗ) = Вн × (МРз+3).} Показывают, сколько ущерба герой способен вынести, прежде чем потеряет сознание или умрет. Текущее значение ЕЗ не может опуститься ниже 0, либо превысить их максимальное значение.
\newline Большинство существ величиной с человека имеют Средний размер. Их ЕЗ = Вн х 3.
\newline Единицы Здоровья расходуются, когда герой получает Повреждения, или теряет Единицы Здоровья по каким-то причинам. 
\paragraph{Чтобы восстановить ЕЗ, герою придется:}
\begin{itemize}
    \item[--] Уйти в Антракт, чтобы восстановить \textbf{|МВн|} (Минимум 1) ЕЗ.
    \item[--] Использовать Интерлюдию "Посещение врача", "Расслабляющее ничегонеделание" или "Сон". Такая Интерлюдия может дополнять Антракт или сочетаться с ним.
    \item[--] Применить Навык Медицины во время Антракта.
    \item[--] Употребить быстродействующие стимуляторы или снадобья.
    \item[--] Использовать Атрибуты, Трюки, феномены и другие умения героя и его союзников.
\end{itemize}

У Вторичных Характеристик нет Модификаторов. Во всех формулах и при проверках используется их полное значение. В случае Энергии используется значение текущей Эн. Единицы Здоровья не используются для каких-либо проверок.
\newline Если при определении величины Вторичной характеристики результат деления получился меньше 1, то значение Вторичной характеристики все равно составит 1. 

\subsection{Проверки Вторичных Характеристик}
Проверка Вторичных характеристики является Проверка с Бонусом, равным значению Вторичной характеристики.

\subsection{Боевые характеристики}
\paragraph{Повреждения (Пв):} результатом успешных проверок Боевых Характеристик (за исключением Защиты) является получение Повреждений целью атаки.
\paragraph{Бонус к Повреждениям (БПв)} дается оружием и особыми способностями. Чем он выше, тем больше шансов Повредить цель. Бонус к Повреждениям является частью Боевых Характеристик Доблести и Меткости, и изменяется в зависимости от того, какое оружие или феномен использует герой.
\paragraph{Естественное оружие и Безоружный Бонус к Повреждениям (ББПв):} отражают ущерб, который герой наносит, используя свое тело – кулаки, пятки, хвост, клыки или рога. ББПв = \textbf{|МРз - 2|}, и может иметь отрицательное значение. Например, у героя Среднего размера он составит -2.
\paragraph{Доблесть (Дб) = |Нв Владения оружием + МСл + МЛв + БПв + МРз|.} Чем выше Доблесть героя, тем он опаснее в ближнем бою.
\paragraph{Рукопашная Доблесть (РДб) = |Нв Рукопашного боя + МСл + МЛв + ББПв + МРз|.} Эта Характеристика используется героем для нанесения ударов ногами, руками, клыками, когтями и другими частями тела, а также для совершения некоторых маневров.
\newline РДб является частным случаем проверки Доблести.
\newline Если правила предписывают проверить Доблесть, примените БПв оружия, которое использует герой.
\paragraph{Меткость (Мт) = |Нв Стрельбы + МЛв + БПв – МРз|.} Чем выше Меткость, тем опаснее герой в дистанционном бою. 
\paragraph{Меткость Метания (ММт) =|Нв Стрельбы + МСл + МЛв + БПв + МРз|.} Она используется при метании предметов, гранат и использовании Метательного оружия (луков и пращей в том числе). 
\paragraph{Меткость Снарядов (МтС) = |Нв Стрельбы + МФх (модификатор Феноменальной характеристики) + БПв - МРз|.} Она используется при активации феноменов – невероятных и сверхъестественных способностей.
\newline ММт и МтС являются частными случаями проверки Меткости. Если правила предписывают проверить Меткость, примените БПв оружия, которое использует герой.
\begin{tcolorbox}
	Обратите внимание, что для Меткости Метания Модификатор Размера добавляется так же, как и для Доблести, а не отнимается, как в обычной Меткости.
\end{tcolorbox}
\paragraph{Защита (Зщ) = |БАЗщ + МЛв + БД + БЩ|.} Чем выше Защита, тем сложнее поразить героя атаками. Помимо МЛв в Защиту входят:
\paragraph{Базовая защита (БАЗщ) = |10 - МРз|.} Она отражает Защиту неподвижного существа. Высокая БАЗщ обычно указывает на небольшой размер или иные факторы, затрудняющие попадание, но не связанные с подвижностью или броней. 
\begin{center}
    \begin{tabular}{ |c|c|c|c|c| }
        \hline
        Размер существа & Базовая защита
        \\ \hline
        Крошечный(К) & 12
        \\ \hline
        Маленький(М) & 11
        \\ \hline
        Средний(С) & 10
        \\ \hline
        Большой(Б) & 9
        \\ \hline
        Огромный(О) & 8
        \\ \hline
        Громадный(Г) & 7
        \\ \hline
    \end{tabular}
\end{center}

\paragraph{Бонус доспеха к Защите (БД):} защита, которую дают доспехи. Бонус доспеха изменяется в зависимости от того, какой доспех носит герой.
\paragraph{Бонус щита к Защите (БЩ):} защита, которую дают щиты. Бонус щита изменяется в зависимости от того, какой щит использует герой.
\newline Ношение некоторых доспехов и щитов ограничивает максимальный модификатор Ловкости, который можно использовать при подсчете Защиты и прочих проверках.
\paragraph{Прочность (Прч):} отражает особенности строения или конструкции существ и объектов, позволяющие противостоять ущербу. Прочность предотвращает полученные Повреждения и вычитается из них.
\newline Прочность не является Боевой Характеристикой, но тесно связана с ними.

\subsection{Проверки Боевых характеристик}
Проверка Боевой характеристики (кроме Защиты) является Проверка с Бонусом, равным значению Боевой характеристики.
\newline Герой может понизить значение боевой характеристики (но не повысить, если характеристика \textit{уже} отрицательная) вплоть до нуля во время атаки, если желает нанести удар не в полную силу.
\newline сложность проверки обычно равна Защите цели, которую выбрал герой.
\newline Подробнее о проверках Боевых Характеристик читайте в разделе "Маневры".

\paragraph{Проверки Защиты:} в Динамических сценах, не подразумевающих детализации – например, когда герой прорывается сквозь разъяренную толпу, бежит под обстрелом или мчится по коридору с ловушками, мастер вправе определить Повреждения героя при помощи проверки Защиты - Проверки с бонусом, равным \textbf{|Зщ – 10|}.

\subsection{Нулевой уровень характеристик}
В приключениях героев подстерегает немало опасностей, и далеко не все из них явные. Яды, болезни, зараженная вода и пища могут понизить значение любой Характеристики до нуля. Понижение Основной Характеристики, даже временное, приводит к понижению зависящей от нее Вторичной.
\newline Нулевой уровень Характеристики означает, что существо автоматически проваливает любые проверки, связанные с ней. Упавшие до нуля Сила, Ловкость и Скорость приводят к состоянию, близкому к параличу (хотя существо все слышит и может наблюдать за происходящим вокруг). Упавшие до нуля Интеллект, Мудрость и Обаяние приводят к коме. Если Реакция или Скорость существа падает до 0, оно впадает в ступор. Во всех этих случаях существо считается находящимся в состоянии Неподвижности. Существо с нулевой Волей покорно выполняет любые отданные ему приказы, даже очевидно самоубийственные. Если существо получит несколько приказов, противоречащих друг другу, то оно постарается выполнить их, удовлетворив максимальное число требований. Если Выносливость существа падает до нуля, то оно немедленно умирает или разваливается на части.
\newline Исключение — Боевые Характеристики. Даже с отрицательными значениями Боевых Характеристик герой может пытаться атаковать противников, каким бы безнадежным это не казалось.
