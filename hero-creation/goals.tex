\subsection{Цели}
\paragraph{}
При помощи таблиц вы можете определить Цели, встающие перед героями в начале игровой встречи. Выбранная Цель может стать как сюжетным стержнем вашей истории, так и побочной (но от этого не менее интересной) линией.
\paragraph{}
Игрок может выбрать или определить случайным образом 1 Цель своего героя в начале игровой встречи. Если герою удается успешно выполнить ее, он получает 1 дополнительное Очко опыта и может тут же взять новую Цель. Выполнение Цели может занять какое-то время, не исключено, что для этого понадобится несколько игровых встреч. Обратите внимание, что Цель предлагает лишь абстрактную идею. Конкретное наполнение приключения зависит от контекста событий, происходящих в вашей истории.
\paragraph{}
Цель может быть не только личной, но и общей. В этом случае она служит сюжетным стержнем, и Очки опыта за ее выполнение начисляются всем героям, участвующим в истории. В некоторых случаях потребуется выяснить, в каких отношениях Цель находится с героями и с кем именно — например, если выпал вариант "Враг" или "Возлюбленный".
\paragraph{}
Число Целей, которых герой пытается достичь одновременно (и за выполнение которых получает Очки опыта), не может превышать его \textbf{|МОб|} (минимум 1).
\paragraph{Выбор Цели:} в начале киньте К20 и Выберите Цель. Затем киньте К20 и Конкретизируйте Цель. После этого киньте К20 и узнайте, что герой должен сделать с Целью. В заключение киньте К20 и определите, из-за чего достижение Цели под угрозой.
\begin{center}
\begin{tabular}{ |c|c|c|c|c|c| }
\hline
\textbf{К20} & 1-4(Информация) & 5-8(Предмет) & 9-12(Разумное существо) & 13-16(Животное) & 17-20(Место) \\ \hline
1-2 & Опасная & Деньги & Друг & Злобное & Руины \\ \hline
3-4 & Бессмысленная & Зелье & Враг & Упрямое & Холм \\ \hline
5-6 & Удивительная & Ценные бумаги & Возлюбленный & Тупое & Дом \\ \hline
7-8 & Не выглядит важной & Механизм & Родственник & Смирное & Озеро \\ \hline
9-10 & Зашифрованная & Оружие & Соперник & Любопытное & Роща \\ \hline
11-12 & Полезная & Драгоценность & Незнакомец & Егозливое & Подземелье \\ \hline
13-14 & Пугает & Доспех & Важная персона & Опасное & Лес \\ \hline
15-16 & Несерьезная & Картина & Чужеземец & Отвратительное & Лаборатория \\ \hline
17-18 & Ценная & Изваяние & Нелюдь & Ядовитое & Болото \\ \hline
19-20 & {Объемная} & {Артефакт} & Кумир & Очень странное & Замок \\ \hline
\end{tabular}
\end{center}


\begin{center}
\begin{tabular}{ |c|c|c|c|c|c| }
\hline
\multicolumn{6}{|c|}{\textbf{Что герой должен сделать с Целью, если она...}} \\ \hline
\textbf{К20}& \textbf{Информация} & \textbf{Предмет} & \textbf{Разумное существо} & \textbf{Животное} & \textbf{Место} \\ \hline

1 & Уточнить & Расколдовать & Расколдовать & Расколдовать & Расколдовать \\ \hline
2 & Восстановить & Восстановить & Исцелить & Исцелить & Восстановить \\ \hline
3 & Отыскать & Отыскать & Отыскать & Отыскать & Отыскать \\ \hline
4 & Скопировать & Скопировать & Наказать & Перевезти & Очистить \\ \hline
5 & Исказить & Вернуть & Вернуть & Вернуть & Осквернить \\ \hline
6 & Передать & Передать & Изобличить & Передать & Укрепить \\ \hline
7 & Скрыть & Укрыть & Укрыть & Укрыть & Подготовить \\ \hline
8 & Выкупить & Выкупить & Выкупить & Выкупить & Выкупить \\ \hline
9 & Продать & Продать & Продать & Продать & Продать \\ \hline
10 & Выкрасть & Выкрасть & Выкрасть & Выкрасть & Освятить \\ \hline
11 & Дополнить & Спрятать & Спрятать & Спрятать & Занять \\ \hline
12 & Проверить & Подменить & Обокрасть & Подменить & Скомпрометировать \\ \hline
13 & Дискредитировать & Изготовить & Уговорить & Спарить & Спасти \\ \hline
14 & Распространить & Применить & Освободить & Освободить & Обследовать \\ \hline
15 & Получить & Получить & Отблагодарить & Передержать & Отдать \\ \hline
16 & Изучить & Изучить & Соблазнить & Изучить & Изучить \\ \hline
17 & Уничтожить & Уничтожить & Убить & Убить & Уничтожить \\ \hline
18 & Захватить & Захватить & Захватить & Захватить & Захватить \\ \hline
19 & Опровергнуть & Подбросить & Опорочить & Подбросить & Отбить \\ \hline
20 & Защитить & Защитить & Защитить & Защитить & Защитить \\ \hline
\end{tabular}
\end{center}

\begin{center}
\begin{tabular}{ |c|c| }
\hline
\textbf{К20} & \textbf{Достижение Цели под угрозой из-за...} \\ \hline
1 & Преступного синдиката \\ \hline
2 & Старых врагов \\ \hline
3 & Бандитов \\ \hline
4 & Капризов природы \\ \hline
5 & Чиновников \\ \hline
6 & Сил правопорядка \\ \hline
7 & Соперников \\ \hline
8 & Нелюдей \\ \hline
9 & Иноземцев \\ \hline
10 & Неведомых врагов \\ \hline
11 & Родственников \\ \hline
12 & Друзей \\ \hline
13 & Любви \\ \hline
14 & Конкурентов \\ \hline
15 & Важной персоны \\ \hline
16 & Высокородного \\ \hline
17 & Безумца \\ \hline
18 & Животных \\ \hline
19 & Чудовища \\ \hline
20 & Сверхъестественных сил \\ \hline
\end{tabular}
\end{center}