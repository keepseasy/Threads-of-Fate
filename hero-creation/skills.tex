\section{Навыки}
Значения Навыков отображают глубину познаний героя в различных областях. В скобках указана Характеристика, Модификатор которой прибавляется к Навыку при проверках, - она означает часть Навыка, которую формируют талант и природные склонности. К некоторым Навыкам могут прибавляться Модификаторы различных Характеристик.
\paragraph{Значение} навыка равно числу Очков опыта, распределенных в Навык.
\paragraph{Максимальное значение Навыка} не может превышать значения Интеллекта героя.
\paragraph{Типы навыков:} Навыки делятся на две категории - Основные и Экспертные.
\paragraph{Основные Навыки} представляют собой виды деятельности, в которых можно достичь успеха, делая выводы из своих побед и неудач. Конечно, многие герои постигают их при поддержке наставника, но мастера-самоучки - не такая уж редкость.
\newline Проверка Основного навыка может быть совершена даже героем, совершенно не разбирающимся в предмете.
\paragraph{Экспертные Навыки} изображают те области знаний, в которых шанс добиться результата без планомерного обучения или исключительного дарования ничтожно мал. Доступ к Навыкам открывают Атрибуты, так как большинство их обладателей имеют специальную подготовку либо врожденный и осознанный ими дар.

\subsection{Проверки Навыков}
Когда герой проверяет Навык, используя Основную характеристику, он совершает Проверку с Бонусом равным \textbf{|[значение Навыка] + [Модификатор Основной Характеристики]|}
\newline Когда герой проверяет Навык, используя Вторичную характеристику, он совершает Проверку с Бонусом равным \textbf{|[значение Навыка] + [значение Вторичной Характеристики]|}
\newline Как правило, Характеристику, которую использует Навык, определяет мастер, но игрок может предложить заменить ее на другую, если у него появились идеи, соответствующие контексту.
\newline Если вы не используете правила Состязания, но герою кто-то активно противостоит сложность проверки равна \textbf{|[значение Навыка оппонента] + [значение Вторичной Характеристики оппонента / Модификатор Основной Характеристики оппонента] + 10|}.

\paragraph{Перед проверкой} навыка определите:
\begin{itemize}
    \item[--] Уместно ли применение Навыка в контексте Сцены?
    \item[--] Может ли применение Навыка принести желаемый результат?
    \item[--] Какая Характеристика будет участвовать в применении Навыка и чем это обусловлено?
    \item[--] Требуются ли для применения Навыка какие-то дополнительные средства (инструменты, снаряжение, участие окружающих)?
    \item[--] Могут ли какие-то дополнительные средства облегчить применение Навыка и понизить Сложность проверки?
\end{itemize}

\paragraph{Альтернативное применение Навыков:} использование Навыков - творческий процесс. Зачастую, добиться желаемого можно множеством разных способов, особенно, если к этому располагает контекст. 

\paragraph{Нулевой уровень Основные Навыков:} если игрок не распределил в Навык героя хотя бы 1 Очко опыта, этот Навык проверяется с Помехой.
\paragraph{Нулевой уровень Экспертных Навыков:} если герою доступен Экспертных навык и игрок не распределил в него хотя бы 1 Очко опыта, этот Навык проверяется с Помехой. 
\newline Если герою недоступен Экспертный навык, он не может совершать соответстующие проверки, но он может использовать статичное значение этого навыка с двумя Помехами.
\paragraph{Нулевой уровень боевых Навыков:} если игрок не распределил хотя бы 1 Очко Опыта во Владение оружием, Рукопашный бой или Стрельбу героя, соответствующие проверки Доблести и Меткости совершаются с Помехой.

\paragraph{Проверка Навыка или проверка Характеристики?} Поле деятельности многих Навыков дублируют профильные Характеристики. Как определить, что выбрать для проверки? Как и во многих других случаях, рекомендуется обратиться к контексту и настроению игры. Обычно проверки Навыков требуются в областях, предполагающих минимальное знакомство с процессом. Более тривиальные задачи допускают проверки Характеристик. Разумеется, если у героя есть Навык, он может применить его в любом случае, значительно упрощая себе задачу.
\newline Герою хватит проверки Характеристики, чтобы:
\begin{itemize}
    \item[--] Вскарабкаться по узловатому стволу старого дерева (Ловкость);
    \item[--] Договориться с приятелем (Обаяние);
    \item[--] Пробежать трусцой пару километров (Выносливость);
    \item[--] Запомнить четверостишье (Интеллект);
    \item[--] Выкопать небольшую яму (Сила).
\end{itemize}
Герою потребуется Навык, чтобы:
\begin{itemize}
    \item[--] Вскарабкаться по отвесной скале или гладкому столбу (Атлетика (Ловкость));
    \item[--] Успокоить вооруженного дебошира (Общение (Обаяние));
    \item[--] Пробежать пару километров в полной боевой выкладке (Атлетика (Выносливость));
    \item[--] Запомнить содержание небольшой книги (Наблюдательность (Интеллект));
    \item[--] Выкопать длинную траншею или глубокую могилу (Атлетика (Сила)).
\end{itemize}

\subsection{Перечень основных навыков}
\genAndGet{skills}{skills}{Базовый}

\subsection{Перечень экспертных навыков}
\genAndGet{skills}{skills}{Экспертный}
