\section{РАЗМЕРЫ СУЩЕСТВ}
\paragraph{}
Во время приключений герои могут встретить множество существ, размером превышающих человека или ощутимо уступающих ему: от сказочных драконов и фей, до гиганских человекоподобных роботов и мозговых червей. За эталон принят человек ростом от 150 до 210 см. Он имеет Среднюю категорию размера. Размер не влияет на Основные характеристики существа напрямую, однако большие существа — легкая мишень для атак любого рода, а маленькие существа вынуждены использовать маленькое оружие (зачастую наносящее меньше ущерба). Герой-человек может по договоренности с мастером начать игру Маленьким (карлик) или Большим (великан).
\newline
Существа занимают определенную область в зависимости от своих размеров. Это вовсе не означает, что существо занимает эту область целиком (хотя бывает и такое). Несомненно одно — в этой области существо может и будет мешать передвижению недругов. Обратите внимание, что высота в холке четвероногих существ зачастую меньше, чем рост существ соответствующего размера, указанный в таблице.
\paragraph{Модификатор Размера (МРз):} размер, отличный от Среднего, может быть выгоден в одних ситуациях и мешать в других. Небольшие существа более подвижные, а крупным не требуется много усилий(по их меркам), чтобы сдвинуть препятствие в сторону.
\newline
Модификатор размера добавляется к Скорости, проверкам Доблести, Силы и навыкам от Силы. И вычитается из Защиты, проверок Меткости, Ловкости и навыков от Ловкости.
\newline
Некоторые способности учитывают размеры существ. У этих способностей есть свойство Размер Имеет Значение(РИЗ). В этом случае Большое существо считается как 2 Средних, Огромное — как 3 Средних, а Громадное — как 4!
\begin{center}
\begin{tabular}{ |c|c|c|c|c| }
\hline
Размер & МРз & Возможный рост & Занимаемая область
\\ \hline
Крошечный(К) & -2 & 0.01-0.65 метра & 0.5 × 0.5 метра
\\ \hline
Маленький(М) & -1 & 0.66-1.49 метра & 1 × 1 метра
\\ \hline
Средний(С) & 0 & 1.5-2.1 метра & 2 × 2 метра
\\ \hline
Большой(Б) & +1 & 2.2-3 метра & 3 × 3 метра
\\ \hline
Огромный(О) & +2 & 3.01-9.99 метра & 5 × 5 метров
\\ \hline
Громадный(Г) & +3 & 10 метров и больше & 7 × 7 метров или даже больше
\\ \hline
\end{tabular}
\end{center}
\paragraph{Исполинские(И) существа и устройства.} Иногда существо или устройство настолько велики, а герои перед ним настолько незначительны, что они не могут взаимодействовать друг с другом обычными способами. Исполинские существа не принимают участие в Сценах, они и есть Сцена, на которой может происходить действие, а герои и существа, даже Громадные, могут взаимодействовать только с частью Исполина, а не с ним целиком.