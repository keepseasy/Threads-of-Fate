\section{Грани и амплуа}
Грани и Амплуа дадут вам множество сюжетных зацепок, расскажут о прошлом героя и о том, на что он надеется в будущем. Многие вопросы, которые неизбежно возникнут после определения Граней, намеренно оставлены без ответов — их предстоит найти игрокам и мастеру!
\begin{tcolorbox}
Фактически, Грани представляют собой глубоко нишевые Уникальные ходы и Недостатки. Они рассчитаны в первую очередь на долгую игру, в которой характер героя меняется и развивается — как и мир вокруг него. Конечно, вы можете использовать их в играх на одну встречу, чтобы добавить образу героя колорита или даже построить вокруг них завязку. При этом стоит держать в голове, что Грани найдут применение далеко не в каждой истории.
\end{tcolorbox}
\paragraph{Амплуа} описывает героя одной емкой фразой, и служит хорошим дополнением к образу, созданному Атрибутами, Трюками и Недостатками. Амплуа совсем не обязано соответствовать перечню Атрибутов. Например, Дверг может выбрать Амплуа Воина и Ремесленника, игнорируя Амплуа Нелюдя, а полуэльф может использовать Амплуа Нелюдя, даже если не приобрел Атрибут Эльфа. Амплуа служат ориентиром для выбора Граней героя и не выполняют никакой игромеханической функции.
\paragraph{Грани} — значимые факты биографии, точки напряжения истории, конфликты героя, внутренние или внешние, а вместе с тем — превосходный источник идей для портрета героя. Грани состоят из \textbf{Орла} — условно положительной стороны Грани (до "но" в описании), и \textbf{Решки} — потенциально негативной стороны Грани (после "но" в описании). Игроки могут придумывать Грани самостоятельно — это поможет обогатить портрет героя деталями, которые подчеркивают жанр и настроение вашей истории.
В отличие от Недостатков и Темной стороны Атрибутов, Грани
рассчитаны на развитие ситуации со временем. Также они могут
прямо подсказать игроку и мастеру, какие Неприятности преследуют героя, каковы его социальный статус, интересы и круг общения, кто ему друг, а кто — враг.
\paragraph{Число Граней:} игрок может определить случайным образом или выбрать число Граней, равное числу Атрибутов героя, из наиболее подходящих ему Амплуа. Для героя без Атрибутов игрок может выбрать одну Грань из любого Амплуа. Орел может быть введен в игру при помощи общедоступного Хода
"Повезло" и обрыва Нити (одной или нескольких на усмотрение мастера) в ситуациях, когда герой может извлечь из этого выгоду.
\paragraph{Решка} вводится в игру как Каприз Судьбы.
Например, если герой обладает Гранью из перечня Бродяги (у героя множество родственников, но никто из них не добился успеха в жизни), игрок может оборвать Нить, объявить, что герой встретил в незнакомом городе троюродного брата и получил кров, стол или информацию. В то же время Грань ясно указывает, что герой вряд ли получит слишком много. При этом игрок может осложнить жизнь герою — протянуть к нему Нить и столкнуть лицом к лицу с восторженным племянником-недорослем, который желает путешествовать вместе с ним!
\paragraph{Смена Амплуа:} мастер может позволить заменить одно Амплуа на другое, если игрок того желает и это обусловлено логикой развития истории. Например, разочаровавшийся в идеалах Паладин может стать Бродягой, а Воин, открывший свое дело, превратится в Дельца. Смена Амплуа не вынуждает героя менять и Грани. Смену Граней, так же, как и смену Недостатков, должно определять развитие характера и образа героя. Разумеется, игрок может и вовсе отказаться от Граней героя, если сочтет нужным.
\newline
\genAndGet{roles}{roles}{Амплуа}