\section{Грани и амплуа}
Грани и Амплуа дадут вам множество сюжетных зацепок, расскажут о прошлом героя и о том, на что он надеется в будущем. Вопросы, которые неизбежно возникнут после определения Граней, намеренно оставлены без ответов - их предстоит найти игрокам и мастеру!
\begin{tcolorbox}
    Фактически, Грани представляют собой глубоко нишевые Уникальные Ходы и Недостатки. Они рассчитаны на долгую игру, в которой характер героя меняется и развивается. Конечно, вы можете использовать их в играх на одну встречу, чтобы придать герою колорит, или даже построить вокруг них завязку. Стоит держать в голове, что Грани найдут применение далеко не в каждой истории. 
\end{tcolorbox}
\paragraph{Амплуа} описывает героя емкой фразой, и служит дополнением к образу, созданному Атрибутами, Трюками и Недостатками. Амплуа не обязано соответствовать Атрибутам. Ученый может выбрать Амплуа Бродяги или Врача, игнорируя Амплуа Ученого, а герой без Атрибута может использовать Амплуа Жулика, даже если не приобрел Атрибут Плута или Коммивояжера. Амплуа служат ориентиром для выбора Граней и не выполняют игромеханических функции.
\paragraph{Грани} - Грани - факты биографии, точки напряжения истории, конфликты героя, внутренние или внешние, а вместе с тем - источник идей. Грани состоят из \textbf{Орла} - условно положительной стороны Грани (до "но" в описании), и \textbf{Решки} - потенциально негативной стороны Грани (после "но" в описании). 
\newline В отличие от Недостатков и Темной стороны Атрибутов, Грани рассчитаны на развитие ситуации со временем. Также они могут прямо подсказать игроку и мастеру, какие Неприятности преследуют героя, каковы его статус, интересы и круг общения, кто ему друг, а кто - враг.
\begin{tcolorbox}
    Разумеется, Грани могут использоваться вне контекста Амплуа. Если вам понравилась одна из них - смело записывайте ее в лист (если такую же еще не выбрал кто-то из ваших соигроков).
\end{tcolorbox}
\paragraph{Орел} вводится в игру Ходом "Повезло" в ситуациях, когда герой может извлечь из этого выгоду. 
\paragraph{Решка} вводится в игру как Каприз Судьбы.
\paragraph{Число Граней:} игрок может определить случайным образом или выбрать 0-1 Грань из наиболее подходящих герою Амплуа. 
\paragraph{Смена Амплуа:} одно Амплуа можно заменить на другое, если игрок того желает и это обусловлено развитием истории. Разочаровавшийся в идеалах Законник может стать Бродягой, а Солдат, открывший свое дело, превратится в Дельца. Смена Амплуа не обязывает менять и Грани. Разумеется, игрок может и вовсе отказаться от Граней, если сочтет нужным.
\newline
\genAndGet{roles}{roles}{Амплуа}