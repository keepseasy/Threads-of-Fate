\section{После определения основных и вторичных характеристик}
\begin{itemize}
\item[--] Выберите для героя 2 Атрибута. Герой может отказаться от 1 Атрибута и получить 5 дополнительных Очков опыта. Если герой отказывается от 2 Атрибутов сразу, он получает 10 дополнительных Очков опыта (всего!).
\item[--] Выберите для героя 2 Трюка.
\item[--] Выберите для героя 0—2 Недостатка (на старте не рекомендуется повторять их у героев разных игроков).
\item[--] Выберите для героя число Граней, не превышающее число Атрибутов (на старте не рекомендуется повторять их у героев разных игроков).
\item[--] Выберите для героя 0—2 Уз (на старте не рекомендуется повторять их у героев разных игроков).
\item[--] Распределите 10 Очков опыта по Навыкам героя.
\end{itemize}
\begin{tcolorbox}
Атрибуты и Трюки наделяют героя огромным количеством способностей. Если вы хотите отразить в игре процесс становления героев, а не авантюру уже состоявшихся искателей приключений, можно начать игру с одним Трюком и одним Атрибутом.
\end{tcolorbox}
