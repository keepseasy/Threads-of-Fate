\section{Трюки}
Герой так же может получать Трюки, не привязанные к Атрибутам. Эти трюки требуют меньше очков опыта для приобритения и работают так же, как и Трюки внутри Атрибутов.

\section{Боевые Трюки}
Трюки, ориентированные на боевые столкновения. Они дают значительные преимущества в боевых ситуациях, однако в других сценах ими воспользоваться будет \textit{сложнее}.
\genAndGet{tricks}{tricks}{Боевой}

\section{Социальные Трюки}
Трюки, ориентированные на социальные взаимодействия. С их помощью можно заручиться поддержкой статистов, произвести хорошее Впечатление или даже избежать конфликта до его начала. Однако, когда присутствующие в сцене похватались за оружие, Социальные Трюки уже \textit{скорее всего} не помогут.
\genAndGet{tricks}{tricks}{Социальный}

\section{Трюки Могущества}
Трюки, влияющие на Феномены и Энергию героя. Не обращайте внимания на эти трюки, если вы не собираетесь создавать мага, псионика или супергероя.
\genAndGet{tricks}{tricks}{Могущество}

\section{Вспомогательные Трюки}
Эти Трюки не имеют сильной специализации на тех или иных Сценах или же несут вспомогательную функцию, которая делает жизнь героя проще.
\genAndGet{tricks}{tricks}{Вспомогательный}
