\section{Трюки}
Трюки – уловки, умения и качества, помогающие герою. Они не так масштабны, как Атрибуты, но не стоит их недооценивать. В некоторых ситуациях Трюк способен не только облегчить жизнь, но и спасти ее. 
\newline Трюки не могут быть приобретены несколько раз, если в описании не указано обратного.
\newline Трюки, зависящие от Модификаторов Характеристик не могут использоваться, если требуемый Модификатор или сумма с ним нулевая или отрицательная. Исключения указаны в описаниях Трюков.

\subsection{Боевые Трюки}
Трюки, ориентированные на боевые столкновения. Они дают значительные преимущества в боевых ситуациях, однако в других сценах ими воспользоваться будет \textit{сложнее}.
\genAndGet{tricks}{tricks}{Боевой}

\subsection{Социальные Трюки}
Трюки, ориентированные на социальные взаимодействия. С их помощью можно заручиться поддержкой статистов, произвести хорошее Впечатление или даже избежать конфликта до его начала. Однако, когда присутствующие в сцене похватались за оружие, Социальные Трюки уже \textit{скорее всего} не помогут.
\genAndGet{tricks}{tricks}{Социальный}

\subsection{Трюки Могущества}
Трюки, влияющие на Феномены и Энергию героя. Лучше всего подходят для магов, псиоников или супергероев.
\genAndGet{tricks}{tricks}{Могущество}

\subsection{Вспомогательные Трюки}
Эти Трюки не имеют сильной специализации на тех или иных Сценах или же несут вспомогательную функцию, которая делает жизнь героя проще.
\genAndGet{tricks}{tricks}{Вспомогательный}

\printindex[tricks]