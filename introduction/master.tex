\section*{Если вы мастер…}

\paragraph{Готовьте завязку, а не сюжет.} Все, что безусловно нужно игре — отправная точка. Остальное сделают игроки, кубики и воображение. \textbf{Обозначьте возможности и цену}. Лгите героям, но не игрокам. Игрокам стоит знать, ради чего герои рискуют, каковы шансы на победу, как именно можно достичь цели… и с чем придется расстаться по пути.
\paragraph{Используйте Капризы Судьбы.} Расшевелите игроков, вводя в игру Недостатки, Темные стороны и Решки их героев. Не давайте героям опомниться, а игрокам — заскучать.
\paragraph{Не будьте всеведущим.} Позвольте игрокам вас удивить. К тому же, чем больше в вашей игре белых пятен, тем больше возможностей для сотворчества. Игрокам будет непросто придумать что-то, если для этого чего-то не осталось места. Используйте проверки Неприятностей, если контекст не дает однозначного ответа на возникший вопрос.
\paragraph{Помогайте, не заставляйте.} Если по каким-то причинам динамика игры падает, а у игроков нет идей — подкиньте и идей, и событий. При этом не стоит заменять идеи игроков своими и прибегать к мастерскому праву вето слишком часто.