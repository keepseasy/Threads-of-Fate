\section{Симбионты}
\paragraph{Симбионты} - это подвид Инструментов могущества, который имеет некоторые особенности:
\begin{itemize}
\item У Симбионтов нет своего Запаса энергии - они всегда используют Энергию владельца на активацию своих Функций.
\item Стоимость Функций Симбионтов указана не в Зарядах, а в Энергии. Работа Симбионтов это очень энергозатратный процесс.
\end{itemize}
\begin{tcolorbox}
Симбионты - это не обязательно живые организмы. Это может быть техническое устройство или артефакт, котрый использует внутреннюю Энергию своего владельца во время функционирования.
\end{tcolorbox}

\genAndGet{tools}{tools}{Симбионт}

\tbd
%\genAndGet{tools}{tools-symbionts}
\section{Импланты}
\paragraph{Импланты} - это подвид Симбионтов, которые в дополнение имеют следующие способности:
\begin{itemize}
\item Имплант нельзя просто снять или надеть - для этого требуется помощь специалиста, и порой очень дорогостоящая.
\item установленные Импланты не учитываются в нагрузке героя даже если сами по себе они имеют значительный вес.
\item Имплант найти гораздо легче, чем другие Инструменты Могущества - у них почти всегда есть СП.
\end{itemize}

\subsection{Встроенное оружие}
В некоторых мирах Герой имеет возможность встроить оружие в свое тело. СП такого оружия возрастает на 2, оно занимает одну ячейку имплантов и получает свойства Естественное и Потайное.

\genAndGet{tools}{tools}{Имплант}