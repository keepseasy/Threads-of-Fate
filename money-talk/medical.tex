\subsection{Медикаменты и яды}
\paragraph{Интоксикация.} получение (Я)довитых Пв не приводит к потере ЕЗ напрямую и подсчитывается отдельно от остальных Пв. Интоксикация является суммой всех Ядовитых Пв, полученных героем, и отражает то, сколько яда попало в организм и как долго он может сопротивляться яду.
\paragraph{Максимальное значение Интоксикации} составляет \textbf{|Максимальные ЕЗ героя х 1.5|}. 
\paragraph{Интоксикация и смерть:} как только значение Интоксикации героя превышает \textbf{|Максимальные ЕЗ героя х 1.5|}, он должен совершить проверку Вн против \textbf{|15|}. При провале герой погибает (как правило, в страшных мучениях). При успехе он впадает в кому и становится Неподвижным до тех пор, пока его Интоксикация не уменьшится.
\begin{tcolorbox}
    Да, проверять Выносливость придется с Помехой за Отравление.
\end{tcolorbox}
\paragraph{Токсичность (Токс):} количество Ядовитых повреждений, которое наносит Лекарство или Яд при применении.
\paragraph{Первичный эффект:} Разовый эффект Лекарства или Яда. Для сопротивления эффекту герой должен преуспеть в проверке Вн против \textbf{|10+Токс|}.
\paragraph{Отравление:} пока значение Интоксикации превышает текущие ЕЗ героя, он получает состояние "Отравлен". 
\paragraph{Снятие состояния Отравления:} герой должен добиться того, чтобы значение Интоксикации не превышало его текущие ЕЗ. Он может достичь этого, как повысив текущие ЕЗ, так и понизив Интоксикацию.
\paragraph{Побочки:} целебные зелья, таинственные эликсиры и мощные стимуляторы зачастую являются не очень-то полезными. Особенно при частом применении. 
\newline Как только герой становится Отравлен, на него немедленно накладываются эффекты Побочек Лекарств и Ядов, которые наносили ему Ядовитые Пв в этой Сцене. 
\newline Если на героя подействовало лекарство или яд, пока герой Отравлен, Побочки начинают действовать \textit{сразу}.
\newline Эффекты Побочек прекращаются, как только герой перестает быть Отравлен (если в описании Побочек не сказано иного).
\begin{tcolorbox}
    Игрок может отказаться от совершения проверки Вн, если считает, что Первичный эффект пойдет герою на пользу. Судьбе виднее! Учтите, что иногда решение придется принимать наугад, до того, как эффекты войдут в игру. Например, если у пузырька с таблетками стерлась этикетка. Или когда единственный источник информации об эффектах зелья - приготовивший его шаман. Разумеется, у Судьбы есть свои способы добиться нужного результата - о них вы уже читали в главе "Нити, Ходы и Капризы".
\end{tcolorbox}
\paragraph{Антракт и побочки.} Антракт завершает действие всех Побочек, если в описании Яда или Лекарства не сказано обратного. Если при этом Интоксикация героя выше, чем текущие ЕЗ, он все еще в состоянии "Отравлен".
\paragraph{Интоксикация и Отдых:} во время Интерлюдий и Антракта герой понижает Интоксикацию на столько же единиц, сколько ЕЗ восстанавливает. Если герой применяет стимулятор или зелье, то он не понижает Интоксикацию, если это прямо не указано в описании препарата. 
\newline Если герой обращается за помощью к врачу или целителю, то снижение Интоксикации является отдельной Услугой. 

\subsection{Медикаменты}
\paragraph{Время приема (ВП):} время, необходимое для употребления одной порции Лекарства. У Ядов отсутствует ВП, т.к. оно сильно зависит от формы отравляющего вещества. 
\genAndGet{drugs}{drugs}{Лекарство}
\printindex[potions]

\subsection{Яды}
\paragraph{Отравленное оружие:} Яд на оружии действует в течение колличества атак, равных его Токсичности. Это правило распространяется и на оружие ближнего боя, и на дальнобойное оружие. Для дальнобойного оружия это значит, что порция Яда была распределена равномерно между боеприпасами. 
\newline Получив хотя бы 1 Пв, цель дополнительно получает Ядовитые Пв в размере Токсичности Яда и должна сопротивляться его Первичному эффекту.
\paragraph{Условия приема} их четыре - вдох, контакт, порез, проглатывание.
\newline Существует огромное количество ядов, - как природных, так и рукотворных, со схожими эффектами. Как правило, у отравителя нет проблем с тем, чтобы добыть яд в нужной ему форме. Для разрешения неоднозначных ситуаций используется таблица Доступности товаров и услуг.
\paragraph{КУ Ядовитых повреждений:} если Ядовитая атака наносит КУ, Побочки Яда сразу входят в игру и действуют (если эффект можно растянуть во времени), пока герой не уйдет в Антракт, получит антидот или значение Интоксикации не упадет до 0.
\genAndGet{drugs}{drugs}{Яд}
\printindex[poisons]
\begin{tcolorbox}
    Как правило, те, кто использует Яды для всяких злодейских дел, не скупятся, и сразу скармливают жертве несколько доз.
    \newline Для тех же, кто ищет в Ядах силу - например, исказителей с Трюком "Третий глаз", галлюциноген - самый удачный выбор.
\end{tcolorbox}
