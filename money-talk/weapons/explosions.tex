\subsection{Гранаты и взрывы}
В бою всегда найдется место бутылкам с горючей смесью, емкостями с едким газом и старым добрым гранатам. И это, и многое другое, можно и нужно метать во врага. А потом наблюдать с безопасного расстояния, как он мечется в попытках погасить пламя, выкашливает свои легкие, истекает кровью, изрешеченный осколками. Или как бутыль предательски падает в траву, газ уносит ветер, а гнилой запал не срабатывает.
\paragraph{Метание гранаты.} Общие характеристики гранаты, как метательного снаряда, описаны в таблице дальнобойного оружия.

\paragraph{Бросок в землю:} гранаты можно метать, поражая \textbf{|Зщ 10|} участка грунта, если только герой не собирается причинить ущерб самим броском. Земля, Неподвижна относительно героя, а потому он получает Преимущество на проверку Мт.
\paragraph{Промах:} не волнуйтесь, если только не выпала Осечка, граната все равно взорвется. Граната отклоняется на Х метров, где Х равен промаху героя по Зщ цели - ловкий противник может успеть отбросить гранату, а бронированный - отбить щитом или доспешным рукавом! Определите направление, в котором отклонилась граната, при помощи проверки Неприятностей. В этом случае граната может пролететь большее расстояние, чем максимальная дистанция броска.
\trouble
{А была ли граната}{Граната взрывается в воздухе, не причинив никому вреда.}
{Перелет}{Граната перелетает цель на Х метров.}
{Мимо}{Граната отклоняется в сторону от цели на Х метров.}
{Недолет}{Граната не долетает до цели на Х метров!}

\paragraph{Сила Взрыва (СВ):} все существа, попавшие в радиус Взрыва, совершают проверку Зщ протв СВ. 
\paragraph{Центр Взрыва:} точка, из которой распространяется Взрыв.
\paragraph{Радиус Взрыва (РВ):} число метров, на которое распространяется Взрыв из центра Взрыва во все стороны. Цель проверки Мт также считается находящейся в области Взрыва и получает Пв и прочие эффекты и от него тоже.
\paragraph{Эффекты:} начинают действовать сразу после взрыва и применяются ко всем существам и предметам, попавшим в РВ, даже если они не получили Пв. 
\paragraph{Другие опасности Взрывов:} Взрывы опасны не только Повреждениями. Доспехи, щиты и прочие предметы, закрепленные на теле попавшего во Взрыв, получают Пв = \textbf{|СВ - Прч|} и могут быть уничтожены. 
\begin{tcolorbox}
    А еще, если Взрыв достаточно силен, попавшие в него существа (с пронзительными воплями) и предметы (относительно беззвучно) разлетаются в разные стороны. 
\end{tcolorbox}

\paragraph{Газ:} газ распространяется в РВ и остается там какое-то время. Для определения длительности действия газа герои (нападающие или попавшие в РВ) проверяют Неприятности. Вдохнувшие газ страдают от эффектов яда, громко кашляют и чихают - если они еще в состоянии кашлять и чихать. 
\trouble
{Штиль}{Газ рассеивается через 10 минут.}
{Бриз}{Газ рассеивается через 5 минут.}
{Ветер}{Газ рассеивается через 1 минуту.}
{Шквал}{Газ рассеивается по истечении полного Круга.}
\tbd

\subsection{Список гранат}
\genAndGet{explosives}{explosives}{Взрывчатка}