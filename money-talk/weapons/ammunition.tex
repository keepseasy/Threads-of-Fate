\subsection{Боеприпасы дальнобойного оружия}
Разные оружия применяют разные боеприпасы. Однако со временем производители начали вводить унифицированные боеприпасы для разных калибров, что позволило использовать одни и те же боеприпасы для не совсем одинакового вооружения.
\paragraph{Тип} боеприпаса определяет, в какие пушки можно его заряжать.
\paragraph{Количество} определяет, сколько едениц боеприпаса можно приобрести, купив стандартную коробку боеприпасов. Если в этой графе стоит прочерк, боеприпас нельзя приобрести в стандартизированном варианте. Правила покупки боеприпасов будут указаны отдельно.
\begin{tabular}{|c|p{3cm}|p{10cm}|c|}
\hline
Тип & Оружие, которое использует боеприпасы & Описание боеприпасов & Кол-во\\ \hline
П & пистолеты и пистолеты-пулеметы & пули малого калибра, которые обычно легко купить & 30\\ \hline
Р & револьверы и крупнокалиберные пистолеты & тяжелые пули и дорогие пули & 15\\ \hline
В & винтовки & быстрые пули с хорошей пробивающей способностью & 20\\ \hline
Д & дробовики & дробь и картечь. Реже - пули & 4\\ \hline
О & пулеметы и крупнокалиберные орудия & действительно большие калибры. Размеры настолько высоки, что в них можно поместить взрывчатку & 10\\ \hline
Г & гранатометы и ракетные установки & Унифицированные гранаты и ракеты. Поражающий эффект зависит от начинки & 1\\ \hline
Э & энергетическое оружие & Батареи всех пород и мастей. Если у оружия класса Э нет свойства Потребления, оно тратит 1 Заряд за выстрел. & -\\ \hline
Б\textsuperscript{ф} & кинетические ускорители & Металические болты с сверхтвердым сердечником. Дешевы в изготовлении, но если их хорошенько разогнать, становятся крайне смертоносны. & 40\\ \hline
М & метательное & Оружие использует метательный боеприпас и приводится в действие силой стрелка. & -\\ \hline
У & уникальный боеприпас & Оружие использует не стандартизированный боеприпас, который подойдет только для него. СП 10 зарядов для такого вида оружия равна \textbf{|СП оружия — 10|}. Стоимость 1 заряда для этого оружия равна \textbf{|СП оружия — 12|} (минимум 1). & -\\ \hline
\end{tabular}

Кроме стандартных боеприпасов можно встретить и специализированные, которые придают дополнительные свойства или улучшают характеристики используемого оружия. В таблице указаны наиболее распространенные типы боеприпасов и стоимость их стандартных коробок.
\newline Значения, Дистанции, БПв, ТПв, КУ, указанные в таблице относительные - они изменяют на указанное значение характеристики использоваемого оружия.
\newline В столбце \textbf{Тип} указаны типы боеприпасов, для которых возможна модификация
\newline В столбце СП указана полная стоимость стандартной коробки боеприпасов.
\newline В столбце КУ отрицательные значения означают расширение диапазона КУ, а положительные - его сужение. Итоговое значение КУ оружия не может превышать 20 и быть меньше 1. Если боеприпас изменяет КУ за пределы этих значений, то они становятся равными 20 и 1 соответственно.

\begin{longtable}{|p{3cm}|p{2.5cm}|c||c|c|c|c||c|}\hline
Название & Особые свойства & Тип & Дистанция & БПв & ТПв & КУ & СП\\ \hline
Стандартные & - & - & - & - & - & - & 8\\ \hline
Утяжеленные & Оружие получает свойство Кувалда & РВОГ & -10/-20 & - & +Д & +1 & 10\\ \hline
Зажигательные & - & РВДО & - & +1/+1 & +О & -1 & 11\\ \hline
Бронебойные & Класс Оружия +1 & РВДОГБ & - & +1/+1 & - & - & 11\\ \hline
Пустотелые & - & ПРВОБ & - & -1/-1 & - & -1 & 9\\ \hline
Подкалиберные & - & ПРВДО & +10/+20 & -1/-1 & - & - & 9\\ \hline
Дозвуковые & Патроны, которые не производят хлопка при выстреле. Если оружие оборудовано глушителем, выстрел не демаскирует стрелка & ПРВО & -5/-10 & -1/-1 & - & - & 9\\ \hline
\end{longtable}