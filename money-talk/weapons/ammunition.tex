\subsection{Боеприпасы дальнобойного оружия}
Разные оружия применяют разные боеприпасы. Однако со временем производители начали вводить унифицированные боеприпасы для разных калибров, что позволило использовать одни и те же боеприпасы для не совсем одинакового вооружения.
\paragraph{Тип} боеприпаса определяет, в какие пушки можно его заряжать.
\paragraph{Количество} определяет, сколько зарядов можно приобрести, купив стандартную коробку боеприпасов. Если в этой графе стоит прочерк, боеприпас нельзя приобрести в стандартизированном варианте.
\begin{center}\begin{tabular}{|c|p{3cm}|p{10cm}|c|}
\hline
Тип & Оружие, использующее боеприпасы & Описание боеприпасов & Кол-во\\ \hline
П & Пистолеты, пистолеты-пулеметы & Малый калибр, низкая цена & 30 \\ \hline
Р & Револьверы, крупнокалиберные пистолеты & Тяжелые и дорогие пули & 15 \\ \hline
В & Винтовки & Быстрота, высокая пробивающая способность & 20 \\ \hline
Д & Дробовики & Дробь, картечь, реже - пули & 4 \\ \hline
О & Пулеметы и крупнокалиберные орудия & Действительно большие калибры. Настолько, что в них можно поместить взрывчатку & 10 \\ \hline
Г & Гранатометы и ракетные установки & Унифицированные гранаты и ракеты. Поражающий эффект зависит от начинки & 1 \\ \hline
Э & Энергетическое оружие & Батареи всех мастей. Если у оружия класса "Э" нет свойства "Потребление", оно тратит 1 Заряд за выстрел & - \\ \hline
Б\textsuperscript{ф} & Кинетческие ускорители & Металлические болты со сверхпрочным сердечником. Дешевы в изготовлении, смертоносны на больших скоростях. & 40 \\ \hline
М & Метательное оружие & Использует метательный боеприпас и приводится в действие силой стрелка & - \\ \hline
У & Уникальный боеприпас & Оружие использует не стандартизированный боеприпас, который подойдет только для него. Если СП не указана в описании оружия, то СП 10 зарядов для такого вида оружия равна \textbf{|СП оружия — 10|}. Стоимость 1 заряда для этого оружия равна \textbf{|СП оружия — 12|} (минимум 1). & * \\ \hline
% Ф & Феномены & Оружие использует Энергию героя для работы. & - \\ \hline
\end{tabular}\end{center}
* если у метательного оружия есть свойство Снаряды. Иначе каждую еденицу оружия надо приобретать отдельно.

\paragraph{} Кроме стандартных боеприпасов можно разжиться и специализированными. Они дают дополнительные свойства или улучшают свойства оружия. В таблице указаны наиболее распространенные типы боеприпасов и стоимость их стандартных коробок. Значения Дистанции, БПв, ТПв, КУ в таблице изменяют стандартные параметры оружия на указанное значение. 
\newline В столбце \textbf{Тип} указаны типы боеприпасов, для которых возможна модификация.
\newline В столбце \textbf{СП} указана стоимость стандартной коробки боеприпасов. 
\newline В столбце \textbf{КУ} отрицательные значения означают расширение диапазона КУ, а положительные - его сужение. Итоговое значение КУ оружия не может превышать 20 и быть меньше 1. Если боеприпас выводит КУ за пределы этих значений, то они становятся равными 20 и 1 соответственно. 

\begin{longtable}{|p{3cm}|p{2.5cm}|c||c|c|c|c||c|}\hline
Название & Особые свойства & Тип & Дистанция & БПв & ТПв & КУ & СП\\ \hline
Стандартные & - & - & - & - & - & - & 8\\ \hline
Утяжеленные & Оружие получает свойство "Кувалда" & РВОГ & -10/-20 & - & +Д & +1 & 10\\ \hline
Зажигательные & - & РВДО & - & +1/+1 & +О & -3 & 11\\ \hline
Бронебойные & - & РВДОГБ & - & +2/+2 & - & +2 & 11\\ \hline
Пустотелые & - & ПРВОБ & - & -1/-1 & - & -5 & 9\\ \hline
Подкалиберные & - & ПРВДО & +10/+20 & -1/-1 & - & - & 9\\ \hline
Разрывные & Полученные целью Пв удваиваются & ПРВДО & -10/-20 & -2/-2 & +Р & +1 & 12\\ \hline
Дозвуковые & Патроны не производят хлопка. Стрелок не обнаруживает себя при промахе & ПРВО & -5/-10 & -1/-1 & - & - & 9\\ \hline
\end{longtable}
