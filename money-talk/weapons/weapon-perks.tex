\subsection{Cвойства оружия}
\subsubsection{Бронебойное} Оружие ополовинивает БД и БЩ цели.
\subsubsection{Винтовка} За каждый пропуск Очереди во время Прицеливания шанс КУ возрастает на 1. Максимальный бонус к КУ не может превышать \textbf{|1 + МИн|} стрелка (минимум 1).
\subsubsection{Возврат Х} При выпадении Х или большего числа во время проверки Мт оружие возвращается в руку к владельцу, даже если цель получила Пв.
\subsubsection{Гарпун} Когда герой наносит КУ цели, оружие остается в ее теле. Запишите ЕЗ, потерянные целью в результате проверки, нанесшей КУ.
\newline Если гарпун оснащен веревкой или цепью, то пока он в теле жертвы, герой может отказаться от Перемещения и повалить жертву, либо подтянуть к себе на свой МСл или МЛв метров.  Для этого он должен успешно проверить Сл, Лв или Атлетику (Сл, Лв) против \textbf{|10 + 2*[МРз цели] + [МСл цели]|}. 
\newline Герой может отказаться от Перемещения  и вырвать гарпун, успешно проверив Сл или Атлетику(Сл) против \textbf{|10 + [ЕЗ, потерянные целью при КУ]|}. Цель повторно теряет ЕЗ, потерянные при КУ.
\newline Проверка совершается с 1 Преимуществом за каждую 1, на которую МРз нападающего больше МРз цели, либо с 1 Помехой за каждую 1, на которую МРз нападающего меньше МРз цели.
\newline При провале герой выпускает веревку. При Критическом провале герой выпускает веревку и падает.
\newline Жертва может освободиться, успешно проверив Силу или Атлетику(Сл) против \textbf{|10 + [потерянные при КУ ЕЗ]|} и повторно теряет ЕЗ, потерянные при КУ. 
\newline Проверка совершается с 1 Преимуществом за каждую 1, на которую МРз жертвы больше МРз нападающего, либо с 1 Помехой за каждую 1, на которую МРз жертвы меньше МРз нападающего.
\subsubsection{Пробивание} Если цель получила Опасную рану, стрелок может атаковать объект, стоящий на линии выстрела за целью. Требуется новая проверка Мт. Число дополнительных целей на одной линии не может превышать \textbf{|1 + МИн|} стрелка (минимум 1). Обратите внимание, что в этом случае снаряд не может застрять внутри цели даже при выпадении КУ, хотя все остальные эффекты КУ применяются, как обычно.
\subsubsection{Граната} Оружие или боеприпас взрывается сразу или через некоторое время после попадания по цели. Эффект взрыва зависит от начинки и будет описан в особых свойствах оружия.
\subsubsection{Громоздкое} Используется с Помехой в тесных помещениях, густых зарослях и в толпе. Если оружие одновременно и Громоздкое, и Длинное, герой получает 2 Помехи. Герой не может использовать Громоздкое оружие ближнего боя лежа.
\subsubsection{Двуручное} Требует 2 рук для использования.
\subsubsection{Дистанция X/Y} Оружие с этим свойством является дальнобойным. \textbf{X} - максимальная Ближняя Дистанция оружия в метрах, \textbf{Y} - максимальная Дальняя Дистанция оружия в метрах. В графе БПв оружия через черту указан урон для Ближней и Дальней дистанции соответственно.
\subsubsection{Длинное} Увеличивает Боевой контакт владельца на 1 метр (т.е., до 2 метров для Среднего существа). Цели в 1 метре от себя герой атакует с Помехой. Также Длинное оружие используется с Помехой в тесных помещениях, густых зарослях, в толпе и других подобных условиях.
\subsubsection{Естественное} Оружие является частью тела героя и использует РДб для проверок. Его нельзя уронить или выбить, в том числе маневром "Разоружение".
\subsubsection{Кавалерийское} Плученные целью Пв удваиваются, если нападающий верхом и применил Разбег.
\subsubsection{Крюк} Оружие может использоваться для Захвата.
\subsubsection{Кастет} Позволяет использовать как Навык Рукопашного боя, так и Навык Владения оружием в формуле Дб. При этом герой является вооруженным.
\subsubsection{Кувалда} Если полученные целью Пв превышают \textbf{|2*МРз + МСл|} цели, она сбита с ног и падает. Цели, не имеющие значения Сл (или ног, с которых их можно сбить), имунны к этому эффекту. 
\subsubsection{Легкое} Позволяет совершать серии молниеносных выпадов и эффективно нападать с оружием в каждой руке.
\newline Дистанционное оружие с этим свойством позволяет эффективнее вести огонь, имея оружие в каждой руке.
\subsubsection{Накопление заряда} Совершая Прицеливание, герой повышает БПв оружия на 2 за каждую пропущенную Очередь, до максимума в +6. Если герой теряет бонусы Прицеливания, бонус Накопления заряда сохраняется. 
\newline Так же герой может Прицеливаться из этого оружия, не заявляя цель маневра. В этом случае он получает только бонус Накопления заряда.
\subsubsection{Метательное} Может быть поднято или извлечено из цели и использовано повторно. Использует ММт при Дистанционных атаках. 
\newline У Метательного оружия нет свойства Перезарядка 
\newline В таблицах свойство Метательное указано в столбце ТМС, как Тип оружия.
\newline Оружие, не предназначенное для метания, может быть брошено с Помехой. Максимальная дистанция для броска такого оружия - 5. Подразумевается бросок, наносящий цели Пв, а не максимальная дальность в принципе.
\subsubsection{Надежное} Благодаря простоте конструкции или защите от дурака, оружие устойчиво к Износу. Его проверки Износа совершаются с Преимуществом. Все оружие ближнего боя обладает этим свойством.
\subsubsection{Огнемет Х} Поражает все объекты, находящиеся в радиусе Х от цели атаки. Мт проверяется лишь раз, все пораженные объекты получают Пв, исходя из нее. При КУ эффекты применяются ко всем пораженным объектам.
\subsubsection{Отдача} Герой должен отказаться от Перемещения, если желает сделать из оружия 5 или более выстрелов.
\subsubsection{Перезарядка} Для перезарядки оружия с этим свойством герой должен отказаться от Действия или Перемещения. Оружие не может использоваться при Быстрой атаке. Арбалеты и пороховое оружие требуют 2 свободных рук при перезарядке.
\subsubsection{Потайное} Герой проверяет Скрытность и Ловкость рук с Преимуществом, когда прячет оружие.
\subsubsection{Потребление Х} Оружие с этим свойством вместо тратит Х Зарядов или Боеприпасов за 1 выстрел или удар.
\subsubsection{Противотанковое} Оружие предназначено для поражения тяжелобронированных целей и игнорирует их Прч. У целей, не имеющих высокого значения Прч, не так много шансов пережить попадание. Если значение Прч цели меньше 5 и она должна получить Пв, цель совершает проверку Внезапной Смерти со штрафом, равным величине успеха при попадании.
\subsubsection{Сеть} Атаки Сетью позволяют совершать Захват. Для Захвата герой проверяет Дб в Боевом контакте и ММт при дистанционных атаках, игнорируя БД и БЩ цели. При КУ цель становится Неподвижной, пока не будет освобождена.
\newline Если сеть оснащена веревкой или цепью, то пока цель Захвачена, герой может отказаться от Перемещения и повалить жертву, либо подтянуть ее к себе на свой МСл или МЛв метров.  Для этого он должен успешно проверить Сл, Лв или Атлетику (Сл, Лв) против |10 + 2*МРз цели + МСл цели|. 
\newline Проверка совершается с 1 Преимуществом за каждую 1, на которую МРз нападающего больше МРз цели, либо с 1 Помехой за каждую 1, на которую МРз нападающего меньше МРз цели. При провале герой выпускает веревку. При Критическом провале герой выпускает веревку и падает. 
\subsubsection{Снаряды} Оружие с этим свойством считается Метательным. Однако герой не метает оружие целиком, а использует заранее подготовленные снаряды. Оружием нельзя эффективно атаковать в ближнем бою, но стоимость выстрела гораздо ниже. СП 10 снарядов для оружия равна |СП оружия — 10|. Стоимость 1 заряда для оружия равна |СП оружия — 12| (минимум 1). 
\subsubsection{Сверхдальняя дистанция} Cнайперское оружие и ракетометы могут совершать выстрелы на Сверхдальнюю дистанцию - до 1500 м. Использется БПв дальней дистанции, Мт проверяется с 2 Помехами. Выстрел получает Осечку 5. Ее выпадение означает, что выстрел произведен, но из-за неких внешних факторов заряд не достиг цели.
\subsubsection{Снайперское} При стрельбе на Дальнюю дистанцию оружие игнорирует штрафы Зон поражения, если герой Прицеливается. На ближней дистанции герой получает Помеху на Мт.
\subsubsection{Сошки(с)} Оружие оснащено сошками для стрельбы с упора. Если носитель имеет возможность установить сошки на какую-либо поверхность, используйте значение тСл с пометкой (с), в противном случае используйте значение после черты. Если у оружия нет значения тСл без символа (с), значит, стрельба с рук невозможна.
\subsubsection{Удавка} Этим свойством обладает оружие, достаточно гибкое, чтобы захлестнуть шею.
\newline Для применения свойства необходимо две руки. Оружие может использоваться для Захватов. Удвойте Пв, полученные Захваченной целью при удушении. Существа, чьи ЕЗ достигли 0 в результате удушения, теряют сознание, а не умирают, если нападающий того пожелает.
\newline Существа с БД 6 или больше, не могут быть атакованы Удавкой в шею - она надежно защищена.
\subsubsection{Универсальное} Может использоваться в 1 или 2 руках. БПв и тСл указаны для одной и двух рук через запятую.
\subsubsection{Упредительный удар} Если нападающий вошел к герою в Боевой контакт, герой вправе израсходовать Быстрое действие и провести маневр "Атака" по нападающему или его средству передвижения. Свойство может применяться вне Очереди героя.
\subsubsection{Фехтовальное} Герой может заменить МСл на ММд при подсчете Дб.
\subsubsection{Хрупкое} Оружие уязвимо к самым незначительным повреждениям и не предназначено для парирования и блокирования. Ополовиньте ЕЗ оружия, до минимума в 1. При этом Прочность оружия может достигнуть 0!
\subsubsection{Цеп} Игнорирует БЩ (хотя может заявлять щит как область поражения). 
\subsubsection{Символ *} Обозначает качества оружия, не указанные в таблице, не входящие в унифицированный перечень Свойств и перечисленные в описательном блоке.
\subsubsection{Символ Ф} Обозначает оружие, уместное лишь в фантастическом антураже.

