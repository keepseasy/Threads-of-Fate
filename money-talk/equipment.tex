\subsection{Общие свойства Имущества}
\paragraph{Сложность приобретения (СП):} абстрактная ценность предмета.
\paragraph{Вес снаряжения:} во всех таблицах указан вес снаряжения в килограммах для существ Среднего размера. Чтобы узнать, как изменится вес предмета для существа другого размера, сверьтесь с таблицей ниже:
\begin{center}\begin{tabular}{ |c|c| }\hline
    \textbf{Размер существа} & \textbf{Множитель веса} \\ \hline
    Крошечный & 1/3 \\ \hline
    Маленький & 2/3 \\ \hline
    Средний & 1 \\ \hline
    Большой & 4/3 \\ \hline
    Огромный & 5/3 \\ \hline
    Громадный & 2 \\ \hline
\end{tabular}\end{center}

\paragraph{Осечка Х:} определяет, насколько ненадежно снаряжение из-за конструктивных особенностей или ненадлежащей эксплуатации. При любых проверках, использующих снаряжение с Осечкой, проверка считается проваленной, когда на К20 выпадает число меньше или равное Х. 
\newline Если герой использует несколько элементов снаряжения с Осечкой, используется наихудшая. Когда герой носит броню или щит с Осечкой, то они водействует на все Активные Проверки.
\paragraph{Поношенный:} Предметы, которые герои добыли в приключениях, как правило, уже видали виды. Все Поношенные предметы имеют Осечку 5. Значение Осечки таких предметов не может опуститься ниже 5 при обычном ремонте.
\newline Как только у предмета появляется Осечка 5 или выше, он автоматически становится Поношенным.
\paragraph{Осечка и СП.} Пока предмет не сильно износился, герой может продать его по сходной цене. Цена продажи не Поношенных предметов снижается на 1 относительно таблицы.
\newline СП продажи и покупки Поношенных предметов снижается на 2.
\newline СП продажи и покупки дополнительно понижается на 1 за каждую еденицу осечки выше 5.
\paragraph{Редкий} предмет непросто отыскать, равно как и получить Редкую услугу. СП Редкого предмета или Услуги возрастает на 5. Герои получают Помеху к проверке Доступности товаров и Услуг, разыскивая Редкий предмет или Услугу. 
\newline Редкий - контекстное свойство. Его наличие у предмета определяется экспозицией и логикой повествования. Вода в пустыне, лед в разгар лета, сухая древесина в сезон дождей - типичные Редкие предметы. В то же время в процветающем городе Редкими могут не быть услуги дантиста или пластического хирурга.

\subsection{Износ имущества}
Если вовремя не проводить техобслуживание, все рано или поздно выходит из строя – оружие, доспехи, приборы. Даже такая простая вещь как топор может не выдержать несчадного использования.

\paragraph{Проверка Износа} предметов происходит в конце любой Сцены, если выполнено одно или несколько из условий ниже.
\paragraph{Проверка Износа оружия происходит когда:} 
\begin{itemize}
    \item[--] Герой совершает Критический Промах, используя оружие; 
    \item[--] Оружие получает Пв;
    \item[--] Проверка указана в описании ситуации или способности.
\end{itemize}
\paragraph{Проверка Износа доспеха, щита или интегрированной защиты происходит, когда:}
\begin{itemize}
    \item[--] Снаряженный щитом и/или облаченный в доспех герой совершает Критический Промах;
    \item[--] Щит или доспех получают Пв.
    \item[--] Снаряженный щитом и/или облаченный в доспех герой подвергается КУ.
\end{itemize}
\paragraph{Проверка Износа доспеха, щита, оружия или устройства также происходит, если:}
\begin{itemize}
    \item[--] Герой использует экипировку не по назначению и очевидно опасным для предмета образом. Например, колет дрова лопатой, открывает замок ножом, взбивает сливки вентилятором или перевозит тяжести в спорткаре;
    \item[--] Расходники для устройства – патроны, топливо, батареи, изготовлены при помощи свойства или Хода Мусорщика.
\end{itemize}
Проверка Износа является проверкой Неприятностей и определяет, насколько ухудшилось состояние снаряжения.
\begin{itemize}
    \item[--] Снаряжение со свойством Надежное дает Преимущество на проверку Износа.
    \item[--] Снаряжение со свойством Хрупкое дает Помеху на проверку Износа.
\end{itemize}

\trouble
{Крепкая штука}{Устройство счастливо избежало неполадок.}
{Заело}{Небольшая проблема - любой герой разберется за пару секунд. Израсходовав Перемещение и Действие, герой проверяет Ремонт или Ловкость рук против \textbf{|15|}. При провале устройство может использоваться, но его Осечка возрастает на 1.}
{Заклинило}{Проблема в механизме – в бою точно не поправить. Герой проверяет Ремонт против \textbf{|15|}. При успехе устройство может использоваться, но его Осечка возрастает на 1. При провале устройство может использоваться, но его Осечка возрастает на 2. }
{Куча хлама}{Пыль, время, скверный уход и неумелые модификации взяли свое. Устройство не может использоваться - разве что, в качестве источника запчастей.}

\paragraph{Ломай меня полностью:} если в конце Сцены устройство должно совершить несколько проверок Износа по разным причинам, проверьте Износ лишь раз, но добавьте 1 Помеху за каждый повод после первого. 

\subsection{Ремонт снаряжения}
\paragraph{Ремонт Осечек:} при ремонте Осечек герой должен приобрести и потратить материалы с СП, равной половине Осечки и успешно проверить Ремонт или Эксплуатацию против \textbf{|10 + значение Осечки|}. При провале материалы потрачены впустую. В случае Критического провала предмет приходит в негодность и  не полежит ремонту. 
\begin{tcolorbox}
    Если Осечка является конструктивной особенностью предмета – как у плазменного оружия, избавиться от нее получится только при помощи Ученого и Хода «Эврика». Само собой, такая Осечка учитывается при ремонте и отражает опасность работы или невероятную сложность механизма.
\end{tcolorbox}

\subsection{Восстановление ЕЗ предметов:} требует проверки Ремонта против \textbf{|5 + Потерянные ЕЗ|}. Герой должен потратить материалы с СП, равной половине потерянных ЕЗ. В случае провала проверки материалы потрачены впустую. В случае Критического провала предмет приходит в негодность и не полежит ремонту.
\subsection{Время ремонта:} герой ремонтирует одну вещь в течение Интерлюдии или любое число вещей в течение Антракта, если контекст ситуации не противоречит этому. 

\subsection{Мелкий ремонт:} иногда предмет не требуется ремонтировать целиком – достаточно подновить фасад и поддержать рабочие функции. В этом случае, для восстановления ЕЗ предмета герой проверяет Ремонт против \textbf{|15|}. Число восстановленных ЕЗ равно величине успеха проверки. За каждую восстановленную 1 ЕЗ герой должен потратить материалы со СП 1 из своих загашников (не забывайте о счетчике Карманных расходов). Если герой не в состоянии «оплатить» ремонт здесь и сейчас, он восстанавливает столько ЕЗ, сколько может позволить оплатить. Таким образом предмету можно восстановить не боолее 10 ЕЗ за Интерлюдию и 20 ЕЗ за Антракт.
