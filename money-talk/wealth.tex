\section{Богатство}
\paragraph{}
Звонкие монеты, ценные материалы и сверкающие каменья... Ради них герои отправляются в опасные походы, ими платят ремесленнику и брадобрею, их бросают к ногам красавиц и могучих правителей. Хотя некоторые герои недальновидно прячут Богатство в сундук и закапывают в землю. Но речь, конечно же, пойдет не о них.
\newline
Богатство отражает финансовое благосостояние героя. В ходе игры Богатство может возрастать или уменьшаться.
\paragraph{Начальный уровень Богатства героя — 5}, минимальный уровень Богатства — 0. Игрок может увеличивать Богатство с помощью траты Очков опыта (в том числе после начала игры), однако оно может увеличиваться и благодаря продаже ценностей. Некоторые Атрибуты и Трюки также влияют на начальный уровень Богатства или дают временные бонусы к нему.
\paragraph{Уровни Богатства:}
\paragraph{0 — Нищий.} Герою нечем заплатить ни за черствый хлеб, ни за самую дешевую ночлежку. Герой не может приобретать услуги и предметы с СП 11 или больше.
\paragraph{1—4 — В долгах, как в шелках.} Герой живет в режиме жесткой экономии. Мясо на его столе — редкий гость.
\paragraph{5—10 — Средний класс.} Герой может побаловать себя время от времени... Впрочем, не слишком часто. Он все еще латает свою одежду вместо того, чтобы купить новую.
\paragraph{11—15 — Пошел в гору.} Герой твердо стоит на ногах. Он может без проблем позволить себе излишества в еде и одежде.
\paragraph{16—20 — Толстосум.} Герой способен оплатить работу лучших ремесленников, приобретать породистых лошадей, борзых и произведения искусства. Также его не сильно обременит наем хорошенькой горничной или привратника-ветерана. за состоянием счетов — это делают многочисленные приказчики. Особняк, карета, надомный лекарь, личная стража и красивые любовницы прилагаются.
\paragraph{31 и более — Купается в золоте.} Тот самый момент, когда герой подумывает о покупке маленькой уютной страны или найме армии для ее захвата!
\paragraph{Проверка Богатства:} товары и услуги имеют \textbf{Сложность приобретения (СП)} — число, заданное правилами или установленное мастером. Чтобы определить, может ли герой приобрести товар или услугу, бросьте К20 и прибавьте к выпавшему числу значение Богатства героя. Если результат больше или равен СП, герою удалось приобрести желаемое. Его Богатство понижается в зависимости от СП оплаченного:
\begin{itemize}
\item[--] СП 15 или больше — 1 Богатство в дополнение к прочим потерям
Богатства.
\item[--] СП на 1—5 больше, чем Богатство героя, — потеря 1 Богатство.
\item[--] СП на 6—10 больше, чем Богатство героя, — потеря 2 Богатств.
\item[--] СП на 11—15 больше, чем Богатство героя, — потеря 4 Богатств.
\item[--] СП на 16 и больше, чем Богатство героя, — потеря 8 Богатств.
\end{itemize}
Если СП товара или услуги меньше или равен Богатству героя, бросок не требуется — герой просто получает желаемое. Он все еще теряет 1 Богатство, если СП покупки 15 и больше. 
\paragraph{Карманные расходы:}
Если герой не потерял Богатство при покупке товара или услуги, это значит, что он буквально использовал мелочь из своего кармана. Но мелочь в караманах когда-нибудь заканчивается и приходится уже разменивать крупные купюры. Если совершил покупок на Карманные расходы больше, чем его текущее Богатство, он тут же теряет 1 Богатство, а счетчик Карманных расходов обнуляется.

\paragraph{Быстрая покупка:} является Быстрой проверкой Богатства. Предполагается, что в этом случае герой платит сразу, не торгуясь. Герой с Богатством 0 не может сделать этого.
\paragraph{Комплексные покупки:} иногда герой вынужден совершать особо внушительные траты за ограниченный временной промежуток — например, когда снаряжает армию, готовит экспедицию за сокровищами или подкупает толпу жадных бюрократов. В этом случае сложите СП покупок для определения сложности проверки и понизьте Богатство героя по обычным правилам.
\paragraph{Неслыханная щедрость:} получить Преимущество на проверку, но при успехе теряет на 1 Богатство больше. Так или иначе, Богатство понижается только при успешном приобретении услуги или предмета. В противном случае герой впустую потратил время на препирательства с продавцом или бесплодную борьбу с собственной жадностью.
\paragraph{Покупки в начале игры:} в начале игры герои покупают все необходимые предметы по отдельности — игнорируйте правила Комплексной покупки. Приобретение предметов происходит после распределения Очков опыта. В начале игры все снаряжение герой приобретает по правилам Быстрой покупки.
\paragraph{Покупка в складчину:} герои могут подкинуть друг другу деньжат по правилам Взаимопомощи (подробнее об этом читайте в разделе «Проверки»). Помощники теряют Богатство по обычным правилам. Не забывайте, что для них СП покупки меньше на 5.
\paragraph{Продажа предметов и услуг:} сначала определите СП предмета или услуги. Если предмет был в употреблении, понизьте его СП на 1. За каждую еденицу Осечки, полученную из-за Износа снаряжения понизтье его СП дополнительно на 1. Затем повысьте Богатство продающего героя на столько, на сколько он понизил бы его, купив этот предмет. Комплексные продажи возможны, хотя у героя, скорее всего, потребуют большую скидку!
\paragraph{Поторгуемся?} Успешная проверка Торговли позволяет сделать одно из следующего (по выбору игрока):
\begin{itemize}
\item[--] Получить Преимущество на проверку Богатства.
\item[--] Дать Преимущество на проверку Богатства другому герою, если герой с Торговлей помогает торговаться.
\item[--] Повысить Богатство на дополнительный 1 в случае успеха проверки, если герой продает или помогает продавать.
\item[--] Уменьшить потерю Богатства на 1 в случае успеха проверки, если герой покупает или помогает покупать.
\end{itemize}
Провал проверки Торговли приводит к одному из следующего (по выбору игрока):
\begin{itemize}
\item[--] Герой получает Помеху на проверку Богатства.
\item[--] Другой герой получает Помеху на проверку Богатства, если герой с Торговлей помогает торговаться.
\item[--] Богатство повышается на 1 меньше, чем должно в случае успеха проверки, если герой продает или помогает продавать. В худшем случае герой потеряет предмет и не повысит свое Богатство.
\item[--] Потеря Богатства увеличивается на 1 в случае успеха проверки, если герой покупает или помогает покупать.
\end{itemize}
\paragraph{Бартер:} Есть места, где Богатство героев не имеет значения. Местных не интересуют блестяшки и бумажки, а о таком концепте, как виртуальные деньги и кредиты они и думать не хотят. В таких местах за место привычной торговли выступает старый-добрый Бартер.
\paragraph{}Во время Бартера герой предлагает свои вещи, имеющие СП вместо Богатства для того, чтобы приобрести желаемое у торговца. Проверка Бартера в этом случае является проверкой Нулевого Богатства против сложности 10. За каждую еденицу разницы СП между товаром героя и товаром продавца сложность увеличивается или уменьшается на 5. Например, если герой хочет обменять Арбалет(СП 10) на Дробовик(СП 11), то сложность проверки будет 15.
\paragraph{}В случае провала Проверки сделка не состоится - повторные проверки с участием тех же вещей невозможны, но можно попробовать обменять другие вещи на желаемое.
\paragraph{}При бартере Боеприпасов считается, что обмен происходит коробками - поштучно торговля не пойдет.
\paragraph{}Если герой хочет обменять сразу несколько едениц товара, то за каждое удвоение СП приобретаемого или отдаваемого товара будет увеличено на 1. Например, если герой хочет использовать в Бартере 8 Кинжалов(СП 3), то это будет считаться как СП 6.
\paragraph{}Бартер - примитивное средство торговли, поэтому вряд ли за одну сделку удастся обменять одновременно Кинжал, Кастет и Пистолет на Бронежилет. Придется найти промежуточную валюту или попытаться обменять более равноценные вещи.
\paragraph{}Герой может использовать навык Торговли при Бартере, но тогда сложность всех проверок возрастает на 5. Торговцу, привыкшему говорить языком денег будет труднее убедить тех, кто концепцию денег не понимает и не принимает.
