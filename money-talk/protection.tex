\subsection{Защита}
Весь перечень того, что в состоянии защитить тело от ударов и выстрелов - от вонючей шкуры дикаря до тактического силового доспеха. 
\paragraph{Бонус к Защите:} доспехи дают герою Бонус доспеха к Защите (БД), а щиты дают герою Бонус щита к Защите (БЩ).
\newline Герой не может надеть 2 комплекта доспехов одновременно, но может одновременно использовать 2 щита. В этом случае он получает БЩ обоих щитов.
\paragraph{Ограничитель модификатора Ловкости (оМЛв):} в большинстве своем доспехи и щиты - тяжелые конструкции, сковывающие движения. Многие доспехи и щиты ограничивают МЛв, доступный герою при Защите и совершении проверок. Так, герой с 20 Лв (МЛв +5), надевший кольчужную рубаху, будет добавлять к Зщ и проверкам Навыков, основанных на Ловкости, лишь +3, потому что ограничитель МЛв кольчужной рубахи составляет +3. Герой с высокой Лв, надевший доспех с низким оМЛв, столкнется с понижением Дб и Мт.
\paragraph{Требуемая Выносливость (тВн):} герой не может носить доспех или щит, если не обладает достаточной для этого Выносливостью. Сила не играет здесь важной роли - герой может поднять доспех и выдержать его вес, но выдохнется после нескольких минут активных действий. Разумеется, он может облачиться в доспех, чтобы повыпендриваться перед девчонками или попозировать фотографу, но в бою выглядит жалко. Все активные проверки такого героя совершаются с Помехой, а все атаки по нему совершаются с Преимуществом.
\paragraph{ЕЗ доспеха или щита = |тВн|.} Если доспех имеет несколько значений тВн, для подсчета ЕЗ используется максимальное.	
\paragraph{Прочность доспехов и щитов = |1/2 ЕЗ доспеха или щита|.} 
\paragraph{БПв щита.} Большинство щитов можно использовать, как оружие с КУ 20 и ТПв Д. Если в этом столбце указан прочерк, то бить щитом нельзя.
\paragraph{Доспехи и Осечка:} доспехи и щиты так же, как и оружие, подвержены износу и могут иметь свойство \textbf{"Осечка Х"}. Если герой, снаряженный доспехом или щитом, совершает Активную проверку и выбрасывает на К20 число, равное Х или меньше, проверка проваливается. 
\paragraph{Количество помех для проверок Скрытности (ПС):} защита может шуметь при движении - звон сочленений, скрежет пластин, шипение сервоприводов и гул силовой установки - все это мешает герою  скрывать свое присутствие.

\paragraph{Халтура:} дешевый, кое-как изготовленный щит или доспех тяжел и неудобен. Его тВн увеличивается на 1 (это не влияет на ЕЗ предмета), а оМЛв понижается на 1. Доспех или щит имеет 1/2 ЕЗ и 1/2 Прч и Осечку 5. Если щит или доспех не имел оМЛв, то он получает оМЛв +4. Когда герой получает КП на Активную проверку, доспех разрушается. Понизьте СП на 5. Если СП достигает 0, это означает, что изготовление доспеха требует Интерлюдии, которую герой тратит на сбор материалов, и Антракта на соединение их друг с другом.
\newline Халтурные доспехи и щиты - отличный выбор для тех, кто живет одним днем, но не слишком торопится умирать. То есть для абсолютного большинства жителей разрушенного мира, вынужденных иногда принимать участие в боевых действиях. Профессиональные воины пользуются Халтурной защитой лишь в крайних случаях.
\paragraph{Работа мастера.} Доспех или щит, изготовленный мастером, безупречно подходит заказчику. ТВн уменьшается на 1 (это не влияет на ЕЗ предмета), а оМЛв повышается на 1. СП повышается на 5.
\paragraph{Шедевр} представляет собой безупречный экземпляр мастерской работы и получает все преимущества работы мастера. ЕЗ предмета увеличиваются в 1.5 раза. В дополнение БД или БЩ возрастает на 1, а ПС понижается на 1. СП повышается на 10.
\newline Работа мастера и шедевр - штучные изделия. Все, кроме хозяина, при ношении такого доспеха получают все логически возможные эффекты халтуры, пока доспех не будет подогнан сведущим технарем под нового владельца.
\paragraph{Надевание и снятие доспехов и щитов:} время, необходимое для этого, зависит от оМЛв доспеха или щита:
\begin{itemize}
    \item[--] Доспех или щит не имеет оМЛв, время надевания и снятия составляет 1 минуту.
    \item[--] ОМЛв доспеха +2 и выше, время надевания составляет 5 минут, а время снятия составляет 1 минуту.
    \item[--] ОМЛв +1 и ниже, время надевания составляет 10 минут, а время снятия составляет 5 минут.
    \item[--] Кто-то помогает герою, то время снятия и надевания сокращается вдвое.
\end{itemize}
\paragraph{Доспехи для больших и маленьких существ:} в таблицах представлены доспехи и щиты для существ Среднего размера. Для существ иного размера доспехи изготовлются под заказ. 
\newline В этом случает, прибавьте МРз существа к тВн, БД, БЩ и СП доспехов и щитов. Стоимость доспехов для маленьких существ не меняется - сокращается стоимость материалов, но возрастает сложность работы. Увеличение СП за размер складывается с изменением СП за Халтуру, Работу мастера или Шедевр.
\paragraph{Атаки по доспехам и щитам:} доспех или щит могут быть выбраны зоной поражения (подробнее смотрите маневр "Сломать снаряжение" в разделе "Маневры"). Носитель получает Пв только в том случае, если доспех или щит получают Пв, которые не смогли полностью поглотить их Прч и ЕЗ.
\newline Если доспех или щит уничтожены, носитель теряет их БД и БЩ, однако его МЛв все еще ограничен оМЛв доспеха или щита.
\paragraph{Иммунитет, Сопротивление и Уязвимость к Повреждениям:} доспехи и щиты обладают иммунитетом к Ядовитым Пв, Сопротивлением к Колющим, Проникающим и Ледяным Пв и Уязвимы к Едким Пв. 
\begin{tcolorbox}
    При встрече с тяжелобронированным противником будет хорошей идеей сначала уничтожить его доспех и щит, вместо того чтобы пытаться поразить заоблачную Зщ. Хотя, тогда снаряжение не выйдет продать. 
\end{tcolorbox}

\subsection{Перечень Доспехов}
\genAndGet{protection}{protection}{Доспех}

\subsection{Перечень Щитов}
\genAndGet{protection}{protection}{Щит}


