\section{Инструменты Могущества}
В этом разделе описаны могущественные артефакты, высокотехнологичные прототипы и технологические чудеса, которые позволяют герою достичь и преодолеть предел своих возможностей. Обычно Инструменты Могущества встречаются только в фантастическом антураже.

\paragraph{Название} Инструмента отражает его дух, историю или возможности. Редко названия Инструментов бывают обыденными.
\paragraph{Базовый предмет} Зачастую Инструмент построен на основе уже существующего, а иногда и весьма распространенного в мире, предмета, оружия или элемента экипировки. Обычно Инструмент сохраняет все характеристики Базового предмета и лишь усиливает их своими способностями.
\paragraph{Запас энергии: } количество Зарядов Инструмента которые можно тратить на активацию его Функций. Способ восстановления этой энергии может сильно отличаться в зависимости от антуража.
\paragraph{Сложность Приобретения(СП): }некоторые Инструменты научились довольно ловко воспроизводить и их даже можно увидеть на полках магазинов! Но если в описании нет СП, то Инструмент нельзя просто купить. Придется искать его в древних руинах, секретных лабораториях, а иногда и выпрашивать у богов. Если герой захочет продать Инструмент без СП, то количество Богатства, которое он получит от продажи будет определено исключительно его красноречием и умением торговаться.
\paragraph{Описание }Инструмента говорит о его виде и о нарративе его использования. Механика его работы описана в Трюках, Функциях, Изъянах и Ходах Инструмента.
\paragraph{Трюки }Инструмента - это его свойства, которые герой может использовать когда захочет или же они работают постоянно.
\paragraph{Функции }Инструмента требуют трату Зарядов для активации. В описании каждой Функции Инструмента в скобках указана её \textbf{Стоимость}.
\paragraph{Ходы }Инструмента позволяют совершать невозможное, но требуют за это обрыв Нитей героя. В описании каждого Хода Инструмента обязательно указана его \textbf{Стоимость}.
\paragraph{Изъяны }Инструмента указывают на любое негативное влияние Инструмента на его владельца. Это может быть как незначительное неудобство, так и серьезное проклятие.
\genAndGet{powerTools}{powerTools}
