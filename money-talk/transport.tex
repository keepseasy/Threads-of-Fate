\section{Транспорт}
В этом разделе описаны механизмы и существа, которых можно использовать, как транспортные средста (ТС).
\paragraph{Символ "Ф"} обозначает вид транспорта, уместный лишь в фантастическом антураже.
\paragraph{Символ «*»} обозначает качества ТС, перечисленные в описательном блоке.
\paragraph{Ограничение Модификатора Ловкости(оМЛв):} крупные ТС ворочаются неохотно, а разогнавшись, не спешат тормозить. оМЛв определяет максимальный МЛв, который водитель может добавить к Эксплуатации при управлении ТС.
\paragraph{Ограничение Модификатора Ловкости(оМЛв) для животных:} не вся животина охотно подчиняется командам погонщика или наездника. Максимальный МЛв, который может герой добавить к проверкам Обращения с животными равен МЛв самого животного.
\paragraph{Проходимость транспортного средства(П):} ТС не может Путешествовать по местности с Опасностью, превышающей его Проходимость. 
\newline Проходимость животных увеличивается на 1, если они не несут груза или всадников.

\paragraph{Защита транспортного средства:} предполагается, что транспортные средства из таблицы уже оснащены защитными приспособлениями в пределах разумной целесообразности. 
\newline МЛв водителя прибавляется (или отнимается, если он отрицательный) к Зщ ТС.
\paragraph{Маневренность (М):} во многих ситуациях важны не только Проходимость и Скорость ТС, но и его способность избегать препятствий. Маневренность ТС = \textbf{|Проходимость — МРз|}. Маневренность добавляется к навыку водителя (или отнимается, если она отрицательная) при проверке ЭксплуатацииЭ для совершения сложных поворотов и перестроений.
\paragraph{Cкорость ТС во время Путешествий (Ск):} она значительно ниже максимума, который можно выжать из транспорта и измеряется в км/ч. 
\newline Если в графе присутствует \textbf{символ М}, это значит, что обычная скорость ТС равна максимальной и оно не может двигаться Маршем (если только скорость самого медленного участника каравана не в два раза меньше, чем Ск этого ТС).
\paragraph{Ск ТС во время Боевых сцен:} она численно равна Ск ТС во время Путешествий, но измеряется в метрах/клетках на тактической карте, которые ТС может преодолеть за свою Очередь. Это отражает необходимость маневрировать и лавировать между другими участниками Сцены (или давить их). Механическое ТС может совершать маневры ближнего боя, если оснащение и контекст позволяют это. Проверки Дб в этом случае заменяются ЭксплуатациейЭ водителя.
Животные используют в Боевых сценах карточки из раздела «монстры и статисты».
\paragraph{Расход топлива (Р)} определяет сколько Зарядов Топлива (или килограммов фуража для животных) тратит тот или иной вид транспорта на каждые 10 км пути.
\paragraph{Грузоподъемность/Вес (Г/В)} определяет количество полезной нагрузки, которую может везти ТС, и вес самого ТС для определения возможностей буксировки.
\paragraph{Буксировка:} предполагается, что ТС может буксировать вес, не превышающий его грузоподъемность в 10 раз.
\paragraph{Перегрузка:} если нагрузка ТС превышает Грузоподъемность не более, чем в 2 раза, либо буксируемый вес превышает вес ТС не более, чем в 2 раза, водитель получает Помеху на Эксплуатацию.
\paragraph{Сильная перегрузка:} если нагрузка ТС превышает Грузоподъемность не более, чем в 3 раза, либо буксируемый вес превышает вес ТС не более, чем в 3 раза, водитель получает 2 Помехи на Эксплуатацию.
Повреждение ТС:} после потери 1/3 ЕЗ ТС теряет половину Ск. После потери 2/3 ЕЗ его Эксплуатация проверяется с Помехой.
\newline Если транспортное средство одномоментно теряет 1/5 или более ЕЗ, водитель должен проверить Эксплуатацию против \textbf{|15|}. При провале транспортное средство глохнет.
\newline Когда зона поражения одномоментно получает Пв, равные 1/4 или более от максимальных ЕЗ ТС, она уничтожается. Все проверки водителя, связанные с повреждением зоны, считаются Критически проваленными. 
\paragraph{Зоны поражения ТС:} при атаке ТС могут быть выбраны Зоны поражения, критичные для его работы. Нападающий должен указать зону и проверить Дб или Мт с -2.
\newline Если Зона поражения одномоментно получает Пв, составляющие 1/5 или более от максимальных ЕЗ ТС, она временно выходит из строя. Лишние Пв теряются, но водитель должен проверить Эксплуатацию против \textbf{|15|}. При провале транспортное средство глохнет. Помимо этого:
\begin{itemize}
    \item[--] Повреждение Двигателя и трансмиссии приводят к тому, что ТС глохнет. Требуется успешно проверить Ремонт против \textbf{|15|}, чтобы снова завести его;
    \item[--] Повреждение Ходовой части приводит к потере управления. Водитель проверяет Эксплуатацию против \textbf{|15|}. При провале ТС терпит крушение;
    \item[--] Повреждение Орудия или Манипулятора выводит их из строя. Для того, чтобы восстановить функционал, требуется успешно проверить Ремонт против \textbf{|15|};
    \item[--] Повреждение Кабины или Пассажирского отсека сбрасывает водителя или пассажиров. Они должны успешно проверить Лв или Атлетику(Лв) против \textbf{|15|}. При провале герои выпадают из транспорта и получают Дробящие Пв, равные \textbf{|[величине провала]*[Опасность местности]|}. Если водитель и пассажиры пристегнуты ремнями безопасности, они совершают проверки с Помехой.
\end{itemize}
\paragraph{Атаки по водителю и пассажирам:} элементы конструкции ТС служат сносным укрытием для находящихся внутри. Поэтому водитель и пассажиры:
\begin{itemize}
    \item[--] Находятся в Мягком укрытии, если ТС имеет Прч 5 – 10.
    \item[--] Находятся в Твердом укрытии, если ТС имеет Прч 11 и больше.
\end{itemize}
Некоторые ТС обладают полностью закрытыми корпусами, и управляются изнутри при помощи телеметрических приборов. В этом случае находящиеся внутри не могут быть выбраны целью.
\paragraph{Инос ТС:} техника может выдержать многое, но многое может и не выдержать. Износ ТС проверяется в конце Сцены, если:
\begin{itemize}
  \item[--] ТС перегружено;
  \item[--] ТС Сильно перегружено;
  \item[--] ТС передвигалось по местности с Опасностью, превышающей его Проходимость;
  \item[--] ТС используется очевидно опасным или нецелевым образом. Для Техники размером Б и меньше это включает намеренный наезд на существо величиной с человека или крупного пса.
\end{itemize}
Животные не проверяют Износа, но получают Пв в зависимости от контекста Сцены.

\subsection{Гужевой транспорт}
Животные способны преодолевать водные преграды, а некоторые - еще и летать!
\genAndGet{transport}{transport}{Животное}

\subsection{Наземный транспорт}
\genAndGet{transport}{transport}{Наземный}

\subsection{Водный транспорт}
Проходимость этого транспорта означает то, насколько глубокая у него посадка. Чем выше проходимость, в тем более мелких реках и ручьях может передвигаться транспорт.
% \paragraph{Буксировка водного транспорта} Так как на воде обычно нет значительных перепадов высот, а вода не сильно сопротивляется движению, любые плавсредства можно буксировать, если их вес (вместе с барахлом) не превышает вес Буксира в ??? раз.
\paragraph{Перегрузка водного транспорта} В отличае от наземного транспорта, перегруженный водный транспорт не ломается, а начинает \textbf{тонуть}.
\tbd проверка неприятностей
\paragraph{Сильная перегрузка водного транспорта} приводит к немедленному затоплению ТС.

\genAndGet{transport}{transport}{Водный}

\subsection{Воздушный транспорт}
Проходимость летающих транспортных средств учитывается только при взлете и посадке. В эту категорию так же входит транспорт, способный находиться в открытом косомсе, но не предназначенный для межпланетных и межзвездных перелетов.
\paragraph{Буксировка воздушного транспорта} возможна только на земле.
\paragraph{Перегрузка воздушного транспорта.} Воздушный транспорт не терпит перегрузки. Все проверки Эксплуатации получают Осечку 9, а в случае провала ТС падает на землю.
\paragraph{Сильная перегрузка воздушного транспорта} не позволяет ему взлететь в принципе, хотя он может продолжать катиться по дорогам, если это позволяет Наземная Проходимость.
\genAndGet{transport}{transport}{Воздушный}

% \subsection{Космический транспорт}
% Космолеты имеет настолько большую скорость перемещения, что погони становятся бессмысленными, а перемещение между любыми двумя точками планеты они совершают меньше, чем за сутки, а скорость межпланетного и межзвездного перемещения очень сильно зависит от выбранного сеттинга, поэтому в таблице скорость космического транспорта не указана. Однако космический транспорт все еще имеет Проходимость, которая указвает на то, в насколько сложных условиях этот транспорт может совершать взлет и посадку.
% \newline космические корабли с проходимостью 0 не способны совершать посадку на поверхность планеты - вместо этого они используют средства орбитальной транспортировки, такие, как телепорты или орбитальные челноки.
% \newline космические корабли с проходимостью 1 способны совершать взлет и посадку с поверхности планеты только со специально подготовленных для них космодромов.
% \genAndGet{transport}{transport}{Космический}

\subsection{Исполинский транспорт}
Некоторые ТС достигают невероятных размеров. Обслуживать такой транспорт, а тем более владеть им, не по карману даже самому обеспеченному герою. Но герои могут взять этот транспорт в аренду или получить в пользование от организации - покровителя.
\newline Героям не нужно знать, сколько стоит, как тяжело обслуживается и насколько грузоподъемен транспорт, на котором они отправляются в путешествие, для приключения это - лишние детали. В описании Исполинского транспорта есть только его \textbf{Скорость} и \textbf{Проходимость} - этого достаточно, для того чтобы определить длительность и событийное наполнение пути. 
\newline Во время Остановок, Исполинский транспорт не участвует в Сценах целиком, а является элементом окружения героев. Сцена может развернуться и внутри транспорта.
\paragraph{Старинные Исполины}
\genAndGet{transport-gigantic}{transport-gigantic}{Старинный}
\paragraph{Современные Исполины}
\genAndGet{transport-gigantic}{transport-gigantic}{Современный}
% \paragraph{Фантастические Исполины}
% \genAndGet{transport-gigantic}{transport-gigantic}{Фантастический}