\section{Транспорт}
В этом разделе описаны механизмы и живые существа, которых можно использовать, как транспорт
\newline Символ <<Ф>> обозначает вид транспорта, уместный лишь в фантастическом антураже.
\newline Если в столбце размера указана литера \textbf{Ж}, то этот вид транспорта является животным.
\paragraph{Ограничение Модификатора Ловкости(оМЛв):} Крупные транспортные средства ворочаются неохотно, а разогнавшись, не спешат тормозить оМЛв определяет максимальны МЛв, который водитель может добавить к проверкам Эксплуатации при управлении транспортным средством.
\paragraph{Ограничение Модификатора Ловкости(оМЛв) для животных:} Не вся животина резво подчиняется командам погонщика или наездника. Максимальный МЛв, который может герой добавить к проверкам Общения с животными равен МЛв самого животного.
\paragraph{Проходимость транспортного средства(П):} транспорт не может передвигаться по местности с Опасностью, превышающей его Проходимость. Проходимость животных увеличивается на 1, если они не несут груза или всадников.
\paragraph{Защита транспортного средства:} предполагается, что транспортные средства из этой уже оснащены всеми возможными защитными приспособлениями в пределах разумной целесообразности. Модификатор Ловкости водителя прибавляется (или отнимается, если он отрицательный) к Защите ТС.
\paragraph{Перегруженный транспорт:} если нагрузка транспортного средства превышает Грузоподъемность не более чем в 2 раза, водитель получает Помеху на проверки Эксплуатации. Если нагрузка транспортного средства превышает Грузоподъемность не более чем в 3 раза, водитель получает 2 Помехи на Эксплуатацию.
\paragraph{Буксировка:} предполагается, что транспортное средство может буксировать вес, не превышающий свой собственный. Если буксируемый вес превышает вес транспортного средства не более чем в 2 раза, водитель получает Помеху на Эксплуатацию. Если буксируемый вес превышает вес транспортного средства не более чем в 3 раза, водитель получает 2 Помехи на Эксплуатацию.
\paragraph{Маневренность.} Во многих ситуациях важны не только проходимость и скорость автомобиля, но и его способность избегать препятствий. Маневренность транспорта равна \textbf{|Проходимость — Модификатор размера транспортного средства|}.Маневренность добавляется к навыку
водителя (или отнимается, если она отрицательная) при проверке Эксплуатации для совершения сложных поворотов и перестроений.
\paragraph{Потеря ЕЗ транспортным средством:} после потери \textbf{1/3 ЕЗ}
транспортное средство теряет половину своей Скорости. После
потери \textbf{2/3 ЕЗ} проверки Эксплуатации совершаются с Помехой.
\newline
Если транспортное средство одномоментно теряет \textbf{1/4 или более своих ЕЗ}, водитель должен совершить проверку Эксплуатации против 15. При провале транспортное средство глохнет.
\newline
Попадания по двигателю или ходовой части транспортного средства могут быть весьма опасны! Если зона поражения одномоментно теряет \textbf{1/5 или более от максимальных Единиц Здоровья}, она выходит из строя. Лишние Повреждения теряются, но водитель должен совершить проверку Эксплуатации против 15. При провале транспортное средство глохнет.
\newline{Ремонт транспортного средства:} восстановление каждых 5 ЕЗ требует 1 часа работы, наличия Ремонта у механика и имеет СП 1.

\paragraph{Cкорость(Ск)} Скорость трансопорта во время путешествий(в км/ч). Численно она равна Ск транспорта во время Боевых Сцен и она \textbf{значительно ниже} максимальной скорости, которую можно выжать из этого транспорта. Если у транспорта в этой графе присутствует литера М, это значит, что его обычная скорость равна максимальной и этот транспорт не может двигаться \textbf{Маршем}(если только скорость самого медленного члена каравана не в два раза меньше, чем скорость этого транспорта).
\paragraph{Расход топлива(Р)} определяет сколько Зарядов Топлива(или килограмм фуража для животных) тратит тот или иной вид транспорта не каждые 10 км пути.
\paragraph{Грузоподъемность/Вес(ГВ)} определяет количество полезной нагрузки(за исключением полных баков топлива), которую может вести транспорт без перегрузки и вес самого транспорта для определения возможностей буксировки.

\subsection{Гужевой транспорт}
Животные способны преодолевать водные преграды, а некоторые - еще и летать!
\genAndGet{transport}{transport}{Животное}

\subsection{Наземный транспорт}
\genAndGet{transport}{transport}{Наземный}

\subsection{Водный транспорт}
Проходимость этого транспорта означает то, насколько глубокая у него посадка. Чем выше проходимость, в тем более мелких реках и ручьях может передвигаться транспорт.
\genAndGet{transport}{transport}{Водный}

\subsection{Воздушный транспорт}
Проходимость летающих транспортных средств учитывается только при взлете и посадке. В эту категорию так же входит транспорт, способный находиться в открытом косомсе, но не предназначенный для межпланетных и межзвездных перелетов.
\genAndGet{transport}{transport}{Воздушный}

\subsection{Космический транспорт}
Космолеты имеет настолько большую скорость перемещения, что погони становятся бессмысленными, а перемещение между любыми двумя точками планеты они совершают меньше, чем за сутки, а скорость межпланетного и межзвездного перемещения очень сильно зависит от выбранного сеттинга, поэтому в таблице скорость космического транспорта не указана. Однако космический транспорт все еще имеет Проходимость, которая указвает на то, в насколько сложных условиях этот транспорт может совершать взлет и посадку.
\newline космические корабли с проходимостью 0 не способны совершать посадку на поверхность планеты - вместо этого они используют средства орбитальной транспортировки, такие, как телепорты или орбитальные челноки.
\newline космические корабли с проходимостью 1 способны совершать взлет и посадку с поверхности планеты только со специально подготовленных для них космодромов.
\genAndGet{transport}{transport}{Космический}

\subsection{Исполинский транспорт}
Транспортные средства могут достигать поистине невероятных размеров. Обслуживать такой транспорт, а тем более владеть им не по карману даже самому обеспеченному герою. Но герои могут взять этот транспорт в аренду или получить в пользование от организации, на которую они в данный момент работают.
\newline Героям не нужно знать, сколько стоит, как тяжело обслуживается и насколько грузоподъемен корабль, на котором они отправляются в путешествие, для приключения это - лишние детали. В описании Исполинского транспорта есть только его Скорость и Проходимость - этого достаточно, для того чтобы определить длительность и событийное наполнение похода.
\newline Во время Остановок, Исполинский транспорт не учавствует в Сценах целиком, а является частью окружения, в котором действуют герои. А иногда Сцена может быть целиком внутри транспорта, на котором путешествуют герои.
\paragraph{Старинные Исполины}
\genAndGet{transport-gigantic}{transport-gigantic}{Старинный}
\paragraph{Современные Исполины}
\genAndGet{transport-gigantic}{transport-gigantic}{Современный}
\paragraph{Фантастические Исполины}
\genAndGet{transport-gigantic}{transport-gigantic}{Фантастический}