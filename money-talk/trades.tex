Звонкие монеты, ценные материалы и сверкающие каменья... Ради них герои отправляются в опасные походы, ими платят ремесленнику и брадобрею, их бросают к ногам красавиц и могучих правителей. Хотя некоторые герои недальновидно прячут Богатство в сундук и закапывают в землю. Но речь, конечно же, пойдет не о них.
\newline Богатство отражает финансовое благосостояние героя. В ходе игры Богатство может возрастать или уменьшаться.
\paragraph{Начальный уровень Богатства героя:} 5.
\paragraph{Минимальный уровень Богатства:} 0.
\paragraph{Будешь должен.} Если в результате снижения Богатство героя должно опуститься ниже минимального уровня, оно становится равным 0, а герой получает Недостаток "Долг", "Враги" или "Преступник". Если у него уже был один из этих недостатков, он немедленно входит в игру.
\paragraph{Пополнение Богатства:} игрок вправе увеличить Богатство героя с помощью траты Очков опыта (даже в начале игры), однако оно может вырасти благодаря продаже ценностей. Некоторые Атрибуты и Трюки влияют на начальный уровень Богатства или дают временные бонусы к нему.
% \paragraph{Уровни Богатства:}
% \paragraph{0 - Нищий.} Герою нечем заплатить ни за черствый хлеб, ни за самую дешевую ночлежку. Герой не может приобретать услуги и предметы с СП 11 или больше.
% \paragraph{1-4 - В долгах, как в шелках.} Герой живет в режиме жесткой экономии. Мясо на его столе - редкий гость.
% \paragraph{5-10 - Средний класс.} Герой может побаловать себя время от времени... Впрочем, не слишком часто. Он все еще латает свою одежду вместо того, чтобы купить новую.
% \paragraph{11-15 - Пошел в гору.} Герой твердо стоит на ногах. Он может без проблем позволить себе излишества в еде и одежде.
% \paragraph{16-20 - Толстосум.} Герой способен оплатить работу лучших ремесленников, приобретать породистых лошадей, борзых и произведения искусства. Также его не сильно обременит наем хорошенькой горничной или привратника-ветерана. за состоянием счетов - это делают многочисленные приказчики. Особняк, карета, надомный лекарь, личная стража и красивые любовницы прилагаются.
% \paragraph{31 и более - Купается в золоте.} Тот самый момент, когда герой подумывает о покупке маленькой уютной страны или найме армии для ее захвата!
\paragraph{Проверка Богатства:} товары и услуги имеют \textbf{Сложность приобретения (СП)}. СП отражает абстрактную ценность предмета или услуги.
\paragraph{Провал проверки Богатства} означает, что герою не удалось договориться о сделке. Он не приобретает желаемый предмет, но и не понижает свое Богатство.
\paragraph{Понижение Богатства при покупке.} При успехе проверки герой получает желаемое, а его Богатство понижается в соответствии с таблицей.
\begin{center}\begin{tabular}{ |c|c| }\hline
    \textbf{СП покупки} & \textbf{Снижение Богатства} \\ \hline
    15 или больше & дополнительно -1 \\ \hline
    на 1-5 больше, чем Богатство героя & -1 \\ \hline
    на 6-10 больше, чем Богатство героя & -2 \\ \hline
    на 11-15 больше, чем Богатство героя & -4 \\ \hline
    на 16 и больше, чем Богатство героя & -8 \\ \hline
\end{tabular}\end{center}
\paragraph{Неслыханная щедрость:} герой получает Преимущество на проверку Богатства, но при успехе дополнительно теряет 1 Богатство.
Если СП товара или услуги меньше или равен Богатству героя, бросок не требуется - герой просто получает желаемое. Он все еще теряет 1 Богатство, если СП покупки 15 и больше.
\paragraph{Быстрая покупка} является Эффективным показателем Богатства, т.е. проверка не совершается, а ее результатом считается 10.  Предполагается, что используя Быструю покупку, герой взвешивает все «за» и «против», и платит сразу, не торгуясь. Герой с Богатством 0 не может сделать этого.
\paragraph{Покупка без проверки Богатства:} когда СП товара или услуги равна Богатству героя или меньше его, бросок не требуется — герой просто получает желаемое. Он все еще теряет 1 Богатство, если СП покупки 15 и больше.
\paragraph{Карманные расходы и их учет:} если герой не потерял Богатство при покупке товара или услуги, это значит, что он использовал рухлядь из своего кармана. Но рухлядь в караманах порой заканчивается. 
\newline Величина счетчика Карманных расходов равна значению Богатства героя. Если герой совершил больше покупок на Карманные расходы, чем его начение Богатства, он тут же теряет 1 Богатство, а счетчик Карманных расходов обнуляется. Так же счетчик обнуляется, если герой уходит в Антракт.
\paragraph{Покупка в складчину:} герои могут подкинуть друг другу барахлишка по правилам Взаимопомощи. Помощники теряют Богатство по обычным правилам. Не забывайте, что для них СП покупки меньше на 5.

\subsection{Покупки в начале игры:}
В начале игры герои покупают все необходимые предметы по отдельности - игнорируйте правила Комплексной покупки. Приобретение предметов происходит после распределения Очков опыта. В начале игры все снаряжение герой приобретает по правилам Быстрой покупки.

\subsection{Комплексные покупки}
Иногда герой вынужден внушительно раскошеливаться в течение Сцены – например, когда снаряжает охотничий отряд, готовит экспедицию на древнюю базу или подкупает толпу жадных бюрократов. 
\newline Комплексная покупка является Испытанием. Для ее совершения вам понадобится:
\begin{itemize}
    \item[--] Определить участников Испытания.
    \item[--] Определите суммарную СП товаров и услуг.
    \item[--] Игроки сами выбирают Счетчик победы – любое число (мастер вправе ограничить величину этого числа, ведь обычно время поджимает).
    \item[--] Разделите суммарную СП товаров и услуг на Счетчик победы
    \item[--] Полученное при делении число – Сложность каждой из Развивающих проверок. Счетчик поражения устанавливает мастер.
    \item[--] Приступайте к совершению проверок, как при обычном Испытании:
        \begin{itemize}
            \item[$\bullet$] \textbf{Для Допускающих} действий потребуются Общение, Торговля или Богатство. Это изображает подготовку сделки и ее «сопровождение» – устройство встречи, посулы, мелкие услуги, подарки, флирт, лесть, ложь, угрозы, слезные мольбы и т.д.
            \item[$\bullet$] \textbf{При Развивающих} действиях герои проверяют Богатство. Каждая проверка Богатства вызывает немедленное понижение его значения в соответствии с правилами. Если Испытание будет провалено, потери Богатства не возмещаются. Возможно, героев ловко обжулили, либо они поиздержались, умасливая продавца, либо задаток не возвращается по условиям сделки.
            \item[$\bullet$] \textbf{Сберегающие} действия представляют собой способы дополнительно сбить цену в процессе сделки и уговоры ненадолго придержать товар. Скорее всего, для этого героям понадобится Общение, Торговля и уместные в ситуации Атрибуты и Трюки.
        \end{itemize}
\end{itemize}

\begin{tcolorbox}
    Даже одиночные предметы с высокой СП могут приобретаться, как Комплексная покупка – до тех пор, пока это насыщает историю действием и деталями.
\end{tcolorbox}

\paragraph{Цена провала:} в случае неудачи Комплексной покупки герои  все еще могут получить жклаемое - если выберут Расплату за каждую единицу, на которую счетчик Победы меньше Счетчика поражения.
\paragraph{Варианты Расплаты:}
\begin{itemize}
    \item[--] Некомплект. В покупках недостает важных мелочей, но в суматохе никто этого не замечает.
    \item[--] Некачественные товары. Техника рассыпается в руках, а от припасов пованивает. Странно, почему на это не обратили внимания при оплате?
    \item[--] Происки недругов. Информация о том, что группа готовится к чему-то масштабному, расползается по округе. Это может привести к появлению конкурентов в погоне за сокровищами, попытке ограбления, и другим подобным проблемам.
    \item[--] Недостаток важного оборудования. Критически важные для героев предметы отсутствуют или неисправны. И ведь герои точно помнят, что все это паковали и проверяли!
    \item[--] Задержки поставок. Товар в наличии, но надо немного подождать. Достаточно впрочем, долго, чтобы это создало героям проблемы. Нет-нет, выставочные образцы не продаются!
\end{itemize}

\subsection{Продажа предметов и услуг}
Cовершается следующим образом: 
\begin{enumerate}
    \item Определите СП предмета или услуги.
    \item Повысьте Богатство продающего героя на столько, на сколько он понизил бы его значение, успешно купив предмет с получившимся СП.
\end{enumerate}
\begin{tcolorbox}
    Обычно конекст достаточно четко говорит о том, может ли статист купить предмет или услугу. В случае сомнений, воспользуйтесь проверкой Неприятностей. Общение и Выступление пригодятся любителям агрессивной рекламы. Сама продажа не требует каких-либо проверок.
\end{tcolorbox}
\paragraph{Комплексные продажи:} если герою нужно продать большую партию товаров сразу, их СП складывается. При этом СП набора снижается согласно таблице:
\begin{center}\begin{tabular}{ |c|c| }\hline
    \textbf{Суммарное количество предметов} & \textbf{Общее снижение СП} \\ \hline
    2 предмета & -3 \\ \hline
    3 предмета & -6 \\ \hline
    4 предмета & -10 \\ \hline
    5 предметов & -15 \\ \hline
    6 предметов & -21 \\ \hline
    7 предметов & -28 \\ \hline
    8 предметов & -36 \\ \hline
    9 предметов & -45 \\ \hline
    10 и больше предметов & -55 \\ \hline
\end{tabular}\end{center}
Затем следуйте обычным правилам продажи. Продавать предметы большими партиями редко бывает выгодно, если у героя уже достаточно большое значение Богатства.
    
\paragraph{Делим на всех!} СП предмета/предметов при продаже может быть разделен на любое количество долей, если сразу несколько героев претендуют на процент с продажи. Раздел СП происходит после понижения СП за факт продажи и износ.

\subsection{Бартер}
В мире есть места, где Богатство героев не имеет значения. Аборигенов не интересуют стеклянные бусы, галстуки, кружевные чулочки и непристойные картинки (вообще-то, очень даже интересуют, но статисты не готовы отдать за них полезные вещи). О таких придумках, как векселя и валюта, они и слышать не хотят. Здесь место торговли занимает старый добрый Бартер.
\paragraph{Проверка} Бартера совершается следующим образом:
\begin{enumerate}
    \item Определите суммарную СП предлагаемых героем товаров и услуг(СП продажи).
    \item СП продажи на число предметов (но не услуг) в предложении.
    \item Определите суммарную СП приобретаемых у статиста товаров и услуг(СП покупки).
    \item Совершите проверку \textbf{|СП продажи|} против \textbf{|10+СП покупки|}
\end{enumerate}
Провал проверки Бартера означает, что герой и статист не договорились, и сделка не состоится. Повторные проверки с участием тех же предметов невозможны (в этой Сцене так уж точно), но герой вправе попробовать обменять на желаемое другие вещи из своих закромов.
\begin{tcolorbox}
    Проверка Бартера является проверкой Богатства во всех отношениях. За одним исключением - при ее успехе или провале герои не повышают и не понижают значение своего Богатства.
\end{tcolorbox}

\paragraph{}Герой может использовать навык Торговли при Бартере, но тогда сложность всех проверок возрастает на 5. Торговцу, привыкшему говорить языком денег будет труднее убедить тех, кто концепцию денег не понимает и не принимает.

\subsection{Торг уместен}
Успешная проверка Торговли позволяет сделать одно из следующего (по выбору игрока):
\begin{itemize}
\item[--] Получить Преимущество на проверку Богатства.
\item[--] Дать Преимущество на проверку Богатства другому герою, если герой с Торговлей помогает торговаться.
\item[--] Повысить Богатство на дополнительный 1 в случае успеха проверки, если герой продает или помогает продавать.
\item[--] Уменьшить потерю Богатства на 1 в случае успеха проверки, если герой покупает или помогает покупать.
\end{itemize}
Провал проверки Торговли приводит к одному из следующего (по выбору игрока):
\begin{itemize}
\item[--] Герой получает Помеху на проверку Богатства.
\item[--] Другой герой получает Помеху на проверку Богатства, если герой с Торговлей помогает торговаться.
\item[--] Богатство повышается на 1 меньше, чем должно в случае успеха проверки, если герой продает или помогает продавать. В худшем случае герой потеряет предмет и не повысит свое Богатство.
\item[--] Потеря Богатства увеличивается на 1 в случае успеха проверки, если герой с Торговлей покупает или помогает покупать.
\end{itemize}

\subsection{Доступность товаров и услуг}
Далеко не всегда герои могут приобрести необходимое, даже если есть чем расплатиться. Если требуется выяснить, доступны ли нужные героям товары или услуги, совершите проверку Неприятностей.
\trouble
{Избыток}{Статист охотно продаст товар или возьмется за услугу – он не испытывает в товаре недостатка, а услуга не слишком его обременяет. Используйте СП из таблицы.}
{Наличие}{Несколько экземпляров товара есть в продаже. Статист расстанется с товаром или возьмется за услугу за честную цену. Повысьте СП на 2.}
{Дефицит}{Статист владеет товаром или необходимыми для услуги навыками, но не желает продавать товар или оказывать услугу – разве что за крупный барыш. Повысьте СП в 2 раза.}
{Нет}{У статиста нет необходимого товара или он не может оказать услугу.}
