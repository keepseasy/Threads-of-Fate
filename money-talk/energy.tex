\subsection{Энергия и топливо}

Энергия - это кровь цивилизации, в какой бы форме она не проявлялась. Запасы топлива в цистернах, магические кристаллы, излучающие энергию, емкие батареи - эти источники использовались и будут использоваться для того, чтобы разнообразные устройства, средства передвижения и вооружение продолжали работать.

\paragraph{СП Топлива:} относительно невелика - СП 5 за 100 зарядов. Топливо может применяться для заправки транспорта. Чтобы преобразовать Топливо в Энергию для оружия, устройств, или восполнения Эн героя, потребуется специальный генератор.

\paragraph{Экзотическое топливо:} некоторые предметы требуют особого топлива, которое не распространено широко в мире. Правила расхода и приобритения топлива в этих случаях будут указаны отдельно.

\paragraph{Заряды Топлива и Энергии} транспорт, устройства и оружие потребляют Заряды (Топлива и Энергии соответственно).

\paragraph{Энергия и Вторичная характеристика Энергии} Эн героев и статистов является гораздо более крупной единицей измерения, чем Заряды Топлива и Заряды Энергии. В 1 Эн содержится 100 Зарядов Энергии (Зр). Для передачи Эн в устройства и обратно требуется специальный адаптер питания.
\paragraph{КПД передачи.} Адаптеры питания не идеальны и пре конвертации Энергии в Заряды и обратно возникают потери. КПД определяет, сколько действительно Зарядов или Эн получит герой после конвертации.
\paragraph{Потребление Х:} устройство потребляет \textbf{|Х|} Зр за указанный период работы.

\paragraph{Энергетические ячейки:} энергетическое оружие и устройства питаются универсальными батареями, емкость которых зависит от размера. После использования всех Зарядов батарею нельзя перезарядить.
\begin{center}
\begin{tabular}{|c|c|c|c|}
\hline
Количество Зарядов & Размер батареи & Вес батареи & СП \\ \hline
10 & Пластинка (1 см3) & 0.1 & 3 \\ \hline
30 & Миниатюрная (5 см3) & 1 & 6 \\ \hline
100 & Крошечная (15 см3) & 3 & 9 \\ \hline
300 & Маленькая (40 см3) & 7 & 12 \\ \hline
500 & Средняя (70 cм3) & 15 & 15 \\ \hline
\end{tabular}
\end{center}
