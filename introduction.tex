\chapter*{Вступление}
\paragraph{}Эта игра о Судьбе, которая управляет всем, кроме свободной воли. Эта игра о победах, веду-щих к поражениям, и о поражениях, ведущих к победам. Эта игра о борьбе с самим собой и о Ни-тях, протянутых над бездной. Эта игра о том, как просто отнять чужую жизнь, как легко расстаться со своей, и как сложно порой избежать и того, и другого. 
\newline Вам предстоит выступить в роли отважных (или не очень) героев, и Судьбы, возносящей их к вершинам мира и низвергающей в бездны отчаяния. Судьбе нравится наблюдать, как герои стал-киваются с неприятностями и преодолевают их. Иногда Судьба самолично вмешивается в события, заставляя героев исполнять ее капризы, или же, наоборот, помогая. Что уготовано героям – слава, богатство, любовь или безвестная смерть в придорожной канаве? Все в ваших руках – руках Судь-бы!

\paragraph{}Для игры вам понадобится три К20 (двадцатигранных кубика), минимум один приятель, готовый с вами играть, карандаши, ластик, блокнот и немного воображения. Один из вас должен взять на себя обязанности мастера (ведущего), остальные станут игроками. Игра проходит в форме беседы. Игроки и мастер обмениваются репликами, описывающими события в воображаемом пространстве. Правила служат для того, чтобы упорядочить это пространство и сделать его общим. Они подскажут участникам игры, когда меч воина бессильно отскакивает от щита, а когда - поражает цель, или насколько шпиону сложно вскарабкаться по замшелой крепостной стене, и осветят многие другие неоднозначные моменты. Игрок придумывает биографию и внешность героя, описывает его действия, то есть играет роль героя и его Судьбы, временами благосклонной, временами безжалостной, а временами - безразличной. Это означает, что иногда герой будет терпеть неудачи, влипать в неприятности, сталкиваться с трудностями лишь потому, что игрок так решил. Мастер изображает окружающий героев мир и описывает его реакцию на их действия (или бездействие).

\paragraph{}Успехи и неудачи героев определяются решениями, которые игроки принимают в ходе игры, и бросками К20. Несмотря на то, что мастер закладывает основы сюжета, игроки - полноправные соавторы. Правила подразумевают, что игроки и мастер готовы работать над историей сообща, прислушиваться к желаниям друг друга и договариваться в случае разногласий.

\begin{tcolorbox}
    Главная задача свода правил – помочь игрокам и мастеру совместно выстроить прав-доподобную и интересную историю, финал и детали которой заранее не известны никому из них. Правила не стремятся симулировать объективную реальность во всем ее многообразии, хотя могут это сделать, если у вашей игровой команды возникнет подобный запрос.
\end{tcolorbox}

\section*{Как играть в <<Нити Судьбы>>}
Игра рассказывает историю о героях, победы которых зачастую имеют цену, а неудачи — последствия. Нити Судьбы и совершение Ходов Судьбы — единственный способ преуспеть без всяких оговорок. Мир вокруг героев изображается широкими мазками, а значимые детали определяются во время игры при помощи проверок Неприятностей и все тех же Ходов Судьбы. Помимо того, есть несколько простых принципов, которые позволят мастеру и игрокам высвободить весь потенциал системы.
\section*{Если вы мастер…}

\paragraph{Готовьте завязку, а не сюжет.} Все, что безусловно нужно игре — отправная точка. Остальное сделают игроки, кубики и воображение. \textbf{Обозначьте возможности и цену}. Лгите героям, но не игрокам. Игрокам стоит знать, ради чего герои рискуют, каковы шансы на победу, как именно можно достичь цели… и с чем придется расстаться по пути.
\paragraph{Используйте Капризы Судьбы.} Расшевелите игроков, вводя в игру Недостатки, Темные стороны и Решки их героев. Не давайте героям опомниться, а игрокам — заскучать.
\paragraph{Не будьте всеведущим.} Позвольте игрокам вас удивить. К тому же, чем больше в вашей игре белых пятен, тем больше возможностей для сотворчества. Игрокам будет непросто придумать что-то, если для этого чего-то не осталось места. Используйте проверки Неприятностей, если контекст не дает однозначного ответа на возникший вопрос.
\paragraph{Помогайте, не заставляйте.} Если по каким-то причинам динамика игры падает, а у игроков нет идей — подкиньте и идей, и событий. При этом не стоит заменять идеи игроков своими и прибегать к мастерскому праву вето слишком часто.
\section*{Если вы игрок...}

\paragraph{Знайте своего героя.} Помните о том, чего он может добиться сам, а где ему понадобится ваша помощь — помощь Судьбы!
\paragraph{Действуйте.} Настоящие герои не будут сидеть и ждать у моря погоды, они будут действовать, даже если Судьба точно знает, что их дело обречено.
\paragraph{Создавайте герою проблемы.} Используйте Успех с Неприятностями, вводите в игру Недостатки, Темные стороны и Решки вашего героя, принимайте худшие варианты Неприятностей — протягивайте к героям Нити! Благополучие и безопасность — плохая основа для запоминающейся истории.
\paragraph{Используйте Капризы Судьбы на других героях.} Если вы видите хорошую возможность для ввода Недостатка, Темной стороны или Решки героя другого игрока — используйте ее. Покажите, насколько Судьба своенравна!
\paragraph{Применяйте Ходы Судьбы.} Поддержите героя, если он в этом нуждается! Изучите перечень общедоступных Ходов, не упускайте из виду Уникальные ходы. Помните — когда Судьба на стороне героя, он способен на все.
\section*{Если вы мастер или игрок...}

\paragraph{Будьте готовы к компромиссам.} История, которая создается на игре, - общая, и каждый участник - ее равно важная часть. Помните об этом.
\paragraph{Правильных решений нет, но всегда есть последствия.} Вы - Судьба, своенравная, капризная, порой жестокая, реже - милосердная. Ваша цель - наблюдать за тем, как герои выпутываются из проблем, которые вы же для них и устроили, и поразвлечься вволю. Худшее, что можно сделать на игре - лишить героев последствий их действий, какими бы ужасными эти последствия ни были.
\paragraph{Уважайте результаты бросков.} Кубики - полноправные соавторы истории. Не игнорируйте результаты бросков, иначе рискуете превратиться из игроков в настольную ролевую игру в писательский коллектив. В этом нет ничего плохого, но это совсем другой вид развлечений.
\paragraph{Доверяйте друг другу.} Взаимное доверие - основа игры, которой все будут довольны. Не забывайте - ваша цель не победа в настольной ролевой игре, ваша цель - запоминающаяся история. И если кто-то использует Каприз Судьбы или вводит в игру Недостаток своего героя, то он участвует в создании истории, а не подставляет других героев под удар.
\paragraph{Наслаждайтесь игрой!}