\section{Неприятности}
Зачастую проблемы создают герои, но иногда Неприятности сами находят их. Неприятности изображают неблагоприятные события, которые могут случиться, а могут и пройти стороной. В потенциально опасной сцене, такой как прогулка по огромной свалке, обыск древнего убежища или прыжок в море с обрыва, мастер может инициировать проверку Неприятностей, если контекст ситуации недостаточно ясно говорит о том, что герою ничто не угрожает.

\begin{tcolorbox}
    Неприятности почти всегда оставляют герою шанс спастись и не убивают его сразу. Все случаи, когда Неприятности приводят к смерти героя описаны отдельно в правилах.
\end{tcolorbox}

Перед проверкой Неприятностей определите:
\begin{itemize}
    \item[--] Причину проверки;
    \item[--] Значимые детали проверки;
    \item[--] Что случится при Успехе;
    \item[--] Что случится при Катастрофе.
\end{itemize}

\begin{tcolorbox}
    Причина и исходы Неприятностей может оставаться тайной для игроков. Но поскольку Судьба всеведуща, авторский коллектив рекомендует проговаривать эти моменты.
\end{tcolorbox}

\paragraph{Герою потребуется проверить Неприятности, когда он:}
\begin{itemize}
    \item[--] Крадется по ядовитой трясине;
    \item[--] Поднимает в воздух древний самолет;
    \item[--] Покупает мясное буррито у незнакомого лотошника;
    \item[--] Выпивает в подозрительной компании;
    \item[--] Прыгает в море с обрыва.
\end{itemize}

\subsection{Проверка неприятностей} это, обычно, бросок К20 без модификаторов. Результат проверки определяется по таблице:
\trouble
{Успех}%success name
{Герой вышел сухим из воды. Ну, на то он и герой.}%success description
{Затруднение}%difficulties name
{Герой вовремя заметил надвигающиеся трудности. Скорее всего, он сумеет их избежать. Скорее всего.}%difficulties description
{Проблема}%troubles name
{Герой оказался в сложном, но не безвыходном положении.}%troubles description
{Катастрофа}%fiasco name
{Герой на волосок от смерти. Ему будет непросто выкрутиться.}%fiasco description

\paragraph{КУ и КП.} к проверки Неприятностей не применяются.

\paragraph{Модификация проверки Неприятностей.} Некоторые правила позволяют или предписывают модифицировать результат Неприятностей, прибавляя или отнимая от него некое число. Возможна ситуация, когда герой не способен получить лучший или худший вариант проверки из-за суммы модификаторов.

\subsection{Приятные Неприятности.} Эта механика так же может использоваться и для определения того, оказался ли под рукой у героя необходимый предмет, есть ли поблизости разыскиваемое героем заведение и так далее, если контекст не дает исчерпывающего ответа на вопрос. В этом случае проверка определит:
\begin{itemize}
    \item[--] Есть ли чистая вода в округе;
    \item[--] Продаются ли в лавке пули нужного калибра;
    \item[--] Можно ли доверять проводнику;
    \item[--] Легко ли взломать старую решетку;
    \item[--] Умеет ли читать дочка отшельника.
\end{itemize}
В этом случае проверка неприятностей будет выглядеть так: 
\trouble
{Да}%success name
{Герой получает желаемое или может получить желаемое, приложив незначительные усилия или потратив немного ресурсов.}%success description
{Да, но}%difficulties name
{Герой может получить желаемое, приложив усилия или потратив ресурсы.}%difficulties description
{Нет, но}%troubles name
{Герой может получить желаемое, только если приложит серьезные усилия или потратит значительные ресурсы.}%troubles description
{Нет}%fiasco name
{Герой не может получить желаемое. Ему придется искать другие пути.}%fiasco description

\subsection{Скрытая угроза}
Иногда последствия действий героев безобидны лишь на первый взгляд. Скрытая Угроза представляет проверку Неприятностей, результаты которой позволяют мастеру определить контекст загодя, или по-новому трактовать события уже завершенной Сцены:
\trouble
{Неожиданная слава}%no sweat name
{Герои будут вознаграждены за смелость, отзывчивость и доброту, а нерешительность, равнодушие и жестокость не возымеют далеко идущих последствий.}%no sweat description
{Круги на воде}%tough day name
{Действия героев не приведут к значительным последствиям. Полученные знакомства мимолетны, враги незлопамятны, а хозяева вещей, которые герои прибрали к рукам нескоро заметят пропажу.}%tough day description
{Тень затмения}%we have trouble name
{Проблема, которую все же реально заметить, пока не станет слишком поздно. Сцена еще может обернуться сущим кошмаром, но герои выйдут сухими из воды, если не будут хлопать ушами.}%we have trouble description
{Петля на шее}%fiasco name
{Герои попали в переплет. Убитые разбойники имели влиятельных покровителей, найденные предметы были кем-то спрятаны, а спасенная красотка обокрала караван и сбежала!}%fiasco description
Проявления Скрытой угрозы неочевидны, а иногда вовсе незаметны для героев, хотя Судьба вправе подсказать им линию поведения или принять иные меры. 

\subsection{Общие Неприятности}
Иногда возникают ситуации, в которых Неприятности напрямую касаются всех участников Сцены, - например, заглох двигатель самолета или шериф считает героев подельниками.
\newline Для того, чтобы откупиться от Общих неприятностей достаточно 1 Нити любого отдельного героя, однако если игроки принимают Каприз Судьбы "Катастрофа!", Нити протягиваются ко всем героям и \tbd персонам, присутствующим в Сцене. О Капризах Судьбы читайте в соответствующем разделе правил.

\subsection{Неприятности под Контролем.}
Обычно Неприятности случаются внезапно, но иногда герою выпадает шанс смягчить эффект. Если мастер считает, что есть способ как-то повлиять на исход, герой может совершить проверку Характеристики или Навыка, уместного в контексте ситуации. Сложность проверки устанавливается мастером как обычно. Величина успеха прибавляется к результату проверки Неприятностей, а величина провала вычитается из него.
\newline Например, в трущобах герой может постараться не мозолить окружающим глаза и использовать Скрытность против сложности, установленной мастером. Это все еще не помешает герою случайно наткнуться на грабителей, но может \textit{понизить шанс} такой встречи. Если герой прошел Скрытность на 3, то к числу, выпавшему при проверке Неприятностей, будет прибавляться 3. Если герой провалил Скрытность на 2, из числа, выпавшего при проверке Неприятностей, будет вычтено 2, а его неумелые попытки выглядеть незаметно привлекут внимание окружающих.
\newline Если герой прошел проверку Характеристики или Навыка, и при сложении ее результата с результатом проверки Неприятностей, получилось число больше 20, то считайте этот результат "Успехом". Если герой провалил проверку Характеристики или Навыка, и при вычитании ее результата из результата проверки Неприятностей, получилось число меньше 1, то считайте, что случилась "Катастрофа".
\paragraph{Неприятности и Впечатление.} Если правила предписывают совершить проверку Неприятностей под контролем Впечатления, \textbf{|текущее значение Впечатления от героя - 10|} прибавляется к числу, выпавшему при проверке Неприятностей.

\subsection{Когда использовать проверку Неприятностей?}
Эта механика позволяет мастеру создавать игровые события и факты без предварительной подготовки, руководствуясь контекстом сцены, и при этом разделять повествовательные права с игроками. Она не заменяет собой проверки Навыков (хотя ситуации, в которых такая замена будет уместна, могут возникнуть). Проверка Неприятностей позволит легко и быстро узнать, в каком настроении вернулся с охоты молодой вождь дикарей, надежен ли информатор, есть ли поблизости укрытие от надвигающейся песчаной бури. 

\begin{tcolorbox}
    Обычно проверки Неприятностей инициирует мастер. Ему же придется судить о том, насколько проверка вообще необходима. Разумеется, если у мастера и игроков уже готовы ответы на все вопросы, проверка Неприятностей вряд ли будет использоваться часто. Но даже в таких случаях не стоит полностью исключать ее из игры.
\end{tcolorbox}
