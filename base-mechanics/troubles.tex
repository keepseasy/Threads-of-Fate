\section{Неприятности}

Зачастую проблемы создают герои, но иногда Неприятности сами находят их. Неприятности изображают неблагоприятные события, которые могут случиться, а могут и пройти стороной. В потенциально опасной сцене, такой как прогулка по огромной свалке, обыск древнего убежища или прыжок в море с обрыва, мастер может инициировать проверку Неприятностей, если контекст ситуации недостаточно ясно говорит о том, что герою ничто не угрожает.
\trouble
{Успех}%success name
{Герой вышел сухим из воды. Ну, на то он и герой.}%success description
{Затруднение}%difficulties name
{Герой вовремя заметил надвигающиеся трудности. Скорее всего, он сумеет их избежать. Скорее всего...}%difficulties description
{Проблема}%troubles name
{Герой оказался в сложном, но не безвыходном положении.}%troubles description
{Катастрофа}%fiasco name
{Герой на волосок от смерти. Из грязного кабака вывалилась толпа головорезов, трухлявый пол просел под ногами, а в воде притаились острые камни. Герою будет непросто выкрутиться.}%fiasco description
\begin{tcolorbox}
Неприятности всегда должны оставлять герою хотя бы мизерный шанс спастись и не должны убивать его сразу.
\end{tcolorbox}
Эта механика может использоваться и для определения того, оказался ли под рукой у героя необходимый предмет, есть ли поблизости разыскиваемое героем заведение и так далее, если контекст не дает исчерпывающего ответа на вопрос. В этом случае проверка неприятностей будет выглядеть так: 
\trouble
{Да}%success name
{Герой получает желаемое или может получить желаемое, приложив незначительные усилия или потратив немного ресурсов.}%success description
{Да, но}%difficulties name
{Герой может получить желаемое, приложив усилия или потратив ресурсы.}%difficulties description
{Нет, но}%troubles name
{Герой может получить желаемое, только если приложит серьезные усилия или потратит значительные ресурсы.}%troubles description
{Нет}%fiasco name
{Герой не может получить желаемое. Ему придется искать другие пути.}%fiasco description
Например, скрываясь от преследования, герой забирается в незнакомый дом и пытается найти там оружие. Проверка Неприятностей покажет, держит ли хозяин в доме хоть что-то, похожее на оружие. В противном случае герою придется довольствоваться посудой или ломать мебель для того, чтобы вооружиться!
\paragraph{Общие Неприятности:} Иногда возникают ситуации, в которых Неприятности напрямую касаются всех участников Сцены, – например, заглох двигатель самолета или шериф считает героев подельниками.
\newline Для того, чтобы откупиться от Общих неприятностей достаточно 1 Нити любого отдельного героя.
\newline При Капризе Судьбы "Катастрофа!", принятом игроками, Нити протягиваются ко всем героям и персонам, присутствующим в Сцене.
\paragraph{Неприятности под Контролем:} Обычно Неприятности случаются внезапно, но иногда герою выпадает шанс смягчить (или усугубить) эффект. Если мастер считает, что существует способ как-то повлиять на исход, герой может совершить проверку Характеристики или Навыка, уместного в контексте ситуации. Сложность проверки устанавливается мастером как обычно. Величина успеха прибавляется к результату проверки Неприятностей, а величина провала вычитается из него. 1 и 20 на кубике при проверке Неприятностей считаются результатами "Катастрофа" и "Успех" вне зависимости от модификаторов.
\newline Если герой прошел проверку Характеристики или Навыка, и при сложении ее результата с результатом проверки Неприятностей, получилось число больше 20 (такое очень даже вероятно), то считайте этот результат "Успехом". Если герой провалил проверку Характеристики или Навыка, и при вычитании ее результата из результата проверки Неприятностей, получилось отрицательное число (а такое тоже не исключено!), то считайте, что случилась "Катастрофа".
\newline Например, в трущобах герой может постараться не мозолить окружающим глаза и использовать Скрытность против сложности, установленной мастером. Это все еще не помешает герою случайно наткнуться на грабителей, но может \textit{понизить шанс} такой встречи. Если герой прошел Скрытность на 3, то к числу, выпавшему при проверке Неприятностей, будет прибавляться 3. Если герой провалил Скрытность на 2, из числа, выпавшего при проверке Неприятностей, будет вычтено 2, а его неумелые попытки выглядеть незаметно привлекут внимание окружающих.
\paragraph{Неприятности и Впечатление.} Если правила предписывают совершить проверку Неприятностей под контролем Впечатления, \textbf{|текущее значение Впечатления от героя – 10|} прибавляется к числу, выпавшему при проверке Неприятностей.
\paragraph{Когда использовать проверку Неприятностей?} Эта механика позволяет мастеру создавать игровые события и факты без предварительной подготовки, руководствуясь контекстом сцены, и при этом разделять повествовательные права с игроками. Она не заменяет собой проверки Навыков (хотя ситуации, в которых такая замена будет уместна, могут возникнуть). Проверка Неприятностей позволит легко и быстро узнать, в каком настроении вернулся с охоты молодой вождь дикарей, умеет ли читать дочка отшельника, есть ли поблизости скалы, где можно укрыться от песчаной бури.
\begin{tcolorbox}
Как правило, проверки Неприятностей инициирует мастер. Ему же придется судить о том, насколько проверка вообще необходима в контексте Сцены. Разумеется, если у мастера и игроков уже готовы ответы на все вопросы, проверка Неприятностей вряд ли будет использоваться часто. Но даже в таких случаях не стоит полностью исключать эту механику из игры.
\end{tcolorbox}