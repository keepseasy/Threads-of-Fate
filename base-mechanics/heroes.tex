\section{Герои, статисты и персоны}
\paragraph{Герои:} так называются персонажи под управлением игроков. Они — протагонисты истории, которую совместно создают мастер и игроки, они избраны Судьбой (для чего именно — вопрос открытый). Герой — ведущая роль на сцене мироздания, точка приложения сил Судьбы. Именно к героям протянуты Нити Судьбы — страховочная сеть над бездной... или ниточки своенравного кукловода.
\paragraph{Статисты:} это персонажи и существа под управлением мастера. Они — второстепенные лица истории. Главное отличие статистов от героев — невозможность прибегнуть к помощи Судьбы. Статисты могут обладать Атрибутами, Трюками и Недостатками, а также совершать Уникальные ходы без обрыва Нитей, принимая все возможные последствия.
\paragraph{Персоны:} это статисты, роли которых сопоставимы по значимости с ролями самих героев. Это не имеет никакого отношения к могуществу или социальному статусу. Правитель, пославший героев на войну, — всего лишь статист, он не принимает значимого участия в истории (хотя участвует в ее завязке). От действий короля мало что зависит — он остался за кулисами, укрывшись в неприступном убежище. А вот находчивый помощник одного из героев — без сомнений, персона. Его успехи и неудачи очень даже влияют на развитие истории. Хоть слуга и не является главным действующим лицом, Судьба присматривает за ним... в полглаза. Персоны также могут использовать Нити Судьбы.

\begin{tcolorbox}
    Нередко Судьба готовит для персон особое место в своих планах. Героям не стоит удивляться, увидев живым и здоровым злодея, которого они с таким трудом одолели неделю назад (хотя мастеру лучше не злоупотреблять сюжетным иммунитетом!).
\end{tcolorbox} 

%"Нити Судьбы" — игра о приключениях группы друзей или как минимум единомышленников. Правила не предполагают конфликтов между героями, хотя могут реализовать их технически. Если мастер и игроки согласны с тем, что герои противостоят друг другу, то все, что герой может сделать со статистом, он может сделать и с другим героем.

\paragraph{Простой статист или персона?} Вопрос скорее философский, чем игромеханический. Статус персоны непостоянен. Если персона по неким причинам отходит на второй план и перестает активно участвовать в развитии истории, то превращается в простого статиста, и наоборот. Если у вашей игровой команды возникают сложности с определением персон, обсудите следующие вопросы:
\begin{itemize}
\item[--] Будет ли героям сложнее добиться цели, если статист не поможет им?
\item[--] Связывают ли статиста и героев чувства дружбы, любви или долга?
\item[--] Готов ли кто-то из героев рискнуть жизнью ради статиста?
\end{itemize}
Если вы ответили "да" на все три вопроса, перед вами самая настоящая персона, однако \textit{Персоны-антагонисты} чаще определяются мастером в соответствии с его личными представлениями о сюжетной роли и важности статиста.

\section{Герои и Судьба}
\paragraph{}Судьба не олицетворяет какую-то \textit{определенную} высшую силу. Прежде всего, Судьба – элемент роли игрока, который решает, когда герой получит поддержку, а когда – нет. Какой облик примет Судьба – госпожи удачи, стечения обстоятельств, сюжетной брони или вмешательства ино-планетных прогрессоров – зависит от контекста вашей истории. Он же подскажет, насколько оче-видно вмешательство Судьбы для героев и окружающих их статистов.
\newline Судьба всеведуща. Почти. Это означает, что игроки вправе получить доступ к любому уста-новленному внутриигровому факту, либо установить его при помощи игромеханических средств, если это важно. Также игровая механика не подразумевает совершения мастером или игроками бросков, результат и последствия которых скрыты от кого-то из них.
\newline Обратите внимание, что пока факт не установлен и не вошел в игру, очевидно, доступа к нему не имеют ни мастер, ни игроки. Так же некоторые способности и правила позволяют по-новому интерпретировать уже установленные факты, поэтому сюрпризы вовсе не исключены – для всех.
