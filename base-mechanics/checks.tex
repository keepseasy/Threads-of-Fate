\section{Проверки}
\paragraph{Проверки} - это броски кубика К20, изображающие усилия героя, физические, волевые или умственные. Чем большее число выпало на кубике, тем больше шанс, что герой преуспеет. Различные факторы могут повысить или понизить шансы героя на успех. Они называются \textbf{бонусами} и \textbf{штрафами} и отображаются числами, которые прибавляются к результату броска или отнимаются от него.
\paragraph{Проверка |А| против |Б|.} В этой формулировке \textbf{|А|} означает бонус, который прибавляется к проверке, а \textbf{|Б|} - это сложность, которую надо преодолеть. Проверка считается успешной, если сумма \textbf{|А + результат броска К20|} больше или равна \textbf{|Б|}.

Перед совершением проверки определите следующее:
\begin{itemize}
\item[--]Цель, которую герой пытается достичь совершением проверки.
\item[--]Сложность проверки, исходя из цели героя и контекста
ситуации, если она не определена правилами.
\item[--]Цену провала проверки, если она не определена правилами.
\item[--]Наличие Преимуществ и Помех.
\item[--]Наличие бонусов или штрафов.
\item[--]Допустимость Успеха с Неприятностями и потенциальные Неприятности, если он возможен.
\end{itemize}


\paragraph{Величина успеха и провала.} Некоторые проверки имеют градации - например, проверки Доблести и Меткости. В этом случае важна разность между результатом проверки и заданной сложностью - величина успеха или провала.

\paragraph{Преимущества и Помехи:} порой обстоятельства складываются неблагоприятно или, наоборот, благоволят герою. В этом случае бросьте дополнительный кубик за каждую Помеху или Преимущество. Выберите меньший результат, если герой находится под действием Помехи, и больший, если герой обладает Преимуществом. Одновременное действие 1 Помехи и 1 Преимущества сводит их на нет. 
\newline Герой не может страдать больше чем от 2 Помех за бросок, как не может реализовать больше 2 Преимуществ за бросок. То есть максимальное число кубиков в броске - 3.
\paragraph{Активные проверки} подразумевают некие осознанные и целенаправленные действия героя. Например, если герой:
\begin{itemize}
    \item[--] Стреляет из пистолета (Меткость);
    \item[--] Преследует вора (Скорость);
    \item[--] Заучивает текст (Интеллект);
    \item[--] Роет окоп (Атлетика (Сила));
    \item[--] Ремонтирует машину (Ремонт (Мудрость)).
\end{itemize}
\paragraph{Пассивные проверки} изображают события, напрямую связанные с героем, но происходящие вне зависимости от его желания. Например, если герой:
\begin{itemize}
    \item[--] Преодолевая болевой шок от ранения (Выносливость);
    \item[--] Пялясь по сторонам без конкретной цели (Наблюдательность (Мудрость));
    \item[--] Стараясь связно мыслить после попойки (Воля);
    \item[--] Усваивая поступающую информацию (Интеллект);
    \item[--] Пытаясь справиться с тяжелой болезнью (Выносливость).
\end{itemize}

\paragraph{Эффективные значения навыков и характеристик.} В некоторых случаях, например, когда нужно отразить длительные усилия героя, повторяющиеся достаточно регулярно или для упрощения состязаний героев. Не кидайте кубик, вместо этого прибавьте значению Навыка значение по таблице в зависимости от количества Помех или Преимуществ к проверке:
\begin{center}
\begin{tabular}{|l|l|}
\hline
Условия проверки & Значение броска К20 \\ \hline
2 Помехи & 5 \\ \hline
1 Помеха & 7 \\ \hline
Нет Помех и Преимуществ & 10 \\ \hline
1 Преимущество & 12 \\ \hline
2 Преимущества & 15 \\ \hline
\end{tabular}
\end{center}
Эффективный показатель пригодится вам, когда герой:
\begin{itemize}
    \item[--] Дежурит на часах (Наблюдательность (Мудрость));
    \item[--] Простукивает стены в поисках тайника (Наблюдательность (Интеллект));
    \item[--] Запоминает детализированную карту местности (Наблюдательность (Интеллект));
    \item[--] Развлекается на вечеринке, пытаясь не перебрать со спиртным (Общение (Выносливость));
    \item[--] Совершает длительный переход (Атлетика (Выносливость)).
\end{itemize}

\paragraph{Массовые проверки статистами.} 
Иногда группа статистов подвергается эффекту, требующему множества отдельных бросков: отряд городского ополчения оказывается в зоне действия отравляющих газов, группа охотников забредает в болото, торговый караван попадает под оползень. В этом случае мастер может совершить проверку один раз и применить для каждого статиста соответствующие бонусы и штрафы, чтобы определить результат.
\paragraph{Критический провал (КП) и Критический успех (КУ):} 
Выпавшие на К20 1 и 20 отражают ошеломительные провалы и успехи на грани возможного. В таких случаях мастер может ввести в игру дополнительные эффекты броска. 
\newline При выпадении 1 на кубике мастер может засчитать автоматический провал, даже если суммарный результат превысил сложность проверки. При выпадении 1 во время проверок Доблести или Меткости цель никогда не теряет Единицы Здоровья.
\newline При выпадении 20 на кубике мастер может засчитать автоматический успех проверки, даже если результат не превысил сложность задачи. При выпадении 20 во время проверок Доблести или Меткости цель всегда теряет минимум 1 Единицу Здоровья.
\newline Критические провалы профессионалов и Критические успехи дилетантов кому-то могут показаться нелогичными. С другой стороны, это мощный повествовательный инструмент, который не стоит игнорировать. Объяснив, из-за чего спасовал профи и преуспел дилетант, вы насытите вашу историю интереснейшими подробностями.

\paragraph{Перебросы:} некоторые правила и способности позволяют перебрасывать проверки, которые не понравились игроку. Любая проверка может быть переброшена не больше 2 раз, вне зависи-мости от способностей, позволяющих это делать. Если проверка включает несколько К20, из-за Помех или Преимуществ, перебрасываются все К20 участвующие в броске.

\paragraph{Какую сложность выбрать?} Игровые испытания показали, что оптимальная сложность задач для героев, даже весьма опытных, составляет от \textbf{15} до \textbf{20}. Используйте меньшую сложность, когда хотите продемонстрировать превосходство героев над статистами, для большинства из которых сложность в \textbf{10} - довольно серьезный вызов, или если желаете оставить небольшой, но все-таки осязаемый, шанс на провал. Сложности, превышающие \textbf{20}, обычно требуют от героев вмешательства Судьбы.
\newline Мастер должен сообщить игроку целевую сложность при совершении проверки. На основании этой информации игрок принимает решения об использовании Ходов Судьбы и способностей героя.
\subsection{Примерная сложность задач}
\begin{center}
\begin{tabular}{ |c|c| }
\hline
\textbf{Задача} & \textbf{Сложность} \\ \hline
Примитивная & 5 \\ \hline
Повседневная & 10 \\ \hline
Придется попотеть & 15 \\ \hline
Работа для эксперта & 20 \\ \hline
Вызов для эксперта & 25 \\ \hline
На грани возможного & 30 \\ \hline
\end{tabular}
\end{center}

\paragraph{Успех с Расплатой:} если герой не прошел проверку, игрок может предложить ввести в игру Расплату, позволяющую тому преуспеть или сопутствующую успеху. Мастер может запретить Успех с Расплатой, если, по его мнению, герой приобретет значительно больше, чем потеряет, или на проверку выпал \textbf{КП}.
\paragraph{Успех с Расплатой и Навыки.} Если герой не распределил Очки опыта в Навык, который он проверял, при использовании Успеха с Расплатой выберите 1 дополнительную Расплату. 
\newline Если герой достигает успеха только при выпадении 20 на кубике (или правила не позволяют ему совершить проверку, как в случае Экспертных навыков), при использовании Успеха с Расплатой выберите 1 дополнительную Расплату. 
\newline \tbd Герой может применять Успех с Расплатой и при использовании Экспертных навыков, к которым не имеет доступа, хотя фактически К20 не бросается и используется эффективное значение навыка.

Возможная Расплата:
\begin{enumerate}
\item \textbf{Временные затраты.} Выполнение задачи требует гораздо больше времени, чем планировал герой. Наверняка из-за этого он:
\begin{itemize}
    \item[--] Упустит уникальный шанс;
    \item[--] Опоздает на судьбоносную встречу;
    \item[--] Небрежно подготовится к значимому событию;
    \item[--] Не завершит другое важное дело;
    \item[--] Замешкается, вытаскивая оружие.
\end{itemize}

\item \textbf{Возможность для недругов.} Успех героя позволяет недругам приблизиться к своей цели или даже достичь ее. Причиной этого могло стать то, что:
\begin{itemize}
    \item[--] Героя не оказалось в нужном месте в нужное время;
    \item[--] Героя отвлекло выполнение задачи;
    \item[--] Успех героя был подстроен недругами;
    \item[--] Героя подвела интуиция;
    \item[--] Герой неверно расставил приоритеты.
\end{itemize}

\item \textbf{Герой под ударом.} Герой преуспел, но оказался в затруднительном положении. Возможно, его:
\begin{itemize}
    \item[--] Избрали главной целью в дистанционном бою;
    \item[--] Окружили в рукопашной схватке;
    \item[--] Сочли подозрительным местные силы правопорядка;
    \item[--] Уличили в осквернении святыни;
    \item[--] Предметы для бартера испорчены или очевидно опасны.
\end{itemize}
\item \textbf{Невыгодная позиция.} Успех вынуждает героя занять невыгодную позицию. Например:
\begin{itemize}
    \item[--] Менять колесо под беглым огнем;
    \item[--] Биться с врагами, стоя спиной к обрыву;
    \item[--] Преследовать статиста среди обезумевшей толпы;
    \item[--] Затягивать трос, повиснув головой вниз на движущемся грузовике;
    \item[--] Играть в карты, обнимая роскошную блондинку.
\end{itemize}
\item \textbf{Оповещение недругов.} Выполнение задачи привлекает к герою нежелательное внимание. Не исключено, что он:
\begin{itemize}
    \item[--] Активировал сигнализацию;
    \item[--] Споткнулся о кота;
    \item[--] Расчихался от едкой пыли;
    \item[--] Вскрикнул, прищемив палец;
    \item[--] Ругнулся по привычке. 
\end{itemize}
\item \textbf{Союзники под ударом.} Успех героя приводит к тому, что его товарищи оказываются в затруднительном положении. Обычно эти события связаны напрямую, если герой сознательно под-ставил друзей ради общей цели или сиюминутной выгоды. Впрочем, логической или сюжетной связи между ними может и не быть. Так или иначе, теперь союзники:
\begin{itemize}
    \item[--] Находятся в невыгодной позиции;
    \item[--] Захвачены в плен;
    \item[--] Испытывают недостаток в чем-либо;
    \item[--] Попали под подозрение;
    \item[--] Получили ранения.
\end{itemize}
\begin{tcolorbox}
    Неприятность \textbf{"Союзники под ударом"} подразумевают прежде всего значимых для героев статистов и персон, но если таковых нет или они отсутствуют в Сцене, то по предварительной договоренности игроков один герой может подвергнуть опасности другого. Судьба своенравна и жестока, не забывайте!
\end{tcolorbox}

\item \textbf{Ослабление эффекта.} Герой выполнил задачу, но в конце Сцены статус-кво будет восстановлен, так как:
\begin{itemize}
    \item[--] Наспех залатанный двигатель скоро заглохнет;
    \item[--] Ржавый лом не задержит гидравлический пресс надолго;
    \item[--] Грубый жгут остановил кровь, но его придется снять, чтобы избегнуть некроза;
    \item[--] Хлипкая баррикада не сможет выносить натиск голодных упырей бесконечно;
    \item[--] Компьютер будет заблокирован вскоре после взлома системы.
\end{itemize}

\item \textbf{Перерасход ресурсов.} Герой преуспел, но потратил гораздо больше ресурсов, чем планировал:
\begin{itemize}
    \item[--] Забыв переключить режим стрельбы на одиночный;
    \item[--] Сделав щедрый подарок;
    \item[--] Напоив всех в кабаке за свой счет;
    \item[--] Подложив побольше взрывчатки;
    \item[--] Проведя жаркую ночь с роковой красоткой.
\end{itemize}

\begin{tcolorbox}
    Под ресурсами подразумеваются не только материальные ценности. Ими является все, что герой может израсходовать и затем восполнить. В том числе, Характеристики, Энергия, Богатство и т. д.
\end{tcolorbox}

\item \textbf{Утрата.} Герой достиг цели, но потерял ключевой предмет или на время лишился важной способности (Уникального Хода, Функции, Трюка и т. д.), когда:
\begin{itemize}
    \item[--] Пуля угодила в докторский саквояж;
    \item[--] Доспех разъела кислотная жижа,;
    \item[--] Тачка сдохла после многочасовой гонки;
    \item[--] Замыкание сделало имплантаты бесполезными;
    \item[--] Аллергическая реакция изуродовала красивое личико.
\end{itemize}

\item \textbf{Ущерб.} Герой добился своего, но теряет число Единиц Здоровья, достаточное для получения Опасной раны, вне зависимости от имеющихся у него защитных средств:
\begin{itemize}
    \item[--] Ввязавшись в уличную драку;
    \item[--] Не успев покинуть зону взрыва;
    \item[--] Отключив опасный механизм;
    \item[--] Прикрыв друзей своим телом;
    \item[--] Победив в соревновании выпивох.
\end{itemize}

\item \textbf{На долгую память.} Успех не проходит для героя бесследно. Он обзаводится новым Недостатком, и теперь:
\begin{itemize}
    \item[--] Вспоминает былое, когда разболится \textbf{Старая рана};
    \item[--] \textbf{Пьянствует}, пытаясь утопить прошлое в выпивке;
    \item[--] Вздрагивает от любого шороха, мучимый \textbf{Фобией};
    \item[--] Боится взглянуть в зеркало изза жутких уродливых \textbf{Шрамов};
    \item[--] Верит в свою неуязвимость, заработав \textbf{Безбашенность}.
\end{itemize}

\end{enumerate}

\paragraph{Взаимопомощь.} Герои могут помогать друг другу, если контекст и логика ситуации это до-пускают. В бою правила взаимопомощи работают иначе (смотрите маневр "Финт" и правило "Все на одного").
\begin{enumerate}
    \item Выберите героя, который совершает основную проверку.
    \item Определите тех, кто ему помогает. 
    \item Определите сложность основной проверки и понизтье ее на 5 для помощников. 
    \item Помощники совершают свои проверки. Каждый успех помощника дает 1 Преимущество герою, совершающему основную проверку, а каждый провал - Помеху. Не забывайте, что герой не может иметь более 2 Преимуществ или 2 Помех на бросок.
    \item Герой, выбранный в первом пункте, совершает проверку с причитающимися Преимуществами или Помехами. 
\end{enumerate}

\subsection{Когда бросать кубик?}
Не бросайте К20, если герой занят рутинными делами, не ограничен во времени и ресурсах и при этом имеет хотя бы 1 Очко опыта в Навыке. Не бросайте К20, если успех или неудача не имеют значимых последствий.
\paragraph{}Герою точно не понадобится проверка, если он:
\begin{itemize}
\item[--]готовит скромный ужин;
\item[--]разводит костер сухим теплым вечером;
\item[--]выполняет рутинный уход за автомобилем;
\item[--]мирно выпивает в кабаке;
\item[--]стирает свою одежду.
\end{itemize}

Бросайте К20, если герой рискует чем-то важным - репутацией, богатством, жизнями друзей (или своей собственной). Бросайте К20, если герой ограничен во времени и ресурсах.
\paragraph{}Но герою обязательно придется совершить бросок, если он:
\begin{itemize}
\item[--]творит кулинарный шедевр из гнилой тушенки, проросшей картошки и вялой моркови;
\item[--]разводит костер из отсыревшего хвороста под противным моросящим дождиком;
\item[--]латает двигатель, имея под рукой лишь спички, желуди и немного медной проволоки;
\item[--]пьянствует напропалую накануне собственной свадьбы;
\item[--]очищает одежду от засохшей крови и радиоактивной пыли.
\end{itemize}
\begin{tcolorbox}
Проверка всегда подразумевает шанс успеха или провала. Если все в игровой команде считают задачу слишком простой или слишком сложной для героя - просто опишите исход.
\newline Какие проверки потребуется сделать герою и нужны ли они, вам подскажут мастер, соигроки и контекст Сцены.
\end{tcolorbox}

\subsection{Состязания}
Обычно для определения успеха или неудачи действия достаточно бросить К20 - все факторы уже включены в бросок. Однако иногда мастер может решить добавить в ситуацию остроты. В этом случае:
\begin{itemize}
    \item Участники Состязания совершают проверки со всеми сопутствующими бонусами и штрафами. 
    \item Результаты сравниваются между собой, а не с абстрактной сложностью проверки. Участник Состязания с наибольшим результатом достигает цели или мешает другому участнику достичь своей. 
    \newline Если проверка имеет градации успеха, определите степень успеха победителя относительно результата проигравшего. Если противники получают равные результаты, то ни одна из сторон не может взять верх, и ситуация остается такой же, как и до броска. 
    \item При использовании Успеха с Расплатой участник получает минимально необходимый успех (то есть побеждает на 1). Если оба противника применили Успех с Расплатой, побеждает тот, кто выберет больше Расплат. При одинаковом количестве выбранных Расплат объявляется ничья, никто не достигает своей цели. В любом случае, в игру входят ВСЕ выбранные противниками Расплаты, в том порядке, который подсказывают логика и контекст ситуации. 
\end{itemize}
\begin{tcolorbox}
Авторский коллектив рекомендует использовать Состязание, только если герою противостоит персона или другой герой.
\end{tcolorbox}

\subsection{Испытания}
Иногда для успеха недостаточно одной проверки, а отдельный провал не обязательно приводит к немедленному ухудшению ситуации. В таких случаях героям предстоит выдержать Испытание - форму комплексной проверки. Так же Испытание поможет упорядочить Динамические сцены, не являющиеся Боевыми.
\newline При Испытании определите:
\begin{itemize}
\item[--]Цель Испытания - чего герои достигнут, преуспев;
\item[--]Цену провала - что произойдет, если герои не справятся;
\item[--]Количество успехов, необходимое для достижения цели - Счетчик победы;
\item[--]Количество провалов, достаточное для неудачи - Счетчик поражения;
\item[--]Всех участников Испытания;
\item[--]Проверки, которые будут совершать участники, и соответствующие им типы Действий.
\end{itemize}
В Испытании вправе участвовать только герои и персоны. Статисты могут пассивно давать бонусы и штрафы к их проверкам. Герои, которые не участвуют в Испытании, могут совершать проверки по правилам Взаимопомощи.
\newline Структура Испытания схожа с Боевой сценой, хоть и не ограничена столь жесткими рамками (при Испытании у героев может быть достаточно времени, чтобы починить автомобиль или произнести продолжительную речь). Герои выполняют действия в порядке установленной очереди и совершают проверки, логически вытекающие из конекста ситуации и описания игроков.
\newline Действия героев в Испытании делятся на 3 типа:
\begin{itemize}
    \item[]\textbf{Допускающее.} Не влияет на Счетчики напрямую, но позволяет другому герою совершить свое действие. Обычно для допуска других героев к Развивающему действию достаточно одного успеха Допускающего, если контекст не располагает к иному. Герой выполняет такое действие, когда:
    \begin{itemize}
        \item[--] Сдерживает натиск упырей в рукопашной;
        \item[--] Прикрывает команду огнем;
        \item[--] Не дает опуститься защитному экрану;
        \item[--] Блокирует гидравлику мусорного пресса;
        \item[--] Отвлекает охранника непринужденной беседой.
    \end{itemize}
\item[]\textbf{Развивающее.} Это действие приближает благополучное завершение Испытания. Его успех повышает Счетчик победы на 1, а провал - повышает Счетчик поражения на 1. Герой выполняет такое действие, когда:
\begin{itemize}
    \item[--] Взламывает системы защиты древнего механизма одну за другой;
    \item[--] Прореживает наступающий отряд беглым огнем;
    \item[--] Заливает из огнетушителя очаги возгорания;
    \item[--] Поэтапно готовит к запуску стратегические ядерные ракеты;
    \item[--] Пережимает крупные артерии, не давая пациенту истечь кровью.
    \end{itemize}
\item[]\textbf{Сберегающее.} Такое действие помогает исправить последствия неудач. Его успех понижает Счетчик поражения на 1, а провал - понижает Счетчик победы на 1. Герой выполняет такое действие, когда:
\begin{itemize}
    \item[--] Подхватывает выскользнувшую из руки деталь;
    \item[--] Вовремя замечает обходной маневр противника;
    \item[--] Отвлекает врагов на себя;
    \item[--] Крепко держит страховочную веревку сорвавшегося товарища;
    \item[--] Вовремя останавливает союзника, вводящего неверный пароль.
\end{itemize}

\begin{tcolorbox}
    Развивающее и Сберегающее действие не обязательно требует предваряющего ему Допускающего действия.\tbd
\end{tcolorbox}
\begin{tcolorbox}
Сберегающее действие возможно, только если ошибка совершена и может быть исправлена (т.е. Счетчик поражения возрос на 1 или больше) - либо когда герой заявил маневр Выжидания и внимательно следит за совершающим Развивающее действие соратником.
\end{tcolorbox}
\end{itemize}

\paragraph{КУ и КП в Испытаниях:}
\begin{itemize}
    \item[--] КУ на Допускающее действие позволяет другому герою совершить 1 дополнительное Развивающее или Сберегающее действие;
    \item[--] КУ на Развивающее действие повышает Счетчик победы на дополнительный 1;
    \item[--] КУ на Сберегающее действие понижает Счетчик поражения на дополнительный 1;
    \item[--] КП на Допускающее действие понижает Счетчик поражения на 1;
    \item[--] КП на Развивающее действие повышает Счетчик поражения на дополнительный 1;
    \item[--] КП на Сберегающее действие понижает Счетчик победы на дополнительный 1.
\end{itemize}

\paragraph{Время поджимает:} Испытание требует от героев не только эффективности, но и расторопно-сти. В конце каждого Круга Испытания Счетчик поражения возрастает на 1.
\paragraph{Завершение Испытания:} Испытание продолжается, пока Счетчик победы или поражения не достигнет заданных в начале Испытания значений. Как только одно из двух значений будет достигнуто, Испытание заканчивается, даже если Круг не завершен и еще не все герои выполнили свои действия в нем.
