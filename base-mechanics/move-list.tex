\section{Перечень Ходов}
Ходы, описанные ниже, доступны любому герою или персоне:
\begin{enumerate}
\item \textbf{Повезло!}
\begin{itemize}
\item[--] 1 Нить. Откажитесь от последствий Каприза Судьбы. Если вы отказываетесь от последствий Каприза, то к вашему герою не протягивается Нить.
\item[--] 1 Нить. Получите наилучший вариант при проверке Неприятностей. Оборвите Нить до броска кубика и считайте, что при проверке выпало 20.
\item[--] 1 Нить. Перебросьте проверку 1 раз. Вы не обязаны принимать второй бросок, если он хуже первого. Любой кубик может быть переброшен не больше 2 раз. Перебрасывая проверки с Помехами или Преимуществами, перебросьте все кубики.
\item[--] 1 Нить. Сохраните жизнь персоне или статисту, провалившему проверку Выносливости при Опасной ране или Смерти.
\item[--] 2 Нити. Получите успех на проверку героя. Оборвите Нити до того, как брошен кубик. Если бросок подразумевает различные эффекты в зависимости от степени успеха, герой получает минимально необходимый успех. Вы не можете использовать этот Ход, если герой достигает успеха только при выпадении 20 на кубике.
Так же вы можете провалить проверку героя. Вы можете использовать Ход, даже если герой достигает успеха автоматически (скажем, из-за высокого Навыка и отсутствия риска). Если бросок подразумевает разные эффекты в зависимости от степени провала, герой получает минимально необходимый провал.
\item[--] 4 Нити. Получите Критический успех на проверку героя. Для тех эффектов, которые это учитывают, считается, что на кубике выпало 20. Вы можете использовать этот Ход, даже если герой достигает успеха только при выпадении 20 на кубике, и в этом случае он получает все дополнительные эффекты Критического успеха.
Точно так же вы можете Критически провалить проверку героя. Вы можете использовать Ход, даже если герой достигает успеха автоматически (например, из-за высокого Навыка и отсутствия рисков). Для тех эффектов, которые это учитывают, считается, что на кубике выпало 1.
\end{itemize}

\item \textbf{В нужном месте в нужное время.}
\begin{itemize}
\item[--] 1 или больше Нитей (на усмотрение мастера). Добавьте в сцену предмет, или элемент обстановки, или статиста. Ржавый нож, спрятанный в тюремном матрасе, незаметная дверь в тупике, охранник, прибежавший на крики о помощи, — приемлемые варианты. Мастер вправе заблокировать Ход или потребовать обрыва большего числа Нитей, если вы предлагаете нечто маловероятное или не соответствующее жанру и настроению игры. Игроки могут использовать этот Ход, чтобы помочь героям других игроков.
\item[--] 1 или больше Нитей (на усмотрение мастера). Сделайте игровую заявку ретроспективно. Игроки могут использовать этот Ход, чтобы помочь героям других игроков. Чем дальше во времени отстоит возможность реализации такой заявки, тем больше Нитей придется оборвать. Одно дело, когда герой еще вчера заходил в магазин, где мог приобрести все необходимое, и совсем другое, когда герой несколько месяцев странствует в пустоши, где не растет ничего, кроме жухлой травы!
\item[--] 1 или больше Нитей (на усмотрение мастера). Введите в игру Орла вашего героя. Подробнее об Орле читайте в разделе «Грани и Амплуа».
\end{itemize}
\item \textbf{Старый знакомый.}
\begin{itemize}
\item[--] 1 или больше Нитей (на усмотрение мастера). Добавьте в сцену \textit{симпатизирующего герою} статиста — родственника, друга, должника и т. п. Не забывайте, статист не может появиться из ниоткуда! Мастер вправе заблокировать Ход или потребовать обрыва большего числа Нитей, если вы предлагаете нечто маловероятное или не соответствующее жанру и настроению игры.
\item[--] 1 или больше Нитей (на усмотрение мастера). Добавьте факт биографии уже присутствующего в игре статиста — Недостаток, грязный секрет, черту характера или событие из прошлого. Вам потребуется объяснить, откуда герой узнал об этом.
\item[--] 1 Нить. Введите в игру известный вам Недостаток персоны или статиста. Герой, чья Нить оборвана, обязан принимать участие в сцене (и посильно содействовать вводу Недостатка в игру). Имейте в виду, что в этом случае к персоне будет протянута 1 Нить.
\end{itemize}
\item \textbf{Рука Судьбы.}
\begin{itemize}
\item[--] 1 Нить. Примените Каприз Судьбы на герое другого игрока.
\item[--] 1 Нить. Герой немедленно восстанавливает 10 Единиц Маны. ЕМ героя не могут превышать максимальной величины его ЕМ. Совершение Хода в Боевой сцене требует Быстрого действия.
\item[--] От 1 до 3 Нитей (или больше на усмотрение мастера). Статист по выбору игрока, оборвавшего Нити своего героя, попадает в Неприятности. 1 оборванная Нить приведет к результату «Ну и денек!», 2 оборванных Нити — к результату «Кажется, у меня проблема!», 3 оборванных Нити — к результату «Катастрофа!». По взаимной договоренности игроки могут оборвать Нити нескольких разных героев, чтобы достичь нужного результата. Герои не обязаны присутствовать в одной сцене со статистами, попадающими в Неприятности. Мастер вправе заблокировать Ход или потребовать обрыва большего числа Нитей, если игроки предлагают нечто маловероятное или не соответствующее жанру и настроению игры.
\end{itemize}
\item \textbf{Ход Атрибута.} Каждый Атрибут позволяет сделать герою уникальный Ход, который недоступен остальным. Стоимость может отличаться в зависимости от атрибута, так же герой может совершать Ход Атрибута без обрыва Нитей. В этом случае на все проверки, которые должен совершить герой для совершения Хода \textit{нельзя подействовать} другими Ходами и Капризами.
\end{enumerate}
