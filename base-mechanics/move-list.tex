\section{Перечень Ходов}
Ходы, описанные ниже, доступны любому герою или персоне:
\begin{enumerate}
\item \textbf{Повезло!}
\begin{itemize}
\item[--] 1 Нить. Откажитесь от последствий Каприза Судьбы. Если вы отказываетесь от последствий Каприза, то к вашему герою не протягивается Нить.
\item[--] 1 Нить. Получите наилучший вариант при проверке Неприятностей. Оборвите Нить до броска кубика и считайте, что при проверке выпало 20.
\item[--] 1 Нить. Перебросьте проверку 1 раз. Вы не обязаны принимать второй бросок, если он хуже первого. Любой кубик может быть переброшен не больше 2 раз. Перебрасывая проверки с Помехами или Преимуществами, перебросьте все кубики.
\begin{tcolorbox}
Переброс выглядит хорошей идеей, но не всегда является таковой. Если вам необходим успех героя, лучше оборвать Нити и позволить ему преуспеть. Авторский коллектив рекомендует использовать перебросы во время атакующих Маневров и прочих проверок, имеющих градации успеха, либо когда герой находится под действием Преимуществ.
\end{tcolorbox}
\item[--] 1 Нить. Сохраните жизнь персоне, провалившей проверку Выносливости при Опасной ране или смерти, либо статисту, получившему Опасную рану или умершему. Спасенный таким образом статист (или персона) прерывает участие в Сцене (обычно – теряя сознание или страдая от ран).
\item[--] 2 Нити. Получите успех на проверку героя. Оборвите Нити до того, как брошен кубик. Если бросок подразумевает различные эффекты в зависимости от степени успеха, герой получает минимально необходимый успех. Вы не можете использовать этот Ход, если герой достигает успеха только при выпадении 20 на кубике.
\newline Так же вы можете провалить проверку героя. Вы можете использовать Ход, даже если герой достигает успеха автоматически (например, из-за высокого Навыка и отсутствия рисков). Если бросок подразумевает различные эффекты в зависимости от степени провала, герой получает минимально необходимый провал. 
\item[--] 4 Нити. Получите Критический успех на проверку героя. Для тех эффектов, которые это учитывают, считается, что на кубике выпало 20. Вы можете использовать этот Ход, даже если герой достигает успеха только при выпадении 20 на кубике, и в этом случае он получает все дополнительные эффекты Критического успеха.
\newline Так же вы можете Критически провалить проверку героя. Вы можете использовать Ход, даже если герой достигает успеха автоматически (например, из-за высокого Навыка и отсутствия рисков). Для тех эффектов, которые это учитывают, считается, что на кубике выпало 1.
\end{itemize}
\begin{tcolorbox}
Зачем нужны автоматические провалы проверок? Действительно, такое применение Нитей кажется не самым очевидным. Но вы не только герой, вы его Судьба. Пока герой живет полной событиями жизнью, вы строите и развиваете сюжет. Наверняка герой будет счастлив завершить хлопотную приключенческую карьеру, обзавестись фермой, семьей и тихим безопасным хобби. Но хотите ли этого вы? Если у вас еще есть планы на героя, не давайте ему стать слишком благополучным.
\newline Помимо этого, игровые правила периодически вынуждают героя совершать действия, которые кажутся нежелательными игроку. Изменить их сюжетную направленность и эмоциональную окраску также помогут автоматические провалы проверок.
\end{tcolorbox}
\item \textbf{В нужном месте в нужное время.}
\begin{itemize}
\item[--] 1 или больше Нитей (на усмотрение мастера). Добавьте в сцену предмет, или элемент обстановки, или статиста. Ржавый нож, спрятанный в тюремном матрасе, незаметная дверь в тупике, охранник, прибежавший на крики о помощи, — приемлемые варианты.
\item[--] 1 или больше Нитей (на усмотрение мастера). Сделайте игровую заявку ретроспективно. Чем дальше во времени отстоит возможность реализации такой заявки, тем больше Нитей придется оборвать. Одно дело, когда герой еще вчера заходил в магазин, где мог приобрести все необходимое, и совсем другое, когда герой несколько месяцев странствует в пустоши, где не растет ничего, кроме жухлой травы!
\item[--] 1 или больше Нитей (на усмотрение мастера). Введите в игру Орла вашего героя. Подробнее об Орле читайте в разделе <<Грани и Амплуа>>.
\item[--] 1 Нить. Ваш герой появляется в текущей Сцене, если он в данный момент не задействован в другой. Это позволит ему принять участие в драке, отпустить едкий комментарий или стать свидетелем события, вместо того, чтобы простаивать где-то вне рамок повествования. Мастер вправе заблокировать Ход, если никто не может объяснить, как герой попал в Сцену, или это противоречит ее контексту.
\end{itemize}
\begin{tcolorbox}
Игроки могут обрывать Нити своих героев, чтобы помочь героям \textit{других} игроков получить эффекты Хода <<В нужном месте в нужное время>>.
\end{tcolorbox}
\item \textbf{Старый знакомый.}
\begin{itemize}
\item[--] 1 или больше Нитей (на усмотрение мастера). Добавьте в сцену \textit{симпатизирующего герою} статиста — родственника, друга, должника и т. п. Не забывайте, статист не может появиться из ниоткуда!
\item[--] 1 или больше Нитей (на усмотрение мастера). Добавьте факт биографии уже присутствующего в игре статиста — Недостаток, грязный секрет, черту характера или событие из прошлого. Вам потребуется объяснить, откуда герой узнал об этом, либо сослаться на контекст и детали экспозиции Сцены.
\item[--] 1 Нить. Введите в игру известный герою Недостаток/Темную сторону Атрибута/Решку персоны или статиста. Герой, Нить которого оборвана, обязан принимать участие в Сцене и посильно содействовать вводу Недостатка/Темной стороны Атрибута/Решки в игру, либо к этому должны располагать контекст и детали экспозиции Сцены. Имейте в виду, что в этом случае к персоне будет протянута 1 Нить.
\end{itemize}
\item \textbf{Рука Судьбы.}
\begin{itemize}
\item[--] 1 Нить. Герой немедленно восстанавливает \textbf{|2 + Модификатор обаяния|} Энергии. Энергия героя не может превышать максимальной величины его Характеристики. Совершение Хода в Боевой сцене требует Быстрого действия.
\item[--] От 1 до 3 Нитей (или больше на усмотрение мастера). Статист по выбору игрока, оборвавшего Нити своего героя, попадает в Неприятности. 1 оборванная Нить приведет к результату <<Ну и денек!>>, 2 оборванных Нити – к результату <<Кажется, у меня проблема!>>, 3 оборванных Нити – к результату <<Катастрофа!>>. По взаимной договоренности игроки могут оборвать Нити нескольких разных героев, чтобы достичь нужного результата. Герои не обязаны присутствовать в одной сцене со статистами, попадающими в Неприятности.
\end{itemize}
\begin{tcolorbox}
Если герои завершают игровую встречу с неиспользованными Нитями, Рука Судьбы – отличная возможность подпортить денек антагонистам.
\end{tcolorbox}
\item \textbf{Единственный и неповторимый.}
\newline Каждый Атрибут позволяет герою сделать Уникальный ход, который недоступен героям и статистам без этого Атрибута. Стоимость может отличаться в зависимости от описания конкретного Хода.
\newline Герой может совершать Ход Атрибута и без обрыва Нитей. В этом случае на все проверки, которые должен совершить герой для успеха Хода, нельзя повлиять другими Ходами и Капризами (но Трюки, Функции и особые способности Атрибутов разрешены, если в их описании не указано обратного).
\item \textbf{Сделка с Судьбой.} Герой может передать любое количество Нитей другому герою, однако за каждую переданную Нить к Мастеру протягивается Темная Нить.
\item \textbf{Дежавю.}
\newline 2 Нити за каждого героя(но не статиста или персону) в сцене. Если игрокам не по нраву результат действий героев, они можгут объявить о применении Дежавю и вернуть их в начало Сцены.
\newline Это решение может быть принято только коллективно - если хоть один игрок считает, что события развиваются благоприятно, интересно или просто выгодно для героев, то он может заблокировать использование Хода.
\newline Этот ход может быть оплачен коллективно, то есть любые герои, учавствующие в сцене могут оборвать Нити для того чтобы добавить их в оплату хода.
\newline После применения Хода герои начинают Сцену заново со всеми доступными им на тот момент ресурсами, (такими, как патроны, медикаменты, Энергия, Единицы здоровья и доверие окружающих) кроме потраченных Нитей. Разумеется, герои не подозревают о вмешательстве Судьбы в свои дела, всего лишь испытывают мучительно ускользающее чувство воспоминаний о будущем.
\end{enumerate}
