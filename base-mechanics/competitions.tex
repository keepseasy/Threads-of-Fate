\section{Состязания}
Обычно для определения успеха или неудачи действия героя или статиста достаточно бросить кубик – все положительные и отрицательные факторы включены в бросок. Однако иногда мастер может решить добавить в ситуацию остроты. В этом случае оба участника Состязания совершают проверки Навыков или Характеристик со всеми уместными бонусами и штрафами, но суммы сравнивают не с заранее определенной сложностью проверки, а результаты проверок друг друга между собой. Участник Состязания с наибольшим результатом достигает своей цели. Если проверка имеет градации успеха, определите степень успеха победителя относительно результата проигравшего.
\newline Если в Состязании противники получают равные результаты, то ни одна из сторон не может взять верх, и ситуация остается такой же, как и до броска.
\newline При использовании в Состязании Успеха с неприятностями участник получает минимально необходимый успех (то есть побеждает на 1). Если оба противника применили Успех с неприятностями, побеждает тот, кто выберет больше Неприятностей. Однако в игру входят ВСЕ выбранные противниками Неприятности!
\begin{tcolorbox}
Рекомендуется использовать Состязание, только если герою противостоит персона или другой герой.
\end{tcolorbox}