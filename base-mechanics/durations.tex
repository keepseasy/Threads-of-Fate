\section{Течение времени}
Во время приключений герои будут получать различные эффекты, продолжительность которых может растягиваться на несколько сцен. Наиболее частые продолжительности следующие:
\begin{itemize}
\item[--] Одна или несколько \textbf{Очередей} в Боевой Сцене. Это самая короткая продолжительность.
\item[--] Один или несколько \textbf{Кругов} в Боевой Сцене. Началом отсчета для этих эффектов является начало Очереди героя или статиста, который его применил.
\item[--] Одна или несколько \textbf{Сцен}. Действие большинства Феноменов, Трюков и Ходов ограничено именно этим событийным промежутком. После чего их предстоит активировать повторно.
\item[--] До следующей \textbf{Интерлюдии}. Эффекты, от которых можно избавиться, переведя дыхание и немного отдохнув. Если герой находится под эффектом, который длиться определенное количество Сцен, Интерлюдия считается за одну Сцену.
\item[--] До следующего \textbf{Антракта}. Только полноценный отдых может избавить героя от этого эффекта. Антракт завершает действие любых эффектов, если в описании эффекта не указано обратного.
\item[--] Специальные условия. Некоторые эффекты длятся, пока не будет выполнено определенное действие или не произойдет определенное событие. В описании эффекта должно быть указано, что его действие не прерывается Антрактом.
\end{itemize}
\begin{tcolorbox}
Эффекты, длящиеся несколько Очередей и Кругов имеют значение только в Боевых Сценах. В других сценах они настолько мимолетны, что не имеют значительного влияния на ход событий.
\end{tcolorbox}