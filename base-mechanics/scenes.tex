\section{Cцены}
Сценой называется эпизод с участием героев. Из множества таких эпизодов и состоит игра. Разговор с информатором, погоня, обыск павших в битве врагов — все это сцены. Действие многих способностей героев ограничено длительностью сцены.
Игровая сцена состоит из следующих этапов:
\paragraph{}
\begin{enumerate}

 \item Мастер описывает наполнение сцены — декорации и события вокруг героев.
\begin{itemize}
\item[--] Что за народ ходит около места встречи с информатором? Привлекло ли внимание появление героев?
\item[--] Героев преследуют по земле или по воздуху? Стараются ли их поймать или пытаются убить во время погони?
\item[--] Все павшие мертвы, или среди них есть те, кого еще можно спасти? Не помешает ли кто-то героям собирать трофеи?
\end{itemize}
При создании наполнения сцены мастер может принимать решения, основанные на импровизации и контексте ситуации, или определить некоторые детали при помощи игромеханических инструментов — проверок Неприятностей, Впечатлений и ввода в игру Недостатков героев. Подробнее об этом читайте в соответствующих разделах книги.
\linebreak
Описание сцены вовсе не должно быть литературным, хотя хороший слог мастера, безусловно, выгодно скажется на атмосфере игры. Главное, донести до игроков информацию, которую они смогут использовать при описании действий своих героев.
 \item Игроки задают уточняющие вопросы (если это требуется) и описывают действия своих героев.
\begin{itemize}
\item[--] Герои предложат информатору деньги за нужные им сведения или попытаются добиться своего при помощи угроз?
\item[--] Герои спрячутся от преследователей или заманят их в ловушку?
\item[--] Герои помогут раненому врагу бескорыстно или сохранят ему жизнь в надежде на выкуп или помощь?
\end{itemize}
На этом этапе игроки не только заявляют, что делают их герои, но и решают, вмешается ли в события Судьба. Они могут воспользоваться Нитями Судьбы и совершить Ходы Судьбы, повлияв на наполнение сцены и ее контекст. Подробнее об этом читайте в разделе «Нити, Ходы и Капризы».
\item Мастер объявляет, какие проверки должны совершить герои (и должны ли вообще), определяет их сложность и описывает последствия действий героев.
\begin{itemize}
\item[--] Информатор охотно принимает деньги и делится тем, что знает, или зовет подмогу?
\item[--]  Погоня отстала потеряв след или же застряла в ловушке?
\item[--]  Раненый обещает отплатить добром за добро или втайне готовится к мести?
\end{itemize}
Далеко не каждое действие и решение героев требует проверок. Определение ее принципиальной необходимости — одна из обязанностей мастера. Подробнее о проверках читайте в разделе «Проверки».
\linebreak \textbf{На этом этапе игроки также могут применять Ходы Судьбы, в том числе связанные с проверками.} Подробнее об этом читайте в разделе «Нити, Ходы и Капризы».
\end{enumerate}
Новая сцена начинается, когда предыдущая так или иначе получает логическое завершение: герои узнали  то, что хотели (или бежали под градом ударов), погоня уничтожена (или остался далеко позади), а поле битвы обследовано вдоль и поперек (или герои поглощены хлопотами с раненым). Если герои по неким причинам разделились, каждый из них станет участником отдельной сцены.