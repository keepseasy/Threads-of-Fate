\section{Cцены}
Сценой называется эпизод с участием героев. Из множества таких эпизодов и состоит игра. Разговор с информатором, погоня, обыск павших в битве врагов - все это сцены. Действие многих способностей героев ограничено длительностью сцены.
Игровая сцена состоит из следующих этапов:
\paragraph{}
\begin{enumerate}

 \item Мастер описывает наполнение сцены - декорации и события вокруг героев.
\begin{itemize}
\item[--] Что за народ ходит около места встречи с информатором? Привлекло ли внимание появление героев?
\item[--] Героев преследуют по земле или по воздуху? Стараются ли их поймать или пытаются убить во время погони?
\item[--] Все павшие мертвы, или среди них есть те, кого еще можно спасти? Не помешает ли кто-то героям собирать трофеи?
\end{itemize}
При создании наполнения сцены мастер может принимать решения, основанные на импровизации и контексте ситуации, или определить некоторые детали при помощи игромеханических инструментов - проверок Неприятностей, Впечатлений и ввода в игру Недостатков героев. Подробнее об этом читайте в соответствующих разделах книги.
\linebreak
Описание сцены вовсе не должно быть литературным, хотя хороший слог мастера, безусловно, выгодно скажется на атмосфере игры. Главное, донести до игроков информацию, которую они смогут использовать при описании действий своих героев.
 \item Игроки задают уточняющие вопросы (если это требуется) и описывают действия своих героев.
\begin{itemize}
\item[--] Герои предложат информатору деньги за нужные им сведения или попытаются добиться своего при помощи угроз?
\item[--] Герои спрячутся от преследователей или заманят их в ловушку?
\item[--] Герои помогут раненому врагу бескорыстно или сохранят ему жизнь в надежде на выкуп или помощь?
\end{itemize}
На этом этапе игроки не только заявляют, что делают их герои, но и решают, вмешается ли в события Судьба. Они могут воспользоваться Нитями Судьбы и совершить Ходы Судьбы, повлияв на наполнение сцены и ее контекст. Подробнее об этом читайте в разделе "Нити, Ходы и Капризы".
\item Мастер объявляет, какие проверки должны совершить герои (и должны ли вообще), определяет их сложность и описывает последствия действий героев.
\begin{itemize}
\item[--] Информатор охотно принимает деньги и делится тем, что знает, или зовет подмогу?
\item[--]  Погоня отстала потеряв след или же застряла в ловушке?
\item[--]  Раненый обещает отплатить добром за добро или втайне готовится к мести?
\end{itemize}
Далеко не каждое действие и решение героев требует проверок. Определение ее принципиальной необходимости - одна из обязанностей мастера. Подробнее о проверках читайте в разделе "Проверки".
\linebreak \textbf{На этом этапе игроки также могут применять Ходы Судьбы, в том числе связанные с проверками.} Подробнее об этом читайте в разделе "Нити, Ходы и Капризы".
\end{enumerate}
Новая сцена начинается, когда предыдущая так или иначе получает логическое завершение: герои узнали  то, что хотели (или бежали под градом ударов), погоня уничтожена (или остался далеко позади), а поле битвы обследовано вдоль и поперек (или герои поглощены хлопотами с раненым). Если герои по неким причинам разделились, каждый из них станет участником отдельной сцены.

\section{Разновидности сцен}
Для простоты управления повествованием сцены разделены на несколько разновидностей. Каждая из них имеет свой темп и особенности. 
\paragraph{Динамическая сцена.} Это наиболее распространенный вид сцен. Ситуация вокруг героев постоянно меняется, реагировать на эти изменения приходится незамедлительно, а время поджимает. В Динамические сценах герои активно противодействуют недругам, обстоятельствам и прочим враждебным силам. 
\paragraph{Боевая сцена.} Это подвид Динамической, изображающий вооруженное противостояние героев и статистов. В боевых сценах очередность действий четко структурирована, а их перечень ограничен. Боевая сцена начинается, как только кто-то из героев или статистов предпринимает направленные атакующие действия. Сцена продолжается до тех пор, пока в ней есть герои и статисты, способные и желающие атаковать друг друга.
\paragraph{Интерлюдия.} Это короткая передышка между Динамическими сценами. В интерлюдиях отсутствует непосредственная угроза для героев (хотя она может быть создана их действиями или Капризами Судьбы, ознаменовав начало Динамической сцены). Во время Интерлюдии герои получают шанс подлечиться, поболтать, привести себя в порядок, вздремнуть или иначе устроить свой досуг. 
\paragraph{Антракт.} Это 
длительный перерыв в активном повествовании. Герои в безопасности, им не грозит цейтнот. Антракт дает возможность основательно поправить здоровье, заняться ремонтом снаряжения, ремеслами и прочей рутиной. О таких делах не очень интересно рассказывать и слушать, но без них не обойдется ни одна история.

\begin{tcolorbox}
В некоторых ситуациях, например, когда герою требуется длительное восстановление, имеет значение длительность Антрактов и Интерлюдий. В этом случае считайте, что один Антракт занимает 12 часов, а Интерлюдия - не менее 1 часа внутриигрового времени. 
\end{tcolorbox}

\section{Течение времени}
Во время приключений герои будут получать различные эффекты, продолжительность которых может растягиваться на несколько сцен. Наиболее частые продолжительности следующие:
\begin{itemize}
\item[--] Одна или несколько \textbf{Очередей} в Боевой Сцене. Это самая короткая продолжительность.
\begin{tcolorbox}
Началом отсчета является начало Очереди героя или статиста, который активировал эффект. Обыч-но эти эффекты имеют значение только в Боевых сценах. В других Сценах они настолько мимолет-ны, что почти не оказывают влияния на события.
\end{tcolorbox}
\item[--] Один или несколько \textbf{Кругов} в Боевой Сцене. Началом отсчета для этих эффектов является начало Очереди героя или статиста, который его применил.
\item[--] Одна или несколько \textbf{Сцен}. Действие большинства Феноменов, Трюков и Ходов ограничено именно этим событийным промежутком. После чего их предстоит активировать повторно.
\item[--] До следующей \textbf{Интерлюдии}. Эффекты, от которых можно избавиться, переведя дыхание и немного отдохнув. Если герой находится под эффектом, который длиться определенное количество Сцен, Интерлюдия считается за одну Сцену.
\item[--] До следующего \textbf{Антракта}. Только полноценный отдых может избавить героя от этого эффекта. Антракт завершает действие любых эффектов, если в описании эффекта не указано обратного.
\item[--] Специальные условия. Некоторые эффекты длятся, пока не будет выполнено определенное действие или не произойдет определенное событие. В описании эффекта должно быть указано, что его действие не прерывается Антрактом.
\end{itemize}
