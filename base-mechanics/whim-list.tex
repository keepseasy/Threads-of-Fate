\section{Перечень капризов}
С одной стороны, Капризы Судьбы — это возможность для игроков обмениваться Нитями, с другой — элемент передачи повествовательных прав, с третьей — непостоянство своенравной Судьбы!
\begin{enumerate}
\item \textbf{Во власти страстей.}
Недостаток героя входит в игру. Ввод в игру Недостатков — основной способ получения Нитей Судьбы героем. Подробный перечень возможных Недостатков вы найдете в разделе <<Недостатки>>.
\begin{tcolorbox}
Обратите внимание: если герой-Пьяница тихо-мирно напился, это никак не осложняет его жизнь и ничего не привносит в историю. Но, если Пьяница напился и избил любимого сына местного шерифа, Недостаток, безусловно, вошел в игру и оставил в ней след. Если Любвеобильный герой потискал смазливую официантку в занюханной забегаловке, это вряд ли потянет на приключение. А вот если его застанут в постели с дочерью главаря преступного синдиката, Недостаток сработал, как надо.
\end{tcolorbox}
\item \textbf{Все имеет цену.}
Темная сторона Атрибута героя входит в игру. В отличие от Недостатков, проявления Темной стороны достаточно ситуативны, а потому целиком и полностью отданы фантазии игроков и мастера. Не забывайте, что ввод в игру Темной стороны (так же, как и Недостатка) должен осложнять герою жизнь и создавать возможности для развития сюжета. Примеры Темных сторон каждого из Атрибутов вы найдете в разделе <<Атрибуты>>.
\item \textbf{Решка!}
Решка героя входит в игру. Так же, как и в случае с Темной стороной Атрибутов, Решка — очень специфический способ получения Нитей и всецело зависит от контекста сцены. Подробнее о Решке читайте в разделе <<Грани и Амплуа>>.
\item \textbf{Катастрофа!}
Игрок добровольно сталкивает героя с наихудшим вариантом проверки Неприятностей. Считайте, что на кубике выпало 1. Подробнее об этом читайте в разделе <<Неприятности>>.
\item \textbf{Черная полоса.}
Теоретически, герои могут начать игру без Атрибутов, Недостатков и Граней. Вряд ли герой так уж идеален, но ни одна из страстей не способна захватить его целиком даже на мгновение. И все же он остается игрушкой в руках Судьбы.
\newline Во время игры постоянно возникают ситуации, так или иначе подвергающие героя опасности. Игрок может предложить неблагоприятный вариант развития такой ситуации, например:
\begin{enumerate}
\item[--]Герой неслышно (проверка Скрытности успешно пройдена) подкрадывается к охраннику, но старый настил под ногами героя скрипит и обнаруживает его присутствие. Начинается Боевая сцена.
\item[--]Герой желает заправить грузовик, и контекст Сцены этому не противоречит. Внезапно выясняется, что все топливо еще утром забрали военные. Придется бросить тачку с хабаром и топать по пустошам пешком.
\item[--]На привале герой достает свою флягу (заблаговременно наполненную у колодца) и обнаруживает, что вся вода вытекла через незаметную трещинку. Кажется, впору помирать от жажды или напиться из того подозрительного озерка.
\item[--]Изнывая от жары, герой врубает автомобильный кондиционер на полную, расходует ценную топливо и вскоре обзаводится неприятнейшим насморком. До выздоровления герой получает Недостаток <<Шумный>>.
\item[--]Герой собирается развести костер в лесу, но весь сухостой вымочил вчерашний ливень. Придется лечь спать на холоде, без огня и ужина.
\end{enumerate}
\end{enumerate}
\begin{tcolorbox}
Вход в игру Каприза Судьбы значит, что в истории возник некий свершившийся факт. Но вход Недостатка, Темной стороны или Решки в игру не обязательно означает безусловные проблемы.
\newline Прежде всего, это создание интересной игровой ситуации, которая начинается неблагоприятно для героя, но может принести выгоды позднее, если Судьба в лице игрока подскажет герою верную линию поведения.
\end{tcolorbox}
