\section{Капризы судьбы}
\paragraph{Каприз Судьбы (Каприз) -} это ввод в игру Недостатка, Темной стороны Атрибута, Решки героя или принятие игроком худшего варианта Неприятностей до броска кубика.
\paragraph{Вход Каприза в игру.} Каждый раз, когда Каприз входит в игру, протяните к герою 1 Нить. Вход в игру означает создание ситуации, в которой Каприз осложняет жизнь героя. Игрок описывает затруднение, с которым столкнулся его герой, и, если мастер считает проблему достаточно серьезной, к герою протягивается 1 Нить.
\newline В течение Сцены мастер может (хоть и не обязан делать этого) ввести Капризы Судьбы по одному разу для каждого героя, принимающего участие в Сцене.
\newline Инициация проверки Неприятностей не считается Капризом, а вот попытка навязать герою ее худший вариант является им. Таким образом, навязанный худший вариант Общих неприятностей, исчерпает базовый лимит Капризов за Сцену.
\newline В течение Сцены мастер может (хоть и не обязан делать этого) ввести Капризы Судьбы по одному разу для каждого героя, принимающего участие в Сцене. Мастер не обрывает для этого Нити персон и не тратит Темные Нити. 
\newline Инициация проверки Неприятностей не считается Капризом, а вот попытка навязать герою ее худший вариант является им. Таким образом, навязанный худший вариант Общих неприятностей, исчерпает весь базовый лимит бесплатных Капризов за Сцену.
\newline Мастер так же располагает дополнительными возможностями для ввода Капризов более одного раза за Сцену для одного и того же героя, а именно: 
\begin{itemize}
\item[--]Нити персон-антагонистов. Обрывая их, мастер может применять Ходы из категории «Старый знакомый» в отношении героев. В дополнение к обычным правилам Мастер не вправе навязать герою новый Недостаток без согласия игрока.
\item[--]Темные Нити. Мастер может обрывать их для ввода в игру Капризов чаще одного раза за Сцену для одного героя. Мастер все еще не вправе ввести один и тот же Недостаток, Темную сторону Атрибута и Решку героя более одного раза за Сцену.
\end{itemize}
\begin{tcolorbox}
Предложить, как и описать Каприз Судьбы может любой из игроков или мастер. Не забывайте, игра – совместное творчество! Однако предложенные Капризы не являются обязательными для ввода в игру - окончательное решение принимают мастер и игрок, герой которого станет жертвой Каприза Судьбы.
\newline Мастеру рекомендуется четко разграничить случаи, когда он предлагает Каприз от случаев, когда он их уже вводит в игру.
\end{tcolorbox}
\paragraph{Отказ от Каприза.} Если игрок не желает принимать последствия Каприза, он должен совершить Ход Судьбы "Повезло!". Если Каприз входит в игру, а у героя нет ни одной Нити, ему придется столкнуться с последствиями, хочет он того или нет. Разумеется, в этом случае к герою протягивается Нить.
\begin{tcolorbox}
Не забывайте, что если игрок применил Ход «Повезло!» и отказался от последствий, то Каприз не входит в игру – к герою не протягивается Нить.
\end{tcolorbox}

\section{Перечень капризов}
С одной стороны, Капризы Судьбы — это возможность для игроков обмениваться Нитями, с другой — элемент передачи повествовательных прав, с третьей — непостоянство своенравной Судьбы!
\begin{enumerate}
\item \textbf{Во власти страстей.}
Недостаток героя входит в игру. Ввод в игру Недостатков — основной способ получения Нитей Судьбы героем. Подробный перечень возможных Недостатков вы найдете в разделе "Недостатки".
\begin{tcolorbox}
Обратите внимание: если герой-Пьяница тихо-мирно напился, это никак не осложняет его жизнь и ничего не привносит в историю. Но, если Пьяница напился и избил любимого сына местного шерифа, Недостаток, безусловно, вошел в игру и оставил в ней след. Если Любвеобильный герой потискал смазливую официантку в занюханной забегаловке, это вряд ли потянет на приключение. А вот если его застанут в постели с дочерью главаря преступного синдиката, Недостаток сработал, как надо.
\end{tcolorbox}
\item \textbf{Все имеет цену.}
Темная сторона Атрибута героя входит в игру. В отличие от Недостатков, проявления Темной стороны достаточно ситуативны, а потому целиком и полностью отданы фантазии игроков и мастера. Не забывайте, что ввод в игру Темной стороны (так же, как и Недостатка) должен осложнять герою жизнь и создавать возможности для развития сюжета. Примеры Темных сторон каждого из Атрибутов вы найдете в разделе "Атрибуты".
\item \textbf{Решка!}
Решка героя входит в игру. Так же, как и в случае с Темной стороной Атрибутов, Решка — очень специфический способ получения Нитей и всецело зависит от контекста сцены. Подробнее о Решке читайте в разделе "Грани и Амплуа".
\item \textbf{Катастрофа!}
Игрок добровольно сталкивает героя с наихудшим вариантом проверки Неприятностей. Считайте, что на кубике выпало 1. Подробнее об этом читайте в разделе "Неприятности".
\item \textbf{Черная полоса.}
Теоретически, герои могут начать игру без Атрибутов, Недостатков и Граней. Вряд ли герой так уж идеален, но ни одна из страстей не способна захватить его целиком даже на мгновение. И все же он остается игрушкой в руках Судьбы.
\newline Во время игры постоянно возникают ситуации, так или иначе подвергающие героя опасности. Игрок может предложить неблагоприятный вариант развития такой ситуации, например:
\begin{enumerate}
\item[--]Герой неслышно (проверка Скрытности успешно пройдена) подкрадывается к охраннику, но старый настил под ногами героя скрипит и обнаруживает его присутствие. Начинается Боевая сцена.
\item[--]Герой желает заправить грузовик, и контекст Сцены этому не противоречит. Внезапно выясняется, что все топливо еще утром забрали военные. Придется бросить тачку с хабаром и топать по пустошам пешком.
\item[--]На привале герой достает свою флягу (заблаговременно наполненную у колодца) и обнаруживает, что вся вода вытекла через незаметную трещинку. Кажется, впору помирать от жажды или напиться из того подозрительного озерка.
\item[--]Изнывая от жары, герой врубает автомобильный кондиционер на полную, расходует ценную топливо и вскоре обзаводится неприятнейшим насморком. До выздоровления герой получает Недостаток "Шумный".
\item[--]Герой собирается развести костер в лесу, но весь сухостой вымочил вчерашний ливень. Придется лечь спать на холоде, без огня и ужина.
\end{enumerate}
\end{enumerate}
\begin{tcolorbox}
Вход в игру Каприза Судьбы значит, что в истории возник некий свершившийся факт. Но вход Недостатка, Темной стороны или Решки в игру не обязательно означает безусловные проблемы.
\newline Прежде всего, это создание интересной игровой ситуации, которая начинается неблагоприятно для героя, но может принести выгоды позднее, если Судьба в лице игрока подскажет герою верную линию поведения.
\end{tcolorbox}
