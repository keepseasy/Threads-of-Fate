\section{Капризы судьбы}
\paragraph{Каприз Судьбы (Каприз) -} это ввод в игру Недостатка, Темной стороны Атрибута, Решки героя или принятие игроком худшего варианта Неприятностей до броска кубика.
\paragraph{Вход Каприза в игру.} Каждый раз, когда Каприз входит в игру, протяните к герою 1 Нить. Вход в игру означает создание ситуации, в которой Каприз осложняет жизнь героя. Игрок описывает затруднение, с которым столкнулся его герой, и, если мастер считает проблему достаточно серьезной, к герою протягивается 1 Нить.
\newline В течение Сцены мастер может (хоть и не обязан делать этого) ввести Капризы Судьбы по одному разу для каждого героя, принимающего участие в Сцене.
\newline Инициация проверки Неприятностей не считается Капризом, а вот попытка навязать герою ее худший вариант является им. Таким образом, навязанный худший вариант Общих неприятностей, исчерпает базовый лимит Капризов за Сцену.
\newline В течение Сцены мастер может (хоть и не обязан делать этого) ввести Капризы Судьбы по одному разу для каждого героя, принимающего участие в Сцене. Мастер не обрывает для этого Нити персон и не тратит Темные Нити. 
\newline Инициация проверки Неприятностей не считается Капризом, а вот попытка навязать герою ее худший вариант является им. Таким образом, навязанный худший вариант Общих неприятностей, исчерпает весь базовый лимит бесплатных Капризов за Сцену.
\newline Мастер так же располагает дополнительными возможностями для ввода Капризов более одного раза за Сцену для одного и того же героя, а именно: 
\begin{itemize}
\item[--]Нити персон-антагонистов. Обрывая их, мастер может применять Ходы из категории «Старый знакомый» в отношении героев. В дополнение к обычным правилам Мастер не вправе навязать герою новый Недостаток без согласия игрока.
\item[--]Темные Нити. Мастер может обрывать их для ввода в игру Капризов чаще одного раза за Сцену для одного героя. Мастер все еще не вправе ввести один и тот же Недостаток, Темную сторону Атрибута и Решку героя более одного раза за Сцену.
\end{itemize}
\begin{tcolorbox}
Предложить, как и описать Каприз Судьбы может любой из игроков или мастер. Не забывайте, игра – совместное творчество! Однако предложенные Капризы не являются обязательными для ввода в игру - окончательное решение принимают мастер и игрок, герой которого станет жертвой Каприза Судьбы.
\newline Мастеру рекомендуется четко разграничить случаи, когда он предлагает Каприз от случаев, когда он их уже вводит в игру.
\end{tcolorbox}
\paragraph{Отказ от Каприза.} Если игрок не желает принимать последствия Каприза, он должен совершить Ход Судьбы <<Повезло!>>. Если Каприз входит в игру, а у героя нет ни одной Нити, ему придется столкнуться с последствиями, хочет он того или нет. Разумеется, в этом случае к герою протягивается Нить.
\begin{tcolorbox}
Не забывайте, что если игрок применил Ход «Повезло!» и отказался от последствий, то Каприз не входит в игру – к герою не протягивается Нить.
\end{tcolorbox}