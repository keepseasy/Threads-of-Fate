\section{Капризы судьбы}
\paragraph{Каприз Судьбы (Каприз):} ввод в игру Недостатка, Темной стороны Атрибута, Решки героя или принятие игроком худшего варианта Неприятностей до броска кубика называется Капризом Судьбы.
\paragraph{Вход Каприза в игру:} каждый раз, когда Каприз входит в игру, протяните к герою 1 Нить. Вход в игру означает создание ситуации, в которой Каприз осложняет жизнь героя. Игрок описывает затруднение, с которым столкнулся его герой, и, если мастер считает проблему достаточно серьезной, к герою протягивается 1 Нить.
\paragraph{Капризы Судьбы и мастер:} в течение сцены мастер может использовать Капризы Судьбы один раз для каждого героя, принимающего участие в сцене (хоть и не обязан делать этого). Мастер не должен обрывать для этого Нити персон или тратить какие-то иные ресурсы. Инициация проверки Неприятностей не считается Капризом, а вот попытка навязать герою ее худший вариант является им!


\paragraph{Капризы Судьбы и игрок:} игрок так же может предложить Каприз Судьбы для своего персонажа и если Мастер сочтет его уместным, получить за его ввод в игру 1 Нить


\paragraph{Отказ от Каприза:} если игрок не желает принимать последствия Каприза, он должен совершить Ход Судьбы «Повезло!». Если у героя нет Нитей, чтобы откупиться от Каприза, ему придется столкнуться с последствиями, хочет он того или нет. Разумеется, в этом случае к герою протягивается Нить. Не забывайте, что если игрок применил Ход «Повезло!» и отказался от последствий, то Каприз не входит в игру и к герою не протягивается Нить.