\section{Нити, Ходы и Капризы}
Как бы ни был ловок, умен и могуч герой, именно благоволение Судьбы выдвигает его на ведущие роли. Нити Судьбы отображают невероятное везение героев — то,
что заставляет обывателей пересказывать истории об удивительных приключениях героев и их невероятных подвигах. 
\begin{tcolorbox}
    Нити Судьбы – важнейший элемент игры. Помимо прочего, Нити позволяют игроку объявить о безусловном успехе героя без каких-либо проверок. Благодаря этому абсолютно любой герой легко окажется в центре внимания и повлияет на развитие сюжета.
\end{tcolorbox}

\paragraph{Начальное и максимальное число Нитей.}
\begin{itemize}
    \item[--] Герои начинают игровую встречу с 2 Нитями;
    \item[--] Одновременно к герою или персоне не может быть протянуто больше 5 Нитей;
    \item[--] Все неиспользованные к концу встречи Нити исчезают, и следующую встречу герой начнет с 2 Нитями.
\end{itemize}
Если Судьба в лице мастера и игроков сочтет нужным, новые Нити могут протягиваться к героям и персонам не в начале игровых встреч, а по завершении важных сюжетных вех или даже перед ними. Например, Нити протянутся к героям накануне генерального сражения с вражеской армией или после того, как битва, так или иначе, завершится. В этом случае любой герой, к которому протянуто меньше 2 Нитей, увеличивает их число до 2. Герои, к которым протянуто больше 2 Нитей, сохраняют их. Заметьте, что к персонам-антагонистам также протянутся новые Нити.
\paragraph{Очки опыта и Нити Судьбы.} Очки опыта могут быть использованы героями как для развития, так и для получения благосклонности Судьбы. Выберите один из этих вариантов в начале игровой встречи:
\begin{enumerate}
    \item В начале игровой встречи игрок может протянуть к своему герою до 3 дополнительных Нитей, потратив до 3 Очков опыта (1 Нить за 1 Очко опыта), помимо начальных 2 Нитей.
    \item Игрок может протянуть к своему герою до 5 Нитей, потратив до 5 Очков опыта в Антрактах и Интерлюдиях (1 Нить за 1 Очко опыта). К нему все еще не может быть одновременно протянуто больше 5 Нитей Судьбы.
    \item Непотраченные Очки опыта у героев могут использоваться как Нити Судьбы в любой момент. Обратите внимание, что это сделает героев невероятно могущественными, так как у них могут быть и Очки опыта, и максимальное количество Нитей.
\end{enumerate}

%---------------------------------------------------------------------------------------------------------------------------------------------------
%---------------------------------------------------------------------------------------------------------------------------------------------------
%---------------------------------------------------------------------------------------------------------------------------------------------------
%---------------------------------------------------------------------------------------------------------------------------------------------------
%---------------------------------------------------------------------------------------------------------------------------------------------------

%\paragraph{Статисты, Персоны и Нити.}
%статисты не могут получать Нити ни каким образом.\tbd
%персоны в начале игровой встречи получают 2 Нити, а неиспользованные ранее Нити сгорают.\tbd
%еще они как-то Нити получают.\tbd
%Если статисты или персоны противостоит героям, они могут использовать не только собственные Нити(если они есть), но и Темные Нити, находящиеся в распоряжении Мастера.\tbd
%\paragraph{Ходы и помощь персонам и статистам:} игроки могут обрывать Нити своих героев, чтобы помочь важным для них статистам. При этом для них доступны Ходы из категорий "Повезло" и "В нужном месте в нужное время".

%---------------------------------------------------------------------------------------------------------------------------------------------------
%---------------------------------------------------------------------------------------------------------------------------------------------------
%---------------------------------------------------------------------------------------------------------------------------------------------------
%---------------------------------------------------------------------------------------------------------------------------------------------------
%---------------------------------------------------------------------------------------------------------------------------------------------------

\paragraph{Когда Нити протягиваются и обрываются?}
Игрок \textit{протягивает} к своему герою:
\begin{itemize}
    \item[--] 2 Нити в начале каждой игровой встречи;
    \item[--] 1 дополнительную Нить за каждые Узы, которыми связан герой, в начале каждой игровой встречи;
    \item[--] 1 Нить, когда принимает Каприз Судьбы;
    \item[--] Нити в обмен на Очки опыта, по предварительной договоренности с мастером.
\end{itemize}
Игрок \textit{обрывает} 1 и более Нитей своего героя, когда делает Ход Судьбы.


Мастер \textit{протягивает} к персонам:
\begin{itemize}
    \item[--] 2 Нити в начале каждой игровой встречи;
    \item[--] 1 дополнительную Нить за каждые Узы, которыми связана персона, в начале каждой игровой встречи.
    \item[--] 1 Нить, когда персона принимает Каприз Судьбы.
\end{itemize}
Мастер \textit{обрывает}:
\begin{itemize}
    \item[--] 1 и более Нитей персоны, когда она делает Ход Судьбы;
    \item[--] 1 и более Темных Нитей, когда статист или персона делает Ход Судьбы, но им не хватает собственных Нитей для его совершения;
    \item[--] 1 Темную Нить, когда вводит дополнительный Каприз Судьбы в сцене на героя.
\end{itemize}

\begin{tcolorbox}
Изменения, внесенные в игру с помощью Ходов и Капризов, становятся игровым фактом и частью истории. Они могут (и обязательно будут) иметь далеко идущие последствия.
\end{tcolorbox}

\subsection{Темные Нити.} 
Порой поступки героев явно противоречат логике повествования, внутренней мотивации и даже самой сути мироздания. Но герои на то и герои, чтобы бросать вызовы и принимать их. Конечно же, Судьба не терпит подобных дерзостей. Даже если возмездие не настигнет героев сразу, рано или поздно они поплатятся.
\newline Темные Нити – это Нити в полном распоряжении мастера. Каждый раз, когда мастер считает, что герои выходят за рамки выбранных Уз или как-то иначе перегибают палку, он сообщает об этом игрокам. Если их объяснение логики поступков героев не удовлетворят мастера, он протягивает к себе 1 Темную Нить. 
\begin{tcolorbox}
    "Герои перегибают палку" вовсе не означает "герои делают что-то жестокое, бессмысленное, ужасное или глупое (даже ужасно глупое)". Недовольство Судьбы вызывает последовательное нежелание героя следовать логике жанра (о том, какова эта логика в представлении собравшихся за игровым столом, лучше договориться заранее). Также Судьба может прогневаться на героя, поступки которого лишены какой-либо логики в принципе. Авторский коллектив рекомендует обозначить все подобные моменты до начала игры.
\end{tcolorbox}
\paragraph{Начальное и максимальное число Темных Нитей.}
В начале игровой встречи к мастеру протягивается 1 Темная Нить, если число игроков не превышает 3 или 2 Темных Нити, если игроков 4 и больше.
\newline В распоряжении мастера одновременно может находиться число Темных Нитей, не более  \textbf{|1 + число игроков, присутствующих на игровой встрече|}.

\paragraph{Применение Темных Нитей:}
\begin{itemize}
    \item[--] Когда статист или персона противостоят героям, мастер может использовать Темные Нити так же, как свои собственные.
    \item[--] Также мастер может в любое время обрывать Темные Нити для ввода в игру разнообразных Капризов Судьбы более одного раза за Сцену для одного и того же героя.
    \begin{tcolorbox}
        Когда мастеру необходимо осложнить героям жизнь, не прибегая к Недостаткам, то "Катастрофа" и "Черная полоса" - идеальный выбор.
    \end{tcolorbox}
    Например, в распоряжении мастера 2 Темных Нити. Один раз за Сцену он сталкивает героя с последствиями Недостатка бесплатно, один раз вводит в Сцену Темную сторону его Атрибута (1 Темная Нить) и один раз вводит в Сцену Катастрофу (1 Темная Нить). Мастер не может ввести в Сцену уже навязанный герою Недостаток второй раз – даже за Темную Нить.
\end{itemize}
Если мастер достиг лимита Темных Нитей, герои могут противоречить Узам и обмениваться Нитями, не опасаясь гнева Судьбы. Это значит, что с последствиями сомнительных решений не стоит тянуть. Герои сами напросились, они поплатятся!
\newline Не забывайте, что технически персональная "Катастрофа!" и общая "Катастрофа!" - это один и тот же Каприз. Впрочем, ничто не мешает мастеру устроить персональную Катастрофу для каждого героя в Сцене, если число Темных Нитей позволяет это.
\newline Темные Нити мастера, по сути, прибавляются к Нитям, протянутым к персонам-антагонистам. Если антагонист накопил 5 собственных Нитей, а у мастера есть еще 5 Темных Нитей, фактически, персона располагает 10 (!) Нитями. Героям лучше отложить встречу с таким противником на какое-то время – или положиться на Судьбу.

\subsection{Ходы Судьбы}
\paragraph{Ход Судьбы (Ход)} — прямое вмешательство высшей силы в жизнь героя. Некоторые Ходы доступны лишь героям, имеющим определенные Атрибуты — это  \textbf{Уникальные ходы.} Подробнее об этом читайте в разделе "Атрибуты".
\paragraph{Совершение Хода:} каждый раз, когда Судьба делает Ход, обрывается число Нитей героя, указанное в описании Хода. Обрывая Нити, игроки вправе добавлять в Сцены факты - статистов, детали их биографии, элементы обстановки и многое другое. Игрок может обрывать любое число Нитей своего героя, комбинируя любое число Ходов (если в их описании не указано обратного).
\paragraph{Право вето:} мастер вправе заблокировать Ход или потребовать обрыва большего числа Нитей, если игрок предлагает нечто маловероятное или не соответствующее логике повествования и настроению игры.
\paragraph{Статисты, персоны и Ходы.} Статистам доступны ходы категории "Единственный и неповторимый". Персонам доступны только Ходы категорий "Повезло", "В нужном месте в нужное время", "Старый знакомый" (включая второй Ход категории "Старый знакомый", если игроки согласны) и "Единственный и неповторимый".

\subsection{Перечень Ходов}
Ходы, описанные ниже, доступны любому герою. 
\begin{enumerate}
    \item \textbf{Повезло.}
    \begin{enumerate}
        \item 1 Нить. Откажитесь от последствий Каприза Судьбы. К вашему герою не протягивается Нить.
        \item 1 Нить. Считайте, что при проверке Неприятностей вашего героя выпало 20. Оборвите Нить до броска.
        \item 1 Нить. Перебросьте проверку своего героя 1 раз. Вы не обязаны принимать второй бросок, если он хуже первого.
        \begin{tcolorbox}
            Переброс выглядит хорошей идеей, но не всегда является таковой. Если вам необходим успех героя, лучше сразу оборвать 2 Нити и позволить ему преуспеть. Авторский коллектив рекомендует использовать перебросы во время проверок, имеющих градации успеха, либо когда герой находится под действием Преимуществ.
        \end{tcolorbox}
        \item 1 Нить. Сохраните жизнь персоне или статисту, которые должны умереть (по любым причи-нам), но имеют осязаемые шансы выжить благодаря стечению обстоятельств. Спасенный благода-ря этому Ходу находится При смерти.
        \newline Также вы можете сохранить жизнь статисту или персоне, который должен умереть без всяких проверок, но имеет осязаемые шансы выжить благодаря стечению обстоятельств.
        \item 2 Нити. Получите успех на проверку своего героя до того, как брошен К20. Если бросок подразумевает эффекты, зависящие от степени успеха, герой получает минимально необходимый успех. Ход недоступен, если герой достигает успеха только при выпадении 20, или не вправе фактически совершать проверку.
        \newline Так же вы можете провалить проверку своего героя, даже если он достигает успеха автоматически. Если бросок подразумевает эффекты, зависящие от степени провала, герой получает минимально необходимый провал.
        \item 4 Нити. Получите Критический успех на проверку своего героя. Для всех эффектов считается, что на К20 выпало 20. Вы вправе использовать Ход, даже если герой не может фактически совершать проверку или достигает успеха только при выпадении 20.
        \newline Так же вы можете Критически провалить проверку своего героя, даже если он достигает успеха автоматически. Для всех эффектов считается, что на К20 выпало 1.
    \end{enumerate}

    \begin{tcolorbox}
        Зачем нужны автоматические провалы проверок? Действительно, такое применение Нитей кажется не самым очевидным. Но вы не только герой, вы его Судьба. Пока герой живет жизнью, полной событий, вы строите и развиваете сюжет. Наверняка герой будет счастлив завершить хлопотную приключенческую карьеру, обзавестись фермой, семьей и тихим безопасным хобби. Но хотите ли этого вы? Если у вас еще есть планы на героя, не давайте ему стать слишком благополучным.
        \newline Помимо этого, игровые правила периодически вынуждают героя совершать действия, которые кажутся нежелательными игроку. Изменить их сюжетную направленность и эмоциональную окраску помогут автоматические провалы проверок.
    \end{tcolorbox}

    \item \textbf{В нужном месте в нужное время.}
    \begin{enumerate}
        \item 1+ Нитей. Добавьте в Сцену факт - предмет/элемент обстановки/статиста. Ржавый нож в тюремном матрасе, родник в пустошах, караванщики, ответившие на просьбу о помощи, – приемлемые варианты. 
        \item 1+ Нитей. Сделайте игровую заявку ретроспективно. Например, вы забыли сказать мастеру, что герой несет с собой запас факелов. Возможно, вы просто-напросто решили, что герою не пригодятся факелы. Теперь, если герой отправится исследовать лабиринты древнего убежища, вам придется оборвать 1 Нить, чтобы герой все же взял с собой связку факелов и канистру с горючей смесью.
        \newline Чем дальше во времени отстоит возможность реализации заявки, тем больше Нитей придется оборвать. Одно дело, если герой еще вчера заходил в поселение, где мог приобрести все необходимое, и совсем другое, - когда он месяц кряду торчит в пустошах, где нет ничего, кроме жухлой травы, фонящих руин и стремных чудищ.
        \item 1+ Нитей. Введите в игру Орла вашего героя. Подробнее об Орле читайте в разделе "Грани и Амплуа".
        \item 1 Нить. Ваш герой появляется в текущей Сцене, если уже не задействован в другой. Это позволит герою поучаствовать в драке, отпустить едкий комментарий в споре или стать свидетелем события, вместо того, чтобы простаивать где-то вне рамок повествования.
    \end{enumerate}
    \begin{tcolorbox}
        Игроки могут обрывать Нити своих героев, чтобы помочь героям других игроков получить эффекты Хода "В нужном месте в нужное время". Согласие другого игрока, разумеется, требуется.
    \end{tcolorbox}

    \item \textbf{Старый знакомый.}
    \begin{enumerate}
        \item 1+ Нитей. Добавьте в Сцену симпатизирующего герою статиста – родственника, друга, должника. Не забывайте, статист не может появиться из ниоткуда!
        \item 1+ Нитей. Добавьте факт биографии уже присутствующего в игре статиста – Недостаток, Атрибут, Узы, Грань, Трюк или событие из прошлого. Вам потребуется объяснить, откуда герой узнал об этом, либо сослаться на контекст и детали экспозиции Сцены.
        \item 1 Нить. Введите в игру известный герою Недостаток/Темную сторону Атрибута/Решку персоны или статиста. Герой, Нить которого оборвана, обязан принимать участие в Сцене и посильно содействовать вводу Недостатка/Темной стороны Атрибута/Решки в игру, либо к этому должны располагать контекст и детали экспозиции Сцены.
        \newline Имейте в виду, что в этом случае к персоне будет протянута 1 Нить.
        \item 1 Нить. Целевой статист преодолевает свои Узы до завершения Сцены. Герой, Нить которого оборвана, обязан принимать участие в Сцене и посильно содействовать преодолению Уз, либо к этому должны располагать контекст и детали экспозиции Сцены.
    \end{enumerate}

    \item \textbf{Рука Судьбы.}
	\begin{enumerate}
        \item 1 Нить. Герой может совершить действие, противоречащее его Узам. В этом случае мастер не получает Темную Нить.
        \item 1 Нить. Герой немедленно восстанавливает \textbf{|2 + МОб|} (минимум 1) Энергии. Энергия героя не может превышать максимальной величины Характеристики. Совершение Хода в Боевой сцене расходует Быстрое действие.
        \item 1-3 Нити (или больше на усмотрение мастера). Статист по выбору игрока, оборвавшего Нити своего героя, попадает в Неприятности. 1 оборванная Нить приведет к результату "Затруднение", 2 оборванных Нити – к результату "Проблема", 3 оборванных Нити – к результату "Катастрофа!". Игроки могут оборвать Нити нескольких разных героев, чтобы достичь нужного результата. Герои не обязаны присутствовать в одной сцене со статистами, попадающими в Неприятности.
        \item X Нитей. Оборвав X Нитей своего героя, игрок может совершить Ход Судьбы для персоны или статиста, даже не находящегося в одной Сцене с его героем.
    \end{enumerate}
    \begin{tcolorbox}
        Если герои завершают игровую встречу с неиспользованными Нитями, Рука Судьбы – отличная возможность подпортить денек антагонистам.
    \end{tcolorbox}

    \item \textbf{Единственный и неповторимый.}
    \newline 1+ Нитей. Каждый Атрибут позволяет герою сделать Уникальный ход, который недоступен героям и статистам без этого Атрибута.  
	Герой может применять Ход Атрибута и без обрыва Нитей. В этом случае на все проверки, которые должен совершить герой для успеха Хода, нельзя повлиять другими Ходами и Капризами.

	\item \textbf{Сделка с Судьбой.}
	\begin{enumerate}
        \item \textbf{|X|} Нитей. Оборвав \textbf{|X|} Нитей своего героя, игрок может протянуть \textbf{|X|} Нитей к герою другого игрока. К мастеру протягивается \textbf{|X|} Темных Нитей.
    \end{enumerate}

	\item \textbf{Дежавю.}
	\newline X Нитей. Если игрокам не по нраву результат действий героев, они вправе объявить о применении Дежавю и вернуть героев в начало Сцены. 
    \newline Игроки обрывают X Нитей героев, где X = числу героев, принимавших участие в Сцене. Могут использоваться в любых сочетаниях Нити всех героев (но не персон!) в Сцене.
    \newline Герои начинают Сцену заново, со всеми доступными им на тот момент ресурсами, такими как патроны, медикаменты, Энергия, Единицы Здоровья и доверие окружающих (кроме, собственно, Нитей). Герои не подозревают о вмешательстве Судьбы в свои дела, всего лишь испытывают мучительно ускользающее чувство воспоминаний о будущем.
    \newline Обязательно обсудите использование Дежавю с остальными игроками. Если они считают, что события развиваются интересно или просто выгодно для их героев, то вправе заблокировать использование Хода. 
\end{enumerate}

\subsection{Капризы судьбы}
\paragraph{Каприз Судьбы (Каприз)} – ввод в игру Недостатка, Темной стороны Атрибута, Решки героя или принятие игроком худшего варианта Неприятностей до броска кубика. 
\paragraph{Вход Каприза в игру} означает создание ситуации, в которой Каприз осложняет жизнь героя или персоны. Опишите затруднение, с которым столкнулся персонаж, и, если мастер считает проблему серьезной, к герою протягивается 1 Нить. 
\newline Мастер не вправе ввести один и тот же Каприз для одного и того же героя более одного раза за Сцену. Например, если игрок откупился Нитями от Пьянства и Любвеобилия, то эти Недостатки не обеспокоят его героя до конца текущей сцены.

\paragraph{Ввод Капризов мастером:} в течение одной Сцены мастер может (хоть и не обязан) бесплатно ввести Каприз 1 раз для каждого героя в Сцене. Мастер не обрывает для этого Нити персон и не тратит Темные Нити.
\newline Инициация проверки Неприятностей не считается Капризом, а вот попытка навязать герою/героям ее худший вариант является им. Таким образом, навязанный худший вариант Общих неприятностей исчерпает весь лимит бесплатных Капризов мастера за Сцену.
\newline Однако мастер располагает возможностями для ввода различных Капризов более одного раза за Сцену для одного и того же героя, а именно: 
\begin{enumerate}
    \item Нити персон-антагонистов. Обрывая их, мастер может применять Ходы из категории "Старый знакомый" в отношении героев. Мастер не вправе навязать герою новый Недостаток без согласия игрока.
    \item Темные Нити. Мастер может обрывать их для ввода в игру Капризов чаще одного раза за Сцену для одного героя. Подробнее читайте об этом в разделе "Темные Нити".
    \item Разнообразные способности статистов, принуждающие героя к действиям.
\end{enumerate}
\paragraph{Отказ от Каприза.} Если игрок не желает принимать последствия Каприза, он должен сделать Ход Судьбы "Повезло!". Если Каприз входит в игру, а у героя нет Нитей, ему придется столкнуться с последствиями, хочет игрок того или нет. Разумеется, в этом случае к герою протягивается 1 Нить. 
\begin{tcolorbox}
	Не забывайте, что если игрок применил Ход "Повезло!" и отказался от последствий, то Каприз не входит в игру – к герою не протягивается Нить. 
\end{tcolorbox}

\paragraph{Ввод Каприза игроком:} игрок вправе вводить Капризы для своего героя любое число раз за сцену, если мастер считает возникшее затруднение существенным.
\newline Также любой игрок впрве вводить Капризы для персоны или статиста, при помощи общедоступных Ходов "Старый знакомый" и "Рука Судьбы". Мастер решает, примет ли Персона Каприз, или применит Ход "Повезло!". Статист может отказаться от Каприза только при помощи Темных нитей мастера.

\subsection{Перечень Капризов}
С одной стороны, Капризы Судьбы – это возможность для игроков получить игромеханический ресурс, с другой – элемент передачи повествовательных прав, с третьей – непостоянство своенравной Судьбы!

\begin{enumerate}
    \item \textbf{Во власти страстей.}
    \newline Недостаток героя входит в игру. Ввод в игру Недостатков – основной способ получения Нитей Судьбы героем. Подробный перечень возможных Недостатков вы найдете в разделе "Недостатки". 
    \begin{tcolorbox}
        Обратите внимание: если герой-Пьяница тихо-мирно напился, это никак не осложняет его жизнь и ничего не привносит в историю. Но, если Пьяница напился и избил любимого сына окружного шерифа, Недостаток, безусловно, вошел в игру и оставил в ней след. Если Любвеобильный герой потискал смазливую официантку в занюханной забегаловке, это вряд ли потянет на приключение. А вот если охотники на монстров застанут его в постели с молоденькой мутанточкой, Недостаток сработал, как надо.
    \end{tcolorbox}

    \item \textbf{Все имеет цену.}
    \newline Темная сторона Атрибута или Решка героя входит в игру. Проявления Темной стороны и Решки достаточно ситуативные, а потому целиком и полностью отданы фантазии игроков и мастера. Не забывайте, что ввод в игру Темной стороны или Решки (так же, как и Недостатка) должен создавать для героя трудности и развивать сюжет. Например, Док может столкнуться с нехваткой медикаментов, Гражданин Убежища почувствует себя неуютно в становище дикарей, а за Красавицу дадут хорошую цену на невольничьем рынке.
    \newline Примеры Темных сторон Атрибутов вы найдете в разделе "Атрибуты". Подробнее о Решке читайте в разделе "Грани и Амплуа". 

    \item \textbf{Катастрофа!}
    \newline Герой сталкивается с худшим вариантом проверки Неприятностей. Считайте, что на К20 выпало 1. Подробнее об этом – в разделе "Неприятности".
    \newline "Катастрофа!" для героев – как, правило, невозможность получить что-то в нужный момент, необходимость распылять силы, выбор из двух зол, упущенные шансы и т.д. Например:
    \begin{itemize}
        \item Герой лишен доступа к чему-то необходимому здесь и сейчас – патронам, топливу, еде, лекарствам, поддержки статистов-союзников, либо доступ сопряжен со значительными расходами времени, сил или средств. Поэтому герой:
        \begin{itemize}
	        \item[--] Отдаст драгоценный артефакт за пачку жаропонижающего;
	        \item[--] Выполнит трудную и опасную работу ради заправки машины;
	        \item[--] Будет орудовать топором там, где хватило бы пары очередей из автомата;
	        \item[--] Пойдет на кражу или грабеж ради канистры мутной воды и черствой краюхи хлеба;
	        \item[--] Отправится в одиночку на встречу, где не помешали бы верные товарищи за спиной.
        \end{itemize}
        \item Герой вынужден выбирать, кому из близких он поможет, а кого оставит наедине с проблемой, либо попадает в ситуацию, когда помощь одному автоматически означает угрозу для другого. Герой вынужден:
        \begin{itemize}
	        \item[--] Оставить тяжелораненого друга в пустошах, пообещав вернуться позже;
	        \item[--] Примкнуть к одному из враждующих лагерей, имея близких в обоих из них;
	        \item[--] Доверить присмотр за детьми полуслепой хромой бабушке;
	        \item[--] Решать, кому из больных товарищей достанется единственная порция лекарства;
	        \item[--] Защищать свой город, зная, что семья в смертельной опасности.
        \end{itemize}
        \item Герой опаздывает к началу важного события, и возможности его влияния на развитие серьезно снижаются, либо ведут к значительному расходу времени, сил и средств. Герой упустит шанс:
        \begin{itemize}
	        \item[--] Уладить кофликт без кровопролития;
	        \item[--] Составить непредвзятое мнение о случившемся;
	        \item[--] Выбрать сторону конфликта по своему усмотрению;
	        \item[--] Помочь какой-либо из сторон конфликта;
	        \item[--] Сохранить нейтралитет.
        \end{itemize}
        \item Герой попадает в опасную или затруднительную ситуацию там, где ничто не предвещало проблем. Не повезло, бывает, но теперь герою придется:
        \begin{itemize}
	        \item[--] Искать надежное бомбоубежище в деревенском захолустье;
	        \item[--] Пережидать чудовищную бурю в хижине отшельника;
	        \item[--] Отступать с боем из крепости, считавшейся неприступной;
	        \item[--] Сражаться, едва выскочив из постели;
	        \item[--] Бросить машину в непролазной грязи.
        \end{itemize}
        \item Некое событие, имевшее место в прошлом героя, получает сугубо отрицательную трактовку. В настоящем герой сталкивается с его негативными последствиями, и узнает, что:
        \begin{itemize}
	        \item[--] Его любимая работает на злодея;
	        \item[--] Близкий друг легко предал его;
	        \item[--] Нуждается в редком и дорогом лекарстве;
	        \item[--] Недавний успех был отвлекающим маневром врага;
	        \item[--] Встал на сторону плохих парней.
        \end{itemize}
    \end{itemize}

    \item \textbf{Чуждое влияние.}
    \newline Герой попадает под власть способности, вынуждающей его изменить поведение против воли игрока. Этот Каприз включает в себя воздействие Навыков Общения и Выступления, Уникального Хода Проповедника, Коммивояжера или Красивого статиста, некоторых феноменов и прочих подобных им. Влияющий на героя статист все еще должен преуспеть в соответствующих проверках.
    \newline Чуждое влияние не применяется, если источником способности является другой герой.
    \begin{tcolorbox}
        Способности, вынуждающие героя трепыхаться в мощном захвате, истекать кровью от выстрела в печень, или валяться без сознания и пары-тройки зубов после сокрушительного апперкота, являются не Чуждым влиянием, а прямыми последствиями действий героя и решений игрока.
    \end{tcolorbox}

    \item \textbf{Черная полоса.}
    \newline Герой может начать игру без Атрибутов, Уз и Недостатков. Вряд ли он идеален, но ни одна страсть не захватывает его целиком. И все же он остается игрушкой в руках Судьбы. 
    \newline Во время игры постоянно возникают ситуации, так или иначе подвергающие героя опасности. Игрок или мастер могут предложить неблагоприятный вариант развития такой ситуации. Например:
    \begin{itemize}
        \item[--] Герой неслышно (проверка Скрытности успешно пройдена) подкрадывается к охраннику, но старый настил под ногами героя скрипит и обнаруживает присутствие. Начинается Боевая сцена.
        \item[--] Герой желает заправить грузовик, но хозяин заправки сообщает, что все топливо еще утром национализировали военные. Придется бросить тачку и топать пешком.
        \item[--] На привале герой достает флягу (заблаговременно наполненную у колодца) и обнаруживает, что вся вода вытекла через незаметную трещинку. Кажется, впору помирать от жажды или прихлебнуть из того подозрительного озерка.
        \item[--] Изнывая от жары, герой врубает автомобильный кондиционер на полную, расходует ценное топливо и вскоре обзаводится неприятнейшим насморком. До выздоровления герой получает Недостаток "Шумный".
        \item[--] Герой собирается развести костер в лесу, но весь сухостой вымочил вчерашний ливень. Придется лечь спать на холоде, без огня и ужина.
    \end{itemize}
\end{enumerate}
\begin{tcolorbox}
    Важно помнить: Каприз Судьбы означает, что в повествовании возник свершившийся факт. Пьяница вспоминает о том, как бил шерифского сынка уже утром, когда ничего нельзя исправить, а Любвеобильного героя и вовсе поймали на горячем. С другой стороны, вход Каприза в игру не обязательно означает безусловные проблемы. Прежде всего, это создание интересной ситуации, которая начинается неблагоприятно для героя, но может принести выгоды позднее, если Судьба в лице игрока подскажет герою верную линию поведения. 
\end{tcolorbox}
