\section{Нити, Ходы и Капризы}
Как бы ни был ловок, умен и могуч герой, именно благоволение Судьбы выдвигает его на ведущие роли. Нити Судьбы отображают невероятное везение героев — то, что заставляет обычных людей благоговейно пересказывать истории об их удивительных приключениях и подвигах столетия спустя.
\paragraph{\textit{Нити Судьбы — один из важнейших элементов игры. Это именно тот раздел правил, который стоит изучить с особым тщанием. Помимо прочего, Нити позволяют игроку объявить о безусловном успехе героя без бросков кубика. Благодаря этому абсолютно любой герой легко окажется в центре внимания и повлияет на развитие сюжета!}}
\paragraph{Начальное число Нитей:} герои и персоны начинают игровую встречу с 2 Нитями. Одномоментно к герою или персоне не может быть протянуто больше 5 Нитей Судьбы. Неиспользованные к концу встречи Нити обрываются, и следующую встречу герой или персона опять начнет с 2 Нитями.
\newline
Если Судьба (в лице мастера и игроков) сочтет нужным, новые Нити могут протягиваться к героям и персонам не в начале игровых встреч, а по завершении важных сюжетных вех или даже \textit{перед} ними. Например, Нити протянутся к героям накануне генерального сражения с ордой захватчиков-из-за-океана или после того, как битва, так или иначе, завершится. В этом случае любой герой, к которому протянуто меньше 2 Нитей, увеличивает их число до 2. Герои, к которым протянуто больше 2 Нитей, сохраняют их. Заметьте, что к персонам-антагонистам также протянутся новые Нити!
\paragraph{Очки опыта и Нити Судьбы:} Очки опыта могут быть использованы героями как для развития, так и для получения благосклонности Судьбы. \textbf{Мастер должен выбрать один из этих вариантов в начале игровой встречи:}
\begin{itemize}
\item[--] В начале игровой встречи игрок может протянуть к своему герою до 3 Нитей, потратив до 3 Очков опыта (1 Нить за 1 Очко опыта).
\item[--] Игрок может протянуть к своему герою до 5 Нитей, потратив до 5 Очков опыта, в перерывах между сценами (1 Нить за 1 Очко опыта).
\item[--] Очки опыта, имеющиеся в распоряжении героев, могут быть использованы как Нити Судьбы в любой момент игры. Обратите внимание, что это сделает героев невероятно могущественными, так как теоретически у них могут быть и Очки опыта, и максимальное количество Нитей!
Как протянуть к герою новые Нити, и когда они обрываются:
Судьба помогает герою не просто так. Ей нравится следить за тем, как герой барахтается в неприятностях.
\item[--] Игрок может протянуть к своему герою Нить, когда принимает Каприз Судьбы.
\item[--] Игрок должен оборвать Нить (или несколько, если этого требует ситуация) своего героя, когда делает Ход Судьбы.
\end{itemize}
\paragraph{\textit{Изменения, внесенные в игру с помощью Ходов и Капризов, становятся частью истории и могут иметь далеко идущие последствия.}}

\subsection{Темные Нити.} В некоторых случаях действия героев явно противоречат логике повествования, собственной мотивации или даже внутренней сути мироздания! На то они и герои, чтобы бросать вызов всему, что они встречают. Но Судьба не терпит подобной дерзости и даже если сразу героев не настигнет возмездие за свою дерзость, рано или поздно они обязательно поплатятся.
\newline Каждый раз, когда Мастер считает, что герои перегибают палку, он сообщает об этом игрокам и протягивает к себе 1 Темную Нить.
\newline Мастер может использовать Темные Нити в отношении статистов и персон так же, как используются Нити Судьбы в отношении героев, но только в тех случаях, когда статист или персона противостоят героям.
\newline Мастер может обрывать Темные Нити для ввода в игру Капризов Судьбы более 1 раза за сцену. Тем не менее, мастер все еще не может вводить один и тот же Недостаток, Темную сторону и Решку героя больше 1 раза за сцену. Например, если в распоряжении мастера есть 2 Темных Нити, он может 1 раз столкнуть героя с последствиями Недостатка бесплатно, 1 раз ввести в игру Темную сторону его Атрибута и 1 раз ввести в игру Решку героя, но не может ввести в игру один и тот же Недостаток героя дважды. В распоряжении мастера одновременно может находиться число Темных Нитей, равное \textbf{|1 + число игроков|}.