\section{Нити, Ходы и Капризы}
Как бы ни был ловок, умен и могуч герой, именно благоволение Судьбы выдвигает его на ведущие роли. Нити Судьбы отображают невероятное везение героев — то, что заставляет обычных людей благоговейно пересказывать истории об их удивительных приключениях и подвигах столетия спустя.
\begin{tcolorbox}
Нити Судьбы — один из важнейших элементов игры. Это именно тот раздел правил, который стоит изучить с особым тщанием. Помимо прочего, Нити позволяют игроку объявить о безусловном успехе героя без бросков кубика. Благодаря этому абсолютно любой герой легко окажется в центре внимания и повлияет на развитие сюжета!
\end{tcolorbox}
\paragraph{Начальное и максимальное число Нитей.} герои и персоны начинают игровую встречу с 2 Нитями. Одномоментно к герою или персоне не может быть протянуто больше 5 Нитей Судьбы. Неиспользованные к концу встречи Нити обрываются, и следующую встречу герой или персона опять начнет с 2 Нитями.
\newline
Если Судьба (в лице мастера и игроков) сочтет нужным, новые Нити могут протягиваться к героям и персонам не в начале игровых встреч, а по завершении важных сюжетных вех или даже \textit{перед} ними. Например, Нити протянутся к героям накануне генерального сражения с ордой захватчиков-из-за-океана или после того, как битва, так или иначе, завершится. В этом случае любой герой, к которому протянуто меньше 2 Нитей, увеличивает их число до 2. Герои, к которым протянуто больше 2 Нитей, сохраняют их. Заметьте, что к персонам-антагонистам также протянутся новые Нити.
\paragraph{Очки опыта и Нити Судьбы:} Очки опыта могут быть использованы героями как для развития, так и для получения благосклонности Судьбы. \textbf{Мастер должен выбрать один из этих вариантов в начале игровой встречи:}
\begin{itemize}
\item[--] В начале игровой встречи игрок может протянуть к своему герою до 3 Нитей, потратив до 3 Очков опыта (1 Нить за 1 Очко опыта).
\item[--] Игрок может протянуть к своему герою до 5 Нитей, потратив до 5 Очков опыта, в перерывах между сценами (1 Нить за 1 Очко опыта).
\item[--] Очки опыта, имеющиеся в распоряжении героев, могут быть использованы как Нити Судьбы в любой момент игры. Обратите внимание, что это сделает героев невероятно могущественными, так как теоретически у них могут быть и Очки опыта, и максимальное количество Нитей!
\end{itemize}
\paragraph{Новые Нити и обрыв Нитей.} Судьба помогает герою не просто так. Ей нравится следить за тем, как герой барахтается в неприятностях.
\begin{itemize}
\item[--]Игрок протягивает к своему герою 2 Нити в начале каждой игровой встречи.
\item[--]Игрок протягивает к своему герою дополнительную Нить в начале каждой игровой встречи за каждые Узы, которые есть у героя.
\item[--]Игрок протягивает к своему герою Нити в обмен на Очки опыта по предварительной договоренности с мастером.
\item[--]Игрок протягивает к своему герою Нить, когда принимает Каприз Судьбы.
\item[--]Игрок обрывает Нить (или несколько, если этого требует ситуация) своего героя, когда делает Ход Судьбы.
\end{itemize}

\paragraph{\textit{Изменения, внесенные в игру с помощью Ходов и Капризов, становятся частью истории и могут иметь далеко идущие последствия.}}

\paragraph{Темные Нити.} 
Порой поступки героев явно противоречат логике повествования, внутренней мотивации и даже самой сути мироздания. Но герои на то и герои, чтобы бросать вызовы и принимать их. Конечно же, Судьба не терпит подобных дерзостей. Даже если возмездие не настигнет героев сразу, рано или поздно они поплатятся.
\newline Каждый раз, когда мастер считает, что герои выходят за рамки выбранных Уз или как-то иначе перегибают палку, он сообщает об этом игрокам. Если им не удается пояснить логику поступков героя, мастер протягивает к себе 1 Темную Нить.
\newline В распоряжении мастера одновременно может находиться число Темных Нитей, не более \textbf{|1 + число игроков, присутствующих на игровой встрече|}.
\begin{tcolorbox}
"Герои перегибают палку" вовсе не означает "герои делают что-то жестокое, бессмысленное, ужасное или глупое (даже ужасно глупое)". Недовольство Судьбы вызывает последовательное нежелание героя следовать логике жанра (о том, какова эта логика в представлении собравшихся за игровым столом, лучше договориться заранее). Также Судьба может прогневаться на героя, поступки которого лишены какой-либо логики в принципе. Авторский коллектив рекомендует обозначить все подобные моменты до начала игры.
\end{tcolorbox}
Мастер может использовать Темные Нити в отношении статистов и персон так же, как игроки используют Нити Судьбы в отношении своих героев (то есть выкупать обычные и Критические успехи или провалы, совершать Ходы), но только в тех случаях, когда статист или персона противостоят героям. Темные Нити мастера \textit{прибавляются} к Нитям, протянутым к персонам-антагонистам. Например, если антагонист накопил 5 собственных Нитей, а у мастера есть еще 5 Темных Нитей, фактически, персона располагает 10 (!) Нитями. Героям лучше отложить встречу с таким противником на какое-то время.
\newline Мастер может обрывать Темные Нити для ввода в игру Капризов Судьбы более одного раза за Сцену для одного и того же героя. Тем не менее, мастер все еще не вправе вводить один и тот же Недостаток, Темную сторону или Решку героя больше одного раза за Сцену. 
\newline Например, в распоряжении мастера есть 2 Темных Нити. Он может один раз столкнуть героя с последствиями Недостатка бесплатно, один раз ввести в Сцену Темную сторону его Атрибута (1 Темная Нить) и один раз ввести в Сцену Решку героя (1 Темная Нить), но не может ввести в Сцену тот же Недостаток героя второй раз.