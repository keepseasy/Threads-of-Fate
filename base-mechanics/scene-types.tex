\section{Разновидности сцен}
Для простоты управления повествованием сцены разделены на несколько разновидностей. Каждая из них имеет свой темп и особенности. 
\paragraph{Динамическая сцена.} Это наиболее распространенный вид сцен. Ситуация вокруг героев постоянно меняется, реагировать на эти изменения приходится незамедлительно, а время поджимает. В Динамические сценах герои активно противодействуют недругам, обстоятельствам и прочим враждебным силам. 
\paragraph{Боевая сцена.} Это особый подвид Динамической, изображающий вооруженное противостояние героев и статистов. В боевых сценах очередность действий четко структурирована, а их перечень ограничен.
\paragraph{Интерлюдия.} Это короткая передышка между Динамическими сценами. В интерлюдиях отсутствует непосредственная угроза для героев (хотя она может быть создана их действиями или Капризами Судьбы, ознаменовав начало Динамической сцены). Во время Интерлюдии герои получают шанс перевести духи и подлечить раны, обсудить друг с другом дальнейшие действия группы, отряхнуть снаряжения от пыли сражений или же приобрести снаряжение в ближайшей лавке.
\paragraph{Антракт.} Это длительный перерыв в активном детализированном повествовании. Герои в безопасности, им не грозит цейтнот – до поры. Антракт дает возможность поправить здоровье, заняться ремонтом снаряжения, ремеслами и прочей рутиной. О таких делах не очень интересно рассказывать и слушать, но без них не обойдется ни одна история (и ни один герой). 
\begin{tcolorbox}
В некоторых ситуациях, например, когда герою требуется длительное восстановление, имеет значение длительность Антракта. В этом случае можно считать, что в сутках три Антракта.
\end{tcolorbox}
