\chapter{Бестиарий}
В этом разделе описаны существа, которые вероятнее всего повстречаются героям во время приключений.
\section{Карточка Существа}
У каждого существа есть набор свойств, которые описаны в карточке существа.
\paragraph{Название} существа содержит опциональные метки, которые определяют, распространяются ли некоторые эффекты на это существо.
\paragraph{[Не живое]} существо - это не только нежить, но и големы, механизмы и оживленные магией объекты. На них не действуют некоторые эффекты и феномены. В описании эффекта должно быть явно указано, что он не действует на Не живых существ.
\paragraph{[Легендарное]} существа исключительны во всем. Это не просто сильные существа, их можно пересчитать по пальцам руки во всем мире. Не редко эти существа являются именованными Персонами, даже если это неразумное существо. Герои вряд ли встретят двух одинаковых Легендарных существ одновременно.
\paragraph{Описание} существа говорит о том, как оно выглядит, какие имеет повадки и где возможно оно обитает. Но это не игромеханическое, а, скорее, литературное описание.
\paragraph{Характеристи.} Существа имеют те же Основные и Второстепенные характеристики, что и герои. У существа, особенно у Легендарного, даже могут быть Нити на момент первого появления в игре. Если существо использует Феномены, то в этой секции указано название характеристики, которая у него является Феноменальной.
\paragraph{Атаки}, которыми существо может наносить повреждения, указаны со всеми бонусами и навыками, которыми владеет существо. Если у существа нет Атак, это означает, что его попытки напасть прямым образом обречены на провал и существо использует другие методы, чтобы победить своих врагов.
\newline Если в названии есть звездочка(*), то атака имеет дополнительное свойство, которое можно узнать в описании соответствующего оружия в разделе Богатство и снаряжение или в оружиях существ в бестиарии.
\newline Если в БПв есть звездочка(*), то существо не умеет пользоваться этой атакой и совершает проверку Доблести или Меткости с Помехой.

\paragraph{Боевые навыки} существа такие же, как и у героев: Владение оружием, Стрельба и Рукопашный бой. В скобках указано значение, на которое этот навык развит. Если боевого навыка нет в списке, то у существа он не прокачан.
\paragraph{Навыки}, которые есть у существа. Они выбираются из списка Основных и Экспетных навыков. В скобках указано значение, на которое этот навык развит. Если навыка нет в списке, то у существа он не прокачан.
\paragraph{Феномены}, которые доступны Существу. В скобках указана стоимость феномена, а полное описание можно узнать в описании феномена в раздели Феномены и Исказители или в феноменах существ в бестиарии.
\paragraph{Недостатки} существа, которые ограничивают его поведение и которыми могут воспользоваться герои. В карточке существа дано полное описание всех его Недостатков.
\paragraph{Трюки} существа, с помощью которых оно становится сильнее. В карточке существа дано полное описание всех его Трюков.
\paragraph{Функции} существа, которые могут за Энергию дать ему какое-либо преимущество. В карточке существа дано полное описание всех его Функций и их стоимость.
\paragraph{Ходы} существа, которыми Судьба может вмешиваться в происходящие события. В карточке существа дано полное описание всех его Ходов, а в скобках указана и их стоимость в Нитях. Помните, что герои тоже могут активировать Ходы союзных существ, если потратят свои Нити.

\section{Описание существ}
\genAndGet{monsters}{monsters}

\section{Шаблоны}
Шаблоны существ - это загатовки, которые можно добавлять к готовым существам, чтобы дать им какую-то специализацию. Обычно это профессия или поведенческий архетип.
\newline Шаблоны включают в себя изменение Характеристик и дополнительные Атаки, Навыки, Феномены, Недостатки, Трюки, Функции и Ходы, которое существо получит вместе с Шаблоном.
\paragraph{Характеристики и Навыки(в том числе и боевые)}, указанные в Шаблоне прямо складываются с характеристиками существа, на которое этот шаблон накладывается
\paragraph{Атаки} Шаблона добавляются к существующим атакам Существа, но нужно перерасчитать БПв всех атак в соответствии с измененными Характеристиками и Навыками. БПв Атак Шаблона указан без учета Характеристик и Навыков.
\paragraph{Феномены, Недостатки, Трюки, Функции и Ходы} шаблона добавляются к исходному существу и если они вступают в противоречие с Феноменами, Недостатками, Трюками, Функциями и Ходами существа, то следует дать предпочтение Шаблону.

\section{Описание шаблонов}
\genAndGet{monsters}{monster-templates}