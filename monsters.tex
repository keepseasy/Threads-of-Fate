\chapter{Бестиарий}
В этом разделе описаны существа, которые вероятнее всего повстречаются героям во время приключений.
\section{Карточка Существа}
У каждого существа есть набор свойств, которые описаны в карточке существа.
\paragraph{Название} существа содержит опциональные метки, которые определяют, распространяются ли некоторые эффекты на это существо.
\paragraph{[Не живое]} существо - это не только нежить, но и големы, механизмы и оживленные магией объекты. На них не действуют некоторые эффекты и феномены. В описании эффекта должно быть явно указано, что он не действует на Не живых существ.
\paragraph{[Легендарное]} существа исключительны во всем. Это не просто сильные существа, их можно пересчитать по пальцам руки во всем мире. Не редко эти существа являются именованными Персонами, даже если это неразумное существо. Герои вряд ли встретят двух одинаковых Легендарных существ одновременно.
\paragraph{Описание} существа говорит о том, как оно выглядит, какие имеет повадки и где возможно оно обитает. Но это не игромеханическое, а, скорее, литературное описание.
\paragraph{Характеристи.} Существа имеют те же Основные и Второстепенные характеристики, что и герои. У существа, особенно у Легендарного, даже могут быть Нити на момент первого появления в игре. Если существо использует Феномены, то в этой секции указано название характеристики, которая у него является Феноменальной.
\paragraph{Атаки}, которыми существо может наносить повреждения, указаны со всеми бонусами и навыками, которыми владеет существо. Если у существа нет Атак, это означает, что его попытки напасть прямым образом обречены на провал и существо использует другие методы, чтобы победить своих врагов.
\newline Если в названии есть звездочка(*), то атака имеет дополнительное свойство, которое можно узнать в описании соответствующего оружия в разделе Богатство и снаряжение или в оружиях существ в бестиарии.
\newline Если в БПв есть звездочка(*), то существо не умеет пользоваться этой атакой и совершает проверку Доблести или Меткости с Помехой.

\paragraph{Боевые навыки} существа такие же, как и у героев: Владение оружием, Стрельба и Рукопашный бой. В скобках указано значение, на которое этот навык развит. Если боевого навыка нет в списке, то у существа он не прокачан.
\paragraph{Навыки}, которые есть у существа. Они выбираются из списка Основных и Экспетных навыков. В скобках указано значение, на которое этот навык развит. Если навыка нет в списке, то у существа он не прокачан.
\paragraph{Феномены}, которые доступны Существу. В скобках указана стоимость феномена, а полное описание можно узнать в описании феномена в раздели Феномены и Исказители или в феноменах существ в бестиарии.
\paragraph{Недостатки} существа, которые ограничивают его поведение и которыми могут воспользоваться герои. В карточке существа дано полное описание всех его Недостатков.
\paragraph{Трюки} существа, с помощью которых оно становится сильнее. В карточке существа дано полное описание всех его Трюков.
\paragraph{Функции} существа, которые могут за Энергию дать ему какое-либо преимущество. В карточке существа дано полное описание всех его Функций и их стоимость.
\paragraph{Ходы} существа, которыми Судьба может вмешиваться в происходящие события. В карточке существа дано полное описание всех его Ходов, а в скобках указана и их стоимость в Нитях. Помните, что герои тоже могут активировать Ходы союзных существ, если потратят свои Нити.
\paragraph{Классификация существ} является условным разделением существ на группы. Это разделение существует для того, чтобы проще было подобрать существо из Бестиария под возникшую ситуацию.

\section{Атаки и Феномены существ}
Некоторые атаки и феномены являются особенностями Существ и недоступны героям. Описания этих атак находятся здесь.
\section{Атаки}
\genAndGet{weapons}{weapons}{Существа}
\section{Феномены}
\genAndGet{powers}{powers-monsters}{}

\section{Существа}
\subsection{Бродяги}
Эти разумные существа склонны к странствиям. Они селятся в любых местах, которые находят для себя хоть сколь-нибудь пригодыми для жизни и их представителей можно встретить во всех уголках мира.
\genAndGet{monsters}{monsters}{Бродяга}

\subsection{Аборигены}
Аборигены не любят покидать места в которых живут, поэтому практически невозможно их встетить за пределами их территорий.
\genAndGet{monsters}{monsters}{Абориген}

\subsection{Отшельники}
Затворники, которые не любят не только путешествия, но и других разумных существ. Они изредка появляются в деревнях рядом со своим убежищем, но чаще - живут в уединении в глубинах гор или лесов.
\genAndGet{monsters}{monsters}{Отшельник}

\subsection{Чужаки}
Эти разумные существа никогда не были частью мира - их пригласили или принудили появиться здесь. Появление Чужака в любом месте мира это выдающиеся событие, притягивающие любопытных.
\genAndGet{monsters}{monsters}{Чужак}

\subsection{Мутанты}
Эти существа значительно отличаются от Бродяг, Аборигенов и даже от известных Отшельников. Их разум разительно отличается от того, что принято считать нормой и их считали бы Чужаками, если бы не было достоверно известно, что Мутанты были рождены в этом мире.
\genAndGet{monsters}{monsters}{Мутант}

\subsection{Животные}
Хищная и травоядная фауна, которую можно повсеместно встретить в дикой природе и изредка - в цивилизованных областях.
\genAndGet{monsters}{monsters}{Животное}

\subsection{Чудища}
Выдающихся размеров звери, огромные хищные растения, мифические существа и жуткие, но неразумные монстры, пришедшие из других миров или возникшие в следствие ужасающих экспериментов.
\genAndGet{monsters}{monsters}{Чудище}

\subsection{Нежить}
Мертвецы, возвращенные к жизни нечестивой силой.
\genAndGet{monsters}{monsters}{Нежить}

\subsection{Механизмы}
Ожившие предметы, сложные автоматоны и даже механические люди - эти существа никогда не были рождены, но тем не мение они живут своей неорганической жизнью.
\genAndGet{monsters}{monsters}{Механизм}

\printindex[monsters]


\section{Архетипы}
Архетипы - это существа, которые выбрали иной путь, нежели следование своему Наследию. Обычно это профессия или поведенческий архетип.

\subsection{Лиходеи}
Это бравые ребята с непростой профессией - деньги отнимать. Они не всегда это будут делать с помощью силы, ведь хитрость зачастую - более безопасный и верный для них метод.
\genAndGet{monsters}{monster-templates}{Лиходей}

\subsection{Простолюдины}
Простые и честные люди, делающие свое дело и далекие от интриг и прочих политических изысков
\genAndGet{monsters}{monster-templates}{Простолюдин}

\subsection{Военные}
Служивые люди. Регулярная армия, организованные дружинники, стража, тайная полиция. Их присутствия означает манифестацию местной власти, от губернатора до императора.
\genAndGet{monsters}{monster-templates}{Военный}

\subsection{Наемники}
Свободные люди, которые за удовлетворительную плату встанут на сторону того, из чьего кошеля идут деньги. Однако это не значит, что они будут верны, если им предложат сумму еще больше.
\genAndGet{monsters}{monster-templates}{Наемник}

\subsection{Знать}
Правители, богатеи и купцы, правящие массами.
\genAndGet{monsters}{monster-templates}{Знать}

\subsection{Отщепенцы}
Изгои, юродивые, социопаты. Они сторонятся цивилизации, или же были изгнаны из общества.
\genAndGet{monsters}{monster-templates}{Отщепенец}

\printindex[monster-templates]