% !TEX options=--shell-escape
\documentclass[a4paper]{book}
\setcounter{secnumdepth}{3}
\setcounter{tocdepth}{3}
\usepackage{cmap}
\usepackage[T2A]{fontenc}
\usepackage[utf8]{inputenc}
\usepackage[russian]{babel}
\usepackage[left=2cm,right=2cm,top=2cm,bottom=2cm,bindingoffset=0cm]{geometry}

\usepackage[poster]{tcolorbox}
\usepackage{longtable}
\usepackage{wrapfig}
\usepackage{multicol}
\usepackage{vwcol}

\usepackage{imakeidx}
\indexsetup{level=\subsection,toclevel=section,noclearpage}
\makeindex[title=Алфавитный перечень Трюков, name=tricks]
\makeindex[title=Алфавитный перечень Оружия, name=weapons]
\makeindex[title=Алфавитный перечень Атрибутов, name=attributes]
\makeindex[title=, name=powers]
\makeindex[title=Алфавитный перечень Существ, name=monsters]
\makeindex[title=Алфавитный перечень Архетипов, name=monster-templates]

\usepackage{pythontex}
\usepackage[hypertexnames=true]{hyperref}
\hypersetup{
    colorlinks,
    citecolor=black,
    filecolor=black,
    linkcolor=black,
    urlcolor=black,
}

\tolerance=1
\emergencystretch=\maxdimen
\hyphenpenalty=10000
\hbadness=10000

\newcommand{\trouble}[8]{
\begin{center}
\begin{tabular}{ |p{2.7cm}|p{12cm}| }
\hline
\textbf{Результат проверки Неприятностей} & \textbf{Последствия}
\\ \hline
19-20 & \textbf{#1. }#2
\\ \hline
13-18 & \textbf{#3. }#4
\\ \hline
7-12 & \textbf{#5. }#6
\\ \hline
1-6 & \textbf{#7. }#8
\\ \hline
\end{tabular}
\end{center}
}

\newcommand{\troubleControl}{герой должен совершить проверку \textit{Неприятностей под Контролем} }
\endinput
\troubleControl
\trouble
{}{}%no sweat name/description
{}{}%tough day name/description
{}{}%we have trouble name/description
{}{}%fiasco name/description

\newcommand{\tbd}{\textcolor{orange}{\textbf{\textit{TBD}}}}
\newcommand{\err}{\textcolor{red}{\textbf{\textit{ERR}}}}

\newcommand{\earlyEnd}{
----> current stop <----
\end{document}
\endinput
}

\newcommand{\genAndGet}[3]{
%\pyc{import sys}
%\pyc{sys.stdout.reconfigure(encoding='utf-8')}
\pyc{baseName = "#1"}
\pyc{dataName = "#2"}
\pyc{form = "#3"}
\pyc{str = main(baseName,dataName,form)}
\py{str}
}

\begin{document}
\pyc{from scripts.genFromYaml import main}
%\tableofcontents
%\chapter*{Вступление}
Эта игра о Судьбе, которая управляет всем, кроме свободной воли. Эта игра о победах, ведущих к поражениям, и о поражениях, ведущих к победам. Эта игра о борьбе с самим собой и о Нитях, протянутых над бездной. Эта игра о том, как просто отнять чужую жизнь, как легко расстаться со своей и как сложно порой избежать и того, и другого.
Вам предстоит выступить в роли отважных (или не очень) героев... и Судьбы, возносящей их к вершинам мира и низвергающей в пропасть отчаяния. Судьбе нравится наблюдать, как герои сталкиваются с неприятностями и преодолевают их. Иногда Судьба самолично вмешивается в события, заставляя героев исполнять ее капризы, или же, наоборот, помогает героям, делая свой ход. Что уготовано героям — слава, богатство, любовь или безвестная смерть в придорожной канаве? Все в ваших руках — руках Судьбы!

Для игры вам понадобится три К20 (двадцатигранных кубика), минимум один приятель, готовый с вами играть, карандаши, ластик, блокнот и немного воображения. Один из вас должен взять на себя обязанности мастера (ведущего), остальные станут игроками. Игра проходит в форме беседы. Игроки и мастер обмениваются репликами, описывающими события в воображаемом пространстве. Правила служат для того, чтобы упорядочить это пространство и сделать его общим. Они подскажут участникам игры, когда меч воина бессильно отскакивает от щита, а когда — поражает цель, или насколько шпиону сложно вскарабкаться по замшелой крепостной стене, и осветят многие другие неоднозначные моменты. Игрок придумывает биографию и внешность героя, описывает его действия, то есть играет роль героя и его Судьбы, временами благосклонной, временами безжалостной, а временами — безразличной. Это означает, что иногда герой будет терпеть неудачи, влипать в неприятности, сталкиваться с трудностями лишь потому, что игрок так решил. Мастер изображает окружающий героев мир и описывает его реакцию на их действия (или бездействие).

Успехи и неудачи героев определяются решениями, которые игроки принимают в ходе игры, и бросками К20. Несмотря на то, что мастер закладывает основы сюжета, игроки — полноправные соавторы. Правила подразумевают, что игроки и мастер готовы работать над историей сообща, прислушиваться к желаниям друг друга и договариваться в случае разногласий.

\section*{Как играть в <<Нити Судьбы>>}
Игра рассказывает историю о героях, победы которых зачастую имеют цену, а неудачи — последствия. Нити Судьбы и совершение Ходов Судьбы — единственный способ преуспеть без всяких оговорок. Мир вокруг героев изображается широкими мазками, а значимые детали определяются во время игры при помощи проверок Неприятностей и все тех же Ходов Судьбы. Помимо того, есть несколько простых принципов, которые позволят мастеру и игрокам высвободить весь потенциал системы.
\section*{Если вы мастер…}

\paragraph{Готовьте завязку, а не сюжет.} Все, что безусловно нужно игре — отправная точка. Остальное сделают игроки, кубики и воображение. \textbf{Обозначьте возможности и цену}. Лгите героям, но не игрокам. Игрокам стоит знать, ради чего герои рискуют, каковы шансы на победу, как именно можно достичь цели… и с чем придется расстаться по пути.
\paragraph{Используйте Капризы Судьбы.} Расшевелите игроков, вводя в игру Недостатки, Темные стороны и Решки их героев. Не давайте героям опомниться, а игрокам — заскучать.
\paragraph{Не будьте всеведущим.} Позвольте игрокам вас удивить. К тому же, чем больше в вашей игре белых пятен, тем больше возможностей для сотворчества. Игрокам будет непросто придумать что-то, если для этого чего-то не осталось места. Используйте проверки Неприятностей, если контекст не дает однозначного ответа на возникший вопрос.
\paragraph{Помогайте, не заставляйте.} Если по каким-то причинам динамика игры падает, а у игроков нет идей — подкиньте и идей, и событий. При этом не стоит заменять идеи игроков своими и прибегать к мастерскому праву вето слишком часто.
\section*{Если вы игрок...}

\paragraph{Знайте своего героя.} Помните о том, чего он может добиться сам, а где ему понадобится ваша помощь — помощь Судьбы!
\paragraph{Действуйте.} Настоящие герои не будут сидеть и ждать у моря погоды, они будут действовать, даже если Судьба точно знает, что их дело обречено.
\paragraph{Создавайте герою проблемы.} Используйте Успех с Неприятностями, вводите в игру Недостатки, Темные стороны и Решки вашего героя, принимайте худшие варианты Неприятностей — протягивайте к героям Нити! Благополучие и безопасность — плохая основа для запоминающейся истории.
\paragraph{Используйте Капризы Судьбы на других героях.} Если вы видите хорошую возможность для ввода Недостатка, Темной стороны или Решки героя другого игрока — используйте ее. Покажите, насколько Судьба своенравна!
\paragraph{Применяйте Ходы Судьбы.} Поддержите героя, если он в этом нуждается! Изучите перечень общедоступных Ходов, не упускайте из виду Уникальные ходы. Помните — когда Судьба на стороне героя, он способен на все.
\section*{Если вы мастер или игрок...}

\paragraph{Будьте готовы к компромиссам.} История, которая создается на игре, - общая, и каждый участник - ее равно важная часть. Помните об этом.
\paragraph{Правильных решений нет, но всегда есть последствия.} Вы - Судьба, своенравная, капризная, порой жестокая, реже - милосердная. Ваша цель - наблюдать за тем, как герои выпутываются из проблем, которые вы же для них и устроили, и поразвлечься вволю. Худшее, что можно сделать на игре - лишить героев последствий их действий, какими бы ужасными эти последствия ни были.
\paragraph{Уважайте результаты бросков.} Кубики - полноправные соавторы истории. Не игнорируйте результаты бросков, иначе рискуете превратиться из игроков в настольную ролевую игру в писательский коллектив. В этом нет ничего плохого, но это совсем другой вид развлечений.
\paragraph{Доверяйте друг другу.} Взаимное доверие - основа игры, которой все будут довольны. Не забывайте - ваша цель не победа в настольной ролевой игре, ваша цель - запоминающаяся история. И если кто-то использует Каприз Судьбы или вводит в игру Недостаток своего героя, то он участвует в создании истории, а не подставляет других героев под удар.
\paragraph{Наслаждайтесь игрой!}
%\chapter{Основы механики}

\section{Герои, статисты и персоны}
\paragraph{Герои:} так называются персонажи под управлением игроков. Они - протагонисты истории, которую совместно создают мастер и игроки, они избраны Судьбой (для чего именно - вопрос открытый). Герой - ведущая роль на сцене мироздания, точка приложения сил Судьбы. Именно к героям протянуты Нити Судьбы - страховочная сеть над бездной... или ниточки своенравного кукловода.
\paragraph{Статисты:} это персонажи и существа под управлением мастера. Они - второстепенные лица истории. Главное отличие статистов от героев - невозможность прибегнуть к помощи Судьбы большую часть времени. Статисты могут обладать Атрибутами, Трюками и Недостатками, а также совершать Уникальные ходы без обрыва Нитей, принимая все возможные последствия.
\paragraph{Персоны:} это статисты, роли которых сопоставимы по значимости с ролями самих героев. Это не имеет никакого отношения к могуществу или социальному статусу. Правитель, пославший героев на войну, - всего лишь статист, он не принимает значимого участия в истории (хотя участвует в ее завязке). От действий короля мало что зависит - он остался за кулисами, укрывшись в неприступном убежище. А вот находчивый помощник одного из героев - без сомнений, персона. Его успехи и неудачи очень даже влияют на развитие истории. Хоть слуга и не является главным действующим лицом, Судьба присматривает за ним... в полглаза. Персоны также могут использовать Нити Судьбы.

\begin{tcolorbox}
    Нередко Судьба готовит для персон особое место в своих планах. Героям не стоит удивляться, увидев живым и здоровым злодея, которого они с таким трудом одолели неделю назад (хотя мастеру лучше не злоупотреблять сюжетным иммунитетом!).
\end{tcolorbox} 

%"Нити Судьбы" - игра о приключениях группы друзей или как минимум единомышленников. Правила не предполагают конфликтов между героями, хотя могут реализовать их технически. Если мастер и игроки согласны с тем, что герои противостоят друг другу, то все, что герой может сделать со статистом, он может сделать и с другим героем.

\paragraph{Простой статист или персона?} Вопрос скорее философский, чем игромеханический. Статус персоны непостоянен. Если персона по неким причинам отходит на второй план и перестает активно участвовать в развитии истории, то превращается в простого статиста, и наоборот. Если у вашей игровой команды возникают сложности с определением персон, обсудите следующие вопросы:
\begin{itemize}
\item[--] Будет ли героям сложнее добиться цели, если статист не поможет им?
\item[--] Связывают ли статиста и героев чувства дружбы, любви или долга?
\item[--] Готов ли кто-то из героев рискнуть жизнью ради статиста?
\end{itemize}
Если вы ответили "да" на все три вопроса, перед вами самая настоящая персона, однако \textit{Персоны-антагонисты} чаще определяются мастером в соответствии с его личными представлениями о сюжетной роли и важности статиста.

\section{Герои и Судьба}
\paragraph{}Судьба не олицетворяет какую-то \textit{определенную} высшую силу. Прежде всего, Судьба – элемент роли игрока, который решает, когда герой получит поддержку, а когда – нет. Какой облик примет Судьба – госпожи удачи, стечения обстоятельств, сюжетной брони или вмешательства ино-планетных прогрессоров – зависит от контекста вашей истории. Он же подскажет, насколько оче-видно вмешательство Судьбы для героев и окружающих их статистов.
\newline Судьба всеведуща. Почти. Это означает, что игроки вправе получить доступ к любому уста-новленному внутриигровому факту, либо установить его при помощи игромеханических средств, если это важно. Также игровая механика не подразумевает совершения мастером или игроками бросков, результат и последствия которых скрыты от кого-то из них.
\newline Обратите внимание, что пока факт не установлен и не вошел в игру, очевидно, доступа к нему не имеют ни мастер, ни игроки. Так же некоторые способности и правила позволяют по-новому интерпретировать уже установленные факты, поэтому сюрпризы вовсе не исключены – для всех.

\section{Герои и Судьба}
Судьба не олицетворяет какую-то определенную высшую силу. Прежде всего, Судьба — элемент роли игрока, который решает, когда герой получит поддержку, а когда — нет. Какой облик примет Судьба — своенравного божества, госпожи удачи, стечения обстоятельств или сюжетного иммунитета — зависит от жанра и настроения вашей игры. От них же зависит, насколько очевидно вмешательство высших сил для героев и статистов. Если герои античного эпоса прекрасно осведомлены о покровительстве богов (и об их жестокости), то герои современного фэнтезийного романа могут быть уверены, что все эти успехи — их личная заслуга!
\section{Cцены}
Сценой называется эпизод с участием героев. Из множества таких эпизодов и состоит игра. Разговор с информатором, погоня, обыск павших в битве врагов — все это сцены. Действие многих способностей героев ограничено длительностью сцены.
Игровая сцена состоит из следующих этапов:
\paragraph{}
\begin{enumerate}

 \item Мастер описывает наполнение сцены — декорации и события вокруг героев.
\begin{itemize}
\item[--] Что за народ ходит около места встречи с информатором? Привлекло ли внимание появление героев?
\item[--] Героев преследуют по земле или по воздуху? Стараются ли их поймать или пытаются убить во время погони?
\item[--] Все павшие мертвы, или среди них есть те, кого еще можно спасти? Не помешает ли кто-то героям собирать трофеи?
\end{itemize}
При создании наполнения сцены мастер может принимать решения, основанные на импровизации и контексте ситуации, или определить некоторые детали при помощи игромеханических инструментов — проверок Неприятностей, Впечатлений и ввода в игру Недостатков героев. Подробнее об этом читайте в соответствующих разделах книги.
\linebreak
Описание сцены вовсе не должно быть литературным, хотя хороший слог мастера, безусловно, выгодно скажется на атмосфере игры. Главное, донести до игроков информацию, которую они смогут использовать при описании действий своих героев.
 \item Игроки задают уточняющие вопросы (если это требуется) и описывают действия своих героев.
\begin{itemize}
\item[--] Герои предложат информатору деньги за нужные им сведения или попытаются добиться своего при помощи угроз?
\item[--] Герои спрячутся от преследователей или заманят их в ловушку?
\item[--] Герои помогут раненому врагу бескорыстно или сохранят ему жизнь в надежде на выкуп или помощь?
\end{itemize}
На этом этапе игроки не только заявляют, что делают их герои, но и решают, вмешается ли в события Судьба. Они могут воспользоваться Нитями Судьбы и совершить Ходы Судьбы, повлияв на наполнение сцены и ее контекст. Подробнее об этом читайте в разделе <<Нити, Ходы и Капризы>>.
\item Мастер объявляет, какие проверки должны совершить герои (и должны ли вообще), определяет их сложность и описывает последствия действий героев.
\begin{itemize}
\item[--] Информатор охотно принимает деньги и делится тем, что знает, или зовет подмогу?
\item[--]  Погоня отстала потеряв след или же застряла в ловушке?
\item[--]  Раненый обещает отплатить добром за добро или втайне готовится к мести?
\end{itemize}
Далеко не каждое действие и решение героев требует проверок. Определение ее принципиальной необходимости — одна из обязанностей мастера. Подробнее о проверках читайте в разделе <<Проверки>>.
\linebreak \textbf{На этом этапе игроки также могут применять Ходы Судьбы, в том числе связанные с проверками.} Подробнее об этом читайте в разделе <<Нити, Ходы и Капризы>>.
\end{enumerate}
Новая сцена начинается, когда предыдущая так или иначе получает логическое завершение: герои узнали  то, что хотели (или бежали под градом ударов), погоня уничтожена (или остался далеко позади), а поле битвы обследовано вдоль и поперек (или герои поглощены хлопотами с раненым). Если герои по неким причинам разделились, каждый из них станет участником отдельной сцены.
\section{Особые Cцены}
В то время, как ход событий большинстве сцен может быть произвольным, в Нитях Судьбы есть особые сцены, наполнение которых имеет ограничения.
\subsection{Боевые Сцены}
Боевые сцены требуют более точного контроля времени и последовательности действий героев и статистов. Боевая сцена поделена на Круги, внутри которых каждый действует в свою Очередь. Подробное описание Боевых Сцен указано в разделе Боевые столкновения.
\newline Сцены, в которых требуется более тщательный контроль времени, хоть в них и не происходит сражения могут быть тоже рассмотрены, как Боевые.
\subsection{Интерлюдия}
Во время интерлюдии герои получают небольшую передышку, во время которой они могут перевести дыхание и частично восстановить свои силы.
\subsection{Антракт}
Антракт - большой перерыв между событиями в игре. Он может быть как несколько часов, так и несколько лет, а может быть и тысячелетий. В это время герои восстанавливают большинство своих способностей.

\section{Течение времени}
Во время приключений герои будут получать различные эффекты, продолжительность которых может растягиваться на несколько сцен. Наиболее частые продолжительности следующие:
\begin{itemize}
\item[--] Одна или несколько \textbf{Очередей} в Боевой Сцене. Это самая короткая продолжительность.
\newline Эти эффекты имеют значение только в Боевых Сценах. В других сценах они настолько мимолетны, что не имеют значительного влияния на ход событий.
\item[--] Один или несколько \textbf{Кругов} в Боевой Сцене. Началом отсчета для этих эффектов является начало Очереди героя или статиста, который его применил.
\newline Эти эффекты имеют значение только в Боевых Сценах. В других сценах они настолько мимолетны, что не имеют значительного влияния на ход событий.
\item[--] Одна или несколько \textbf{Сцен}. Действие большинства Феноменов, Трюков и Ходов ограничено именно этим событийным промежутком. После чего их предстоит активировать повторно.
\item[--] До следующей \textbf{Интерлюдии}. Эффекты, от которых можно избавиться, переведя дыхание и немного отдохнув. Если герой находится под эффектом, который длиться определенное количество Сцен, Интерлюдия считается за одну Сцену.
\item[--] До следующего \textbf{Антракта}. Только продолжительный отдых может избавить героя это этого эффекта. Антракт завершает действие всех эффектов, если в описании эффекта не указано иначе.
\item[--] Специальные условия. Некоторые эффекты длятся, пока не будет выполнено определенное действие или не произойдет определенное событие. В описании эффекта должно быть указано, что его действие не прерывается Антрактом.
\end{itemize}
\section{Проверки}
\paragraph{Проверки} — это броски кубика К20, изображающие усилия героя, физические, волевые или умственные. Чем большее число выпало на кубике, тем больше шанс, что герой преуспеет. Например, когда герой совершает проверку Характеристики, игрок бросает К20 и прибавляет к выпавшему результату модификатор Характеристики. Различные факторы могут повысить или понизить шансы героя на успех. Они называются \textbf{бонусами} и \textbf{штрафами} и отображаются числами, которые прибавляются к результату броска или отнимаются от него.
\newline Проверки совершаются против фиксированного числа — \textbf{сложности}, которую задают мастер, контекст ситуации или правила. Если результат равен сложности или превышает ее, герой достиг успеха.
\newline Перед совершением проверки определите следующее:
\begin{itemize}
\item[--]Цель, которую герой пытается достичь совершением проверки.
\item[--]Сложность проверки, исходя из цели героя и контекста
ситуации, если она не определена правилами.
\item[--]Цену провала проверки, если она не определена правилами.
\item[--]Наличие Преимуществ и Помех.
\item[--]Наличие бонусов или штрафов.
\item[--]Допустимость Успеха с Неприятностями и потенциальные Неприятности, если он возможен.
\end{itemize}
\paragraph{Градации успеха:} некоторые проверки имеют градации успеха — например, проверки Доблести и Меткости. В этом случае важна величина разницы между результатом проверки и заданной сложностью.
\paragraph{Преимущества и Помехи:} порой обстоятельства складываются неблагоприятно или, наоборот, благоволят герою. В этом случае бросьте дополнительный кубик за каждую Помеху или Преимущество. Выберите меньший результат, если герой находится под действием Помехи, и больший, если герой обладает Преимуществом. Одновременное действие 1 Помехи и 1 Преимущества сводит их на нет. Герой не может страдать больше чем от 2 Помех за бросок, как не может реализовать больше 2 Преимуществ за бросок. То есть максимальное число кубиков в броске — 3.
\paragraph{Активные проверки} подразумевают некие действия героя — атаку, бег, прыжки, разговор, поиск, размышления. К активным не относятся проверки Наблюдательности, не связанные с целенаправленным поиском, проверки Воли, проверки на Потерю сознания, сопротивление яду и тому подобные.
{\paragraph{Массовые проверки.} Иногда группа статистов подвергается эффекту, требующему множества отдельных бросков: отряд городского ополчения оказывается в зоне действия отравляющих газов, группа охотников забредает в болото, купеческий караван попадает под оползень. В этом случае мастер может совершить проверку один раз и применить для каждого статиста соответствующие бонусы и штрафы, чтобы определить результат.
\paragraph{Пассивные проверки.} В некоторых случаях, например, когда нужно отразить длительные усилия героя, повторяющиеся достаточно регулярно или для упрощения состязаний героев. Не кидайте кубик, вместо этого прибавьте значению Навыка значение по таблице в зависимости от количества Помех или Преимуществ к проверке.
\begin{center}
\begin{tabular}{|l|l|}
\hline
Условия проверки & Значение \\ \hline
2 Помехи & 5 \\ \hline
1 Помеха & 7 \\ \hline
нет Помех и Преимуществ & 10 \\ \hline
1 Преимущество & 12 \\ \hline
2 Преимущества & 15 \\ \hline
\end{tabular}
\end{center}
Вы можете применять Быстрые проверки, когда герой стоит на часах, методично обшаривает стены в поисках потайного лаза или выполняет другие задачи, требующие систематического повторения одних и тех же действий, а так же для избегания совершения многочисленных проверок во время действия.
\paragraph{Критический провал и успех:} выпав на кубике, числа 1 и 20 отражают ошеломительные провалы и успехи на грани возможного. В таких случаях мастер может ввести в игру дополнительные эффекты броска, кроме неудачи или успеха. При выпадении 1 мастер может засчитать автоматический провал проверки, даже если бросок превысил сложность задачи. При выпадении 1 во время проверок Доблести или Меткости цель никогда не теряет Единицы Здоровья, даже если Доблесть или Меткость героя достаточно велики, чтобы поразить Защиту цели. При выпадении 20 на кубике мастер может засчитать автоматический успех проверки, даже если бросок не превысил сложность задачи. При выпадении 20 во время проверок Доблести или Меткости цель всегда теряет минимум 1 Единицу Здоровья, даже если Доблесть или Меткость героя недостаточно велики, чтобы поразить Защиту цели.
\newline Критические провалы профессионалов и Критические успехи дилетантов кому-то могут показаться нелогичными. С другой стороны, это мощный повествовательный инструмент, который не стоит игнорировать. Объяснив, из-за чего спасовал профи и преуспел дилетант, вы насытите вашу историю интереснейшими подробностями.
\section{Примерная сложность задач}
\begin{center}
\begin{tabular}{ |c|c| }
\hline
\textbf{Задача} & \textbf{Сложность} \\ \hline
Примитивная & 5 \\ \hline
Повседневная & 10 \\ \hline
Придется попотеть & 15 \\ \hline
Работа для эксперта & 20 \\ \hline
Вызов для эксперта & 25 \\ \hline
На грани возможного & 30 \\ \hline
\end{tabular}
\end{center}
\paragraph{Какую сложность выбрать?} Игровые испытания показали, что оптимальная сложность задач для героев, даже весьма опытных, составляет от \textbf{15} до \textbf{20}. Используйте меньшую сложность, когда хотите продемонстрировать превосходство героев над статистами, для большинства из которых сложность в \textbf{10} – довольно серьезный вызов, или если желаете оставить небольшой, но все-таки осязаемый, шанс на провал. Сложности, превышающие \textbf{20}, обычно требуют от героев вмешательства Судьбы.
\newline В любом случае не забудьте сообщить игроку целевое число при совершении героем проверки. На основании этой информации игрок принимает решения об использовании Ходов Судьбы и способностей героя.

\paragraph{Успех с Неприятностями:} если герой не прошел проверку, игрок может предложить ввести в игру Неприятность, позволяющую тому преуспеть или сопутствующую успеху. Например, вор открыл замок, но старый механизм пронзительно заскрипел и разбудил стражника. Или воин поразил противника, но дешевый клинок сломался при ударе. Считайте, что герой добился минимально необходимого успеха и автоматически получил вариант "Катастрофа" при проверке Неприятностей. Не протягивайте к герою Нить — его наградой за Неприятность будет успех проверки. Ниже вы найдете возможные примеры Неприятностей, осложняющих успех проверки.
\begin{itemize}
\item[--] \textbf{Возможность для недругов:} успех героя позволяет недругам приблизиться к своей цели или даже достичь ее. Эта Неприятность может оставаться за кадром до тех пор, пока герои не столкнутся с ее последствиями.
\item[--] \textbf{Временные затраты:} выполнение задачи требует больше времени, чем планировал герой.
\item[--] \textbf{Герой под ударом:} герой преуспел, но оказался в затруднительном положении. Его жизнь, здоровье или репутация под угрозой!
\item[--] \textbf{Невыгодная позиция:} успех вынуждает героя занять невыгодную позицию. Несколько последовательных выборов этого варианта могут привести героя на край обрыва, в глухой тупик или под обстрел артиллерийской батареи!
\item[--] \textbf{Оповещение недругов:} выполнение задачи привлекает к герою нежелательное внимание.
\item[--] \textbf{Ослабление эффекта:} герой выполнил задачу, но в самом скором времени статус-кво будет восстановлен.
\item[--] \textbf{Перерасход ресурсов:} герой преуспел, но потратил гораздо больше ресурсов, чем планировал.
\item[--] \textbf{Поломка снаряжения:} герой справился с задачей, но его снаряжение пришло в негодность.
\item[--] \textbf{Союзники под ударом:} успех героя приводит к тому, что его товарищи оказываются в затруднительном положении. Их жизнь, здоровье или репутация под угрозой!
\begin{tcolorbox}
Неприятность \textbf{Союзники под ударом} подразумевают прежде всего значимых для героев статистов и персон, но если таковых нет или они отсутствуют в Сцене, то по предварительной договоренности игроков один герой может подвергнуть опасности другого. Судьба своенравна и жестока, не забывайте!
\end{tcolorbox}
\item[--] \textbf{Ущерб}: герой добился своего, но получил Опасную рану. Герой теряет число ЕЗ, достаточное для получения Опасной раны, вне зависимости от имеющихся у него защитных средств.
\end{itemize}
\paragraph{}Если герой не распределил Очки опыта в Навык, проверку которого он совершает, выберите 1 дополнительную Неприятность из списка.
Если герой достигает успеха только при выпадении 20 на кубике (или правила не позволяют ему совершить проверку в принципе), выберите 1 дополнительную Неприятность из списка.
\paragraph{Успех с Неприятностями и Экспертные навыки:} герой может применять Успех с Неприятностями при использовании Экспертных навыков, к которым не имеет доступа, хотя фактически бросок кубика не совершается.
\begin{tcolorbox}
Мастер может запретить Успех с неприятностями, если, по его мнению, герой в ходе успеха приобретет значительно больше, чем потеряет, или если при проверке на кубике выпала 1.
\end{tcolorbox}
\paragraph{\textit{Успехи с Неприятностями позволят героям преодолеть полосу невезения, а мастеру и игрокам — наблюдать за развитием сюжета, а не за бесконечной чередой неудач. Не ограничивайтесь вариантами из списка, опирайтесь на жанр и настроение вашей игры!}}
\paragraph{Взаимопомощь:} герои могут помогать друг другу, если логика ситуации это допускает. Выберите героя, который будет совершать основную проверку и определите тех, кто ему помогает. Определив сложность проверки, отнимите от нее 5 — это сложность задачи для помощников. Совершите проверку профильной Характеристики или Навыка для каждого из помощников. Если помощник преуспел в проверке, герой, совершающий основную проверку, получает Преимущество. Если помощник потерпел неудачу, герой, совершающий основную проверку, получает Помеху. Не забывайте, что единовременно герой не может иметь более 2 Преимуществ или 2 Помех на бросок.
\newline В бою правила взаимопомощи работают иначе (смотрите маневр "Финт" и правило "Все на одного").
\section{Когда бросать кубик?}
Не бросайте кубик, если герой занят рутинными делами, не ограничен во времени и ресурсах и при этом имеет хотя бы 1 Очко опыта в Навыке. Не бросайте кубик, если успех или неудача не имеют значимых последствий.
Бросайте кубик, если герой рискует чем-то важным – репутацией, богатством, жизнями друзей (или своей собственной). Бросайте кубик, если герой ограничен во времени и ресурсах.
\paragraph{}Герою точно не понадобится проверка, если он…
\begin{itemize}
\item[--]готовит скромный ужин,
\item[--]разводит костер сухим теплым вечером,
\item[--]выполняет рутинный уход за автомобилем,
\item[--]мирно выпивает в кабаке,
\item[--]стирает свою одежду.
\end{itemize}
\paragraph{}Но герою обязательно придется совершить бросок, если он…
\begin{itemize}
\item[--]творит кулинарный шедевр из гнилой тушенки, проросшей картошки и вялой моркови,
\item[--]разводит костер из отсыревшего хвороста под противным моросящим дождиком,
\item[--]латает двигатель, имея под рукой лишь спички, желуди и немного медной проволоки,
\item[--]пьянствует напропалую накануне собственной свадьбы,
\item[--]очищает одежду от засохшей крови и радиоактивной пыли.
\end{itemize}
\paragraph{}Какие проверки потребуется сделать герою, вам подскажут мастер, соигроки и контекст Сцены.
\section{Состязания}
Обычно для определения успеха или неудачи действия героя или статиста достаточно бросить кубик — все положительные и отрицательные факторы включены в бросок. Однако иногда мастер может добавить в ситуацию остроты. В этом случае и мастер, и игрок бросают кубики, прибавляют к ним все необходимые бонусы/штрафы и сравнивают результаты. В состязании побеждает тот, чей финальный результат окажется больше. При использовании этого правила замените в формулах, используемых для противостояния герою, 10 на бросок К20.
\newline
Если в Состязании противники получают равные результаты, то ни одна из сторон не может взять верх, и ситуация остается такой же, как и до броска. Рекомендуется использовать Состязание, только если герою противостоит персона или другой герой.
\newline
Если для состязания используется правило Быстрых проверок, то рекомендуется применять ее к стороне, против которой иницировали Состязание. Например, герой вызвал статиста на арм-реслинг. В этом случае герой совершает обычную Проверку Силы или Атлетики(СЛ), а статист - Быструю проверку Силы или Атлетики(СЛ).
\section{Округление результатов}
Все дробные числа, получившиеся в результате расчетов, округляются в меньшую сторону.
\section{Частное превосходит общее}
Атрибуты, Трюки, некоторые предметы часто позволяют совершать нечто, недоступное обычным людям (даже героям, у которых иной набор снаряжения, Атрибутов и Трюков). Описания Атрибутов, Трюков, предметов и способностей существ могут противоречить основным правилам, создавая исключения. В таких случаях частное правило всегда превалирует над общим.
\section{Неприятности}
Зачастую проблемы создают герои, но иногда Неприятности сами находят их. Неприятности изображают неблагоприятные события, которые могут случиться, а могут и пройти стороной. В потенциально опасной сцене, такой как прогулка по огромной свалке, обыск древнего убежища или прыжок в море с обрыва, мастер может инициировать проверку Неприятностей, если контекст ситуации недостаточно ясно говорит о том, что герою ничто не угрожает.

\begin{tcolorbox}
    Неприятности почти всегда оставляют герою шанс спастись и не убивают его сразу. Все случаи, когда Неприятности приводят к смерти героя описаны отдельно в правилах.
\end{tcolorbox}

Перед проверкой Неприятностей определите:
\begin{itemize}
    \item[--] Причину проверки;
    \item[--] Значимые детали проверки;
    \item[--] Что случится при Успехе;
    \item[--] Что случится при Катастрофе.
\end{itemize}

\begin{tcolorbox}
    Причина и исходы Неприятностей может оставаться тайной для игроков. Но поскольку Судьба всеведуща, авторский коллектив рекомендует проговаривать эти моменты.
\end{tcolorbox}

\paragraph{Герою потребуется проверить Неприятности, когда он:}
\begin{itemize}
    \item[--] Крадется по ядовитой трясине;
    \item[--] Поднимает в воздух древний самолет;
    \item[--] Покупает мясное буррито у незнакомого лотошника;
    \item[--] Выпивает в подозрительной компании;
    \item[--] Прыгает в море с обрыва.
\end{itemize}

\subsection{Проверка неприятностей} это, обычно, бросок К20 без модификаторов. Результат проверки определяется по таблице:
\trouble
{Успех}%success name
{Герой вышел сухим из воды. Ну, на то он и герой.}%success description
{Затруднение}%difficulties name
{Герой вовремя заметил надвигающиеся трудности. Скорее всего, он сумеет их избежать. Скорее всего.}%difficulties description
{Проблема}%troubles name
{Герой оказался в сложном, но не безвыходном положении.}%troubles description
{Катастрофа}%fiasco name
{Герой на волосок от смерти. Ему будет непросто выкрутиться.}%fiasco description

\paragraph{КУ и КП.} к проверки Неприятностей не применяются.

\paragraph{Модификация проверки Неприятностей.} Некоторые правила позволяют или предписывают модифицировать результат Неприятностей, прибавляя или отнимая от него некое число. Возможна ситуация, когда герой не способен получить лучший или худший вариант проверки из-за суммы модификаторов.

\subsection{Приятные Неприятности.} Эта механика так же может использоваться и для определения того, оказался ли под рукой у героя необходимый предмет, есть ли поблизости разыскиваемое героем заведение и так далее, если контекст не дает исчерпывающего ответа на вопрос. В этом случае проверка определит:
\begin{itemize}
    \item[--] Есть ли чистая вода в округе;
    \item[--] Продаются ли в лавке пули нужного калибра;
    \item[--] Можно ли доверять проводнику;
    \item[--] Легко ли взломать старую решетку;
    \item[--] Умеет ли читать дочка отшельника.
\end{itemize}
В этом случае проверка неприятностей будет выглядеть так: 
\trouble
{Да}%success name
{Герой получает желаемое или может получить желаемое, приложив незначительные усилия или потратив немного ресурсов.}%success description
{Да, но}%difficulties name
{Герой может получить желаемое, приложив усилия или потратив ресурсы.}%difficulties description
{Нет, но}%troubles name
{Герой может получить желаемое, только если приложит серьезные усилия или потратит значительные ресурсы.}%troubles description
{Нет}%fiasco name
{Герой не может получить желаемое. Ему придется искать другие пути.}%fiasco description

\subsection{Скрытая угроза}
Иногда последствия действий героев безобидны лишь на первый взгляд. Скрытая Угроза представляет проверку Неприятностей, результаты которой позволяют мастеру определить контекст загодя, или по-новому трактовать события уже завершенной Сцены:
\trouble
{Неожиданная слава}%no sweat name
{Герои будут вознаграждены за смелость, отзывчивость и доброту, а нерешительность, равнодушие и жестокость не возымеют далеко идущих последствий.}%no sweat description
{Круги на воде}%tough day name
{Действия героев не приведут к значительным последствиям. Полученные знакомства мимолетны, враги незлопамятны, а хозяева вещей, которые герои прибрали к рукам нескоро заметят пропажу.}%tough day description
{Тень затмения}%we have trouble name
{Проблема, которую все же реально заметить, пока не станет слишком поздно. Сцена еще может обернуться сущим кошмаром, но герои выйдут сухими из воды, если не будут хлопать ушами.}%we have trouble description
{Петля на шее}%fiasco name
{Герои попали в переплет. Убитые разбойники имели влиятельных покровителей, найденные предметы были кем-то спрятаны, а спасенная красотка обокрала караван и сбежала!}%fiasco description
Проявления Скрытой угрозы неочевидны, а иногда вовсе незаметны для героев, хотя Судьба вправе подсказать им линию поведения или принять иные меры. 

\subsection{Общие Неприятности}
Иногда возникают ситуации, в которых Неприятности напрямую касаются всех участников Сцены, - например, заглох двигатель самолета или шериф считает героев подельниками.
\newline Для того, чтобы откупиться от Общих неприятностей достаточно 1 Нити любого отдельного героя, однако если игроки принимают Каприз Судьбы "Катастрофа!", Нити протягиваются ко всем героям и \tbd персонам, присутствующим в Сцене. О Капризах Судьбы читайте в соответствующем разделе правил.

\subsection{Неприятности под Контролем.}
Обычно Неприятности случаются внезапно, но иногда герою выпадает шанс смягчить эффект. Если мастер считает, что есть способ как-то повлиять на исход, герой может совершить проверку Характеристики или Навыка, уместного в контексте ситуации. Сложность проверки устанавливается мастером как обычно. Величина успеха прибавляется к результату проверки Неприятностей, а величина провала вычитается из него.
\newline Например, в трущобах герой может постараться не мозолить окружающим глаза и использовать Скрытность против сложности, установленной мастером. Это все еще не помешает герою случайно наткнуться на грабителей, но может \textit{понизить шанс} такой встречи. Если герой прошел Скрытность на 3, то к числу, выпавшему при проверке Неприятностей, будет прибавляться 3. Если герой провалил Скрытность на 2, из числа, выпавшего при проверке Неприятностей, будет вычтено 2, а его неумелые попытки выглядеть незаметно привлекут внимание окружающих.
\newline Если герой прошел проверку Характеристики или Навыка, и при сложении ее результата с результатом проверки Неприятностей, получилось число больше 20, то считайте этот результат "Успехом". Если герой провалил проверку Характеристики или Навыка, и при вычитании ее результата из результата проверки Неприятностей, получилось число меньше 1, то считайте, что случилась "Катастрофа".
\paragraph{Неприятности и Впечатление.} Если правила предписывают совершить проверку Неприятностей под контролем Впечатления, \textbf{|текущее значение Впечатления от героя - 10|} прибавляется к числу, выпавшему при проверке Неприятностей.

\subsection{Когда использовать проверку Неприятностей?}
Эта механика позволяет мастеру создавать игровые события и факты без предварительной подготовки, руководствуясь контекстом сцены, и при этом разделять повествовательные права с игроками. Она не заменяет собой проверки Навыков (хотя ситуации, в которых такая замена будет уместна, могут возникнуть). Проверка Неприятностей позволит легко и быстро узнать, в каком настроении вернулся с охоты молодой вождь дикарей, надежен ли информатор, есть ли поблизости укрытие от надвигающейся песчаной бури. 

\begin{tcolorbox}
    Обычно проверки Неприятностей инициирует мастер. Ему же придется судить о том, насколько проверка вообще необходима. Разумеется, если у мастера и игроков уже готовы ответы на все вопросы, проверка Неприятностей вряд ли будет использоваться часто. Но даже в таких случаях не стоит полностью исключать ее из игры.
\end{tcolorbox}

\section{Нити, Ходы и Капризы}
Как бы ни был ловок, умен и могуч герой, именно благоволение Судьбы выдвигает его на ведущие роли. Нити Судьбы отображают невероятное везение героев — то, что заставляет обычных людей благоговейно пересказывать истории об их удивительных приключениях и подвигах столетия спустя.
\paragraph{\textit{Нити Судьбы — один из важнейших элементов игры. Это именно тот раздел правил, который стоит изучить с особым тщанием. Помимо прочего, Нити позволяют игроку объявить о безусловном успехе героя без бросков кубика. Благодаря этому абсолютно любой герой легко окажется в центре внимания и повлияет на развитие сюжета!}}
\paragraph{Начальное число Нитей:} герои и персоны начинают игровую встречу с 2 Нитями. Одномоментно к герою или персоне не может быть протянуто больше 5 Нитей Судьбы. Неиспользованные к концу встречи Нити обрываются, и следующую встречу герой или персона опять начнет с 2 Нитями.
\newline
Если Судьба (в лице мастера и игроков) сочтет нужным, новые Нити могут протягиваться к героям и персонам не в начале игровых встреч, а по завершении важных сюжетных вех или даже \textit{перед} ними. Например, Нити протянутся к героям накануне генерального сражения с ордой захватчиков-из-за-океана или после того, как битва, так или иначе, завершится. В этом случае любой герой, к которому протянуто меньше 2 Нитей, увеличивает их число до 2. Герои, к которым протянуто больше 2 Нитей, сохраняют их. Заметьте, что к персонам-антагонистам также протянутся новые Нити!
\paragraph{Очки опыта и Нити Судьбы:} Очки опыта могут быть использованы героями как для развития, так и для получения благосклонности Судьбы. \textbf{Мастер должен выбрать один из этих вариантов в начале игровой встречи:}
\begin{itemize}
\item[--] В начале игровой встречи игрок может протянуть к своему герою до 3 Нитей, потратив до 3 Очков опыта (1 Нить за 1 Очко опыта).
\item[--] Игрок может протянуть к своему герою до 5 Нитей, потратив до 5 Очков опыта, в перерывах между сценами (1 Нить за 1 Очко опыта).
\item[--] Очки опыта, имеющиеся в распоряжении героев, могут быть использованы как Нити Судьбы в любой момент игры. Обратите внимание, что это сделает героев невероятно могущественными, так как теоретически у них могут быть и Очки опыта, и максимальное количество Нитей!
Как протянуть к герою новые Нити, и когда они обрываются:
Судьба помогает герою не просто так. Ей нравится следить за тем, как герой барахтается в неприятностях.
\item[--] Игрок может протянуть к своему герою Нить, когда принимает Каприз Судьбы.
\item[--] Игрок должен оборвать Нить (или несколько, если этого требует ситуация) своего героя, когда делает Ход Судьбы.
\end{itemize}
\paragraph{\textit{Изменения, внесенные в игру с помощью Ходов и Капризов, становятся частью истории и могут иметь далеко идущие последствия.}}

\subsection{Темные Нити.} В некоторых случаях действия героев явно противоречат логике повествования, собственной мотивации или даже внутренней сути мироздания! На то они и герои, чтобы бросать вызов всему, что они встречают. Но Судьба не терпит подобной дерзости и даже если сразу героев не настигнет возмездие за свою дерзость, рано или поздно они обязательно поплатятся.
\newline Каждый раз, когда Мастер считает, что герои перегибают палку, он сообщает об этом игрокам и протягивает к себе 1 Темную Нить.
\newline Мастер может использовать Темные Нити в отношении статистов и персон так же, как используются Нити Судьбы в отношении героев, но только в тех случаях, когда статист или персона противостоят героям.
\newline Мастер может обрывать Темные Нити для ввода в игру Капризов Судьбы более 1 раза за сцену. Тем не менее, мастер все еще не может вводить один и тот же Недостаток, Темную сторону и Решку героя больше 1 раза за сцену. Например, если в распоряжении мастера есть 2 Темных Нити, он может 1 раз столкнуть героя с последствиями Недостатка бесплатно, 1 раз ввести в игру Темную сторону его Атрибута и 1 раз ввести в игру Решку героя, но не может ввести в игру один и тот же Недостаток героя дважды. В распоряжении мастера одновременно может находиться число Темных Нитей, равное \textbf{|1 + число игроков|}.
\section{Ходы Судьбы}
\paragraph{Ход Судьбы (Ход)} — прямое вмешательство высшей силы в жизнь героя. Некоторые Ходы доступны лишь героям, имеющим определенные Атрибуты — это \textbf{Уникальные ходы.} Подробнее об этом читайте в разделе <<Атрибуты>>.
\paragraph{Совершение Хода:} каждый раз, когда Судьба делает Ход, обрывается число Нитей героя, указанное в описании Хода. Таким образом игроки могут добавлять в сцены статистов, факты их биографии, элементы обстановки и многое другое. Игрок может обрывать любое число Нитей своего героя, комбинируя любое число Ходов (если в описании Хода не сказано обратного).
\paragraph{Ходы и помощь персонам и статистам:} игроки могут обрывать Нити своих героев, чтобы помочь важным для них статистам. При этом для них доступны Ходы из категорий <<Повезло>>> и <<В нужном месте в нужное время>>.
\begin{tcolorbox}
Мастер вправе заблокировать Ход или потребовать обрыва большего числа Нитей, если вы предлагаете нечто маловероятное или не соответствующее жанру и настроению игры.
\end{tcolorbox}

\section{Перечень Ходов}
Ходы, описанные ниже, доступны любому герою или персоне:
\begin{enumerate}
\item \textbf{Повезло!}
\begin{itemize}
\item[--] 1 Нить. Откажитесь от последствий Каприза Судьбы. Если вы отказываетесь от последствий Каприза, то к вашему герою не протягивается Нить.
\item[--] 1 Нить. Получите наилучший вариант при проверке Неприятностей. Оборвите Нить до броска кубика и считайте, что при проверке выпало 20.
\item[--] 1 Нить. Перебросьте проверку 1 раз. Вы не обязаны принимать второй бросок, если он хуже первого. Любой кубик может быть переброшен не больше 2 раз. Перебрасывая проверки с Помехами или Преимуществами, перебросьте все кубики.
\begin{tcolorbox}
Переброс выглядит хорошей идеей, но не всегда является таковой. Если вам необходим успех героя, лучше оборвать Нити и позволить ему преуспеть. Авторский коллектив рекомендует использовать перебросы во время атакующих Маневров и прочих проверок, имеющих градации успеха, либо когда герой находится под действием Преимуществ.
\end{tcolorbox}
\item[--] 1 Нить. Сохраните жизнь персоне, провалившей проверку Выносливости при Опасной ране или смерти, либо статисту, получившему Опасную рану или умершему. Спасенный таким образом статист (или персона) прерывает участие в Сцене (обычно – теряя сознание или страдая от ран).
\item[--] 2 Нити. Получите успех на проверку героя. Оборвите Нити до того, как брошен кубик. Если бросок подразумевает различные эффекты в зависимости от степени успеха, герой получает минимально необходимый успех. Вы не можете использовать этот Ход, если герой достигает успеха только при выпадении 20 на кубике.
\newline Так же вы можете провалить проверку героя. Вы можете использовать Ход, даже если герой достигает успеха автоматически (например, из-за высокого Навыка и отсутствия рисков). Если бросок подразумевает различные эффекты в зависимости от степени провала, герой получает минимально необходимый провал. 
\item[--] 4 Нити. Получите Критический успех на проверку героя. Для тех эффектов, которые это учитывают, считается, что на кубике выпало 20. Вы можете использовать этот Ход, даже если герой достигает успеха только при выпадении 20 на кубике, и в этом случае он получает все дополнительные эффекты Критического успеха.
\newline Так же вы можете Критически провалить проверку героя. Вы можете использовать Ход, даже если герой достигает успеха автоматически (например, из-за высокого Навыка и отсутствия рисков). Для тех эффектов, которые это учитывают, считается, что на кубике выпало 1.
\end{itemize}
\begin{tcolorbox}
Зачем нужны автоматические провалы проверок? Действительно, такое применение Нитей кажется не самым очевидным. Но вы не только герой, вы его Судьба. Пока герой живет полной событиями жизнью, вы строите и развиваете сюжет. Наверняка герой будет счастлив завершить хлопотную приключенческую карьеру, обзавестись фермой, семьей и тихим безопасным хобби. Но хотите ли этого вы? Если у вас еще есть планы на героя, не давайте ему стать слишком благополучным.
\newline Помимо этого, игровые правила периодически вынуждают героя совершать действия, которые кажутся нежелательными игроку. Изменить их сюжетную направленность и эмоциональную окраску также помогут автоматические провалы проверок.
\end{tcolorbox}
\item \textbf{В нужном месте в нужное время.}
\begin{itemize}
\item[--] 1 или больше Нитей (на усмотрение мастера). Добавьте в сцену предмет, или элемент обстановки, или статиста. Ржавый нож, спрятанный в тюремном матрасе, незаметная дверь в тупике, охранник, прибежавший на крики о помощи, — приемлемые варианты.
\item[--] 1 или больше Нитей (на усмотрение мастера). Сделайте игровую заявку ретроспективно. Чем дальше во времени отстоит возможность реализации такой заявки, тем больше Нитей придется оборвать. Одно дело, когда герой еще вчера заходил в магазин, где мог приобрести все необходимое, и совсем другое, когда герой несколько месяцев странствует в пустоши, где не растет ничего, кроме жухлой травы!
\item[--] 1 или больше Нитей (на усмотрение мастера). Введите в игру Орла вашего героя. Подробнее об Орле читайте в разделе "Грани и Амплуа".
\item[--] 1 Нить. Ваш герой появляется в текущей Сцене, если он в данный момент не задействован в другой. Это позволит ему принять участие в драке, отпустить едкий комментарий или стать свидетелем события, вместо того, чтобы простаивать где-то вне рамок повествования. Мастер вправе заблокировать Ход, если никто не может объяснить, как герой попал в Сцену, или это противоречит ее контексту.
\end{itemize}
\begin{tcolorbox}
Игроки могут обрывать Нити своих героев, чтобы помочь героям \textit{других} игроков получить эффекты Хода "В нужном месте в нужное время".
\end{tcolorbox}
\item \textbf{Старый знакомый.}
\begin{itemize}
\item[--] 1 или больше Нитей (на усмотрение мастера). Добавьте в сцену \textit{симпатизирующего герою} статиста — родственника, друга, должника и т. п. Не забывайте, статист не может появиться из ниоткуда!
\item[--] 1 или больше Нитей (на усмотрение мастера). Добавьте факт биографии уже присутствующего в игре статиста — Недостаток, грязный секрет, черту характера или событие из прошлого. Вам потребуется объяснить, откуда герой узнал об этом, либо сослаться на контекст и детали экспозиции Сцены.
\item[--] 1 Нить. Введите в игру известный герою Недостаток/Темную сторону Атрибута/Решку персоны или статиста. Герой, Нить которого оборвана, обязан принимать участие в Сцене и посильно содействовать вводу Недостатка/Темной стороны Атрибута/Решки в игру, либо к этому должны располагать контекст и детали экспозиции Сцены. Имейте в виду, что в этом случае к персоне будет протянута 1 Нить.
\end{itemize}
\item \textbf{Рука Судьбы.}
\begin{itemize}
\item[--] 1 Нить. Герой немедленно восстанавливает \textbf{|2 + Модификатор обаяния|} Энергии. Энергия героя не может превышать максимальной величины его Характеристики. Совершение Хода в Боевой сцене требует Быстрого действия.
\item[--] От 1 до 3 Нитей (или больше на усмотрение мастера). Статист по выбору игрока, оборвавшего Нити своего героя, попадает в Неприятности. 1 оборванная Нить приведет к результату "Ну и денек!", 2 оборванных Нити – к результату "Кажется, у меня проблема!", 3 оборванных Нити – к результату "Катастрофа!". По взаимной договоренности игроки могут оборвать Нити нескольких разных героев, чтобы достичь нужного результата. Герои не обязаны присутствовать в одной сцене со статистами, попадающими в Неприятности.
\end{itemize}
\begin{tcolorbox}
Если герои завершают игровую встречу с неиспользованными Нитями, Рука Судьбы – отличная возможность подпортить денек антагонистам.
\end{tcolorbox}
\item \textbf{Единственный и неповторимый.}
\newline Каждый Атрибут позволяет герою сделать Уникальный ход, который недоступен героям и статистам без этого Атрибута. Стоимость может отличаться в зависимости от описания конкретного Хода.
\newline Герой может совершать Ход Атрибута и без обрыва Нитей. В этом случае на все проверки, которые должен совершить герой для успеха Хода, нельзя повлиять другими Ходами и Капризами (но Трюки, Функции и особые способности Атрибутов разрешены, если в их описании не указано обратного).
\item \textbf{Сделка с Судьбой.} Герой может передать любое количество Нитей другому герою, однако за каждую переданную Нить к Мастеру протягивается Темная Нить.
\item \textbf{Дежавю.}
\newline 2 Нити за каждого героя(но не статиста или персону) в сцене. Если игрокам не по нраву результат действий героев, они можгут объявить о применении Дежавю и вернуть их в начало Сцены.
\newline Это решение может быть принято только коллективно - если хоть один игрок считает, что события развиваются благоприятно, интересно или просто выгодно для героев, то он может заблокировать использование Хода.
\newline Этот ход может быть оплачен коллективно, то есть любые герои, учавствующие в сцене могут оборвать Нити для того чтобы добавить их в оплату хода.
\newline После применения Хода герои начинают Сцену заново со всеми доступными им на тот момент ресурсами, (такими, как патроны, медикаменты, Энергия, Единицы здоровья и доверие окружающих) кроме потраченных Нитей. Разумеется, герои не подозревают о вмешательстве Судьбы в свои дела, всего лишь испытывают мучительно ускользающее чувство воспоминаний о будущем.
\end{enumerate}

\section{Капризы судьбы}
\paragraph{Каприз Судьбы (Каприз) -} это ввод в игру Недостатка, Темной стороны Атрибута, Решки героя или принятие игроком худшего варианта Неприятностей до броска кубика.
\paragraph{Вход Каприза в игру.} Каждый раз, когда Каприз входит в игру, протяните к герою 1 Нить. Вход в игру означает создание ситуации, в которой Каприз осложняет жизнь героя. Игрок описывает затруднение, с которым столкнулся его герой, и, если мастер считает проблему достаточно серьезной, к герою протягивается 1 Нить.
\newline В течение Сцены мастер может (хоть и не обязан делать этого) ввести Капризы Судьбы по одному разу для каждого героя, принимающего участие в Сцене.
\newline Инициация проверки Неприятностей не считается Капризом, а вот попытка навязать герою ее худший вариант является им. Таким образом, навязанный худший вариант Общих неприятностей, исчерпает базовый лимит Капризов за Сцену.
\newline В течение Сцены мастер может (хоть и не обязан делать этого) ввести Капризы Судьбы по одному разу для каждого героя, принимающего участие в Сцене. Мастер не обрывает для этого Нити персон и не тратит Темные Нити. 
\newline Инициация проверки Неприятностей не считается Капризом, а вот попытка навязать герою ее худший вариант является им. Таким образом, навязанный худший вариант Общих неприятностей, исчерпает весь базовый лимит бесплатных Капризов за Сцену.
\newline Мастер так же располагает дополнительными возможностями для ввода Капризов более одного раза за Сцену для одного и того же героя, а именно: 
\begin{itemize}
\item[--]Нити персон-антагонистов. Обрывая их, мастер может применять Ходы из категории «Старый знакомый» в отношении героев. В дополнение к обычным правилам Мастер не вправе навязать герою новый Недостаток без согласия игрока.
\item[--]Темные Нити. Мастер может обрывать их для ввода в игру Капризов чаще одного раза за Сцену для одного героя. Мастер все еще не вправе ввести один и тот же Недостаток, Темную сторону Атрибута и Решку героя более одного раза за Сцену.
\end{itemize}
\begin{tcolorbox}
Предложить, как и описать Каприз Судьбы может любой из игроков или мастер. Не забывайте, игра – совместное творчество! Однако предложенные Капризы не являются обязательными для ввода в игру - окончательное решение принимают мастер и игрок, герой которого станет жертвой Каприза Судьбы.
\newline Мастеру рекомендуется четко разграничить случаи, когда он предлагает Каприз от случаев, когда он их уже вводит в игру.
\end{tcolorbox}
\paragraph{Отказ от Каприза.} Если игрок не желает принимать последствия Каприза, он должен совершить Ход Судьбы <<Повезло!>>. Если Каприз входит в игру, а у героя нет ни одной Нити, ему придется столкнуться с последствиями, хочет он того или нет. Разумеется, в этом случае к герою протягивается Нить.
\begin{tcolorbox}
Не забывайте, что если игрок применил Ход «Повезло!» и отказался от последствий, то Каприз не входит в игру – к герою не протягивается Нить.
\end{tcolorbox}
\section{Перечень капризов}
С одной стороны, Капризы Судьбы — это возможность для игроков обмениваться Нитями, с другой — элемент передачи повествовательных прав, с третьей — непостоянство своенравной Судьбы!
\begin{enumerate}
\item \textbf{Во власти страстей.}
Недостаток героя входит в игру. Ввод в игру Недостатков — основной способ получения Нитей Судьбы героем. Подробный перечень возможных Недостатков вы найдете в разделе «Недостатки».
\newline
Обратите внимание, что герой должен попасть в неприятность из-за своего недостатка.
\item \textbf{Все имеет цену.}
Темная сторона Атрибута героя входит в игру. В отличие от Недостатков, проявления Темной стороны достаточно ситуативны, а потому целиком и полностью отданы фантазии игроков и мастера. Не забывайте, что ввод в игру Темной стороны (так же, как и Недостатка) должен осложнять герою жизнь и создавать возможности для развития сюжета. Примеры Темных сторон каждого из Атрибутов вы найдете в разделе «Атрибуты».
\item \textbf{Решка!}
Решка героя входит в игру. Так же, как и в случае с Темной стороной Атрибутов, Решка — очень специфический способ получения Нитей и всецело зависит от контекста сцены. Подробнее о Решке читайте в разделе «Грани и Амплуа».
\item \textbf{Катастрофа!}
Игрок добровольно сталкивает героя с наихудшим вариантом проверки Неприятностей. Считайте, что на кубике выпало 1. Подробнее об этом читайте в разделе «Неприятности».
\item \textbf{Черная полоса.}
Теоретически, герои могут начать игру без Атрибутов, Недостатков и Граней. Вряд ли герой так уж идеален, но ни одна из страстей не способна захватить его целиком даже на мгновение. И все же он остается игрушкой в руках Судьбы. Во время игры постоянно возникают ситуации, так или иначе подвергающие героя опасности. Игрок может предложить неблагоприятный вариант развития такой ситуации.
\end{enumerate}
\textbf{\textit{Вход в игру Каприза Судьбы значит, что в истории возник некий свершившийся факт. Но вход Недостатка, Темной стороны или Решки в игру не обязательно означает безусловные проблемы.
\newline Прежде всего, это создание интересной игровой ситуации, которая начинается неблагоприятно для героя, но может принести выгоды позднее, если Судьба в лице игрока подскажет герою верную линию поведения.}}

%\section{Конец игровой встречи}




На самом интересном месте

\chapter{СОЗДАНИЕ ГЕРОЯ}
Здесь вы узнаете о том, как герои и статисты устроены с точки зрения игровой механики, что дается им легко, а что вызывает затруднения, и как именно Судьба может направить их и помочь им. Что вам понадобится для создания героя? Прежде всего, мысленно представить его. Откуда он? Каков его характер? Что он любит и что ненавидит? Есть ли у него друзья или враги? Есть ли у него семья? В чем он хорош, а в чем — не очень? Сколько ему лет? Чем зарабатывает на жизнь? И, конечно же, как его зовут? 
\newline
Ответив на эти вопросы, вы получите примерно абзац текста. Это — ваша подсказка. Какие-то детали непременно останутся за ее пределами, но о них вы сможете вспомнить уже во время игры — если захотите. В конце концов, большую часть образа раскрывают действия героя во время игры.
\section{У ГЕРОЯ ЕСТЬ}
\begin{itemize}
\item[--] 6 Основных и 4 Вторичных характеристики, задающие пределы физических и ментальных возможностей.
\item[--] 3 Боевые характеристики, отвечающие за успехи героя в бою.
\item[--] Навыки, отвечающие за степень подготовки к всевозможным жизненным ситуациям.
\item[--] Атрибуты — важнейшие детали образа и таланты, определяющие род занятий.
\item[--] Трюки — ловкие приемы или качества, способные серьезно облегчить жизнь.
\item[--] Недостатки, изображающие слабости и страсти.
\item[--] Ячейки для установки Имплантов(если в мире умеют изготавливать и устанавливать их). Начальное число ячеек равно \textbf{|3 + Модификатор размера|} героя.
\item[--] Амплуа и Грани, определяющие отношения героя с миром и место в нем.
\item[--] Узы, показывающие внутренние убеждения героя.
\item[--] Богатство, отражающее благосостояние.
\end{itemize}

\section{После определения основных и вторичных характеристик}
\begin{itemize}
\item[--] Выберите для героя 2 Атрибута. Герой может отказаться от 1 Атрибута и получить 5 дополнительных Очков опыта. Если герой отказывается от 2 Атрибутов сразу, он получает 10 дополнительных Очков опыта (всего!).
\item[--] Выберите для героя 2 Трюка.
\item[--] Выберите для героя 0—2 Недостатка (на старте не рекомендуется повторять их у героев разных игроков).
\item[--] Выберите для героя число Граней, не превышающее число Атрибутов (на старте не рекомендуется повторять их у героев разных игроков).
\item[--] Выберите для героя 0—2 Уз (на старте не рекомендуется повторять их у героев разных игроков).
\item[--] Распределите 10 Очков опыта по Навыкам героя.
\end{itemize}
\begin{tcolorbox}
Атрибуты и Трюки наделяют героя огромным количеством способностей. Если вы хотите отразить в игре процесс становления героев, а не авантюру уже состоявшихся искателей приключений, можно начать игру с одним Трюком и одним Атрибутом.
\end{tcolorbox}

\section{Характеристики героя}
Очки характеристик: создавая героя, вы распределяете 70 Очков характеристик по Основным характеристикам. Очки распределяются в любых сочетаниях. Например, вы можете распределить 16 очков в любую одну Характеристику, получив значение 16. Если вы распределите 16 Очков по двум Характеристикам, то можете получить значения 9 и 7, 10 и 6 и т. д.

\subsection{Начальная и максимальная величина Характеристик}
При создании героя ни одна из Характеристик не может превышать 18. Игрок не вправе поднять Характеристики выше этого предела, даже если у героя есть свободные Очки Характеристик или Очки опыта. В дальнейшем, ни одна из Характеристик героя не может превысить 20 за счет траты Очков опыта, но может превысить 20 при помощи стимуляторов, аугментаций, специальных способностей и т.д.
\newline Базовые значения Основных Характеристик героев могут быть абсолютно любыми - в пределах от 1 до 20.
\newline Мастер может уменьшать или увеличивать количество Очков характеристик. Следует, однако, помнить, что герои, созданные меньше, чем на 60 очков, не вполне приспособлены к приключениям. 70 Очков характеристик - оптимальный вариант, так как в этом случае придется выбрать, в чем герой будет хорош, а в чем - не очень.
\newline Для антагонистов и прочих значимых фигур мастер вправе использовать любое количество Очков характеристик, даже превышающее то, на которое созданы герои. Для существ под управлением мастера не действует ограничение на максимальную величину Характеристик (хотя его стоит держать в голове).

\subsection{Основные характеристики}
\paragraph{Сила (Сл):} показывает, насколько герой развит физически. От этой Характеристики зависит, какой вес может нести герой, насколько далеко и высоко он прыгает, и ущерб, который он причиняет оружием, использующим мускульную силу. Также параметр Силы влияет на то, какое оружие герой сможет успешно применять.

\paragraph{Ловкость (Лв):} отвечает за быстроту и координацию движений. Эта Характеристика так же важна для воина, как и Сила. Еще она пригодится герою, который собирается стрелять, карабкаться по руинам и водить багги.

\paragraph{Выносливость (Вн):} пригодится любому герою. Высокая Выносливость означает, что герой крепок телом, редко болеет и легко переносит ранения. Также от этой Характеристики зависит, в каких доспехах герой сможет эффективно действовать.

\paragraph{Интеллект (Ин):} помогает запоминать информацию, делать выводы, а также учиться на своих и чужих ошибках. Интеллект необходим любому герою, который желает иметь высокие параметры Навыков - именно он устанавливает предел их роста.
\newline Герой с, например, 2 Интеллектом не сможет распределить в любой Навык больше 2 Очков опыта. Это не помешает ему относительно связно выражать мысли, найти работу, завести семью и вообще наслаждаться жизнью. Особенно, если удастся закорешиться с действительно умными ребятами, которые растолкуют, что к чему. 

\paragraph{Мудрость (Мд):} включает в себя находчивость, наблюдательность, бытовое здравомыслие и глубинные инстинкты. Именно Мудрость поможет герою вовремя заметить опасность… или просто избежать ее.
\newline Благодаря пассивным проверкам Наблюдательности, Мудрость - весьма важная Характеристика. Нужна она и героям, которые желают действовать раньше остальных, так как Мудрость входит в состав Вторичной Характеристики "Реакция".

\paragraph{Обаяние (Об):} позволит наладить контакт с окружающими и понравиться им, не особенно усердствуя. Этот параметр незаменим для того, кто предпочитает действовать исподволь и добиваться своего без применения физического насилия. Помимо этого, Обаяние определяет, в каком ключе герой воспринимает окружающий мир. Высокое Обаяние - залог оптимизма!
\newline Обаяние никак не связано с привлекательностью, хотя обаятельные герои часто кажутся окружающим симпатичными. 

\begin{tcolorbox}
    Игромеханически за внешнюю привлекательность отвечают Атрибут "Красивый", а так же Трюки "Соблазнительный" и "Стильный". 
    \newline Если ваш герой не планирует трепать языком, располагать к себе статистов, использовать Функции и феномены - Обаяние последняя в списке значимых для него Характеристик. Тем, кто не в восторге от героя из-за плохих проверок Впечатления, всегда можно отстрелить уши, отрубить тестикулы или просто от души наварить в торец.
\end{tcolorbox}

\subsection{Модификаторы Характеристик (М)} в большинстве формул используется не полное значение Основных характеристик, а \textbf{Модификатор = | (Основная характеристика - 10) ÷ 2|}.
\begin{center}
\begin{tabular}{ |c|c| }
\hline
\textbf{Основная характеристика} & \textbf{Модификатор} \\ \hline
  1 & -5 \\ \hline
  2-3 & -4 \\ \hline
  4-5 & -3 \\ \hline
  6-7 & -2 \\ \hline
  8-9 & -1 \\ \hline
  10-11 & 0 \\ \hline
  12-13 & 1 \\ \hline
  14-15 & 2 \\ \hline
  16-17 & 3 \\ \hline
  18-19 & 4 \\ \hline
  20 & 5 \\ \hline
\end{tabular}
\end{center}
\subsection{Проверки Основных характеристик}
Проверка Основной характеристики является Проверка с Бонусом, равным модификатору Основной характеристики.

\subsection{Размеры существ}
Правила берут за эталон существо ростом от 150 до 210 см. Оно имеет Средний размер. Но в сюжетах найдется немало отклонений от этого стандарта - мутанты, роботы, киборги и \tbd.
\newline Размер не влияет на Основные характеристики напрямую, однако большие существа - легкая мишень для атак, а маленькие существа вынуждены использовать маленькое оружие, наносящее меньше ущерба.
\newline Существа занимают область в зависимости от размеров. Это не означает, что существо занимает область целиком (хотя бывает и такое). Несомненно одно - в ней существо может и будет мешать передвижению недругов.
\newline Высота в холке четвероногих существ зачастую меньше, чем рост существ соответствующего размера, указанный в таблице.

\paragraph{Каков Размер моего героя?} Игрок вправе начать приключение Крошечным, Маленьким или Большим героем. 

\paragraph{Модификатор размера (МРз):} наравне с Выносливостью определяет, насколько существо или объект восприимчивы к урону. Чем больше цель атаки, тем сложнее нанести ей серьезный вред без специального снаряжения.
\newline Размер, отличный от Среднего, выгоден в одних ситуациях и мешает в других. Небольшому существу легче прятаться и избегать ударов, а крупному - атаковать противника, используя массу и габариты. 
\paragraph{На что влияет МРз?} МРз прибавляется к Скорости, результатам проверок Силы, Доблести, Меткости метания, и Навыков (Сл).
\newline МРз вычитается из БАЗщ, результатов проверок Меткости, Ловкости и Навыков (Лв).
\begin{tcolorbox}
    Не забывайте, минус на минус дает плюс. Это означает, что, например, Крошечное существо фактически повышает свою БАЗщ, Меткостьт, МЛв и навыки от Ловкости на 2.
    %Также обратите внимание, что Меткость Метания крупных существ не изменится, так как МРз одновременно и прибавляется к ней, и отнимается от нее.
\end{tcolorbox}

Некоторые способности учитывают размеры существ. В этом случае Большое существо считается как 2 Средних, Огромное - как 3 Средних, а Громадное - как 4!
\begin{center}
    \begin{tabular}{ |c|c|c|c|c| }
        \hline
        Размер & Модификатор размера & Возможный рост & Занимаемая область
        \\ \hline
        Крошечный(К) & -2 & 0.01-0.65 метра & 0.5 × 0.5 метра
        \\ \hline
        Маленький(М) & -1 & 0.66-1.49 метра & 1 × 1 метра
        \\ \hline
        Средний(С) & 0 & 1.5-2.1 метра & 2 × 2 метра
        \\ \hline
        Большой(Б) & +1 & 2.2-3 метра & 3 × 3 метра
        \\ \hline
        Огромный(О) & +2 & 3.01-9.99 метра & 5 × 5 метров
        \\ \hline
        Громадный(Г) & +3 & 10 метров и больше & 7 × 7 метров или даже больше
        \\ \hline
    \end{tabular}
\end{center}
\paragraph{Исполинские (И) существа и устройства.} Иногда существо или устройство немыслимо велики, а герои перед ними абсолютно незначительны. Исполинские существа не принимают участия в Сценах, они и есть Сцена, на которой разворачивается действие. Герои и существа, даже Громадные, могут взаимодействовать только с частью Исполина, а не с ним целиком.

\subsection{Вторичные характеристики}
\paragraph{Воля (Вл) = (Ин + Мд) ÷ 4.} Отвечает за самоконтроль, сопротивление всевозможным соблазнам и внешним влияниям.

\paragraph{Реакция (Рц) = (Лв + Мд) ÷ 4.} Определяет порядок действия в Боевых сценах и прочих ситуациях, в которых это важно.

\paragraph{Скорость (Ск) = (Лв + Вн) ÷ 4 + МРз.} За 5 секунд герой может преодолеть число метров (или клеток, если используется масштабная карта), равное своей Ск.
\paragraph{Ск и четвероногие существа:} четвероногие существа, такие, как кони, кошки и слоны, увеличивают свою Ск в 2 раза, перемещаясь по земле. Умножьте Ск после прибавления или вычитания МРз. Тараканы, пауки и многоножки считаются четвероногими для определения Ск, а гигантские слизни, улитки и змеи - нет!
\paragraph{Ск и [Полет]:} если существо способно летать, умножьте его Ск на 3 после прибавления или вычитания МРз. Если существо по каким-то причинам не может использовать [Полет], применяйте его обычную Ск.

\paragraph{Энергия (Эн) = (Вн + Об) ÷ 4.} Отражает абстрактный запас внутренних сил. Источником Энергии героя может служить оптимизм, ненависть, неукротимый дух, вездесущий эфир, миниатюрный ядерный реактор и множество других вещей и явлений, как философско-мистического, так и обыденного толка.
\newline Энергия расходуется, когда герой активирует феномен или Функцию Атрибута - невероятные способности, недоступные большинству окружающих. Текущее значение Энергии не может опуститься ниже 0, либо превысить ее максимальное значение.

\paragraph{Чтобы восстановить Энергию, герой должен:}
\begin{itemize}
    \item[--] Уйти в Антракт и восстановить Эн до максимального значения.
    \item[--] Воспользоваться Интерлюдией с пометкой "Отдых", чтобы восстановить \textbf{|МОб|} (минимум 1) Эн.
    \item[--] Употребить быстродействующие стимуляторы или зелья.
    \item[--] Преобразовать топливо в Энергию, если снаряжение или способности героя предоставляют такую возможность. Для того чтобы восстановить 1 Эн, следует потратить 100 Зарядов (Зр).
    \item[--] Применить Ход "Рука Судьбы", чтобы немедленно восстановить \textbf{|2 + МОб|} (минимум 1) Эн.
    \item[--] Трюки и Атрибуты также могут позволить герою восстанавливать Эн. 
\end{itemize}

\paragraph{Единицы Здоровья (ЕЗ) = |Вн*(МРз+3)|.} Показывают, сколько ущерба герой способен вынести, прежде чем потеряет сознание или умрет. Текущее значение ЕЗ не может опуститься ниже 0, либо превысить их максимальное значение.
\begin{tcolorbox}
  Единицы Здоровья являются условностью, позволяющей отслеживать боеспособность героя и его состояние в целом. Впрочем, эта условность достаточно правдоподобна, чтобы пользоваться ею в любых историях.
\end{tcolorbox}
Большинство существ величиной с человека имеют Средний размер. Их ЕЗ = \textbf{|Вн*3|}.
\newline Единицы Здоровья расходуются, когда герой получает Повреждения, или теряет Единицы Здоровья по каким-то причинам. 
\paragraph{Чтобы восстановить ЕЗ, герою придется:}
\begin{itemize}
    \item[--] Уйти в Антракт, чтобы восстановить \textbf{|МВн|} (Минимум 1) ЕЗ.
    \item[--] Использовать Интерлюдию "Посещение врача", "Расслабляющее ничегонеделание" или "Сон". Такая Интерлюдия может дополнять Антракт или сочетаться с ним.
    \item[--] Применить Навык Медицины во время Антракта.
    \item[--] Употребить быстродействующие стимуляторы или снадобья.
    \item[--] Использовать Атрибуты, Трюки, феномены и другие умения героя и его союзников.
\end{itemize}

У Вторичных Характеристик нет Модификаторов. Во всех формулах и при проверках используется их полное значение. В случае Энергии используется значение текущей Эн. Единицы Здоровья не используются для каких-либо проверок.
\newline Если при определении величины Вторичной характеристики результат деления получился меньше 1, то значение Вторичной характеристики все равно составит 1. 

\subsection{Проверки Вторичных Характеристик}
Проверка Вторичных характеристики является Проверка с Бонусом, равным значению Вторичной характеристики.

\subsection{Боевые характеристики}
\paragraph{Повреждения (Пв):} результатом успешных проверок Боевых Характеристик (за исключением Защиты) является получение Повреждений целью атаки.
\paragraph{Бонус к Повреждениям (БПв)} дается оружием и особыми способностями. Чем он выше, тем больше шансов Повредить цель. Бонус к Повреждениям является частью Боевых Характеристик Доблести и Меткости, и изменяется в зависимости от того, какое оружие или феномен использует герой.
\paragraph{Естественное оружие и Безоружный Бонус к Повреждениям (ББПв):} отражают ущерб, который герой наносит, используя свое тело - кулаки, пятки, хвост, клыки или рога. ББПв = \textbf{|МРз - 2|}, и может иметь отрицательное значение. Например, у героя Среднего размера он составит -2.
\paragraph{Доблесть (Дб) = |Нв Владения оружием + МСл + МЛв + БПв + МРз|.} Чем выше Доблесть героя, тем он опаснее в ближнем бою.
\paragraph{Рукопашная Доблесть (РДб) = |Нв Рукопашного боя + МСл + МЛв + ББПв + МРз|.} Эта Характеристика используется героем для нанесения ударов ногами, руками, клыками, когтями и другими частями тела, а также для совершения некоторых маневров.
\newline РДб является частным случаем проверки Доблести.
\newline Если правила предписывают проверить Доблесть, примените БПв оружия, которое использует герой.
\paragraph{Меткость (Мт) = |Нв Стрельбы + МЛв + БПв - МРз|.} Чем выше Меткость, тем опаснее герой в дистанционном бою. 
\paragraph{Меткость Метания (ММт) =|Нв Стрельбы + МСл + МЛв + БПв + МРз|.} Она используется при метании предметов, гранат и использовании Метательного оружия (луков и пращей в том числе). 
\paragraph{Меткость Снарядов (МтС) = |Нв Стрельбы + МФх (модификатор Феноменальной характеристики) + БПв - МРз|.} Она используется при активации феноменов - невероятных и сверхъестественных способностей.
\newline ММт и МтС являются частными случаями проверки Меткости. Если правила предписывают проверить Меткость, примените БПв оружия, которое использует герой.
\begin{tcolorbox}
	Обратите внимание, что для Меткости Метания Модификатор Размера добавляется так же, как и для Доблести, а не отнимается, как в обычной Меткости.
\end{tcolorbox}
\paragraph{Защита (Зщ) = |БАЗщ + МЛв + БД + БЩ|.} Чем выше Защита, тем сложнее поразить героя атаками. Помимо МЛв в Защиту входят:
\paragraph{Базовая защита (БАЗщ) = |10 - МРз|.} Она отражает Защиту неподвижного существа. Высокая БАЗщ обычно указывает на небольшой размер или иные факторы, затрудняющие попадание, но не связанные с подвижностью или броней. 
\begin{center}
    \begin{tabular}{ |c|c|c|c|c| }
        \hline
        Размер существа & Базовая защита
        \\ \hline
        Крошечный(К) & 12
        \\ \hline
        Маленький(М) & 11
        \\ \hline
        Средний(С) & 10
        \\ \hline
        Большой(Б) & 9
        \\ \hline
        Огромный(О) & 8
        \\ \hline
        Громадный(Г) & 7
        \\ \hline
    \end{tabular}
\end{center}

\paragraph{Бонус доспеха к Защите (БД):} защита, которую дают доспехи. Бонус доспеха изменяется в зависимости от того, какой доспех носит герой.
\paragraph{Бонус щита к Защите (БЩ):} защита, которую дают щиты. Бонус щита изменяется в зависимости от того, какой щит использует герой.
\newline Ношение некоторых доспехов и щитов ограничивает максимальный модификатор Ловкости, который можно использовать при подсчете Защиты и прочих проверках.
\paragraph{Прочность (Прч):} отражает особенности строения или конструкции существ и объектов, позволяющие противостоять ущербу. Прочность предотвращает полученные Повреждения и вычитается из них.
\newline Прочность не является Боевой Характеристикой, но тесно связана с ними.

\subsection{Проверки Боевых характеристик}
Проверка Боевой характеристики (кроме Защиты) является Проверка с Бонусом, равным значению Боевой характеристики.
\newline Герой может понизить значение боевой характеристики (но не повысить, если характеристика \textit{уже} отрицательная) вплоть до нуля во время атаки, если желает нанести удар не в полную силу.
\newline сложность проверки обычно равна Защите цели, которую выбрал герой.
\newline Подробнее о проверках Боевых Характеристик читайте в разделе "Маневры".

\paragraph{Проверки Защиты:} в Динамических сценах, не подразумевающих детализации - например, когда герой прорывается сквозь разъяренную толпу, бежит под обстрелом или мчится по коридору с ловушками, мастер вправе определить Повреждения героя при помощи проверки Защиты - Проверки с бонусом, равным \textbf{|Зщ - 10|}.

\subsection{Нулевой уровень характеристик}
В приключениях героев подстерегает немало опасностей, и далеко не все из них явные. Яды, болезни, зараженная вода и пища могут понизить значение любой Характеристики до нуля. Понижение Основной Характеристики, даже временное, приводит к понижению зависящей от нее Вторичной.
\newline Нулевой уровень Характеристики означает, что существо автоматически проваливает любые проверки, связанные с ней. Упавшие до нуля Сила, Ловкость и Скорость приводят к состоянию, близкому к параличу (хотя существо все слышит и может наблюдать за происходящим вокруг). Упавшие до нуля Интеллект, Мудрость и Обаяние приводят к коме. Если Реакция или Скорость существа падает до 0, оно впадает в ступор. Во всех этих случаях существо считается находящимся в состоянии Неподвижности. Существо с нулевой Волей покорно выполняет любые отданные ему приказы, даже очевидно самоубийственные. Если существо получит несколько приказов, противоречащих друг другу, то оно постарается выполнить их, удовлетворив максимальное число требований. Если Выносливость существа падает до нуля, то оно немедленно умирает или разваливается на части.
\newline Исключение - Боевые Характеристики. Даже с отрицательными значениями Боевых Характеристик герой может пытаться атаковать противников, каким бы безнадежным это не казалось.

\section{Начальная и максимальная величина характеристик}
\paragraph{}
{\color{orange}При определении Основных Характеристик героя ни одна из них не может превышать значения 18. Это значит, что игрок не может поднять Характеристики героя выше указанных пределов, даже если у него есть свободные Очки характеристик или Очки опыта. В дальнейшем, ни одна из Основных Характеристик героя не может превышать 20. 
\newline
При определении Вторичных Характеристик, кроме Здоровья, ни одна из них не может превышать значения 9 при старте и максимального значения 12 в последствии.
\newline
Здоровье героя не может превышать значения 60 при старте и максимального значения 80 в последствии.}

Мастер может уменьшать или увеличивать количество Очков характеристик, если это соответствует жанру и настроению игры. Следует, однако, помнить, что герои, созданные меньше чем на 60 очков, не вполне приспособлены к приключениям. 70 Очков характеристик — оптимальный вариант, так как в этом случае придется выбрать, в чем герой будет хорош, а в чем — не очень.
\newline
Для антагонистов и прочих значимых фигур мастер может использовать любое количество очков, даже превышающее то, на которое созданы герои. Для существ под управлением мастера не действует ограничение на максимальные Характеристики (хотя его все же стоит держать в голове).


\section{Основные характеристики}
\paragraph{Сила (Сл)} показывает, насколько герой развит физически. От этой Характеристики зависит, какой вес может нести герой, и ущерб, который он причиняет оружием, использующим мускульную силу. Также параметр Силы влияет на то, какое оружие герой сможет успешно применить в принципе.
\paragraph{Ловкость (Лв)} отвечает за быстроту и координацию движений. Эта Характеристика так же важна для воина, как и Сила. Еще она пригодится герою, который собирается стрелять, срезать кошельки и карабкаться по деревьям.
\paragraph{Выносливость (Вн)} пригодится любому герою. Высокая Выносливость означает, что герой крепок, редко болеет и легко оправляется от ран. Также от этой Характеристики зависит, в какой броне может эффективно действовать герой.
\paragraph{Интеллект (Ин)} помогает запоминать информацию и учиться на своих и чужих ошибках. Интеллект необходим любому герою, который желает иметь высокие параметры Навыков — именно он задает пределы их роста.
\paragraph{Мудрость (Мд)} включает в себя находчивость, наблюдательность, здравомыслие и глубинные инстинкты. Именно Мудрость поможет герою вовремя заметить опасность... или просто избежать ее.
\paragraph{Обаяние (Об)} позволит наладить контакт с окружающими и понравиться им, не особенно усердствуя. Этот параметр незаменим для того, кто предпочитает действовать исподволь и добиваться своего без применения насилия. Помимо того, Обаяние определяет, в каком ключе герой воспринимает окружающий мир. Высокое Обаяние — залог оптимизма! Обаяние никак не связано с внешней привлекательностью героя, хотя обаятельные герои часто кажутся окружающим симпатичными.
\paragraph{Модификаторы Характеристик (М)} в большинстве формул используется не полное значение Основных характеристик, а \textbf{Модификатор = | (Основная характеристика — 10) ÷ 2|}.
\begin{center}
\begin{tabular}{ |c|c| }
\hline
\textbf{Основная характеристика} & \textbf{Модификатор}
\\ \hline
1 & -5
\\ \hline
2-3 & -4
\\ \hline
4-5 & -3
\\ \hline
6-7 & -2
\\ \hline
8-9 & -1
\\ \hline
10-11 & 0
\\ \hline
12-13 & 1
\\ \hline
14-15 & 2
\\ \hline
16-17 & 3
\\ \hline
18-19 & 4
\\ \hline
20 & 5
\\ \hline
\end{tabular}
\end{center}
Проверки Основных характеристик совершаются следующим образом:
\begin{enumerate}
\item Бросьте К20 и прибавьте к нему модификатор Основной характеристики.
\item Сравните получившееся число со сложностью проверки. Герой преуспевает, если число равно сложности или превышает ее.
\end{enumerate}
\section{РАЗМЕРЫ СУЩЕСТВ}
\paragraph{}
Во время приключений герои могут встретить множество существ, размером превышающих человека или ощутимо уступающих ему: от сказочных драконов и фей, до гиганских человекоподобных роботов и мозговых червей. За эталон принят человек ростом от 150 до 210 см. Он имеет Среднюю категорию размера. Размер не влияет на Основные характеристики существа напрямую, однако большие существа — легкая мишень для атак любого рода, а маленькие существа вынуждены использовать маленькое оружие (зачастую наносящее меньше ущерба). Герой-человек может по договоренности с мастером начать игру Маленьким (карлик) или Большим (великан).
\newline
Существа занимают определенную область в зависимости от своих размеров. Это вовсе не означает, что существо занимает эту область целиком (хотя бывает и такое). Несомненно одно — в этой области существо может и будет мешать передвижению недругов. Обратите внимание, что высота в холке четвероногих существ зачастую меньше, чем рост существ соответствующего размера, указанный в таблице.
\paragraph{Модификатор Размера (МРз):} размер, отличный от Среднего, может быть выгоден в одних ситуациях и мешать в других. Небольшие существа более подвижные, а крупным не требуется много усилий(по их меркам), чтобы сдвинуть препятствие в сторону.
\newline
Модификатор размера добавляется к Скорости, проверкам Доблести, Меткости Метания, Силы и навыкам от Силы. И вычитается из Защиты, проверок Меткости, Ловкости и навыков от Ловкости.
\newline
Некоторые способности учитывают размеры существ. У этих способностей есть свойство Размер Имеет Значение(РИЗ). В этом случае Большое существо считается как 2 Средних, Огромное — как 3 Средних, а Громадное — как 4!
\begin{center}
\begin{tabular}{ |c|c|c|c|c| }
\hline
Размер & МРз & Возможный рост & Занимаемая область
\\ \hline
Крошечный(К) & -2 & 0.01-0.65 метра & 0.5 × 0.5 метра
\\ \hline
Маленький(М) & -1 & 0.66-1.49 метра & 1 × 1 метра
\\ \hline
Средний(С) & 0 & 1.5-2.1 метра & 2 × 2 метра
\\ \hline
Большой(Б) & +1 & 2.2-3 метра & 3 × 3 метра
\\ \hline
Огромный(О) & +2 & 3.01-9.99 метра & 5 × 5 метров
\\ \hline
Громадный(Г) & +3 & 10 метров и больше & 7 × 7 метров или даже больше
\\ \hline
\end{tabular}
\end{center}
\begin{tcolorbox}
Не забывайте, минус на минус дает плюс. Это означает, что, например, Крошечное существо фактически повышает свою БАЗщ, Меткостьт, МЛв и навыки от Ловкости на 2.
%Также обратите внимание, что Меткость Метания крупных существ не изменится, так как МРз одновременно и прибавляется к ней, и отнимается от нее.
\end{tcolorbox}
\paragraph{Исполинские(И) существа и устройства.} Иногда существо или устройство настолько велики, а герои перед ним настолько незначительны, что они не могут взаимодействовать друг с другом обычными способами. Исполинские существа не принимают участие в Сценах, они и есть Сцена, на которой может происходить действие, а герои и существа, даже Громадные, могут взаимодействовать только с частью Исполина, а не с ним целиком.
%\documentclass[Нити судьбы.tex]{subfiles}
%\graphicspath{{\subfix{../images/}}}
%\begin{document}

\section{Вторичные характеристики}
\paragraph{Реакция (Рц) = | (Лв + Мд) ÷ 4|.} Реакция определяет порядок действия в боевых сценах и прочих ситуациях, в которых это важно. У Вторичных характеристик нет модификаторов. Во всех формулах используется их полное значение. Если результат деления меньше 1, то значение Вторичной характеристики все равно составит 1.
\paragraph{Воля (Вл) = | (Ин + Мд) ÷ 4|.} Отвечает за самоконтроль, сопротивление враждебным Могуществам и обычным соблазнам.
\paragraph{Скорость (Ск) = | (Лв + Вн) ÷ 4 + МРз|.} За 5 секунд времени герой может преодолеть число метров (или клеток, если при игре используется масштабная карта), равное своей Скорости.
\paragraph{Четвероногие существа:} четвероногие существа, такие как кони, кошки и гиганские ящеры, увеличивают свою Ск в 2 раза, перемещаясь по земле. Умножьте Ск после прибавления МРз. Тараканы, пауки и многоножки считаются четвероногими для определения Ск, а гигантские слизни, улитки и змеи — нет!
\paragraph{Полет:} если существо способно летать, его Ск полета в 3 раза выше, чем наземная Ск. Полет может использоваться до тех пор, пока нагрузка существа не превышает Комфортную. Если существо по каким-то причинам не может использовать полет, применяйте обычную Ск существа.
\paragraph{Единицы Здоровья (ЕЗ) = |Вн × (МРз+3)|.} Эта величина показывает, сколько ущерба герой способен вынести, прежде чем потеряет сознание или умрет.
\paragraph{Энергия (Эн) = | (Вн + Об) ÷ 4|.} Энергия определяет максимальный запас внутренних сил героя, которые он может использовать для творения заклинаний и активации некоторых предметов и атрибутов.
\newline
Текущий уровень энергии не может быть ниже нуля и выше максимальной Эн. Восстановить энергию герой может разными способами:
\begin{itemize}
\item[--] Во время Антракта герой полностью восстанавливает свою Энергию.
\item[--] Во время Интерлюдии герой восстанавливает Энергию в размере \textbf{|МОб|(мин. 1)}.
\item[--] Использовать зелья и стимуляторы для того чтобы восстановить Энергию.
\item[--] Иногда у героя есть возможность преобразовать топливо своего мира в собственную Энергию. Для того, чтобы восстановить 1 Эн нужно потратить 100 Зарядов.
\item[--] Трюки и Атрибуты тоже могут позволить герою восстанавливать Энергию. Способы могут быть как простыми, так и экзотическими. Смотрите описание соответствующих Трюков и Атрибутов для уточнения этих способов.
\end{itemize}
\paragraph{}Проверки Вторичных характеристик совершаются следующим образом:
\begin{enumerate}
\item Бросьте К20 и прибавьте к нему значение Вторичной характеристики.
\item Сравните получившееся число со сложностью проверки. Герой преуспевает, если число равно сложности или превышает ее.
\end{enumerate}
%\end{document}
\section{Боевые характеристики}
\paragraph{Повреждения (Пв):} результатом успешных проверок Боевых характеристик (за исключением Защиты) являются Повреждения — потеря Единиц Здоровья от атак и вредоносных эффектов.
\paragraph{Бонус к Повреждениям (БПв)} обычно дается оружием и Феноменам в форме Снаряда. Чем он выше, тем больше шансов у героя нанести цели Повреждения. Бонус к Повреждениям является частью Боевых характеристик, Доблести и Меткости, и может изменяться в зависимости от того, какое оружие или способность использует герой.
\paragraph{Доблесть (Дб) = |Владение оружием + МРз + МСл + МЛв + БПв|.} Чем выше Доблесть героя, тем он опаснее в ближнем бою.
\newline \textbf{Рукопашная Доблесть} вычисляется по формуле \textbf{|Рукопашный бой + МРз + МСл + МЛв + БПв(Безоружный)|}. Безоружная Доблесть используется для нанесения Безоружных ударов руками, ногами, клыками, когтями и другими частями собсвтенного тела, а так же для совершения некоторых маневров.
\paragraph{Меткость (Мт) = |Стрельба - МРз + МЛв + БПв|.} Чем выше Меткость, тем опаснее герой в дистанционном бою.
\newline \textbf{Меткость Метания} вычисляется по формуле: \textbf{Мт = |Стрельба + МРз + МСл + МЛв + БПв|.} Она используется при метании предметов, гранат и использовании Метательного оружия(Луков в том числе). Обратите внимание, что для Меткости Метания Модификатор Размера добавляется так же, как и для Доблести, а не отнимается, как в обычной меткости.
\newline \textbf{Меткость Снарядов} вычисляется по формуле: \textbf{Мт = |Стрельба - МРз + МФх(модификатор Феноменальной характеристики) + БПв|.} Она используется при использовании Феноменов(см. соответствующую главу.)
\paragraph{Защита (Зщ) = |Базовая защита + МЛв + БД + БЩ|.} Чем выше Защита, тем сложнее поразить героя атаками.
\newline \textbf{Базовая защита(БАЗщ) = |10 - МРз|} определяет, насколько сложно попасть по цели в зависимости от ее размера.
\newline \textbf{БД(Бонус Доспеха)} равна сумме Бонусов к Защите, котрые дают надетые на героя доспехи и толстая шкура, если герой ей обзавелся.
\newline \textbf{БЩ(Бонус Щита)} равен сумме Бонусов к Защите, которые дают используемые героем щиты. Обычно герой может носить не более двух щитов одновременно, но если он обзаведется дополнительными руками или найдет летающий щит, он может использовать и больше.
\paragraph{Прочность (Прч):} камень, сталь, лед и многие другие материалы имеют показатель Прочности. Прочность вычитается из Повреждений, нанесенных цели. Некоторые существа — например, стальные големы и ожившие деревья — также обладают Прочностью! Прочность не является частью Боевых характеристик, но тесно связана с ними.
\newline Прочность, полученная героем из разных источников складывается.
\paragraph{Проверки Боевых характеристик:}
\begin{enumerate}
\item Бросьте К20 и прибавьте к нему значение Доблести или Меткости героя.
\item Сравните получившееся число с Защитой цели. Цель получает 1 Повреждение за каждую 1, на которую атакующий герой преодолел Защиту цели.
\end{enumerate}
Например, если герой с Доблестью 10 атаковал статиста с Защитой 18 и на К20 выпало 14, статист получит 10 +14 — 18 = 6 Повреждений.
\newline
Подробнее о проверках Боевых характеристик читайте в разделе "Маневры".
\paragraph{Проверки Защиты:} в сценах, не подразумевающих
детализированных боевых действий — например, когда герой
прорывается сквозь разъяренную толпу, передвигается под беглым обстрелом или бежит по коридору, наполненному ловушками, мастер может определить полученные героем Повреждения при помощи проверки Защиты.
\begin{enumerate}
\item Бросьте К20 и прибавьте к нему значение \textbf{|Зщ — 10|}.
\item Сравните получившееся число со сложностью проверки. Герой получает 1 Повреждение за каждую 1, на которую провалил проверку.
\end{enumerate}
Например, герой Среднего размера с 16 Защитой передвигается
под беглым огнем вражеских стрелков. Мастер устанавливает
20 сложность проверки. На К20 выпадает 12. 12 +16 — 10 = 18.
Герой получает 2 Повреждения, так как 20 (Установленная
сложность) — 18 (Результат проверки) = 2.
\section{НУЛЕВОЙ УРОВЕНЬ ХАРАКТЕРИСТИК}
В приключениях героев подстерегает немало опасностей, и далеко не все из них явные. Яды, болезни, зараженная вода и пища могут понизить значение любой Характеристики до нуля. Понижение Основной Характеристики, даже временное, приводит к понижению зависящей от нее Вторичной.
\paragraph{}
Нулевой уровень Характеристики означает, что существо автоматически проваливает любые проверки, связанные с ней. Упавшие до нуля Сила, Ловкость и Скорость приводят к состоянию, близкому к параличу (хотя существо все слышит и может наблюдать за происходящим вокруг). Упавшие до нуля Интеллект, Мудрость и Обаяние приводят к коме. Если Реакция или Скорость существа падает до 0, оно впадает в ступор. Во всех этих случаях существо считается находящимся в состоянии Неподвижности. Существо с нулевой Волей покорно выполняет любые отданные ему приказы, даже очевидно самоубийственные. Если существо получит несколько приказов, противоречащих друг другу, то оно постарается выполнить их, удовлетворив максимальное число требований. Если Выносливость существа падает до нуля, то оно немедленно умирает или разваливается на части.
\paragraph{}
Исключение — боевые характеристики. Герой даже с отрицательными значениями боевых характеристик может пытаться атаковать противников, как бы безнадежно это не казалось.
\section{Основные навыки}
Значения Навыков отображают глубину познаний героя в различных областях. В скобках указана Характеристика, модификатор которой чаще всего прибавляется к Навыку при проверках, но могут возникнуть ситуации, при которых на для успеха приходится использовать другие характеристики, например, обычно для того, чтобы хорошо спрятаться нужно использовать навык Скрытность(Лв,Ин), но если герой хочет скрыться под водой, задержав дыхание или спрятаться в углу потолка, держась за стены, будет уместно использовать Скрытность(Вн). К некоторым Навыкам по умолчанию могут прибавляться модификаторы различных Характеристик, в зависимости от области применения. Мастер указывает перед проверкой, какой именно модификатор используется.
\paragraph{Значение навыка (Нв)} равно числу Очков опыта, распределенных в Навык.
\paragraph{Максимальное число Очков опыта в Навыке} не может превышать значение(не модификатор) Интеллекта героя. Когда герой совершает проверку Навыка, игрок бросает К20 и прибавляет к выпавшему результату значение Навыка героя и модификатор Характеристики, связанной с Навыком.
\newline
Если вы не используете правила Состязания, но герою кто-то активно противостоит, проверки Навыков совершаются следующим образом:
\begin{enumerate}
\item Бросьте К20 и прибавьте к нему значение Навыка и модификатор Характеристики.
\item Сравните получившееся число с \textbf{|Навыком оппонента +10|}. Герой преуспевает, если число равно сложности проверки или превышает ее.
\end{enumerate}
Например, герой со Скрытностью 6 и Ловкостью 14 (модификатор +2) пытается прокрасться мимо охранника. Охранник обладает Наблюдательностью 5 и Мудростью 12 (модификатор +1). Значит, для того, чтобы обмануть его бдительность, герою потребуется совершить проверку Скрытности против \textbf{|5 +1 +10 = 16|}. Если герой выбросит на К20 8 и больше, проверка будет успешна, т.к. \textbf{|6 +2 +8 = 16|}. В сумерках избежать внимания охранника будет гораздо легче — герой получит Преимущество. Напротив, если охранник бдительно осматривает пустой узкий коридор, герою придется действовать с Помехой, а то и двумя, если коридор ярко освещен!
\paragraph{Альтернативное применение Навыков:} использование Навыков — творческий процесс. Зачастую, добиться желаемого можно множеством разных способов, особенно, если к этому располагает контекст.
\paragraph{Нулевой уровень навыков:} если игрок не распределил в Навык героя хотя бы 1 Очко опыта, проверка этого Навыка совершается с Помехой.
\paragraph{Нулевой уровень боевых навыков:} если игрок не распределил хотя бы 1 Очко Опыта во Владение оружием, Рукопашный бой или Стрельбу героя, соответствующие проверки Доблести и Меткости совершаются с Помехой.

\genAndGet{skills}{skills-basic}

\section{Экспертные навыки}
Навыки из этой категории отвечают за узкоспециализированные области образования — и тем они ценнее. Некоторые Атрибуты открывают доступ к Экспертным навыкам и позволяют героям использовать их и вкладывать в них Очки опыта.
\newline
Герои, имеющие доступ к Экспертным навыкам, но не распределившие в них Очки опыта, могут совершать проверки этих Навыков с Помехой. Герои, не имеющие доступа к Экспертным навыкам, могут использовать только их статические значения, равные \textbf{|5 + модификатор профильной характеристики|}. Если разница между статическим значением навыка и сложностью проверки составляет 10 и больше, проверка считается Критическим Провалом.
\newline
Широта применения Экспертных навыков определяется мастером. Искусство может включать в себя все разнообразие видов художественной деятельности, а Наука — всю широту научных познаний. Но, с другой стороны, мастер может потребовать от художника уточнить область деятельности, — например, татуировщик, скульптор или аниматор, — серьезно ограничив широту применения Навыка. В случае Науки мастер может потребовать указать раздел или область, с которыми знаком герой.
\begin{tcolorbox}
В перечне представлены только самые часто встречаемые Экспертные навыки. Если игрок или мастер считает, что герою нужно использовать навык, которого нет в списке, его можно добавить в игру, обговорив перед этим его область применения.
\end{tcolorbox}

\genAndGet{skills}{skills}{Экспертный}

\section{Атрибуты}
Атрибуты - важнейшие детали образа героя и во многом отвечают за то, как его воспринимают окружающие. 
\newline Впрочем, и герои, и статисты могут обойтись (и зачастую обходятся) без Атрибутов. Не каждый проповедник - пламенный оратор, не каждая красавица способна очаровывать окружающих, не каждый офицер отдает толковые приказы. Герой вполне может быть проповедником и занимать формальное место в иерархии культа, родиться миловидным, или получить офицерский чин, но при этом не иметь соответствующего Атрибута. Его приобретение будет означать, что на Проповедника снизошла благодать (или он наконец-то научился ладно излагать догматы и полоскать мозги), Красавица расцвела (или поняла, как использовать красоту в достижении целей), а Офицер закалился в боях и заслужил уважение подчиненных (или заставил себя бояться).
Атрибут дает герою следующие возможности:
\begin{itemize}
    \item[--] Набор Экспертных навыков, к которым герой получает доступ при приобритении Атрибута;
    \item[--] Набор свойств, действующих постоянно или вводимых в игру без участия Нитей или Энергии. Условия применения свойств указаны в их описании;
    \item[--] Функции, действующие за счет траты Энергии героя. Стоимость Функции в Эн и прочие условия применения указаны в скобках после названия;
    \item[--] Cнаряжениe с указанными свойствами и СП. Начальное снаряжение имеет полный набор расходников, т.е. транспортные средства заправлены, оружие заряжено и т.д. Начальное снаряжение может быть продано, потеряно, украдено, уничтожено;
    \item[--] Темную Сторону - ситуативный Недостатка, сопутствующий Атрибуту;
    \item[--] Уникального Хода, который входит в игру при обрыве Нитей или с помощью проверок Неприятностей. Стоимость Хода в Нитях и условия его применения, в том числе в Боевых сценах, указаны в скобках после названия.
\end{itemize}

\begin{tcolorbox}
    У Атрибута всегда есть Ход и хотя бы одно Свойство, но далеко не у каждого - набор Экспертных навыков, Функции или Начальное снаряжение. Выбирайте тот, число способностей которого позволит вам комфортно с ним взаимодействовать.
\end{tcolorbox}

\paragraph{Стоимость Начального снаряжения:} некоторое снаряжение имеет фиксированную сложность приобретения (СП), например, Офицерский значок (СП 30) и Энциклопедия (СП 20). Если в описании указано "СП Х и меньше", то герой получает любой предмет указанной категории с СП, равной или меньшей Х. Если в описании указано "суммарно Х СП", то сумма значений СП всех предметов, выбранных героем, не должна превышать Х. Прочее снаряжение герой выбирает и покупает самостоятельно, расходуя Богатство.
\newline Подробнее о СП читайте в главе "Богатство и снаряжение".
\paragraph{Свойства, Функции, Ходы и течение времени:} время, которое занимает применение свойства, Функции или Хода, определяется мастером. Иногда вполне допустимо позволить герою уладить свои дела в Интерлюдии или Антракте. Даже когда временные рамки обозначены, мастер вправе отступить от них, если это уместно в контексте ситуации.
\paragraph{Свойства и Ходы, зависящие от Модификаторов Характеристик} всегда могут быть использованы минимум 1 раз за указанный период или воздействовать минимум на 1 цель, даже если применяемый Модификатор нулевой или отрицательный. Например, Дипломат с 9 Об (МОб -1) 1 раз активировать свойство "Парламентер".
\paragraph{Восполнение способностей героев:} способности некоторых Атрибутов могут быть возобновлены без ухода в Антракт или использования Интерлюдий. Условия восполнения указаны в описании таких Атрибутов. Если условия требуют наличия Услуги, СП восполнения может возрасти.
%\paragraph{Антракт и способности героев:} пребывание в Антракте возобновляет запас всех способностей, число применений которых за игровую встречу ограничено - если контекст не противоречит этому. Это никогда не требует расхода каких-либо ресурсов. Способности, действие которых ограничено рамками игровой встречи, прекращают работать при уходе героя в Антракт.
\paragraph{Уникальный Ход} позволяет герою преуспевать в задачах, сложных или невозможных для героев с другими Атрибутами. Добиться успеха при совершении Хода помогут как своенравная Судьба, так и способности самого героя. 
\newline Некоторые из Ходов ориентированы на применение в бою. Это никоим образом не должно останавливать от использования их в быту, если у игроков и мастера возникла идея, соответствующая настроению игры. И напротив - мирные ходы наверняка найдут применение в Боевых сценах.
\paragraph{1 или более Нитей} Некоторые Уникальные ходы отдают стоимость в Нитях на откуп мастеру. Определяйте стоимость Хода в соответствии с контекстом и логикой ситуации. Помните, что Ходы из перечня "Повезло", позволяющие выкупить успех и Критический успех на любую проверку за 2 и 4 Нити, доступны абсолютно любому герою. Там, где герой без Атрибута может преуспеть при помощи покупки обычного успеха за 2 Нити (или при помощи обычной проверки), герой с применимым к ситуации Атрибутом обойдется 1 Нитью. Если герой без Атрибута может преуспеть, только купив Критический успех за 4 Нити, герой с применимым к ситуации Атрибутом уложится в 2-3 Нити. Успех, за который мастер потребует 5 Нитей, должен изображать нечто умопомрачительное - один из тех случаев, о котором и герой, и свидетели события будут вспоминать всю оставшуюся жизнь. Без сомнений, такой успех быстро обрастет слухами и домыслами, а со временем - легендами. Разумеется, без применимого к ситуации Атрибута такой успех попросту невозможен.
\paragraph{0 Нитей.} Условия некоторых Ходов могут снизить стоимость до 0 Нитей и меньше. В этом случае для успеха все еще требуется обрыв 1 Нити.
\paragraph{Уникальный ход без обрыва Нитей.} Герои и персоны достаточно компетентны, чтобы добиться эффекта Уникальных ходов без вмешательства Судьбы, а у статистов и вовсе нет выбора большую часть времени. В этом случае активация Хода требует проверки Неприятностей под контролем Навыков или Характеристик. 
\begin{tcolorbox}
    Обратите внимание, что если игрок использует Ход героя без обрыва Нитей, он не может использовать любые Ходы Судьбы для влияния на результат, хотя вправе пользоваться Функциями, Трюками, Успехами с Расплатой и т.д.
\end{tcolorbox}
\paragraph{Сложность Уникального Хода без обрыва Нитей.} Если сложность контрольной проверки не указана в описании Хода, ее задает мастер, ориентируясь на таблицу сложности задач. Не рекомендуется устанавливать ее выше |20| - обладающий атрибутом герой неплохо разбирается в том, что делает.
\paragraph{Уникальный Ход без обрыва Нитей и Расплата:} некоторые из Ходов требуют выбрать Расплату даже при провале. Это означает, что герой пожертвовал временем, силами и ресурсами, но в итоге его все равно постигла неудача. Да, случается и такое. Разумеется, Успех с Расплатой может использоваться при проверках, связанных с Уникальным Ходом.
\paragraph{Темная сторона:} за все приходится платить, и возможности Атрибутов - не исключение. Фактически, Темная сторона - это Недостаток в Атрибуте. 
\begin{tcolorbox}
Герой может взять несколько одинаковых Атрибутов, получая все их преимущества согласно описанию. При этом активация Уникального Хода не будет требовать меньшего числа Нитей.
\end{tcolorbox}
\paragraph{Атрибуты, придуманные игроками и замена Уникальных ходов}
По договоренности с мастером игрок может:
\begin{itemize}
    \item[--] Начать игру с Атрибутом, который придумал сам. Возможности Атрибута и Уникального Хода должны быть определены до начала игры.
    \item[--] Заменить Уникальный Ход одного Атрибута на Уникальный Ход другого. Например, игрок может заменить Уникальный Ход Технаря на Уникальный Ход Мусорщика или Гражданина убежища. Замена производится до приобретения Атрибута.
\end{itemize}

\subsection{Атрибуты Наследия}
Некоторые атрибуты отражают не только способности героя, но и его происхождение. Их герой получил от предков - так или иначе. Герой вправе иметь лишь один Атрибут Наследия и обычно не может приобрести его по ходу игры - это то, чему нельзя научиться, а только получить при рождении (или сборке). Если герой - полукровка, и может претендовать на несколько Наследий, ему придется выбрать одно из них, или же отказаться от всех, ступив на путь, совершенно отличный от его предназначения.
\begin{tcolorbox}
	Иногда, впрочем, допустимо приобретение Атрибута Наследия по ходу игры. Гены Биоконструкта просыпаются в прагматичном жителе убежища, путешественник Изменяется, укрывшись на ночь в таинственной пещере, а великого воина после ранения вживляют в корпус Боевого робота. Фантазируйте вместе - и вы придумаете интересное и правдоподобное объяснение случившемуся.
\end{tcolorbox}
\genAndGet{attributes}{attributes}{Наследие}

\subsection{Атрибуты Могущества}
Атрибуты, дающие герою доступ к мистическому дару, псионическим силам и прочим невероятным способностям, ошеломительными даже на фоне других невероятных способностей. Герой может приобретать несколько Атрибутов Могущества, что отражает знакомство с разными аспектами \textbf{Феноменов}.
\paragraph{Феноменальная характеристика (Фх).} Основная характеристика, с помощью которой герою активирует Феномены. Она указана в описании Атрибута, Трюка или Предмета, который используется для активации. Модификатор Феноменальной характеристики (МФх) используется в большинстве формул, описывающих Феномены. 
\newline Если у героя есть несколько источников Феноменальной характеристики, перед активацией Феномена игрок должен заявить, какой источник и соответствующая характеристика будет использоваться.
\begin{tcolorbox}
    Атрибуты Могущества не обязаны иметь мистическую подоплеку. Впрочем, способности проникать в древние информационные библиотеки, ощущать электромагнитные поля, считывать информацию с голографических иероглифов или подключаться к секретной спутниковой сети, иначе, как волшебством, не назовешь. Да и сами обладатели таких возможностей зачастую уверены в их сверхъестественном происхождении.
\end{tcolorbox}

\genAndGet{attributes}{attributes}{Могущество}

\subsection{Боевые Атрибуты}
Атрибуты, ориентированные на боевые столкновения. Они дают значительные преимущества в боевых ситуациях, однако в других сценах ими воспользоваться будет \textit{сложнее}.
\genAndGet{attributes}{attributes}{Боевой}

\subsection{Социальные Атрибуты}
Атрибуты, ориентированные на социальные взаимодействия. С их помощью можно заручиться поддержкой статистов, произвести хорошее Впечатление или даже избежать конфликта до его начала. Однако, когда присутствующие в сцене похватались за оружие, Социальные Атрибуты уже \textit{скорее всего} не помогут.
\genAndGet{attributes}{attributes}{Социальный}

\subsection{Вспомогательные Атрибуты}
Герои с этими Атрибутами прекрасно дополнят любую команду благодаря способностями, серьезно облегчающими жизнь - как им, так и их товарищам.
\begin{tcolorbox}
    В этой категории обретаются самые экзотические Атрибуты. Чем объяснить их способности, вам подскажут жанр и настроение истории, а еще, конечно же, ваши соигроки. Возможно, Двойник - разумная колония нанороботов, Перевертыш - невероятная мутация, а Паразит - биодрон-разведчик.  Хотя не исключено, что все совсем не так, верно? 
\end{tcolorbox}
\genAndGet{attributes}{attributes}{Вспомогательный}

\printindex[attributes]

\section{Трюки}
Трюки – уловки, умения и качества, помогающие герою. Они не так масштабны, как Атрибуты, но не стоит их недооценивать. В некоторых ситуациях Трюк способен не только облегчить жизнь, но и спасти ее. 
\newline Трюки не могут быть приобретены несколько раз, если в описании не указано обратного.
\newline Трюки, зависящие от Модификаторов Характеристик не могут использоваться, если требуемый Модификатор или сумма с ним нулевая или отрицательная. Исключения указаны в описаниях Трюков.

\subsection{Боевые Трюки}
Трюки, ориентированные на боевые столкновения. Они дают значительные преимущества в боевых ситуациях, однако в других сценах ими воспользоваться будет \textit{сложнее}.
\genAndGet{tricks}{tricks}{Боевой}

\subsection{Социальные Трюки}
Трюки, ориентированные на социальные взаимодействия. С их помощью можно заручиться поддержкой статистов, произвести хорошее Впечатление или даже избежать конфликта до его начала. Однако, когда присутствующие в сцене похватались за оружие, Социальные Трюки уже \textit{скорее всего} не помогут.
\genAndGet{tricks}{tricks}{Социальный}

\subsection{Трюки Могущества}
Трюки, влияющие на Феномены и Энергию героя. Лучше всего подходят для магов, псиоников или супергероев.
\genAndGet{tricks}{tricks}{Могущество}

\subsection{Вспомогательные Трюки}
Эти Трюки не имеют сильной специализации на тех или иных Сценах или же несут вспомогательную функцию, которая делает жизнь героя проще.
\genAndGet{tricks}{tricks}{Вспомогательный}

\printindex[tricks]
\section{Недостатки}
Недостатки не обязательно являются отрицательным чертами (с точки зрения героя так уж точно). И все же они способны осложнить жизнь героя, а то и повернуть ее под совершенно непредсказуемым углом. Герой начинает игру с 0-1 Недостатком.
\newline Недостатки могут вводиться в игру по-разному - не стесняйтесь импровизировать. Неряха может выдать местоположение засады характерным запахом, Привязанность невовремя захочет поиграть в дамочку в беде, а Чужак случайно использует оскорбительный жест, заказывая выпивку. Способы испортить жизнь герою ограничены только контекстом и вашей фантазией.
\paragraph{Недостатки, придуманные игроками - отличная идея!} Главное, помните - Недостаток должен осложнять герою жизнь, и быть достаточно широко применимым, иначе в нем нет смысла.
\paragraph{Замена и отказ от Недостатков:} один Недостаток можно сменить на другой, если игрок того желает и это обусловлено развитием характера героя. Например, Вспыльчивый герой, нагрубив не тому человеку, может стать Осторожным, а Любвеобильный, отыскавший ту самую, обзаведется Привязанностью. Игрок вправе и просто вычеркнуть Недостаток, если контекст располагает к этому.
\begin{tcolorbox}
    Помимо прочего, Атрибуты, Трюки и Недостатки призваны создавать образ героя широкими мазками. "Гордая, но слегка Застенчивая Красавица-Прогрессор с Честным лицом и Чистым генофондом" - вполне завершенный образ. Или "Офицер-Ветеран, Мастер защиты, Знаток оружия, Пьяница и Грубиян" - звучит неплохо, верно? А главное, каждое слово имеет под собой игромеханическую основу и будет так или иначе работать на игру.
\end{tcolorbox}
\subsection{Перечень Недостатков}
\genAndGet{tricks}{tricks-flaws}{Недостаток}
\section{Грани и амплуа}
Грани и Амплуа дадут вам множество сюжетных зацепок, расскажут о прошлом героя и о том, на что он надеется в будущем. Многие вопросы, которые неизбежно возникнут после определения Граней, намеренно оставлены без ответов — их предстоит найти игрокам и мастеру!
\begin{tcolorbox}
Фактически, Грани представляют собой глубоко нишевые Уникальные ходы и Недостатки. Они рассчитаны в первую очередь на долгую игру, в которой характер героя меняется и развивается — как и мир вокруг него. Конечно, вы можете использовать их в играх на одну встречу, чтобы добавить образу героя колорита или даже построить вокруг них завязку. При этом стоит держать в голове, что Грани найдут применение далеко не в каждой истории.
\end{tcolorbox}
\paragraph{Амплуа} описывает героя одной емкой фразой, и служит хорошим дополнением к образу, созданному Атрибутами, Трюками и Недостатками. Амплуа совсем не обязано соответствовать перечню Атрибутов. Например, Дверг может выбрать Амплуа Воина и Ремесленника, игнорируя Амплуа Нелюдя, а полуэльф может использовать Амплуа Нелюдя, даже если не приобрел Атрибут Эльфа. Амплуа служат ориентиром для выбора Граней героя и не выполняют никакой игромеханической функции.
\paragraph{Грани} — значимые факты биографии, точки напряжения истории, конфликты героя, внутренние или внешние, а вместе с тем — превосходный источник идей для портрета героя. Грани состоят из \textbf{Орла} — условно положительной стороны Грани (до «но» в описании), и \textbf{Решки} — потенциально негативной стороны Грани (после «но» в описании). Игроки могут придумывать Грани самостоятельно — это поможет обогатить портрет героя деталями, которые подчеркивают жанр и настроение вашей истории.
В отличие от Недостатков и Темной стороны Атрибутов, Грани
рассчитаны на развитие ситуации со временем. Также они могут
прямо подсказать игроку и мастеру, какие Неприятности преследуют героя, каковы его социальный статус, интересы и круг общения, кто ему друг, а кто — враг.
\paragraph{Число Граней:} игрок может определить случайным образом или выбрать число Граней, равное числу Атрибутов героя, из наиболее подходящих ему Амплуа. Для героя без Атрибутов игрок может выбрать одну Грань из любого Амплуа. Орел может быть введен в игру при помощи общедоступного Хода
«Повезло» и обрыва Нити (одной или нескольких на усмотрение мастера) в ситуациях, когда герой может извлечь из этого выгоду.
\paragraph{Решка} вводится в игру как Каприз Судьбы.
Например, если герой обладает Гранью из перечня Бродяги (у героя множество родственников, но никто из них не добился успеха в жизни), игрок может оборвать Нить, объявить, что герой встретил в незнакомом городе троюродного брата и получил кров, стол или информацию. В то же время Грань ясно указывает, что герой вряд ли получит слишком много. При этом игрок может осложнить жизнь герою — протянуть к нему Нить и столкнуть лицом к лицу с восторженным племянником-недорослем, который желает путешествовать вместе с ним!
\paragraph{Смена Амплуа:} мастер может позволить заменить одно Амплуа на другое, если игрок того желает и это обусловлено логикой развития истории. Например, разочаровавшийся в идеалах Паладин может стать Бродягой, а Воин, открывший свое дело, превратится в Дельца. Смена Амплуа не вынуждает героя менять и Грани. Смену Граней, так же, как и смену Недостатков, должно определять развитие характера и образа героя. Разумеется, игрок может и вовсе отказаться от Граней героя, если сочтет нужным.
\newline
\genAndGet{roles}{roles}
\section{Узы}
\paragraph{Узы:} Узы представляют собой утверждения, которыми должен руководствоваться игрок, выбирая линию поведения героя. Это внешние и внутренние обстоятельства, мировоззрение и привычки, ощутимо влияющие на решения героя и определяющие выбор при прочих равных.
\paragraph{Число Уз:} игрок может выбрать для своего героя 0—2 Уз самостоятельно или определить их случайным образом. Когда герой связан Узами, протяните к герою 1 дополнительную Нить Судьбы в начале игровой встречи или сюжетной вехи за каждые Узы героя.
\paragraph{Темные Нити:} если герой совершает действия, противоречащие выбранным Узам, игрок должен оборвать 1 Нить или передать в руки мастера 1 Темную Нить.
\newline
Мастер может использовать Темные Нити в отношении статистов и персон так же, как используются Нити Судьбы в отношении героев, но только в тех случаях, когда статист или персона противостоят героям. Это единственный случай, когда статист может применить Ход Судьбы без проверок и последствий для себя!
\newline
Мастер может обрывать Темные Нити для ввода в игру Капризов Судьбы более 1 раза за сцену. Тем не менее, мастер все еще не может вводить один и тот же Недостаток, Темную сторону и Решку героя больше 1 раза за сцену. Например, если в распоряжении мастера есть 2 Темных Нити, он может 1 раз столкнуть героя с последствиями Недостатка бесплатно, 1 раз ввести в игру Темную сторону его Атрибута и 1 раз ввести в игру Решку героя, но не может ввести в игру один и тот же Недостаток героя дважды. В распоряжении мастера одновременно может находиться число Темных Нитей, равное \textbf{|1 + число игроков|}.
\paragraph{Смена Уз и отказ от них:} игрок может заменить одни Узы на другие или освободить героя от Уз, если это обусловлено логикой развития истории. Не считая специально оговоренных случаев, смена и отказ от Уз происходит в начале игровой встречи.
\paragraph{Выбор Уз и контекст:} Узы очень зависимы от логики происходящего. Если по какой-то причине в течение игровой встречи герой не может даже теоретически столкнуться с затруднениями и ограничениями, вызванными Узами, мастер начинает встречу с 1 Темной Нитью за каждые Узы, которые не будут задействованы.
\newline
Выбор Уз — это не только дополнительная Нить, но и способ сделать характер героя более выпуклым. Если вы задумаетесь над тем, почему герой связан именно этими Узами, то узнаете множество любопытных фактов о его личности. Например, если герой связан Узами единства, он начинает игру с 1 дополнительной Нитью, но должен оборвать 1 Нить или передать мастеру 1 Темную Нить, если его действия могут явно навредить героям остальных игроков. Почему командная работа так важна для него? Быть может, герой — ветеран, знающий, к чему может привести разлад на поле битвы, или в его большой семье все привыкли помогать друг другу. Не исключено даже, что герой — участник преступного синдиката, могущество которого держится на круговой поруке.
\newline
Чтобы определить Узы случайным образом, выберите столбец и бросьте К20.
\begin{center}
\begin{tabular}{ |c|p{7cm}|c|p{7cm}| }
\hline
K20 & \textbf{Узы} & K20 & \textbf{Узы} \\ \hline
1 & Узы единства & 1 & Лишний шум - лишние проблемы \\ \hline
2 & Любовь выдумали поэты и менестрели & 2 & Первый удар - решающий \\ \hline
3 & Честь превыше всего & 3 & Семь раз отмерь, один раз отрежь \\ \hline
4 & Сумел нынче убежать - завтра будешь воевать & 4 & Насилием нельзя изменить мир - только изуродовать \\ \hline
5 & Принимай чужеземцев с радушием & 5 & Слово - лучшее оружие \\ \hline
6 & После нас - хоть потоп & 6 & Цель оправдывает средства \\ \hline
7 & Простолюдины хуже зверей & 7 & Один раз живем \\ \hline
8 & Умеренность и аккуратность & 8 & Закон не ошибается \\ \hline
9 & Вещи не предадут и не обманут & 9 & Знание - сила \\ \hline
10 & Счастья за деньги не купишь & 10 & Старшим виднее \\ \hline
11 & Сделал дело - гуляй смело & 11 & Женщина священна \\ \hline
12 & От стариков никакой пользы & 12 & От чужаков добра не жди \\ \hline
13 & Каждая жизнь бесценна & 13 & Высокий род не оправдывает высокомерия \\ \hline
14 & Сомнения - удел слабаков & 14 & Жестокость внушает уважение \\ \hline
15 & Терпение украшает & 15 & Любовь - величайшее из чудес \\ \hline
16 & Ты - мне, я - тебе & 16 & Предательство постыдно \\ \hline
17 & Мужчина во всем главный & 17 & Никогда не сдавайся \\ \hline
18 & Око за око & 18 & Деньги могут все \\ \hline
19 & Приметы не лгут & 19 & Вовремя предать - значит предвидеть \\ \hline
20 & Я заслуживаю самого лучшего & 20 & Вещи обременяют \\ \hline
\end{tabular}
\end{center}
\paragraph{Узы единства} объясняют, почему герои, порой очень поверхностно знакомые друг с другом, работают в команде и не выступают друг против друга открыто или тайно, даже когда есть возможность извлечь из предательства серьезную выгоду. Узы единства не обязаны связывать всех героев в команде. Иногда один герой искренне радеет за общее дело, а другой ждет удобного момента, чтобы перерезать ему глотку! С другой стороны, в не нацеленной на внутренние конфликты команде Узы единства — хороший способ начать игру с дополнительной Нитью. Узы единства также распространяются и на статистов, которые важны для успеха дела.
\begin{center}
\begin{tabular}{ |c|p{4cm}|p{10cm}| }
\hline
\textbf{К20} & \textbf{Узы единства} & \textbf{Мотив} \\ \hline
1 & Один в поле не воин & Боязнь остаться один на один с некоей угрозой удерживает героя от необдуманных действий \\ \hline
2 & С приказами не спорят & Герой получил приказ, запрещающий ему вредить остальнму, по крайней мере, пока дело не завершено \\ \hline
3 & Лучше делить на всех, чем лишиться всего & Герой не уверен в своих силах и предпочитает получить меньше, но наверняка \\ \hline
4 & Дружба нерушима & Герой испытывает искреннюю симпатию к своим спутникам и считает их друзьями \\ \hline
5 & Без меня они пропадут & Герой чувствует ответственность за своих спутников, хоть и относится к ним слегка снисходительно \\ \hline
6 & Сначала дело, потом - разборки & Герой привык разделять деловые интересы и личную неприязнь. Это не мешает ему считать своих спутников ничтожествами, хотя говорить об этом вслух он вряд ли сочтет разумным \\ \hline
7 & Свары - для любителей & Герой считает себя профессионалом и не позволяет эмоциям взять верх над здравым смыслом. Впрочем, он не будет хвататься за оружие, если кто-то из его спутников действительно \textit{заслужил} хорошую взбучку \\ \hline
8 & Боги ненавидят предателей & Герой убежден, что боги покарают его за предательство \\ \hline
9 & Риск слишком велик & Герой рад бы обогатиться за чужой счет, но боится огласки и преследования властей \\ \hline
10 & Я выше этого & Герой не видет смысла в усобицах. Возможно, он в тайне гордится этим \\ \hline
11 & Для дела важен каждый & Герой уверен, что натянутые отношения с кем-то из спутников (не говоря уж о гибели), серьезно понизят общие шансы на успех \\ \hline
12 & За смирение мне воздастся & Герой верит, что Судьба воздаст ему за терпение по отношению к спутникам, как и за нежелание идти против них \\ \hline
13 & Ценные связи на будущее & Герой стремится сохранить со спутниками хорошие отношения из соображений будущей выгоды \\ \hline
14 & Я здесь не ради выгоды & Успех предприятия на первом месте для героя. Он скорее пожертвует собственной выгодой, чем пойдет на конфликт \\ \hline
15 & Доверие - основа успеха & Герой не представляет командной работы без взаимопомощи. Он полностью доверяет спутникам и ждет от них того же \\ \hline
16 & Честь дороже выгоды & Герой считает предательство бесчестным делом \\ \hline
17 & Я зла не делаю и не помню & Герой редко раздражается и быстро отходит. Причинить спутнику зло, пусть и ради выгоды, кажется ему чудовищным \\ \hline
18 & Просто немыслимо & Герой никогда не задумывался о мести и предательстве в принципе \\ \hline
19 & Мы - команда & Герой считает себя частью единого целого, команды, внутри которой каждый занимается своим делом и получает то, что должно \\ \hline
20 & Я - пример для окружающих & Герой считает себя примером добродетели и печется о своей репутации \\ \hline
\end{tabular}
\end{center}
\section{Языки}
В некоторых сюжетах герои встречают путешественников или сами попадают в далекие страны. В таких случаях у героев может возникнуть желание (или даже насущная необходимость) выучить чужеземный язык.
\paragraph{}
В начале игры герой знает свой родной язык достаточно хорошо, чтобы говорить и писать на нем (если только у игрока нет других идей на этот счет). В дополнение герой владеет числом иноземных языков, равным своему \textbf{МИн}. Каждый язык или наречие сверх этого стоит 1 Очко опыта. Как правило, герой все равно говорит на иноземном языке с акцентом, очевидном для аборигенов.
\section{Начисление очков опыта}
Герой немедленно получает 1 Очко опыта, если он:
\begin{itemize}
\item[--] Принял влияние другого героя или статиста (подробнее об этом читайте в части «Социальные взаимодействия»).
\item[--] Достиг одной из своих Целей. Герой может тут же выбрать новую Цель, если к этому располагает контекст ситуации. Герой получает 1 Очко опыта в конце игровой встречи за каждое выполненное условие. Многократное выполнение одного и того же условия не приносит герою дополнительных Очков опыта.
\end{itemize}
Герой получает 1 Очко опыта в конце игровой встречи за каждое
выполненное условие. Многократное выполнение одного и того же
условия не приносит герою дополнительных Очков опыта.
\begin{itemize}
\item[--] Герой столкнулся с проблемой, вызванной Капризом Судьбы.
\item[--] Герой использовал Ход Судьбы.
\item[--] Герой использовал Уникальный ход своего Атрибута.
\item[--] Герой использовал достоинства своего Атрибута: его Свойство или Функцию.
\item[--] Герой использовал свой Трюк, приобретенный отдельно от Атрибута.
\end{itemize}
\subsection{Цели}
\paragraph{}
При помощи таблиц вы можете определить Цели, встающие перед героями в начале игровой встречи. Выбранная Цель может стать как сюжетным стержнем вашей истории, так и побочной (но от этого не менее интересной) линией.
\paragraph{}
Игрок может выбрать или определить случайным образом 1 Цель своего героя в начале игровой встречи. Если герою удается успешно выполнить ее, он получает 1 дополнительное Очко опыта и может тут же взять новую Цель. Выполнение Цели может занять какое-то время, не исключено, что для этого понадобится несколько игровых встреч. Обратите внимание, что Цель предлагает лишь абстрактную идею. Конкретное наполнение приключения зависит от контекста событий, происходящих в вашей истории.
\paragraph{}
Цель может быть не только личной, но и общей. В этом случае она служит сюжетным стержнем, и Очки опыта за ее выполнение начисляются всем героям, участвующим в истории. В некоторых случаях потребуется выяснить, в каких отношениях Цель находится с героями и с кем именно — например, если выпал вариант «Враг» или «Возлюбленный».
\paragraph{}
Число Целей, которых герой пытается достичь одновременно (и за выполнение которых получает Очки опыта), не может превышать его \textbf{|МОб|} (минимум 1).
\paragraph{Выбор Цели:} в начале киньте К20 и Выберите Цель. Затем киньте К20 и Конкретизируйте Цель. После этого киньте К20 и узнайте, что герой должен сделать с Целью. В заключение киньте К20 и определите, из-за чего достижение Цели под угрозой.
\begin{center}
\begin{tabular}{ |c|c|c|c|c|c| }
\hline
\textbf{К20} & 1-4(Информация) & 5-8(Предмет) & 9-12(Разумное существо) & 13-16(Животное) & 17-20(Место) \\ \hline
1-2 & Опасная & Деньги & Друг & Злобное & Руины \\ \hline
3-4 & Бессмысленная & Зелье & Враг & Упрямое & Холм \\ \hline
5-6 & Удивительная & Ценные бумаги & Возлюбленный & Тупое & Дом \\ \hline
7-8 & Не выглядит важной & Механизм & Родственник & Смирное & Озеро \\ \hline
9-10 & Зашифрованная & Оружие & Соперник & Любопытное & Роща \\ \hline
11-12 & Полезная & Драгоценность & Незнакомец & Егозливое & Подземелье \\ \hline
13-14 & Пугает & Доспех & Важная персона & Опасное & Лес \\ \hline
15-16 & Несерьезная & Картина & Чужеземец & Отвратительное & Лаборатория \\ \hline
17-18 & Ценная & Изваяние & Нелюдь & Ядовитое & Болото \\ \hline
19-20 & {Объемная} & {Артефакт} & Кумир & Очень странное & Замок \\ \hline
\end{tabular}
\end{center}


\begin{center}
\begin{tabular}{ |c|c|c|c|c|c| }
\hline
\multicolumn{6}{|c|}{\textbf{Что герой должен сделать с Целью, если она...}} \\ \hline
\textbf{К20}& \textbf{Информация} & \textbf{Предмет} & \textbf{Разумное существо} & \textbf{Животное} & \textbf{Место} \\ \hline

1 & Уточнить & Расколдовать & Расколдовать & Расколдовать & Расколдовать \\ \hline
2 & Восстановить & Восстановить & Исцелить & Исцелить & Восстановить \\ \hline
3 & Отыскать & Отыскать & Отыскать & Отыскать & Отыскать \\ \hline
4 & Скопировать & Скопировать & Наказать & Перевезти & Очистить \\ \hline
5 & Исказить & Вернуть & Вернуть & Вернуть & Осквернить \\ \hline
6 & Передать & Передать & Изобличить & Передать & Укрепить \\ \hline
7 & Скрыть & Укрыть & Укрыть & Укрыть & Подготовить \\ \hline
8 & Выкупить & Выкупить & Выкупить & Выкупить & Выкупить \\ \hline
9 & Продать & Продать & Продать & Продать & Продать \\ \hline
10 & Выкрасть & Выкрасть & Выкрасть & Выкрасть & Освятить \\ \hline
11 & Дополнить & Спрятать & Спрятать & Спрятать & Занять \\ \hline
12 & Проверить & Подменить & Обокрасть & Подменить & Скомпрометировать \\ \hline
13 & Дискредитировать & Изготовить & Уговорить & Спарить & Спасти \\ \hline
14 & Распространить & Применить & Освободить & Освободить & Обследовать \\ \hline
15 & Получить & Получить & Отблагодарить & Передержать & Отдать \\ \hline
16 & Изучить & Изучить & Соблазнить & Изучить & Изучить \\ \hline
17 & Уничтожить & Уничтожить & Убить & Убить & Уничтожить \\ \hline
18 & Захватить & Захватить & Захватить & Захватить & Захватить \\ \hline
19 & Опровергнуть & Подбросить & Опорочить & Подбросить & Отбить \\ \hline
20 & Защитить & Защитить & Защитить & Защитить & Защитить \\ \hline
\end{tabular}
\end{center}

\begin{center}
\begin{tabular}{ |c|c| }
\hline
\textbf{К20} & \textbf{Достижение Цели под угрозой из-за…} \\ \hline
1 & Преступного синдиката \\ \hline
2 & Старых врагов \\ \hline
3 & Бандитов \\ \hline
4 & Капризов природы \\ \hline
5 & Чиновников \\ \hline
6 & Сил правопорядка \\ \hline
7 & Соперников \\ \hline
8 & Нелюдей \\ \hline
9 & Иноземцев \\ \hline
10 & Неведомых врагов \\ \hline
11 & Родственников \\ \hline
12 & Друзей \\ \hline
13 & Любви \\ \hline
14 & Конкурентов \\ \hline
15 & Важной персоны \\ \hline
16 & Высокородного \\ \hline
17 & Безумца \\ \hline
18 & Животных \\ \hline
19 & Чудовища \\ \hline
20 & Сверхъестественных сил \\ \hline
\end{tabular}
\end{center}

\section{Развитие героя}
После начала игры герой может потратить заработанные Очки опыта следующим образом:
\begin{itemize}
\item[--] Повысить значение любого Навыка на 1, потратив 1 Очко опыта.
\item[--] Повысить Богатство на 1, потратив 1 Очко опыта.
\item[--] Изучить новый язык или наречие, потратив 1 Очко опыта.
\item[--] Изучить новое заклинание, потратив 2 Очка опыта.
\item[--] Повысить на 1 максимальные Единицы Здоровья, потратив 2 Очка опыта.
\item[--] Повысить на 1 максимальные Единицы Маны, потратив 2 Очка опыта.
\item[--] Изучить Трюк, потратив 5 Очков опыта.
\item[--] Повысить на 1 любую Основную или Вторичную характеристику, потратив 5 Очков опыта.
\item[--] Получить любой Атрибут, потратив 10 Очков опыта.
Повышение Основных характеристик приводит к повышению
Вторичных. Например, если герой приобрел 1 единицу Выносливости, то его ЕЗ также вырастут.
\end{itemize}
\paragraph{}
Новые способности не могут взяться из ниоткуда. Чтобы отобразить развитие героя, есть два способа:
\begin{enumerate}
\item Вы можете исходить из уже существующих внутриигровых фактов. Например, прежде чем приобрести Атрибут «Аристократ», герой проявил себя перед государем и заслужил титул, получил его путем вероломных интриг… или же просто купил за внушительную сумму денег. Если герой увеличивает Богатство или Владение оружием, то перед этим он разумно распоряжался своими средствами и принимал участие в битвах.
\item Вы можете создать факт, договорившись с мастером и соигроками о логичном внутриигровом обосновании приобретения вашего героя. Например, в случае Атрибута «Аристократ» герой может оказаться потерянным наследником древнего рода, повышение Богатства произошло за счет выплаты банковских процентов или выигрыша на скачках, а Владение оружием улучшилось благодаря тренировкам, на которые герой тратил свободное время между игровыми встречами.
\end{enumerate}
\chapter{БОГАТСТВО И СНАРЯЖЕНИЕ}
Здесь подробно рассказывается о купле-продаже предметов и ценностей, о снаряжении, которым пользуются герои и статисты, о существах, которые могут служить им скакунами и о том, как разумно пользоваться всем этим.

\section{Богатство}
\paragraph{}
Звонкие монеты, ценные материалы и сверкающие каменья... Ради них герои отправляются в опасные походы, ими платят ремесленнику и брадобрею, их бросают к ногам красавиц и могучих правителей. Хотя некоторые герои недальновидно прячут Богатство в сундук и закапывают в землю. Но речь, конечно же, пойдет не о них.
\newline
Богатство отражает финансовое благосостояние героя. В ходе игры Богатство может возрастать или уменьшаться.
\paragraph{Начальный уровень Богатства героя — 5}, минимальный уровень Богатства — 0. Игрок может увеличивать Богатство с помощью траты Очков опыта (в том числе после начала игры), однако оно может увеличиваться и благодаря продаже ценностей. Некоторые Атрибуты и Трюки также влияют на начальный уровень Богатства или дают временные бонусы к нему.
\paragraph{Уровни Богатства:}
\paragraph{0 — Нищий.} Герою нечем заплатить ни за черствый хлеб, ни за самую дешевую ночлежку. Герой не может приобретать услуги и предметы с СП 11 или больше.
\paragraph{1—4 — В долгах, как в шелках.} Герой живет в режиме жесткой экономии. Мясо на его столе — редкий гость.
\paragraph{5—10 — Средний класс.} Герой может побаловать себя время от времени... Впрочем, не слишком часто. Он все еще латает свою одежду вместо того, чтобы купить новую.
\paragraph{11—15 — Пошел в гору.} Герой твердо стоит на ногах. Он может без проблем позволить себе излишества в еде и одежде.
\paragraph{16—20 — Толстосум.} Герой способен оплатить работу лучших ремесленников, приобретать породистых лошадей, борзых и произведения искусства. Также его не сильно обременит наем хорошенькой горничной или привратника-ветерана. за состоянием счетов — это делают многочисленные приказчики. Особняк, карета, надомный лекарь, личная стража и красивые любовницы прилагаются.
\paragraph{31 и более — Купается в золоте.} Тот самый момент, когда герой подумывает о покупке маленькой уютной страны или найме армии для ее захвата!
\paragraph{Проверка Богатства:} товары и услуги имеют \textbf{Сложность приобретения (СП)} — число, заданное правилами или установленное мастером. Чтобы определить, может ли герой приобрести товар или услугу, бросьте К20 и прибавьте к выпавшему числу значение Богатства героя. Если результат больше или равен СП, герою удалось приобрести желаемое. Его Богатство понижается в зависимости от СП оплаченного:
\begin{itemize}
\item[--] СП 15 или больше — 1 Богатство в дополнение к прочим потерям
Богатства.
\item[--] СП на 1—5 больше, чем Богатство героя, — потеря 1 Богатство.
\item[--] СП на 6—10 больше, чем Богатство героя, — потеря 2 Богатств.
\item[--] СП на 11—15 больше, чем Богатство героя, — потеря 4 Богатств.
\item[--] СП на 16 и больше, чем Богатство героя, — потеря 8 Богатств.
\end{itemize}
Если СП товара или услуги меньше или равен Богатству героя, бросок не требуется — герой просто получает желаемое. Он все еще теряет 1 Богатство, если СП покупки 15 и больше. 
\paragraph{Карманные расходы:}
Если герой не потерял Богатство при покупке товара или услуги, это значит, что он буквально использовал мелочь из своего кармана. Но мелочь в караманах когда-нибудь заканчивается и приходится уже разменивать крупные купюры. Если совершил покупок на Карманные расходы больше, чем его текущее Богатство, он тут же теряет 1 Богатство, а счетчик Карманных расходов обнуляется.

\paragraph{Быстрая покупка:} является Быстрой проверкой Богатства. Предполагается, что в этом случае герой платит сразу, не торгуясь. Герой с Богатством 0 не может сделать этого.
\paragraph{Комплексные покупки:} иногда герой вынужден совершать особо внушительные траты за ограниченный временной промежуток — например, когда снаряжает армию, готовит экспедицию за сокровищами или подкупает толпу жадных бюрократов. В этом случае сложите СП покупок для определения сложности проверки и понизьте Богатство героя по обычным правилам.
\paragraph{Неслыханная щедрость:} получить Преимущество на проверку, но при успехе теряет на 1 Богатство больше. Так или иначе, Богатство понижается только при успешном приобретении услуги или предмета. В противном случае герой впустую потратил время на препирательства с продавцом или бесплодную борьбу с собственной жадностью.
\paragraph{Покупки в начале игры:} в начале игры герои покупают все необходимые предметы по отдельности — игнорируйте правила Комплексной покупки. Приобретение предметов происходит после распределения Очков опыта. В начале игры все снаряжение герой приобретает по правилам Быстрой покупки.
\paragraph{Покупка в складчину:} герои могут подкинуть друг другу деньжат по правилам Взаимопомощи (подробнее об этом читайте в разделе «Проверки»). Помощники теряют Богатство по обычным правилам. Не забывайте, что для них СП покупки меньше на 5.
\paragraph{Продажа предметов и услуг:} сначала определите СП предмета или услуги. Если предмет был в употреблении, понизьте его СП на 1. За каждую еденицу Осечки, полученную из-за Износа снаряжения понизтье его СП дополнительно на 1. Затем повысьте Богатство продающего героя на столько, на сколько он понизил бы его, купив этот предмет. Комплексные продажи возможны, хотя у героя, скорее всего, потребуют большую скидку!
\paragraph{Поторгуемся?} Успешная проверка Торговли позволяет сделать одно из следующего (по выбору игрока):
\begin{itemize}
\item[--] Получить Преимущество на проверку Богатства.
\item[--] Дать Преимущество на проверку Богатства другому герою, если герой с Торговлей помогает торговаться.
\item[--] Повысить Богатство на дополнительный 1 в случае успеха проверки, если герой продает или помогает продавать.
\item[--] Уменьшить потерю Богатства на 1 в случае успеха проверки, если герой покупает или помогает покупать.
\end{itemize}
Провал проверки Торговли приводит к одному из следующего (по выбору игрока):
\begin{itemize}
\item[--] Герой получает Помеху на проверку Богатства.
\item[--] Другой герой получает Помеху на проверку Богатства, если герой с Торговлей помогает торговаться.
\item[--] Богатство повышается на 1 меньше, чем должно в случае успеха проверки, если герой продает или помогает продавать. В худшем случае герой потеряет предмет и не повысит свое Богатство.
\item[--] Потеря Богатства увеличивается на 1 в случае успеха проверки, если герой покупает или помогает покупать.
\end{itemize}
\paragraph{Бартер}
Есть места, где Богатство героев не имеет значения. Местных не интересуют блестяшки и бумажки, а о таком концепте, как виртуальные деньги и кредиты они и думать не хотят. В таких местах за место привычной торговли выступает старый-добрый Бартер.
\paragraph{}Во время Бартера герой предлагает свои вещи, имеющие СП вместо Богатства для того, чтобы приобрести желаемое у торговца. Проверка Бартера в этом случае является проверкой Нулевого Богатства против сложности 10. За каждую еденицу разницы СП между товаром героя и товаром продавца сложность увеличивается или уменьшается на 5. Например, если герой хочет обменять Арбалет(СП 10) на Дробовик(СП 11), то сложность проверки будет 15.
\paragraph{}В случае провала Проверки сделка не состоится - повторные проверки с участием тех же вещей невозможны, но можно попробовать обменять другие вещи на желаемое.
\paragraph{}При бартере Боеприпасов считается, что обмен происходит коробками - поштучно торговля не пойдет.
\paragraph{}Если герой хочет обменять сразу несколько едениц товара, то за каждое удвоение СП приобретаемого или отдаваемого товара будет увеличено на 1. Например, если герой хочет использовать в Бартере 8 Кинжалов(СП 3), то это будет считаться как СП 6.
\paragraph{}Бартер - примитивное средство торговли, поэтому вряд ли за одну сделку удастся обменять одновременно Кинжал, Кастет и Пистолет на Бронежилет. Придется найти промежуточную валюту или попытаться обменять более равноценные вещи.
\paragraph{}Герой может использовать навык Торговли при Бартере, но тогда сложность всех проверок возрастает на 5. Торговцу, привыкшему говорить языком денег будет труднее убедить тех, кто концепцию денег не понимает и не принимает.

\section{Имущество}
В этом разделе будут раскрыты характеристики и механики, присущие любому имуществу, которым могут владеть герои, будть то автомобиль, кинжал или скакун.

\paragraph{Сложность приобретения (СП):} Большинство имущества можно приобрести или продать за хорошую цену. Сложность приобретения определяет абстрактную стоимость снаряжения и аммуниции. Если СП не указана, то снаряжение нельзя приобрести в магазинах - либо оно слишком дешево, чтобы торговцы озаботились иметь эти вещи на складе, либо это уникальные образцы, приобретение которых может быть отдельным приключением!

\paragraph{Осечка} определяет, насколько ненадежным является снаряжение из-за его конструктивных особенностей или же из-за ненадлежащей эксплуатации. При любых проверках с участием снаряженияжения с Осечкой, если на кубике выпало число меньше или равное значению Осечки, проверка автоматически считается проваленной. Если герой использует несколько устройств с Осечкой, то общая Осечка для проверки равна максимальной Осечке среди устройств.
\newline Если герой носит броню с Осечкой, то она действует на все Активные Проверки героя.

\paragraph{Изъяны} Вещи указывают на любое негативное влияние Вещи на его владельца. Это может быть как незначительное неудобство, так и серьезное проклятие. Они делятся на категории по частоте появления:
\begin{enumerate}
\item \textbf{Ноша}. Эффекты Изъяна проявляются даже если герой носит свою Ношу в рюкзаке.
\item \textbf{Издержка}. Эффекты Изъяна проявляются каждый раз, когда герой совершает проверку с использованием этого снаряжения.
\item \textbf{Осечка}. Эффекты Изъяна проявляются только когда при проверке с использованием снаряжения выпала Осечка или Критический Провал.
\item \textbf{Глюк}. Эффекты Изъяна проявляются только при определенных условиях, которые указаны в описании снаряжения.
\end{enumerate}

\paragraph{Вес снаряжения для больших и маленьких существ}
Во всех таблицах указан вес снаряжения для существ Среднего размера. Чтобы определить во сколько раз изменится вес предмета, изготовленного для существа другого размера, сверьтесь с таблицей.
\begin{center}
\begin{tabular}{|c|c|}
\hline
Размер Существа & Множетитель веса \\ \hline
Крошечный & 1/3 \\ \hline
Маленький & 2/3 \\ \hline
Средний & 1 \\ \hline
Большой & 4/3 \\ \hline
Огромный & 5/3 \\ \hline
Громадный & 2 \\ \hline
\end{tabular}
\end{center}

\subsection{Износ и ремонт снаряжения}
Даже самое надежное устройство при ненадлежащей эксплуатации и недостаточном уходе начнет сбоить.
\paragraph{Проверка Износа} является Проверкой неприятностей, которая определяет, насколько ухудшилось состояние снаряжения во время эксплуатации.
\newline Снаряжение со свойством \textbf{Надежное} позволяет совершать эту проверку с Преимуществом.
\newline Снаряжение со свойством \textbf{Хрупкое} добавляет Помеху на проверку Износа.
\begin{tcolorbox}
Проверки Износа происходит каждый раз, когда герой совершает Критический Промах, используя снаряжение или оружие, а так же проверки Износа могут быть явно указаны в описании ситуаций, требующих их. Однако ведущий может потребовать проверку Износа в конце Сцены, если по его мнению герой слишком уж сильно издевается над своей экипировкой.
\end{tcolorbox}
\trouble
{Надежная штука}%no sweat name
{Устройство счастливо избежало неполадок и может быть использовано в дальнейшем.}%no sweat description
{Заело}%tough day name
{Небольшая проблема в механизме. Любой герой разберется с ней за 10 минут. Для того чтобы вернуть работоспособность устройства, совершите проверку Эксплуатации против 15. Устройство может быть использовано после проверки в любом случае, но в случае провала проверки его Осечка возрастает на 1.}%tough day description
{Заклинило}%we have trouble name
{Серьезная проблема в механизме. Для того чтобы вернуть работоспособность устройства, совершите проверку Эксплуатации против 20. Устройство может быть использовано после проверки в любом случае, но в случае провала проверки его Осечка возрастает на 2, а в случае успеха - на 1.}%we have trouble description
{Капитальная поломка.}%fiasco name
{Устройство не может быть использовано до ремонта, а его Осечка возрастает на 5.}%fiasco description
\paragraph{Ремонт Осечек} снаряжения доступен только в том случае, если Осечка была получена из-за Износа. Если Осечка является свойством нового снаряжения, избавиться от нее может только Изобретатель с помощью хода Эврика.
\newline
При ремонте осечек снаряжения герой должен совершить проверку Ремонта или Эксплуатация против \textbf{|15+Осечка|} и потратить материалы с СП, равныой величине Осечки. В случае провала проверки материалы не возвращаются. В случае критического провала снаряжение приходит в негодность и больше не полежит ремонту.
\begin{tcolorbox}
Если герой ремонтирует снаряжение, которое имело свойство Осечки до того, как она возрасла во время эксплуатации, герою все равно нужно учитывать в формулах полную Осечку снаряжения. Это значит, что механизм устройства настолько сложный, что его непросто ремонтировать.
\end{tcolorbox}
\paragraph{Ремонт Потери ЕЗ} снаряжения требует совершения проверки Ремонта против \textbf{|15+Потерянные ЕЗ|} и потратить материалы с СП, равныой величине количеству потерянных ЕЗ. В случае провала проверки материалы не возвращаются. В случае критического провала снаряжение приходит в негодность и больше не полежит ремонту.
\paragraph{Время ремонта:} герой ремонтирует одну вещь в течении Антракта. Если сложность ремонта выше 15, то за каждую дополнительную еденицу сложности ему придется провести один дополнительный Антракт за ремонтом.

\input{money-talk/supplies}
\subsection{Оружие}
От выбора оружия зависит жизнь или смерть героя. Впрочем, иногда оружие - всего лишь элемент социального статуса или стильное дополнение к костюму.
\paragraph{Бонус к Повреждениям (БПв):} опытный боец опасен сам по себе. Однако большинство воинов предпочитают иметь при себе оружие. Оружие дает Бонус к Повреждениям, который прибавляется к Доблести и Меткости. Бонус постоянен и зависит от выбранного оружия. Чем он больше, тем смертоноснее оружие в умелых (и даже не очень умелых) руках.

\paragraph{Дистанция поражения:} оружие ближнего боя применимо, только если цель находится в Боевом контакте героя  (обычно в 1 метре от него, или в соседней клетке, если используется масштабная карта).
\newline У дальнобойного оружия две дистанции - Ближняя, на которой оружие наиболее опасно, и Дальняя, предельная дистанция, на которой возможно поражение цели. Как правило, БПв на Дальней дистанции ниже, чем на Ближней.
\newline Если у дальнобойного оружия нет Дальней дистанции, это значит, что оно не способно нанести существенные повреждения за пределами Ближней дистанции.

\paragraph{Требуемая Сила (тСл):} для успешного использования оружия нужно быть сильным. Хилый герой не натянет лук и не рубанет двуручным топором. тСл для использования оружия указана в его параметрах. Если герой не имеет достаточной Силы для использования оружия, то все атаки им он совершает с Помехой. Герой, не имеющий достаточной Силы, не может сражаться Громоздким оружием вообще. Конечно, он способен размахивать им и даже выглядеть при этом угрожающе, но нанести вред может разве что случайно.

\paragraph{Единицы Здоровья (ЕЗ) оружия = |1/2 тСл|.} Например, большая дубина имеет тСл 13, значит, у нее 6 ЕЗ. Если у оружия несколько параметров тСл, для подсчета ЕЗ используется наибольший.

\paragraph{Прочность (Прч) оружия = |1/2 ЕЗ оружия|.} Прочность вычитается из Пв, которые получает оружие.

\paragraph{Тип Повреждений:} в зависимости от типа, повреждения могут иметь разные эффекты КУ и могут быть увеличены или уменьшены в зависимости от того, какие есть сопротивления или уязвимости у цели. Атаки имеют один из следующих типов Повреждений:
\newline \textbf{(Д)}робящие, \textbf{(Е)}дкие, \textbf{(К)}олющие, \textbf{(Л)}едяные, \textbf{(О)}гненные, \textbf{(П)}роникающие, \textbf{(Р)}убящие, \textbf{(Э)}лектрические, \textbf{(Я)}довитые.
\newline Если тип Повреждений отмечен символом "*", то он зависит от используемых боеприпасов.
\newline Если оружие имеет несколько типов Пв, перечисленных через черту (К/Р или Р/Д), герой может выбирать, какой тип Повреждений наносить при атаке. 
\newline Если типы Пв перечислены слитно (ОЭ или РЯ), то оружие наносит одновременно несколько типов Пв. 
\newline Подробнее о типах Повреждений читайте в главе "Боевые столкновения".

\paragraph{Критический Удар (КУ):} минимальное число на К20, при выпадении которого цель подвергается эффектам КУ. 

\paragraph{Халтура} - это дешевое, кое-как изготовленное оружие, которое имеет 1/2 ЕЗ, 1/2 Прч, -1 к БПв и +2 к Осечке. Такое оружие выходит из строя при Критическом провале Дб или Мт. Понизьте СП на 5. Если СП падают до 0, значит, оружие можно вытащить из соседней мусорной кучи, потратив на это Интерлюдию. Что большинство статистов и делает.
\paragraph{Работа мастера} - штучное изделие, изготовленное на заказ. Имеет +1 к БПв и свойство "Новое". Параметр тСл уменьшается на 1 (это не влияет на ЕЗ предмета). Повысьте СП на 10. 
\paragraph{Шедевр} - больше произведение искусства, чем средство убийства.  Шедевр получает все преимущества Работы мастера. Увеличьте ЕЗ предмета в 1.5 раза. Шанс КУ Шедевра возрастает на 1. Повысьте СП на 15.

\begin{tcolorbox}
    Халтура - распространенное оружие статистов, для которых убиство и грабеж - хобби, а не средство заработка. Это не значит, что они не умеют или боятся им пользоваться. Просто весь их заработок уходит на что-то более практичное и необходимое, вроде инструментов для обработки земли или сырья для самогоноварения.
\end{tcolorbox}

\paragraph{Импровизированное оружие(Импро):} может обладать любыми свойствами и типом Пв. Например, вилы Длинные и Колющие, а оглобля Длинная, Громоздкая и Дробящая. Как правило, Импровизированное оружие напоминает боевое. Из разбитой пивной кружки получится превосходный кастет, а из солдатского ремня - цеп. 
\newline Обычно БПв, Прч и ЕЗ импровизированного оружия ниже, чем у его боевого аналога. 
\newline Если Импровизированное оружие использовалось для нанесения Пв, проверьте его Износ в конце Сцены.	

\paragraph{Удары плашмя:} большинство видов рубящего и колющего оружия можно использовать для ударов плашмя. Это редко имеет смысл в бою насмерть, но неплохо работает, когда противник защищен от колющих и режущих атак, или если надо захватить его относительно невредимым. 
\newline Удары плашмя наносятся с Помехой и имеют Дробящие Пв. Дальнобойное оружие также может использовать это правило - стрелок пускает стрелу или пулю вскользь. Удары плашмя могут (но не обязаны) следовать правилам Несмертельных Пв.

\paragraph{Несмертельные Повреждения (НПв):} иногда героям требуется захватить противника живьем. Для этих ситуаций идеально подходит оружие, наносящее Дробящие Пв. 
\newline При атаке оружием с Дробящими Пв герой может нанести число Несмертельных Пв не превышающее \textbf{|1 + ММд / Наблюдательность(Мд) / Медицина(Мд)|}. Если проверка Дб приводит к получению большего числа Пв, остальные Пв наносятся обычным образом. Игрок должен сообщить о применении НПв до проверки героя.
\newline Когда при нападении цель получает и НПв, и Пв, это может привести к Смерти или Перелому, если Наблюдательность или  МедицинаЭ нападающего недостаточно высоки.
\newline Медицина может быть заменена Ремонтом или Обращением с животными, если цель - механизм или является животным/растением.
\newline Цель получает НПв так же, как обычные Пв. Они могут привести к Опасной ране и прочим эффектам. Сломанные и отрубленные благодаря НПв конечности считаются вывихнутыми, если нападающий того пожелает. Все КУ, нанесенные героем, вызывают Оглушение. Если проверка героя понижает ЕЗ цели до 0, игрок вправе объявить, что она потеряла сознание и осталась жива. Потерянные от НПв ЕЗ восстанавливаются в первой же Интерлюдии (даже если жертва пролежала ее на холодной земле со связанными руками и ногами).

\paragraph{Оружие для больших и маленьких существ:} в таблицах представлено оружие для существ Среднего размера (МРз 0). Для существ иного размера оружие может изготовляться под заказ. 
\newline В этом случает, прибавьте МРз существа к тСл, БПв и СП оружия. Стоимость оружия для маленьких существ не меняется - сокращается стоимость материалов, но возрастает сложность работы. Увеличение СП за размер складывается с изменением СП за Халтуру, Работу мастера или Шедевр.
\newline Если существо ипользует оружие, изготовленное для существа иного размера, вычтите МРз существа из тСл оружия. Это не влияет на ЕЗ оружия.
\begin{tcolorbox}
    И опять Маленькие и Крошечные существа фактически прибавят свой МРз к тСл оружия, ведь минус на минус дает плюс. Странная штука эта математика.
\end{tcolorbox}

\paragraph{Модификатор размера и Боевой контакт:} чем больше существо, тем проще ему атаковать цели, не стоящие к нему вплотную. Совершая атаки в ближнем бою, существо прибавляет свой МРз к Боевому контакту и Дистанции поражения, а его оружие получает свойство "Длинное" и "Упредительный удар". Например, оружие ближнего боя в руках Большого существа с МРз 1 имеет Дистанцию поражения 2, а копье и подобные ему виды вооружения получат Дистанцию поражения 3. МРз не может понизить Дистанцию поражения до 0. То есть даже удар Крошечного существа будет иметь Дистанцию поражения 1.

\paragraph{Тип/Магазин/Скорострельность (ТМС):} объединенный параметр дальнобойных видов оружия, в котором через черту указаны тип, размер магазина и Скорострельность оружия.
\begin{itemize}
    \item Тип определяет, какие боеприпасы использует оружие.
    \item Магазин определяет, сколько выстрелов оружие сделает, прежде чем потребуется Перезарядка.
    \item Скоросрельность указывает, сколько зарядов оружие может выпустить в течение Действия героя. Скорострельность оружия ограничена числом снарядов в Магазине оружия. Например, оружие со Скорострельностью 10 не может выпустить 10 снарядов, если в магазине осталось только 3. В этом случае Скорострельность оружия составит 3, затем ему потребуется Перезарядка.
\end{itemize}
\begin{tcolorbox}
    Если у оружия есть свойство Магазин или Потребление, значит, у оружия есть и свойство Перезарядка, даже если это не указано.
\end{tcolorbox}

\subsection{Cвойства оружия}
\paragraph{Бронебойное:} оружие ополовинивает Бонус доспеха и щита цели. Обратите внимание, что это свойство не влияет на Прочность цели.
\paragraph{Винтовка:} за каждый пропуск Очереди во время Прицеливания
шанс Критического удара возрастает на 1. Максимальный бонус к КУ
не может превышать МИн героя (минимум +1).
\paragraph{Возврат Х:} при выпадении Х или большего числа во время проверки Мт оружие не только наносит Пв, но и возвращается в руку к владельцу.
\paragraph{Граната:} Оружие или боеприпас взрывается сразу или через некоторое время после попадания по цели. Эффект взрыва зависит от начинки и будет описан в особых свойствах оружия.
\paragraph{Громоздкое:} Изъян, из-за которого оружие используется с Помехой в тесных помещениях и густых зарослях. Если оружие одновременно и Громоздкое, и Длинное, герой получает 2 Помехи. Герой не может использовать Громоздкое оружие лежа для атак в ближнем бою.
\paragraph{Двуручное:} требует 2 рук для использования.
\paragraph{Дистанция X/Y:} Оружие с этим свойством является дальнобойным и не предназначено для совершения атак в ближнем бою. \textbf{X} - Ближняя Дистанция оружия, \textbf{Y} - Дальняя Дистанция оружия. В графе БПв для этго оружия через черту написан урон для Ближней и Дальней дистанции соответственно.
\paragraph{Длинное:} позволяет атаковать врага в 2 метрах от героя. Цели, находящиеся ближе, герой атакует с Помехой. Длинное оружие используется с Помехой в тесных помещениях, густых зарослях и других подобных условиях.
\paragraph{Естественное:} оружие, которое является продолжением тела героя. Естественное оружие использует Рукопашную Доблесть для атак в ближнем бою и не может быть выронено(в том числе за счет маневров Разоружение), но все еще может быть повреждено.
\paragraph{Кавалерийское:} удвойте успешно нанесенные Пв, если геройверхом и применяет Атаку с разбега.
\paragraph{Кастет:} позволяет использовать как Навык Рукопашного боя, так и Навык Владения оружием в формуле Дб.
\paragraph{Кувалда:} если нанесенные Пв превышают МСл атакованного существа, оно падает. 
\newline
Маленькие существа падают, если нанесенные Пв превышают 1/2 их МСл, Крошечные — падают, если нанесенные Пв превышают 1/4 их МСл.
\newline
Большие существа падают, если нанесенные Пв превышают их МСл более чем в 2 раза, Огромные — если нанесенные Пв превышают их МСл более чем в 3 раза, Громадные — если нанесенные Пв превышают их МСл более чем в 4 раза.
\newline
Если оружие ближнего боя с этим свойством используется в одной руке, увеличьте необходимые для падения Пв в 2 раза. То есть для того, чтобы сбить с ног существо Среднего размера, понадобится превысить его МСл более чем в 2 раза.
\paragraph{Легкое:} позволяет совершать серии молниеносных выпадов и эффективно атаковать с оружием в каждой руке.
 оружие устойчиво к внешним воздействиям - все проверки Износа для этого оружия совершаются с Преимуществом.
 \paragraph{Накопление заряда:} совершая маневр Прицеливание, герой повышает бПв оружия на 2 за каждую пропущенную очередь (макс +6). Если герой теряет бонусы Прицеливания, бонус Накопления заряда на атаку все равно остается. Так же герой может объявить Прицеливание из этого оружия не заявляя цель маневра. В этом случае он получает только бонус Накопления заряда.
\paragraph{Тип оружия:} Любое дальнобойное оружие нуждается в боеприпасах. Но благодаря унификации герою не нужно носить разные боеприпасы для \textit{каждой} еденицы оружия, которая есть у него в арсенале. Стоимость и свойства боеприпасов описаны в следующем разделе. Тип оружия определяет тип боеприпасов, который использует это оружие.
\paragraph{Магазин/Скорострельность(МСк) Х/Y:} дальнобойное оружие с этим свойством способно выпускать множество снарядов сразу или один за другим и не требовать Перезарядки сразу после первого выстрела. \textbf{X} количество выстрелов, которое может сделать оружие без Перезарядки, а \textbf{Y} — число выстрелов, которое герой может произвести за время своего Действия. Скорострельное оружие может использоваться двумя способами — для поражения одной цели или нескольких. При использовании скорострельного оружия герой может одновременно поразить число целей, не превышающее его \textbf{|ММд+1|} (минимум 2 цели).
\newline
Если Скорострельное оружие используется против одной цели, повысьте БПв оружия на 1 за каждый снаряд после первого, выпущенный по ней. Например, скорострельный арбалет имеет БПв +1 и Скорострельность 3. Если герой трижды стреляет из скорострельного арбалета в одну цель, БПв скорострельного арбалета возрастает до +3 (то есть на 2).
Герой не может совершать Быструю атаку скорострельным оружием, но может использовать место этого Беглый огонь. В разделе Маневры есть полное описание этого маневра.
\begin{tcolorbox}
Если у оружие есть свойство Магазин или Потребление, то это значит, что у оружия есть свойство Перезарядка, даже если это явно не указано. Оружие с МСк надо Перезаряжать, когда закончатся заряды в магазине, а оружие с Потреблением - когда опустошена Энергоячейка.
\end{tcolorbox}
\paragraph{Тип/Магазин/Скорострельность(ТМС):} объедененный столбец, в котором через черту отражены тип оружия, размер магазина и его скорострельность.
\paragraph{Метательное:} может быть использовано и в ближнем и в дальнем бою. При метании оружие обычно падает в область, в которой находилась цель. Оно может быть поднято и использовано повторно. При Дистанционной атаке Метательным оружием герой добавляет свой МСл к БПв оружия. У метательного оружия \textbf{нет} свойства Перезарядка - количество выстрелов в раунд ограничена только Скорострельностью оружия. Если герой держит в каждой руке два одинаковых Метательных оружия, он может метнуть оба в рамках маневра Беглый огонь или Атака, повысив скорострельность оружия на 1, но получив при этом Помеху на этом маневр и дополнительно вторую Помеху, если оружие Громоздкое. Если у героя есть трюк Амбидекстер, он получает на одну Помеху меньше при совершении этого маневра. В таблицах свойство Метательное указано в столбце ТМС как тип оружия.
\begin{tcolorbox}
Используя метательное оружие с дополнительным свойством \textbf{Снаряды} герой не метает оружие целиком, а использует подготовленные для него снаряды. Им нельзя производить атаку в Ближнего боя, однако стоимость выстрела гораздо ниже, когда во врага летит только часть оружия. СП 10 зарядов для такого вида оружия равна \textbf{|СП оружия — 10|}. Стоимость 1 заряда для этого оружия равна \textbf{|СП оружия — 12|} (минимум 1).
\end{tcolorbox}
\paragraph{Огнемет Х}: при атаке огнемет поражает все объекты, находящиеся на в радиусе Х от точки прицеливания. Совершите только одну проверку Меткости и определите повреждения для всех пораженных объектов, исходя из нее. Обратите внимание, что в случае выпадения Критического удара его эффекты применяются ко всем объектам, пораженным огнеметом.
\paragraph{Отдача:} герой должен отказаться от Перемещения, если желает выпустить из оружия с этим свойством 5 или больше зарядов.
\paragraph{Перезарядка:} для перезарядки оружия с этим свойством герой должен отказаться от Действия или Перемещения. Оружие не может использоваться при Быстрой атаке. Арбалеты и пороховое оружие требуют 2 свободных рук при перезарядке.
\paragraph{Потайное:} герой получает Преимущество на проверки Скрытности и Ловкости рук при попытках спрятать оружие.
\paragraph{Потребление Х:} Оружие с этим свойством вместо боеприпасов использует энергию и тратит Х Зарядов за 1 выстрел или удар.
\paragraph{Противотанковое:} оружие предназначено для поражения тяжелобронированных целей. Это оружие игнорирует Прочность цели при попадании. Если у цели нет Прочности, она совершает проверку Внезапной Смерти со штрафом, равным величине успеха при попадании. Для статистов эта проверка считается автоматически проваленной.
\paragraph{Пулемет:} оружие не приспособлено для одиночных выстрелов. Если герой использует оружие для поражения нескольких целей, то должен выпустить минимум 3 пули в каждую.
\paragraph{Снайперское:} при стрельбе на дальнюю дистанцию оружие игнорирует штрафы Зон поражения, если герой проводит Прицеливаясь хотя бы один полный Круг. При стрельбе на ближней дистанции герой получает Помеху.
\paragraph{Сошки(с):} оружие оснащено сошками для стрельбы с упора. Если носитель имеет возможность установить сошки на какую-либо поверхность, используйте значение Минимальной силы с пометкой (с), в противном случае используйте значение после черты. Если у оружия нет значения Минимальной силы без символа (с), значит, стрельба с рук невозможна.
\paragraph{Тип Повреждений:}
\input{battles/damage-types-common}
\newline
Если тип повреждений отмечен *, то он зависит от заряженных в оружие боеприпасов. См. раздел боеприпасов для определения типа повреждений.
\newline
Если оружие имеет несколько типов повреждений через черту(К/Р или Р/Д), герой может выбирать, какой тип повреждений наносить при атаке. Если типы повреждений перечислены слитно(ОЭ или РЯ), то оружие наносит одновременно несколько типов повреждений. Подробнее об этом написано в главе Боевые Столкновения.
\paragraph{Удавка:} этим свойством обладает любое оружие, достаточно гибкое, чтобы захлестнуть конечности противника. Для применения свойства необходимо две руки. Оружие может использоваться для Захватов. Удвойте Пв, успешно нанесенные Удавкой в шею. Существа, чьи ЕЗ опустились до 0 в результате Пв, нанесенных Удавкой в шею, теряют сознание, а не умирают, если атакующий того пожелает. Существа, у которых БД (но не БЩ) составляет 6 или больше, не могут быть атакованы Удавкой в шею — она надежно защищена.
\paragraph{Универсальное:} может использоваться в 1 или 2 руках. Смена хвата является Быстрым действием. В БПв и тСл указаны для одной и двух рук через черту.
\paragraph{Упредительный удар:} если противник входит в зону поражения оружия, герой может потратить Быстрое действие и провести внеочередную атаку по нападающему или его скакуну. Выход из зоны действия оружия и передвижение в ней не провоцируют Упредительный удар.
\paragraph{Фехтовальное:} герой может заменить МСл на ММд при подсчете Дб.
\paragraph{Хрупкое:} Изъян, из-за которого оружие уязвимо к самым незначительным повреждениям и не предназначено для парирования и блокирования. Ополовиньте ЕЗ оружия(минимум 1 ХП). Так же все проверки Износа этого оружия совершаются с Помехой.
\paragraph{Цеп:} игнорирует БЩ, хотя может заявлять щит как область поражения.
\paragraph{Символ "*"} обозначает качества оружия, не указанные в таблице, не входящие в унифицированный перечень Свойств и перечисленные в описательном блоке. \textbf{Символ "Ф"} обозначает оружие, уместное лишь в фантастическом антураже.


\subsection{Боеприпасы дальнобойного оружия}
Разные оружия применяют разные боеприпасы. Однако со временем производители начали вводить унифицированные боеприпасы для разных калибров, что позволило использовать одни и те же боеприпасы для не совсем одинакового вооружения.
\paragraph{Тип} боеприпаса определяет, в какие пушки можно его заряжать.
\paragraph{Количество} определяет, сколько зарядов можно приобрести, купив стандартную коробку боеприпасов. Если в этой графе стоит прочерк, боеприпас нельзя приобрести в стандартизированном варианте.
\begin{center}\begin{tabular}{|c|p{3cm}|p{10cm}|c|}
\hline
Тип & Оружие, использующее боеприпасы & Описание боеприпасов & Кол-во\\ \hline
П & Пистолеты, пистолеты-пулеметы & Малый калибр, низкая цена & 30 \\ \hline
Р & Револьверы, крупнокалиберные пистолеты & Тяжелые и дорогие пули & 15 \\ \hline
В & Винтовки & Быстрота, высокая пробивающая способность & 20 \\ \hline
Д & Дробовики & Дробь, картечь, реже - пули & 4 \\ \hline
О & Пулеметы и крупнокалиберные орудия & Действительно большие калибры. Настолько, что в них можно поместить взрывчатку & 10 \\ \hline
Г & Гранатометы и ракетные установки & Унифицированные гранаты и ракеты. Поражающий эффект зависит от начинки & 1 \\ \hline
Э & Энергетическое оружие & Батареи всех мастей. Если у оружия класса "Э" нет свойства "Потребление", оно тратит 1 Заряд за выстрел & - \\ \hline
Б\textsuperscript{ф} & Кинетческие ускорители & Металлические болты со сверхпрочным сердечником. Дешевы в изготовлении, смертоносны на больших скоростях. & 40 \\ \hline
М & Метательное оружие & Использует метательный боеприпас и приводится в действие силой стрелка & - \\ \hline
У & Уникальный боеприпас & Оружие использует не стандартизированный боеприпас, который подойдет только для него. Если СП не указана в описании оружия, то СП 10 зарядов для такого вида оружия равна \textbf{|СП оружия — 10|}. Стоимость 1 заряда для этого оружия равна \textbf{|СП оружия — 12|} (минимум 1). & * \\ \hline
% Ф & Феномены & Оружие использует Энергию героя для работы. & - \\ \hline
\end{tabular}\end{center}
* если у метательного оружия есть свойство Снаряды. Иначе каждую еденицу оружия надо приобретать отдельно.

\paragraph{} Кроме стандартных боеприпасов можно разжиться и специализированными. Они дают дополнительные свойства или улучшают свойства оружия. В таблице указаны наиболее распространенные типы боеприпасов и стоимость их стандартных коробок. Значения Дистанции, БПв, ТПв, КУ в таблице изменяют стандартные параметры оружия на указанное значение. 
\newline В столбце \textbf{Тип} указаны типы боеприпасов, для которых возможна модификация.
\newline В столбце \textbf{СП} указана стоимость стандартной коробки боеприпасов. 
\newline В столбце \textbf{КУ} отрицательные значения означают расширение диапазона КУ, а положительные - его сужение. Итоговое значение КУ оружия не может превышать 20 и быть меньше 1. Если боеприпас выводит КУ за пределы этих значений, то они становятся равными 20 и 1 соответственно. 

\begin{longtable}{|p{3cm}|p{2.5cm}|c||c|c|c|c||c|}\hline
Название & Особые свойства & Тип & Дистанция & БПв & ТПв & КУ & СП\\ \hline
Стандартные & - & - & - & - & - & - & 8\\ \hline
Утяжеленные & Оружие получает свойство "Кувалда" & РВОГ & -10/-20 & - & +Д & +1 & 10\\ \hline
Зажигательные & - & РВДО & - & +1/+1 & +О & -3 & 11\\ \hline
Бронебойные & - & РВДОГБ & - & +2/+2 & - & +2 & 11\\ \hline
Пустотелые & - & ПРВОБ & - & -1/-1 & - & -5 & 9\\ \hline
Подкалиберные & - & ПРВДО & +10/+20 & -1/-1 & - & - & 9\\ \hline
Разрывные & Полученные целью Пв удваиваются & ПРВДО & -10/-20 & -2/-2 & +Р & +1 & 12\\ \hline
Дозвуковые & При использовании глушителя патроны не производят хлопка. Стрелок не обнаруживает себя при промахе & ПРВО & -5/-10 & -1/-1 & - & - & 9\\ \hline
\end{longtable}


\newpage
\subsection{Перечень оружия ближнего боя}
Если в этой таблице у оружия есть Дистанции, это значит, что его можно как использовать в ближнем бою, так и метать в противника.
\genAndGet{weapons}{weapons}{Рукопашное}

% \newpage
% \subsection{Вспомогательное и импровизированное оружие}
% Все, что не является оружием, можно использовать как оружие. Но некоторые вещи просто напрашиваются на то, чтобы ими кого-нибудь ударили, порезали или проткнули.
% \genAndGet{weapons}{weapons}{Вспомогательное}

\newpage
\subsection{Перечень метательного оружия}
\genAndGet{weapons}{weapons}{Метательное}

\newpage
\subsection{Перечень огнестрельного оружия}
\genAndGet{weapons}{weapons}{Огнестрельное}

\newpage
\subsection{Перечень энергетического дальнобойного оружия}
\genAndGet{weapons}{weapons}{Энергетическое}

\newpage
\subsection{Бластеры\textsuperscript{ф}}
Очередная ипостась электричества на службе цивилизации и прогресса. Теперь бластеры готовы служить любому, кто врубается в их конструкцию.
\newline Бластер является фантастической модификацией дальнобойного оружия, благодаря которой вместо боеприпасов оно потребляет Зр. СП бластеров на 2 выше, чем в таблице оружия, а тип повреждений меняется на ОЭ.
\newline Любой бластер может использоваться в тазер-режиме. Цель не получает Пв, но должна преуспеть в проверке Вн против \textbf{|5 + [Величина успеха проверки Мт]|} или потерять сознание.\newline Потребление бластеров зависит от типа оружия:
\begin{itemize}
\item \textbf{Тип П} -- потребление 1
\item \textbf{Тип Р} -- потребление 3
\item \textbf{Тип В} -- потребление 2
\item \textbf{Тип Д} -- потребление 15
\item \textbf{Тип О} -- потребление 5
\end{itemize}
Бластер нельзя сделать из оружия типа Г,Б,М,У и, очевидно, Э.% и Ф.

\printindex[weapons]

\subsection{Гранаты и взрывы}
\paragraph{}
В историях всегда найдется место бутылкам с горючей смесью, емкостями с едкими летучими веществами и старым добрым гранатам. И это, и многое другое можно метнуть во врага. А потом наблюдать с безопасного расстояния, как он мечется в бессильных попытках погасить пламя, истекает кровью, изрешеченный осколками... или как бутыль предательски падает в траву, а отсыревший запал гаснет.
\paragraph{}
Гранаты — Метательное оружие с БПв = \textbf{|-2|} на Ближней и \textbf{|-4|} на Дальней дистанции. Непосредственно удар гранаты имеет Дробящие Пв и наносит КУ при выпадении 20. Используйте Мт, чтобы определить, поразил ли герой цель. Не забывайте, что гранаты можно метать в землю, поражая Зщ 10, если только герой не собирается нанести Повреждения самим броском!
\newline
Все бомбы и гранаты считаются одним видом оружия.
\paragraph{}
В случае промаха граната все равно взорвется (если только не выпала Осечка). Граната отклоняется на Х метров, где Х равен промаху героя по Зщ цели — ловкий противник может успеть отбросить гранату, а бронированный — отбить щитом или латным рукавом! Определите направление, в котором отклонилась граната, при помощи броска К20. В этом случае граната может пролететь большее расстояние, чем максимальная дистанция броска!
\begin{center}
\begin{tabular}{ |p{2.7cm}|p{12cm}| }
\hline
\textbf{К20} & \textbf{Направление}
\\ \hline
1-5 & Граната перелетела за цель
\\ \hline
6-10 & Граната отклонилась вправо от цели
\\ \hline
11-15 & Граната недолетела до цели
\\ \hline
16-20 & Граната отклонилась влево от цели
\\ \hline
\end{tabular}
\end{center}
\begin{tcolorbox}
Если нужно больше детализации по направлению, можно использовать принцип дартс: область вокруг цели делится на 20 равных секторов и каждому из них сопоставляется число, которое определяет направление отклонения.
\newline
\begin{center}
\includegraphics[width=0.5\textwidth]{darts}
\end{center}
\end{tcolorbox}
\paragraph{Вес} представленных гранат равен 0.5.
\paragraph{Дистанция} у гранат обычно Ближняя 5, Дальняя 20.
\paragraph{Центр Взрыва:} точка, из которой распространяется Взрыв.
\paragraph{Радиус Взрыва(РВ):} число метров, на которое распространяется Взрыв из центра Взрыва во все стороны. Если атака обладает свойством <<Взрыв>>, то цель, против которой совершается проверка Доблести или Меткости, также считается находящейся в области Взрыва и получает Повреждения и прочие эффекты и от него тоже!
\paragraph{Сила Взрыва(СВ):} все существа, попавшие в радиус Взрыва, получают Пв, равные \textbf{|Сила Взрыва — БАЗщ — БД|}.
\paragraph{Газ:} газ распространяется в радиусе Взрыва и остается там какое-то время. Для определения длительности действия газа совершите проверку Неприятностей. Существа, вдохнувшие газ, страдают от эффектов яда, громко кашляют и чихают, если они еще в состоянии кашлять и чихать. Существа, имеющие Иммунитет к Ядовитым Пв, не подпадают под действие газа. В помещении увеличьте временные промежутки вдвое.
\trouble
{Штиль}%no sweat name
{Газ рассеивается чере 10 минут}%no sweat description
{Бриз}%tough day name
{Газ рассеивается через 5 минут}%tough day description
{Порыв ветра}%we have trouble name
{Газ рассеивается через 1 минуту}%we have trouble description
{Шторм}%fiasco name
{Газ рассеивается по истечении полного Круга}%fiasco description

\paragraph{Другие опасности Взрывов:} Взрывы опасны не только Повреждениями. Попавшие во Взрыв доспехи, щиты и прочие предметы, закрепленные на теле попавшего во Взрыв, получают \textbf{|Пв = Сила Взрыва — Прч|} и могут быть уничтожены. Если Взрыв достаточно силен, попавшие в него существа с воплями разлетаются в разные стороны! Если существо получает Повреждения от взрыва, его отбрасывает от центра Взрыва на \textbf{|Сила Взрыва — МСл отброшенного — МЛв отброшенного|} метров. Отброшенный получает столько Пв, сколько метров пролетел. Когда отброшенный сталкивается с другим существом, оно должно совершить проверку Атлетики(Сл) против \textbf{|Зщ + мВн отброшенного|}, чтобы поймать отброшенного или Атлетики(Лв), чтобы увернуться. В случае провала, оба существа получают Пв, равные величине провала. Если другое существо увернулось, отброшенный продолжает полет и получает Пв согласно расстоянию, которое он пролетел. Существо не может быть отброшено на большее число метров, чем получило Пв от Взрыва. Увеличьте расстояние броска в 2 раза за каждую категорию размера меньше Среднего и уменьшите в 2 раза за каждую категорию больше Среднего. Падения и столкновения наносят Дробящие Пв.
\newline
Если при подсчете дистанции отбрасывания получилось 0 или меньше, существо остается на месте.



\subsection{Защита}
Весь перечень того, что в состоянии защитить тело от ударов и выстрелов - от вонючей шкуры дикаря до тактического силового доспеха. 
\paragraph{Бонус к Защите:} доспехи дают герою Бонус доспеха к Защите (БД), а щиты дают герою Бонус щита к Защите (БЩ).
\newline Герой не может надеть 2 комплекта доспехов одновременно, но может одновременно использовать 2 щита. В этом случае он получает БЩ обоих щитов.
\paragraph{Ограничитель модификатора Ловкости (оМЛв):} в большинстве своем доспехи и щиты - тяжелые конструкции, сковывающие движения. Многие доспехи и щиты ограничивают МЛв, доступный герою при Защите и совершении проверок. Так, герой с 20 Лв (МЛв +5), надевший кольчужную рубаху, будет добавлять к Зщ и проверкам Навыков, основанных на Ловкости, лишь +3, потому что ограничитель МЛв кольчужной рубахи составляет +3. Герой с высокой Лв, надевший доспех с низким оМЛв, столкнется с понижением Дб и Мт.
\paragraph{Требуемая Выносливость (тВн):} герой не может носить доспех или щит, если не обладает достаточной для этого Выносливостью. Сила не играет здесь важной роли - герой может поднять доспех и выдержать его вес, но выдохнется после нескольких минут активных действий. Разумеется, он может облачиться в доспех, чтобы повыпендриваться перед девчонками или попозировать фотографу, но в бою выглядит жалко. Все активные проверки такого героя совершаются с Помехой, а все атаки по нему совершаются с Преимуществом.
\paragraph{ЕЗ доспеха или щита = |тВн|.} Если доспех имеет несколько значений тВн, для подсчета ЕЗ используется максимальное.	
\paragraph{Прочность доспехов и щитов = |1/2 ЕЗ доспеха или щита|.} 
\paragraph{БПв щита.} Большинство щитов можно использовать, как оружие с КУ 20 и ТПв Д. Если в этом столбце указан прочерк, то бить щитом нельзя.
\paragraph{Доспехи и Осечка:} доспехи и щиты так же, как и оружие, подвержены износу и могут иметь свойство \textbf{"Осечка Х"}. Если герой, снаряженный доспехом или щитом, совершает Активную проверку и выбрасывает на К20 число, равное Х или меньше, проверка проваливается. 
\paragraph{Количество помех для проверок Скрытности (ПС):} защита может шуметь при движении - звон сочленений, скрежет пластин, шипение сервоприводов и гул силовой установки - все это мешает герою  скрывать свое присутствие.

\paragraph{Халтура:} дешевый, кое-как изготовленный щит или доспех тяжел и неудобен. Его тВн увеличивается на 1 (это не влияет на ЕЗ предмета), а оМЛв понижается на 1. Доспех или щит имеет 1/2 ЕЗ и 1/2 Прч и Осечку 5. Если щит или доспех не имел оМЛв, то он получает оМЛв +4. Когда герой получает КП на Активную проверку, доспех разрушается. Понизьте СП на 5. Если СП достигает 0, это означает, что изготовление доспеха требует Интерлюдии, которую герой тратит на сбор материалов, и Антракта на соединение их друг с другом.
\newline Халтурные доспехи и щиты - отличный выбор для тех, кто живет одним днем, но не слишком торопится умирать. То есть для абсолютного большинства жителей разрушенного мира, вынужденных иногда принимать участие в боевых действиях. Профессиональные воины пользуются Халтурной защитой лишь в крайних случаях.
\paragraph{Работа мастера.} Доспех или щит, изготовленный мастером, безупречно подходит заказчику. ТВн уменьшается на 1 (это не влияет на ЕЗ предмета), а оМЛв повышается на 1. СП повышается на 5.
\paragraph{Шедевр} представляет собой безупречный экземпляр мастерской работы и получает все преимущества работы мастера. ЕЗ предмета увеличиваются в 1.5 раза. В дополнение БД или БЩ возрастает на 1, а ПС понижается на 1. СП повышается на 10.
\newline Работа мастера и шедевр - штучные изделия. Все, кроме хозяина, при ношении такого доспеха получают все логически возможные эффекты халтуры, пока доспех не будет подогнан сведущим технарем под нового владельца.
\paragraph{Надевание и снятие доспехов и щитов:} время, необходимое для этого, зависит от оМЛв доспеха или щита:
\begin{itemize}
    \item[--] Доспех или щит не имеет оМЛв, время надевания и снятия составляет 1 минуту.
    \item[--] ОМЛв доспеха +2 и выше, время надевания составляет 5 минут, а время снятия составляет 1 минуту.
    \item[--] ОМЛв +1 и ниже, время надевания составляет 10 минут, а время снятия составляет 5 минут.
    \item[--] Кто-то помогает герою, то время снятия и надевания сокращается вдвое.
\end{itemize}
\paragraph{Доспехи для больших и маленьких существ:} в таблицах представлены доспехи и щиты для существ Среднего размера. Для существ иного размера доспехи изготовлются под заказ. 
\newline В этом случает, прибавьте МРз существа к тВн, БД, БЩ и СП доспехов и щитов. Стоимость доспехов для маленьких существ не меняется - сокращается стоимость материалов, но возрастает сложность работы. Увеличение СП за размер складывается с изменением СП за Халтуру, Работу мастера или Шедевр.
\paragraph{Атаки по доспехам и щитам:} доспех или щит могут быть выбраны зоной поражения (подробнее смотрите маневр "Сломать снаряжение" в разделе "Маневры"). Носитель получает Пв только в том случае, если доспех или щит получают Пв, которые не смогли полностью поглотить их Прч и ЕЗ.
\newline Если доспех или щит уничтожены, носитель теряет их БД и БЩ, однако его МЛв все еще ограничен оМЛв доспеха или щита.
\paragraph{Иммунитет, Сопротивление и Уязвимость к Повреждениям:} доспехи и щиты обладают иммунитетом к Ядовитым Пв, Сопротивлением к Колющим, Проникающим и Ледяным Пв и Уязвимы к Едким Пв. 
\begin{tcolorbox}
    При встрече с тяжелобронированным противником будет хорошей идеей сначала уничтожить его доспех и щит, вместо того чтобы пытаться поразить заоблачную Зщ. Хотя, тогда снаряжение не выйдет продать. 
\end{tcolorbox}

\subsection{Перечень Доспехов}
\genAndGet{protection}{protection}{Доспех}

\subsection{Перечень Щитов}
\genAndGet{protection}{protection}{Щит}



\subsection{Медикаменты и яды}
\paragraph{Интоксикация.} получение (Я)довитых Пв не приводит к потере ЕЗ напрямую и подсчитывается отдельно от остальных Пв. Интоксикация является суммой всех Ядовитых Пв, полученных героем, и отражает то, сколько яда попало в организм и как долго он может сопротивляться яду.
\paragraph{Максимальное значение Интоксикации} составляет \textbf{|Максимальные ЕЗ героя х 1.5|}. 
\paragraph{Интоксикация и смерть:} как только значение Интоксикации героя превышает \textbf{|Максимальные ЕЗ героя х 1.5|}, он должен совершить проверку Вн против \textbf{|15|}. При провале герой погибает (как правило, в страшных мучениях). При успехе он впадает в кому и становится Неподвижным до тех пор, пока его Интоксикация не уменьшится.
\begin{tcolorbox}
    Да, проверять Выносливость придется с Помехой за Отравление.
\end{tcolorbox}
\paragraph{Токсичность (Токс):} количество Ядовитых повреждений, которое наносит Лекарство или Яд при применении.
\paragraph{Первичный эффект:} Разовый эффект Лекарства или Яда. Для сопротивления эффекту герой должен преуспеть в проверке Вн против \textbf{|10+Токс|}.
\paragraph{Отравление:} пока значение Интоксикации превышает текущие ЕЗ героя, он получает состояние «Отравлен». 
\paragraph{Снятие состояния Отравления:} герой должен добиться того, чтобы значение Интоксикации не превышало его текущие ЕЗ. Он может достичь этого, как повысив текущие ЕЗ, так и понизив Интоксикацию.
\paragraph{Побочки:} целебные зелья, таинственные эликсиры и мощные стимуляторы зачастую являются не очень-то полезными. Особенно при частом применении. 
\newline Как только герой становится Отравлен, на него немедленно накладываются эффекты Побочек Лекарств и Ядов, которые наносили ему Ядовитые Пв в этой Сцене. 
\newline Если на героя подействовало лекарство или яд, пока герой Отравлен, Побочки начинают действовать \textit{сразу}.
\newline Эффекты Побочек прекращаются, как только герой перестает быть Отравлен (если в описании Побочек не сказано иного).
\begin{tcolorbox}
    Игрок может отказаться от совершения проверки Вн, если считает, что Первичный эффект пойдет герою на пользу. Судьбе виднее! Учтите, что иногда решение придется принимать наугад, до того, как эффекты войдут в игру. Например, если у пузырька с таблетками стерлась этикетка. Или когда единственный источник информации об эффектах зелья – приготовивший его шаман. Разумеется, у Судьбы есть свои способы добиться нужного результата – о них вы уже читали в главе «Нити, Ходы и Капризы».
\end{tcolorbox}
\paragraph{Антракт и побочки.} Антракт завершает действие всех Побочек, если в описании Яда или Лекарства не сказано обратного. Если при этом Интоксикация героя выше, чем текущие ЕЗ, он все еще в состоянии «Отравлен».
\paragraph{Интоксикация и Отдых:} во время Интерлюдий и Антракта герой понижает Интоксикацию на столько же единиц, сколько ЕЗ восстанавливает. Если герой применяет стимулятор или зелье, то он не понижает Интоксикацию, если это прямо не указано в описании препарата. 
\newline Если герой обращается за помощью к врачу или целителю, то снижение Интоксикации является отдельной Услугой. 

\subsection{Медикаменты}
\paragraph{Время приема (ВП):} время, необходимое для употребления одной порции Лекарства. У Ядов отсутствует ВП, т.к. оно сильно зависит от формы отравляющего вещества. 
\genAndGet{drugs}{drugs}{Лекарство}
\printindex[potions]

\subsection{Яды}
\paragraph{Отравленное оружие:} Яд на оружии действует в течение колличества атак, равных его Токсичности. Это правило распространяется и на оружие ближнего боя, и на дальнобойное оружие. Для дальнобойного оружия это значит, что порция Яда была распределена равномерно между боеприпасами. 
\newline Получив хотя бы 1 Пв, цель дополнительно получает Ядовитые Пв в размере Токсичности Яда и должна сопротивляться его Первичному эффекту.
\paragraph{Условия приема} имеют значение для ядов. их четыре – вдох, контакт, порез, проглатывание.
\paragraph{КУ Ядовитых повреждений:} если Ядовитая атака наносит КУ, Побочки Яда сразу входят в игру и действуют (если эффект можно растянуть во времени), пока герой не уйдет в Антракт, получит антидот или значение Интоксикации не упадет до 0.
\genAndGet{drugs}{drugs}{Яд}
\printindex[poisons]
\begin{tcolorbox}
    Как правило, те, кто использует Яды для всяких злодейских дел, не скупятся, и сразу скармливают жертве несколько доз.
    \newline Для тех же, кто ищет в Ядах силу – например, исказителей с Трюком «Третий глаз», галлюциноген – самый удачный выбор.
\end{tcolorbox}

\section{Инструменты Могущества}
В этом разделе описаны могущественные артефакты, высокотехнологичные прототипы и технологические чудеса, которые позволяют герою достичь и преодолеть предел своих возможностей. Обычно Инструменты Могущества встречаются только в фантастическом антураже.

\paragraph{Название} Инструмента отражает его дух, историю или возможности. Редко названия Инструментов бывают обыденными.
\paragraph{Базовый предмет} Зачастую Инструмент построен на основе уже существующего, а иногда и весьма распространенного в мире, предмета, оружия или элемента экипировки. Обычно Инструмент сохраняет все характеристики Базового предмета и лишь усиливает их своими способностями.
\paragraph{Запас энергии: } количество Зарядов Инструмента которые можно тратить на активацию его Функций. Способ восстановления этой энергии может сильно отличаться в зависимости от антуража.
\paragraph{Сложность Приобретения(СП): }некоторые Инструменты научились довольно ловко воспроизводить и их даже можно увидеть на полках магазинов! Но если в описании нет СП, то Инструмент нельзя просто купить. Придется искать его в древних руинах, секретных лабораториях, а иногда и выпрашивать у богов. Если герой захочет продать Инструмент без СП, то количество Богатства, которое он получит от продажи будет определено исключительно его красноречием и умением торговаться.
\paragraph{Описание }Инструмента говорит о его виде и о нарративе его использования. Механика его работы описана в Трюках, Функциях, Изъянах и Ходах Инструмента.
\paragraph{Трюки }Инструмента - это его свойства, которые герой может использовать когда захочет или же они работают постоянно.
\paragraph{Функции }Инструмента требуют трату Зарядов для активации. В описании каждой Функции Инструмента в скобках указана её \textbf{Стоимость}.
\paragraph{Ходы }Инструмента позволяют совершать невозможное, но требуют за это обрыв Нитей героя. В описании каждого Хода Инструмента обязательно указана его \textbf{Стоимость}.
\paragraph{Изъяны }Инструмента указывают на любое негативное влияние Инструмента на его владельца. Это может быть как незначительное неудобство, так и серьезное проклятие.
\genAndGet{power-tools}{power-tools}

\section{Импланты}
Импланты - это подвид Инструментов могущества, который имеет некоторые особенности:
\begin{itemize}
\item Имплант нельзя просто снять или надеть - для этого требуется помощь специалиста, и порой очень дорогостоящая.
\item У Имплантов нет своего Запаса энергии - они всегда используют Энергию владельца на активацию своих Функций.
\item Стоимость Функций Имплантов указана не в Зарядах, а в Энергии. Работа имплантов это очень энергозатратный процесс.
\item Имплант найти гораздо легче, чем другие Инструменты Могущества - у них почти всегда есть СП.
\end{itemize}

\genAndGet{powerTools}{implants}
\section{Транспорт}
В этом разделе описаны механизмы и существа, которых можно использовать, как транспортные средста (ТС).
\paragraph{Символ "Ф"} обозначает вид транспорта, уместный лишь в фантастическом антураже.
\paragraph{Символ «*»} обозначает качества ТС, перечисленные в описательном блоке.
\paragraph{Ограничение Модификатора Ловкости(оМЛв):} крупные ТС ворочаются неохотно, а разогнавшись, не спешат тормозить. оМЛв определяет максимальный МЛв, который водитель может добавить к Эксплуатации при управлении ТС.
\paragraph{Ограничение Модификатора Ловкости(оМЛв) для животных:} не вся животина охотно подчиняется командам погонщика или наездника. Максимальный МЛв, который может герой добавить к проверкам Обращения с животными равен МЛв самого животного.
\paragraph{Проходимость транспортного средства(П):} ТС не может Путешествовать по местности с Опасностью, превышающей его Проходимость. 
\newline Проходимость животных увеличивается на 1, если они не несут груза или всадников.

\paragraph{Защита транспортного средства:} предполагается, что транспортные средства из таблицы уже оснащены защитными приспособлениями в пределах разумной целесообразности. 
\newline МЛв водителя прибавляется (или отнимается, если он отрицательный) к Зщ ТС.
\paragraph{Маневренность (М):} во многих ситуациях важны не только Проходимость и Скорость ТС, но и его способность избегать препятствий. Маневренность ТС = \textbf{|Проходимость — МРз|}. Маневренность добавляется к навыку водителя (или отнимается, если она отрицательная) при проверке ЭксплуатацииЭ для совершения сложных поворотов и перестроений.
\paragraph{Cкорость ТС во время Путешествий (Ск):} она значительно ниже максимума, который можно выжать из транспорта и измеряется в км/ч. 
\newline Если в графе присутствует \textbf{символ М}, это значит, что обычная скорость ТС равна максимальной и оно не может двигаться Маршем (если только скорость самого медленного участника каравана не в два раза меньше, чем Ск этого ТС).
\paragraph{Ск ТС во время Боевых сцен:} она численно равна Ск ТС во время Путешествий, но измеряется в метрах/клетках на тактической карте, которые ТС может преодолеть за свою Очередь. Это отражает необходимость маневрировать и лавировать между другими участниками Сцены (или давить их). Механическое ТС может совершать маневры ближнего боя, если оснащение и контекст позволяют это. Проверки Дб в этом случае заменяются ЭксплуатациейЭ водителя.
Животные используют в Боевых сценах карточки из раздела «монстры и статисты».
\paragraph{Расход топлива (Р)} определяет сколько Зарядов Топлива (или килограммов фуража для животных) тратит тот или иной вид транспорта на каждые 10 км пути.
\paragraph{Грузоподъемность/Вес (Г/В)} определяет количество полезной нагрузки, которую может везти ТС, и вес самого ТС для определения возможностей буксировки.
\paragraph{Буксировка:} предполагается, что ТС может буксировать вес, не превышающий его грузоподъемность в 10 раз.
\paragraph{Перегрузка:} если нагрузка ТС превышает Грузоподъемность не более, чем в 2 раза, либо буксируемый вес превышает вес ТС не более, чем в 2 раза, водитель получает Помеху на Эксплуатацию.
\paragraph{Сильная перегрузка:} если нагрузка ТС превышает Грузоподъемность не более, чем в 3 раза, либо буксируемый вес превышает вес ТС не более, чем в 3 раза, водитель получает 2 Помехи на Эксплуатацию.
Повреждение ТС:} после потери 1/3 ЕЗ ТС теряет половину Ск. После потери 2/3 ЕЗ его Эксплуатация проверяется с Помехой.
\newline Если транспортное средство одномоментно теряет 1/5 или более ЕЗ, водитель должен проверить Эксплуатацию против \textbf{|15|}. При провале транспортное средство глохнет.
\newline Когда зона поражения одномоментно получает Пв, равные 1/4 или более от максимальных ЕЗ ТС, она уничтожается. Все проверки водителя, связанные с повреждением зоны, считаются Критически проваленными. 
\paragraph{Зоны поражения ТС:} при атаке ТС могут быть выбраны Зоны поражения, критичные для его работы. Нападающий должен указать зону и проверить Дб или Мт с -2.
\newline Если Зона поражения одномоментно получает Пв, составляющие 1/5 или более от максимальных ЕЗ ТС, она временно выходит из строя. Лишние Пв теряются, но водитель должен проверить Эксплуатацию против \textbf{|15|}. При провале транспортное средство глохнет. Помимо этого:
\begin{itemize}
    \item[--] Повреждение Двигателя и трансмиссии приводят к тому, что ТС глохнет. Требуется успешно проверить Ремонт против \textbf{|15|}, чтобы снова завести его;
    \item[--] Повреждение Ходовой части приводит к потере управления. Водитель проверяет Эксплуатацию против \textbf{|15|}. При провале ТС терпит крушение;
    \item[--] Повреждение Орудия или Манипулятора выводит их из строя. Для того, чтобы восстановить функционал, требуется успешно проверить Ремонт против \textbf{|15|};
    \item[--] Повреждение Кабины или Пассажирского отсека сбрасывает водителя или пассажиров. Они должны успешно проверить Лв или Атлетику(Лв) против \textbf{|15|}. При провале герои выпадают из транспорта и получают Дробящие Пв, равные \textbf{|[величине провала]*[Опасность местности]|}. Если водитель и пассажиры пристегнуты ремнями безопасности, они совершают проверки с Помехой.
\end{itemize}
\paragraph{Атаки по водителю и пассажирам:} элементы конструкции ТС служат сносным укрытием для находящихся внутри. Поэтому водитель и пассажиры:
\begin{itemize}
    \item[--] Находятся в Мягком укрытии, если ТС имеет Прч 5 – 10.
    \item[--] Находятся в Твердом укрытии, если ТС имеет Прч 11 и больше.
\end{itemize}
Некоторые ТС обладают полностью закрытыми корпусами, и управляются изнутри при помощи телеметрических приборов. В этом случае находящиеся внутри не могут быть выбраны целью.
\paragraph{Инос ТС:} техника может выдержать многое, но многое может и не выдержать. Износ ТС проверяется в конце Сцены, если:
\begin{itemize}
  \item[--] ТС перегружено;
  \item[--] ТС Сильно перегружено;
  \item[--] ТС передвигалось по местности с Опасностью, превышающей его Проходимость;
  \item[--] ТС используется очевидно опасным или нецелевым образом. Для Техники размером Б и меньше это включает намеренный наезд на существо величиной с человека или крупного пса.
\end{itemize}
Животные не проверяют Износа, но получают Пв в зависимости от контекста Сцены.

\subsection{Гужевой транспорт}
Животные способны преодолевать водные преграды, а некоторые - еще и летать!
\genAndGet{transport}{transport}{Животное}

\subsection{Наземный транспорт}
\genAndGet{transport}{transport}{Наземный}

\subsection{Водный транспорт}
Проходимость этого транспорта означает то, насколько глубокая у него посадка. Чем выше проходимость, в тем более мелких реках и ручьях может передвигаться транспорт.
% \paragraph{Буксировка водного транспорта} Так как на воде обычно нет значительных перепадов высот, а вода не сильно сопротивляется движению, любые плавсредства можно буксировать, если их вес (вместе с барахлом) не превышает вес Буксира в ??? раз.
\paragraph{Перегрузка водного транспорта} В отличае от наземного транспорта, перегруженный водный транспорт не ломается, а начинает \textbf{тонуть}.
\tbd проверка неприятностей
\paragraph{Сильная перегрузка водного транспорта} приводит к немедленному затоплению ТС.

\genAndGet{transport}{transport}{Водный}

\subsection{Воздушный транспорт}
Проходимость летающих транспортных средств учитывается только при взлете и посадке. В эту категорию так же входит транспорт, способный находиться в открытом косомсе, но не предназначенный для межпланетных и межзвездных перелетов.
\paragraph{Буксировка воздушного транспорта} возможна только на земле.
\paragraph{Перегрузка воздушного транспорта.} Воздушный транспорт не терпит перегрузки. Все проверки Эксплуатации получают Осечку 9, а в случае провала ТС падает на землю.
\paragraph{Сильная перегрузка воздушного транспорта} не позволяет ему взлететь в принципе, хотя он может продолжать катиться по дорогам, если это позволяет Наземная Проходимость.
\genAndGet{transport}{transport}{Воздушный}

% \subsection{Космический транспорт}
% Космолеты имеет настолько большую скорость перемещения, что погони становятся бессмысленными, а перемещение между любыми двумя точками планеты они совершают меньше, чем за сутки, а скорость межпланетного и межзвездного перемещения очень сильно зависит от выбранного сеттинга, поэтому в таблице скорость космического транспорта не указана. Однако космический транспорт все еще имеет Проходимость, которая указвает на то, в насколько сложных условиях этот транспорт может совершать взлет и посадку.
% \newline космические корабли с проходимостью 0 не способны совершать посадку на поверхность планеты - вместо этого они используют средства орбитальной транспортировки, такие, как телепорты или орбитальные челноки.
% \newline космические корабли с проходимостью 1 способны совершать взлет и посадку с поверхности планеты только со специально подготовленных для них космодромов.
% \genAndGet{transport}{transport}{Космический}

\subsection{Исполинский транспорт}
Некоторые ТС достигают невероятных размеров. Обслуживать такой транспорт, а тем более владеть им, не по карману даже самому обеспеченному герою. Но герои могут взять этот транспорт в аренду или получить в пользование от организации - покровителя.
\newline Героям не нужно знать, сколько стоит, как тяжело обслуживается и насколько грузоподъемен транспорт, на котором они отправляются в путешествие, для приключения это - лишние детали. В описании Исполинского транспорта есть только его \textbf{Скорость} и \textbf{Проходимость} - этого достаточно, для того чтобы определить длительность и событийное наполнение пути. 
\newline Во время Остановок, Исполинский транспорт не участвует в Сценах целиком, а является элементом окружения героев. Сцена может развернуться и внутри транспорта.
\paragraph{Старинные Исполины}
\genAndGet{transport-gigantic}{transport-gigantic}{Старинный}
\paragraph{Современные Исполины}
\genAndGet{transport-gigantic}{transport-gigantic}{Современный}
% \paragraph{Фантастические Исполины}
% \genAndGet{transport-gigantic}{transport-gigantic}{Фантастический}

\chapter{Боевые столкновения}
\paragraph{}
Эта часть полностью посвящена сражениям и их последствиям.
Здесь рассказывается о действиях, доступным героям в бою
и о ранах, которые герои могут нанести и получить.
\section{Тип Повреждений}
в зависимости от типа, повреждения могут иметь разные эффекты КУ и могут быть увеличены или уменьшены в зависимости от того, какие есть сопротивления или уязвимости у цели. Атаки имеют один из следующих типов Повреждений:
\newline \textbf{(Д)}робящие, \textbf{(Е)}дкие, \textbf{(К)}олющие, \textbf{(Л)}едяные, \textbf{(О)}гненные, \textbf{(П)}роникающие, \textbf{(Р)}убящие, \textbf{(Э)}лектрические, \textbf{(Я)}довитые.
\newline
Хотя задача большинства атак - понижение ЕЗ цели, достигается она по-разному. Пытаться повредить рапирой каменную стену - не самая здравая идея, здесь лучше сработают молот или хотя бы дубина. Зато меткий укол в глаз способен не только ослепить противника, но уложить его на месте!
\paragraph{Уязвимость, Сопротивление, Иммунитет и Родная стихия:} Уязвимость к определенному типу Пв увеличивает успешно нанесенный урон в 2 раза. Сопротивление к определенному типу Пв уменьшает успешно нанесенный урон в 2 раза. Иммунитет к определенному типу Пв делает существо абсолютно невосприимчивым к этому типу Пв. Тип Пв, являющийся для цели Родной стихией, не отнимает ее ЕЗ, а восстанавливает их на то число, которое цель потеряла бы, не имея Родной стихии. ЕЗ цели все еще не могут превышать максимальной величины.
\paragraph{Атаки с несколькими типами Повреждений:} некоторые виды оружия могут наносить Пв разных типов - например, меч может наносить как Рубящие, так и Колющие Пв. В этом случае выберите тип Пв перед атакой.
\newline
Также существуют виды оружия, которые имеют несколько типов Повреждений одновременно - например, удар горящим факелом нанесет Огненные и Дробящие Пв. Возможна ситуация, в которой цель будет иметь Уязвимость к нескольким типам Пв атаки одновременно или даже Уязвимость к одному из них и Сопротивление к другому.
\newline
Если цель имеет Уязвимость к нескольким типам Пв атаки сразу, удвойте успешно нанесенные Пв за каждую Уязвимость. Например, если гигантский слизень имеет Уязвимость к Огненным Пв и Дробящим Пв, то удар горящим факелом нанесет в 4 раза больше Пв, чем обычно.
\newline
Уязвимость и Сопротивление компенсируют друг друга. Например, если цель имеет Сопротивление к Огненным повреждениям и уязвимость к Дробящим, то атака факелом будет наносить обычное количество повреждений по цели.
\newline
Иммунитет к любому типу Пв, которые включает атака, не позволяет ей наносить Повреждения цели.
\paragraph{Эффекты КУ} от разных типов повреждений:
\begin{itemize}
\item[--] \textbf{Дробящие:} получивший КУ Оглушен.
\item[--] \textbf{Едкиe:} получивший КУ начинает Растворяться. Состояние наступает даже в том случае, если цель не получила Пв при атаке.
\item[--] \textbf{Колющие:} получивший КУ страдает от Внутреннего кровотечения.
\item[--] \textbf{Ледяные:} получивший КУ становится Неподвижным до конца своей следующей Очереди.
\item[--] \textbf{Огненные:} получивший КУ Загорается.
\item[--] \textbf{Проникающие:} снаряд засел внутри тела! Получивший КУ страдает от Внутреннего кровотечения. После завершения боя он страдает от Агонии до тех пор, пока снаряд не будет извлечен (Врачевание против 15).
\newline
Если герой подвергается магическому лечению до того, как снаряд извлечен, он больше не страдает от Агонии и Внутреннего кровотечения, но боль дает о себе знать. Все активные проверки героя совершаются с Помехой до тех пор, пока снаряд не извлечен. Последующая сложность проверки Врачевания для извлечения снаряда возрастает до 20.
\item[--] \textbf{Рубящие:} получивший КУ страдает от Кровотечения.
\item[--] \textbf{Электрические:} получивший КУ сбит с ног.
\item[--] \textbf{Ядовитые:} на получившего КУ накладывается эффект яда.
\end{itemize}
\section{Состояния}
Физические  воздействия могут повлиять на состояние (и поведение) героя, как в бою, так и вне его. Эффекты двух
одинаковых состояний, происходящих из различных источников, не складываются, за исключением Внутреннего Кровотечения, Возгорания, Кровотечения, Отравления и Растворения. Эффекты различных состояний воздействуют на героя одновременно.
\paragraph{Агония:} герой охвачен мучительной болью. Герой может действовать только в случае успешной проверки Вл против 20. В боевых сценах совершайте проверку каждую Очередь героя, в остальных — один раз за сцену. При провале все, на что способен герой — лежать, вопить и поносить свою злую Судьбу. При успехе проверки герой держит себя в руках, но не может совершать Маневры и передвигается с половинной Ск. Состояние длится до следующего Антракта или пока герой не будет исцелен чарами (если Агония вызвана потерей ЕЗ), не получит квалифицированную врачебную помощь (если Агония вызвана Переломом) или сильное обезболивающее.
\paragraph{Внутреннее кровотечение:} жизненно важные органы героя повреждены. Каждую свою Очередь герой теряет |5 — МВн| ЕЗ (минимум 2 ЕЗ) до тех пор, пока не будет исцелен чарами или не получит квалифицированную врачебную помощь (Врачевание против 15).
\paragraph{Возгорание:} в начале своей Очереди горящее существо или предмет теряет 5 ЕЗ. Горящее существо может пропустить свою Очередь, чтобы потушиться. В этом случае до начала следующей Очереди существа атаки по нему совершаются по правилам Внезапного нападения. Состояние длится \textbf{|5 — МЛв цели|} Кругов (минимум 1 Круг).
\paragraph{Кровотечение:} герой истекает кровью. Каждую свою Очередь герой теряет число ЕЗ, равное БПв оружия, вызвавшего КУ(мин 1). Состояние длится до тех пор, пока герой не будет исцелен или перевязан (Врачевание против 10).
\paragraph{Неподвижность:} по каким-то причинам герой не может двигаться и защищать себя. Он может быть схвачен, заморожен, опутан сетью или просто спит. Герой теряет бонус защиты щита и бонус ловкости к защите. Все атаки по герою совершаются с Преимуществом.
\paragraph{Оглушение:} герой ненадолго теряет связь с реальностью (обычно из-за доброго удара по голове или под дых). Герой не может совершать Действие в свою следующую Очередь. В дальнейшем герой совершает все свои активные проверки с Помехой число Очередей, равное \textbf{|5 — МВн|} (минимум 1 Очередь).
\paragraph{Ослепление:} герой ничего не видит. Все атаки по нему совершаются по правилам Внезапного нападения. Герой атакует с 2 Помехами. Он не может атаковать цели, находящиеся от него на расстоянии (в метрах) большем, чем его Наблюдательность.
\paragraph{Отравление:} герой находится под действием яда. Смотрите описание яда для определения эффекта.
\paragraph{Ошеломление:} герой приходит в замешательство и не может совершать Действие в свою следующую Очередь. В дальнейшем герой совершает все свои активные проверки с Помехой число Очередей, равное \paragraph{|5 — МИн|} (минимум 1 Очередь). Также это состояние может быть причиной недосыпа, злоупотребления алкоголем или наркотическими зельями. Тогда избавиться от него поможет только полноценный отдых.
\paragraph{Ранен:} если ЕЗ героя достигают 2/3 от максимальных, то он Ранен. Его Ск ополовинивается. Обычно это состояние связано с потерей ЕЗ героем, но иногда оно возможно и в иных случаях — например, если герой страдает от вывиха или последствий болевого приема.
\paragraph{Растворение:} в начале своей Очереди существо или предмет получает 1 Пв вне зависимости от своей Прч. Если жертва была одета, она может остановить действие эффекта, пропустив 1 Очередь и сорвав с себя одежду, хотя снятие доспеха может занять куда больше времени! Если же одежды на жертве не было... Ей понадобится помощь квалифицированного медика и проверка Медицины против 15. Растворение вызывается множеством самых различных субстанций и веществ, поэтому способ нейтрализации, который сработал в прошлый раз, в следующий может сделать только хуже! Проверка Медицины необходима для каждого нового источника Растворения.
\newline
Если меры не были приняты, состояние заканчивается через \textbf{|10 + Пв, нанесенные вызвавшей состояние атакой|} Кругов. В конце сцены совершите проверку Неприятностей, чтобы узнать, пришло ли в негодность снаряжение жертвы (при желании можно совершить отдельную проверку для каждого предмета).
\trouble
{Отдушка}%no sweat name
{Никаких последствий, кроме специфического запаха, который исчезнет после хорошей стирки}%no sweat description
{Патина}%tough day name
{Незначительные повреждения. Герой может исправить их самостоятельно, совершив проверку соответствующего НавыкаЭ против 15 или заплатив ремесленнику 1/4 СП предмета (минимум 1 СП). До завершения ремонта предмет получает Осечку 6.}%tough day description
{Ржа}%we have trouble name
{Серьезные повреждения, все еще поддающиеся ремонту.Герой может исправить их самостоятельно, совершив проверку Ремонта против 20 или заплатив ремесленнику 1/2 СП предмета (минимум 1 СП). До завершения ремонта предмет получает Осечку 10.}%we have trouble description
{Труха}%fiasco name
{Снаряжение приведено в полнейшую негодность. Возможно, удастся всучить его подвыпившему старьевщику и выручить пару медяков.}%fiasco description
\paragraph{Серьезно ранен:} если ЕЗ героя достигают 1/3 от максимальных, то он Серьезно ранен. Все его активные проверки совершаются с Помехой. Обычно это состояние связано с потерей ЕЗ героем, но иногда оно возможно и в иных случаях — например, если герой страдает от вывиха или последствий болевого приема.
\paragraph{Сон:} герой спит. Все проверки Наблюдательности героя совершаются с Помехой. Пока герой спит, он Неподвижен. Если герой проснулся и вынужден сразу же действовать, он Ошеломлен. Героя автоматически разбудит достаточно громкий звук — выстрел из пистолета или звон медного колокольчика. Более тихие звуки потребуют проверки Наблюдательности (с Помехой)!
\paragraph{Удушье:} герой задыхается. Все его проверки совершаются с Помехой. Герой Потеряет сознание, если состояние продлится дольше, чем \textbf{|Вн × 10|} секунд, и умрет вне зависимости от величины его ЕЗ, если состояние продлится дольше, чем \textbf{|Вн × 20|} секунд.
\paragraph{Ужас:} герой дрожит от ужаса. Он совершает все активные проверки с Помехой, пока источник его ужаса находится поблизости (в зоне видимости или слышимости). В дальнейшем состояние длится число Очередей, равное \textbf{|10 — Вл героя|}.
\paragraph{Усталость:} все активные проверки совершаются героем с Помехой. Атаки по герою совершаются с Преимуществом.
\section{Подавляющее Превосходство и Внезапная Смерть}
Если разница в размере существ составляет 3 категории или больше, у большего существа есть Подавляющее Превосходство над меньшим. Любая успешная атака существа с Подавляющим Превосходством грозит Внезапной Смертью меньшему существу, одновременно с этим любая атака меньшего существа настолько незначительна, что не может нанести повреждений большему даже при Критическом Ударе. Так же меньшее существо не может наложить эффектов КУ на большее существо с Подавляющем превосходством.
\paragraph{Класс Защиты и Повреждений.} Некоторые виды вооружений и защиты созданы специально для того, чтобы бороться с крупными противниками и защищаться от них. Существо и его атаки можгут считаться большего или меньшего размера, для определения условий Подавляющего превосходства, если снаряжение, которое оно использует имеет Класс Защиты или Класс Повреждений.

\paragraph{Внезапная Смерть.} Иногда урон, который предстоит получить герою или статисту настолько велик, что никакая защита его не спасет, только чудо. Статисты умирают сразу(если судьба героев не вступится за них), герои и персоны совершают проверку Неприятностей.
\trouble
{Ни царапины}%no sweat name
{Герой чудом пережил прямое попадание. Хотя постойте - она мимо прошла!}%no sweat description
{Задело}%tough day name
{Героя зацепило осколками и ударной волной. Он теряет половину текущих ЕЗ и должен совершить проверку Опасной раны, даже если потерял меньше 1/4 максимальных ЕЗ.}%tough day description
{Потрепало}%we have trouble name
{Героя приложило крупным осколком. Его ЕЗ становятся равными 0 и он теряет сознание.}%we have trouble description
{Уничтожило}%fiasco name
{От героя осталось только воспоминание. Даже чудотворец не сможет вернуть его к жизни.}%fiasco description
\section{Раны и их последствия}
\paragraph{}
Единицы Характеристик (ЕХ) и их потеря: некоторые атаки - как правило, яды, отнимают не ЕЗ жертвы, а понижают ее Характеристики. Например, ядовитая паучья слюна понижает Ловкость жертвы на 2. Под потерей ЕХ понимаются все подобные случаи.
\newline
Если герой пережил Повреждения, Единицы Здоровья и Единицы Характеристик могут быть восстановлены до максимума посредством отдыха или медицинской помощи. О восстановлении ЕЗ и ЕХ подробнее читайте в разделе "Отдых".
\paragraph{Опасная рана:} если герой одномоментно получает Пв, превышающие \textbf{|1/5 от максимальных ЕЗ|}, он должен совершить проверку Вн против 15. Если герой проходит проверку, то он продолжает сражаться. В противном случае он Теряет сознание до конца Сцены.
\paragraph{Болевой шок:} если в течение Круга герой одномоментно теряет число ЕЗ, превышающее его \textbf{|Вл|}, то следующая активная проверка героя совершается с Помехой. Разумеется, это относится к живым существам, которые в принципе способны испытывать боль.
\paragraph{Потеря сознания:} герой, потерявший сознание, находится в этом состоянии до конца \textit{следующей} Сцены, после чего приходит в себя. Если не замерзнет, не истечет кровью, не будет добит победителями или съеден дикими тварями.
\newline
Герой немедленно приходит в сознание, если он восстанавливает хотя бы 1 ЕЗ.
\paragraph{Смерть:} как только герой теряет последнюю ЕЗ, он должен совершить проверку Вн против 15. Если герой проходит проверку, то Теряет сознание и остается жив (и обзаводится парой впечатляющих шрамов). В противном случае герой умирает. Он приходит в сознание в следующей Сцене в 0 ЕЗ, страдает от ран и находится в Агонии до следующей Интерлюдии!
\paragraph{Смерть и статисты:} для ускорения боевых сцен мастер может считать проверки Выносливости статистов при Потере конечностей, Опасных ранах и Смерти автоматически проваленными и совершать их только в тех сценах, которые он и игроки считают важными. Также мастер может делать проверки, если игроки желают захватить статиста живым.
\subsection{Тяжелые травмы}
\paragraph{Переломы:} попадание по руке или ноге может вызвать Перелом. Когда конечность одномоментно получает Пв, превышающие \textbf{|1/5 от максимальных ЕЗ|}, она считается сломанной. Переломы считаются Опасной раной.
\paragraph{Потеря конечностей:} если конечность одномоментно получает Пв, превышающие \textbf{|2/5 от максимальных ЕЗ|}, то она отрублена, размолота в кашу или приведена в полнейшую негодность каким-то иным образом. Повреждения, превышающие требуемое для уничтожения конечности число, теряются. Герой, потерявший конечность, находится в состоянии Кровотечения. Само собой, Потеря конечности считается Опасной раной!
\paragraph{Однорукие герои:} однорукие герои не могут использовать Двуручное оружие. Если рука обрублена ниже локтя, герой может закрепить на ней щит или одноручное оружие.
\paragraph{Одноногие герои:} одноногий герой с костылем или протезом считается перемещающимся по Трудному ландшафту. Да, это значит, что одноногий герой с костылем или протезом и Чувством равновесия не испытывает никаких неудобств при движении. Костыль может использоваться для атаки! Без костыля или протеза одноногий герой перемещается на 1/3 своей Ск. Одноногие герои не могут использовать Громоздкое оружие.
\paragraph{Безногие герои:} безногие герои понижают свой размер на 1 категорию. Ск безногих героев составляет 1. Если у безногого героя есть какое-то средство перемещения - например, тачка с колесиками - при помощи Перемещения он может двигаться на число метров, равное своему МЛв (минимум на 2 метра). Безногий герой атакует в ближнем бою с Помехой. В ближнем бою враги атакуют безногого героя с Преимуществом. Безногие герои не могут использовать Громоздкое и Длинное оружие.
\paragraph{Потеря глаз:} герой, лишившийся двух глаз, перманентно находится в состоянии Ослепления. Одноглазый герой совершает проверки Меткости с Помехой.
\section{Боевые сцены}
Боевая сцена начинается, если один или несколько героев подверглись атаке, либо напали на кого-то сами. В начале боевой сцены:
\begin{enumerate}
\item Определите, подвергся ли кто-то из участников Внезапному нападению. Обычно это сопряжено с проваленными проверками Наблюдательности против Скрытности противника. Если никто из участников сражения не пытался передвигаться скрытно и ожидал нападения, фактор внезапности вряд ли стоит учитывать. С другой стороны, Стремительный герой, герой с трюком <<Гопля!>> или герой с высокой Реакцией вполне может застать противников врасплох, молниеносно выхватив оружие и атаковав!
\item Определите позиции участников сцены относительно друг друга и окружающих предметов. На обсуждение этого стоит потратить несколько минут (или даже больше), чтобы игроки четко представляли возможности героев и их противников.
\item Начало Круга. Все участники сцены действуют в порядке, определяемом их Реакцией. Возможно, для завершения битвы понадобится больше одного Круга.
\end{enumerate}
\paragraph{Круг:} в бою герои и статисты действуют по Очереди. Один полный \textbf{Круг} (то есть Очереди всех героев и статистов, принимающих участие в сцене) занимает \textbf{5 секунд}. Подразумевается, что герои и статисты действуют в бою одновременно, однако для удобства игры круг разделен на Очереди. Статисты под управлением мастера действуют в одну общую Очередь, однако мастер может разделить их Очереди в соответствии с параметрами Реакции. Общая Очередь ориентируется на статиста с наименьшей Реакцией.
\paragraph{Очередность в бою:} первым действует герой или статист с наибольшей Реакцией. Если у двух героев или статистов одинаковая Реакция, первым действует тот, у кого больше Ловкость. Если и они равны, первым действует выигравший Состязание в Реакции.
\paragraph{Очередь:} Очередь героя состоит из Перемещения, Действия и Быстрого действия. В течение своей Очереди герой может:
\begin{itemize}
\item[--] \textbf{Совершить Перемещение.} Для Перемещения используется значение Ск — столько метров/клеток может преодолеть герой. Герой может отказаться от Перемещения, чтобы получить дополнительное Быстрое действие.
\item[--] \textbf{Совершить Действие.} Во время Действия герой использует предметы, атакует, творит чары, убеждает окружающих прекратить кровопролитие или делает что-то еще, требующее концентрации внимания.
\item[--] \textbf{Совершить Быстрое действие.} Быстрое действие включает односложные фразы, быстрые жесты, шаги в пределах 1 метра и броски предметов без намерения кому-то повредить или поразить конкретную цель. Хрупкие предметы могут сломаться или разбиться! У героя есть лишь одно Быстрое действие за круг, но мастер может отступать от правила, если считает ситуацию располагающей к этому. Иногда выполнение Быстрого действия возможно и в чужую Очередь. Такие случаи указаны в описании соответствующих Трюков или Атрибутов.
\end{itemize}
\begin{tcolorbox}
Герой может отказатьсяот Действия, чтобы совершить дополнительное Перемещение или дополнительное Быстрое действие.
\newline
Так же герой может отказаться от Перемещения, чтобы совершить дополнительное Быстрое действие.
\end{tcolorbox}
Герой может отказаться от Перемещения или Действия, чтобы подняться с земли, или взять предмет, или аккуратно положить предмет, или привести оружие в боевую готовность, или убрать оружие в ножны, или перезарядить оружие со свойством <<Перезарядка>>, или вскочить в седло.
\newline
Перемещение, Действие и Быстрое действие используются в любых комбинациях. Например, совершая Быструю атаку, герой со Скоростью 6 может за одну Очередь пройти на 1, атаковать, потом пройти на 2, атаковать еще раз, затем пройти на 3 и уронить предмет.
\subsection{Детали боевой сцены}
\paragraph{Внезапное нападение.} Если герой не заметил врага или заметил в последний момент (то есть провалил проверку Наблюдательности против Скрытности врага), он захвачен врасплох. Герой не может Перемещаться и Действовать независимо от своей Реакции, а также вычитает МЛв и БЩЗщ из своей Зщ. Если герой дожил до своей следующей Очереди, он сражается по обычным правилам.
\paragraph{Все на одного:} сражение с несколькими противниками требует от воина высочайшего мастерства. Если герой сражается с несколькими противниками, в начале Очереди он должен выбрать, кому из них он уделяет больше внимания. Выберите число противников, равное \textbf{|1 + ММд героя|} (минимум 1). Они атакуют героя, как обычно. Все противники сверх этого числа атакуют героя с Преимуществом.
\paragraph{Ложись!:} герой может упасть, использовав Быстрое действие. Дистанционные атаки по лежащему герою совершаются с Помехой, атаки в ближнем бою по нему совершаются с Преимуществом. Лежащий герой может ползти с 1/2 Ск и атаковать в ближнем бою с Помехой. В остальном герой действует по обычным правилам.
\paragraph{Невидимки:} если герой атакует невидимого (из-за чар, укрытия или по иным причинам) противника, бросок совершается с Помехой. Если невидимого противника нет в атакуемой области, герой автоматически промахивается. Невидимые существа атакуют по правилам Внезапного нападения. Если существо невидимо благодаря Скрытности, а не волшебству, то, атаковав, оно обнаружит себя вне зависимости от успеха атаки.
\paragraph{Перемещение через занятые области:} если герой желает пройти через область, занятую враждебным существом, он должен пройти проверку Атлетики против \textbf{|БАЗщ + Дб противника|}. В случае успеха герой передвигается через занятую область, в случае провала падает рядом с противником. Перемещение героя на этом заканчивается.
\paragraph{Трудный ландшафт.} Иногда перемещение героя затруднено густым подлеском, глубоким снегом, скользким льдом, качающейся палубой. В такой ситуации Ск героя ополовинивается.
\subsection{Зоны поражения}
Герой может выбрать для атаки любую из перечисленных зон. Если при атаке не заявлена специфическая Зона поражения, удар наносится в торс.
\newline Сложность маневра Возрастает на указанную в таблице Зон поражения, но при успехе маневра возможен дополнительный эффект.
\begin{center}
\begin{tabular}{|c|c|}
\hline
Зона поражения & Сложность \\ \hline
Торс & 0 \\ \hline
Конечность & 2 \\ \hline
Пах & 3 \\ \hline
Шея & 5 \\ \hline
Голова & 5 \\ \hline
Глаз & 7 \\ \hline
\end{tabular}
\end{center}
\paragraph{Торс:} попадание по торсу не вызывает никаких эффектов, кроме синяков, шрамов, шишек и потери Единиц Здоровья.
\paragraph{Конечность:} попадание по руке или ноге может вызвать Перелом или Потерю конечности.
\paragraph{— Переломы:} попадание по руке или ноге может вызвать Перелом. Когда конечность одномоментно получает Пв, превышающие \textbf{|1/5 от максимальных ЕЗ|}, она считается сломанной. Переломы считаются Опасной раной.
\paragraph{— Потеря конечностей:} если конечность одномоментно получает Пв, превышающие \textbf{|2/5 от максимальных ЕЗ|}, то она отрублена, размолота в кашу или приведена в полнейшую негодность каким-то иным образом. Повреждения, превышающие требуемое для уничтожения конечности число, теряются. Герой, потерявший конечность, находится в состоянии Кровотечения. Само собой, Потеря конечности считается Опасной раной!
\paragraph{Пах:} мужчина, получивший удар в пах, должен пройти проверку Вн против \textbf{|10 + полученные Пв|}. При провале он не может действовать и перемещаться число Очередей, равное числу, на которое провалил проверку (хотя может орать, сквернословить и совершать Быстрые действия). Все атаки по нему совершаются с Преимуществом.
\paragraph{Шея:} если шея одномоментно получает Пв, превышающие \textbf{|1/4 от максимальных ЕЗ + МВн|}, то жертва умирает. Без проверок. В случае Колющего или Проникающего удара смерть наступает от обильного кровотечения, Дробящие атаки ломают позвоночник, Рубящие удары обезглавливают.
\newline
Разумеется, это относится к живым гуманоидным существам, у которых мозг находится в голове. Умертвия, разумные грибы, трехголовые драконы и тому подобные создания вполне могут существовать и без головы (или одной из них).
\paragraph{Голова:} получивший удар в голову должен пройти проверку Вн против \textbf{|10 + полученные Пв|}. При провале жертва Теряет сознание. Если в голову нанесена Опасная рана, жертва должна преуспеть в проверке Вн против 15 или умереть.
\paragraph{Глаз} не может быть выбран для поражения Громоздким оружием. 2 и более Пв приводят к потере глаза. Существо, потерявшее все свои глаза, Ослеплено. Крупных существ ослепить сложнее — Большое существо лишается глаза при получении глазом 3 Пв, огромное — 4 Пв, гигантское — 5 Пв. Маленькие и крошечные существа лишаются глаза при получении в глаз 1 Пв. Лишние Пв наносятся в голову.
\newline
Если атака имеет Колющие или Проникающие Пв, удвойте успешно нанесенные Пв — атака поражает не только глаз, но и мозг! В этом случае получивший удар должен пройти проверку Вн против \textbf{|10 + полученные Пв|} или Потерять сознание. Проверка совершается с Помехой.
\newline
Если Колющая или Проникающая атака наносит Опасную рану в глаз, жертва должна преуспеть в проверке Вн против 15 или умереть. Проверка совершается с Помехой.
\newline
Зачастую глаза — самое уязвимое место на теле огромных монстров!

\section{Маневры}
Во время Действия герой может совершить Маневр. Перечисленные ниже Маневры может выполнить любой герой, если он снаряжен соответствующим образом. Некоторые Маневры могут совмещаться друг с другом или дополнительно требовать траты Перемещения или Быстрого действия.
\paragraph{Совершение Маневра:}
\begin{enumerate}
\item Выберите цель. Герой не может поразить цель за пределами досягаемости оружия или заклинания.
\item Определите штрафы и бонусы к проверке. Как правило, это штраф зоны поражения или штрафы за размер предмета при Разоружении или Поломке оружия. Также перед совершением проверки определите, есть ли у героя Преимущество или Помеха на нее.
\item Совершите необходимую проверку и определите последствия. Получает цель Маневра Повреждения, или герой промахивается, нанесен ли Критический Удар и т. д. Например, успех при Разоружении или Поломке оружия лишит противника оружия или уничтожит (или повредит) ценный предмет на его теле, успех при Захвате даст герою возможность справиться с противником, не нанося Повреждений, или просто сломать ему хребет.
\end{enumerate}
\paragraph{Зоны Поражения}
\subsection{Атака}
Герой атакует в ближнем бою. Герой может атаковать противника в 1 метре от себя (или в 2 метрах, если его оружие Длинное) и должен совершить проверку Дб против \textbf{|Зщ цели|}. Количество нанесенных Пв равно величине успеха проверки. Неподвижные цели герой атакует с Преимуществом. Если в результате подсчета получается 0 или меньше, то удар соскользнул с доспеха или просто не достиг цели. Герой может понизить свою Дб на любое число (минимум до 0), если желает нанести удар не в полную силу.
\subsection{Атака с разбега}
Герой преодолевает по прямой расстояние в интервале от своей \textbf{|Ск +1|} до своей \textbf{|Ск × 2|} и атакует с Преимуществом. До начала его следующей Очереди все атаки по нему совершаются с Преимуществом. Атака с разбега не может совмещаться с Быстрой атакой, Выжиданием и Плетением чар, но может совмещаться с другими маневрами. Если герой совершает Сокрушительную атаку с разбега, он получает
2 Преимущества. Но так же и враги при атаках по нему до начала его следующей Очереди!
\newline
Перемещение героя уже учтено в этом Маневре, то есть при его использовании герой может преодолеть расстояние, не превышающее его \textbf{|Ск × 2|}. Герой не может совершить Перемещение и затем использовать Атаку с разбега!
\subsection{Быстрая атака}
Герой совершает 2 атаки с Помехой. Если герой имеет оружие в каждой руке и оба оружия Легкие, только одна атака совершается с Помехой. Каждой из этих атак герой может совершать Дистанционную атаку, Захватывать, Ломать снаряжение, Разоружать, Сбивать с ног, Толкать или выполнять Финт.
\subsection{Выжидание}
Герой ожидает совершения некоего действия кем-то из окружающих. Любой иной маневр может быть выполнен как Выжидающий.
\newline
Очередность событий при этом определяется Реакцией. Чтобы
опередить противников, герой должен преуспеть в проверке
Реакции против \textbf{|10 + Рц статиста|}.
\subsection{Захват}
Для маневра нужна как минимум 1 свободная рука. Герой проходит проверку Дб (Рукопашный бой) против \textbf{|БАЗщ + Дб(Рукопашный бой) противника|}. При успехе противник не получает Повреждений, но становится Захваченным. Маневр может сочетаться с Атакой с разбега, Быстрой атакой (это имеет смысл, если герой пытается захватить сразу двух
противников) и Сокрушительной атакой. Если Захват успешен, все атаки в ближнем бою против схваченного получают Преимущество, пока он не освободится. Некоторые Трюки и виды оружия позволяют проводить Захват при помощи Владения оружием. В этом случае замените во всех формулах Рукопашный бой на Владение оружием для инициатора Захвата. Если у героя есть Трюк «Знаток оружия», то он может использовать во всех формулах Владение оружием, даже являясь Захваченным!
\begin{tcolorbox}
Если герой использует для хватания или удерживания две руки, он получает +2 ко всем проверкам захвата. Если у героя больше двух рук, то он получает еще +2 к проверкам за каждую пару рук.
\end{tcolorbox}
\paragraph{Схвативший может} (в том числе в ту же Очередь, когда совершен Захват) cовершать любые действия со следующими дополнениями и исключениями:
\begin{itemize}
\item[--] Атаковать схваченного любым одноручным оружием (даже дистанционным). Схваченный не добавляет МЛв и Щит к своей Зщ.
\item[--] Атаковать другие цели по обычным правилам, если у него есть свободная рука с одноручным оружием ближнего боя или одноручным дистанционным оружием.
\item[--] Перемещаться на половину своей Ск. Схвативший передвигается без ограничений, если на 2 категории и более превышает Размером схваченного.
\item[--] Сменить зону, за которую удерживает противника. Герой должен пройти проверку Захвата по новой выбранной Зоне. В случае провала, он продолжает держать цель в Захвате, но зона, за которую он держит цель не меняется.
\item[--] Блокировать противника. До начала своей следующей Очереди схвативший не дает схваченному выполнять любые действия (даже говорить). Все что может делать схваченный — это пытаться вырваться. Схвативший не может перемещаться и совершать никаких действий, кроме Быстрых.
\item[--] Душить схваченного. Для этого противник должен быть схвачен за шею. Схвативший совершает проверку Дб(Рукопашный бой) против \textbf{|Дб(Рукопашный бой) + МВн схваченного|}. Величина успеха равна нанесенным Пв. Если в ходе удушения ЕЗ схваченного достигает 0, он Теряет сознание. Обратите внимание, что отрицательный МВн прибавляется к Пв от удушения.
\item[--] Бросить схваченного. Герой не может бросать существ и предметы, вес которых превышают его \textbf{|комфортную нагрузку × 2|}. Дальность броска не может быть дальше, чем \textbf{|МСл+МЛв|} бросающего. Увеличьте максимальное расстояние броска в 2 раза за каждую категорию размера, на которую бросающий больше бросаемого, и уменьшите в 2 раза за каждую категорию, на которую бросающий меньше бросаемого.
\newline Для того, чтобы бросить схваченного в конкретную точку, бросающей должен совершить проверку Меткости для Метательного оружия с БПв 0. При падении бросаемый получает столько Пв, сколько метров пролетел. Если точке, куда совершен бросок, находится другое существо, то бросаемый падает ему под ноги.
\newline Для того, чтобы попасть по другому существу, нужно совершить проверку Меткости с Помехой для метательного оружия с БПв равным \textbf{|МРз + Броня бросаемого|}.
\item[--] Отпустить схваченного.
\end{itemize}
\paragraph{Схваченный может} совершать перечисленные действия:
\begin{itemize}
\item[--] Проводить атакующие Маневры по схватившему с Помехой. Для атаки может использоваться только Легкое оружие, шипы на доспехе, а также кулаки (зубы, когти, клювы, щупальца и т. п.). В остальном атаки проводятся по обычным правилам.
\item[--] Творить Феномены, если может соблюсти необходимые для этого условия (что бывает затруднительно, если героя держат). Все проверки, необходимые для успеха Феномена, совершаются с Помехой.
\item[--] Пытаться вырваться. Чтобы вырваться, схваченный должен пройти проверку Дб (Рукопашный бой), Атлетики (Сл, Лв), Сл или Лв против \textbf{|БАЗщ + Дб (Рукопашный бой) схватившего|}. Попытка вырваться считается Действием.
\item[--] Достать Легкое оружие. Схваченный может отказаться от Перемещения, чтобы достать оружие, несмотря на то, что фактически обездвижен.
\item[--] Совершать Быстрые действия.
\end{itemize}
\paragraph{Схваченный не может} совершать следующие действия:
\begin{itemize}
\item[--] Передвигаться, если только не превышает схватившего размером на 2 или больше. Также существа, размером превышающие схватившего, могут атаковать и другие цели (без Помехи), кроме схватившего. 
\end{itemize}
Схвативший/схваченный автоматически получают Пв, если на доспехе схватившего/схваченного есть шипы. Пв равны разнице бонуса доспехов схватившего и схваченного. Например, если герой в кольчужной рубахе (Броня +3) схватил противника в шипованном кольчато-пластинчатом доспехе (Броня +6), то герой получит 3 Пв (противник, само собой, не получает Пв). Шипы наносят Пв в начале каждой Очереди схватившего/схваченного, пока схвативший не отпустит противника. Для того чтобы удерживать противника, покрытого шипами, схвативший должен пройти проверку Вл против \textbf{|10 + полученные Пв|}. При провале он отпускает его в начале своей следующей Очереди!
\subsection{Защитная стойка}
Герой сосредочен на обороне. Все атаки по нему до начала его следующей Очереди совершаются с Помехой. Лежащий на земле герой также может выбирать этот маневр!
\subsection{Провокация}
Этот Маневр может сочетаться с любым другим и требует Быстрого действия. При помощи оскорбительных фразочек и еще более оскорбительных жестов герой привлекает к себе внимание врага. Совершив проверку Общения (Об) против \textbf{|10 + Вл цели|}, герой провоцирует противника и становится целью его следующей атаки. В некоторых ситуациях герою точно не обойтись без Луженой глотки!
Если жертвы не слышат героя или не видят его, или не понимают язык, на котором он говорит, проверка совершается с Помехой. Если героя и не видят, и не слышат, Провокация не сработает! Маневр действует только на одну цель одновременно, но если у героя больше одного Быстрого действия, он может Провоцировать несколько раз за Очередь, привлекая внимание разных противников!
\begin{tcolorbox}
Провокация действует только в боевых сценах (то есть, когда бой уже начался) и имеет смысл в тех случаях, когда враги могут атаковать оскорбившего их героя. В противном случае они выберут другую цель. 
\end{tcolorbox}
\subsection{Разоружение}
Чтобы выбить предмет из рук противника, герой должен пройти проверку Дб против \textbf{|10 + Щит + Дб противника|}. Герой получает штраф к Дб за размер предмета-цели (сверьтесь с таблицей ниже). Если противник держит предмет в 2 руках, герой получает дополнительные -2 к проверке. Маневр не может повредить предмету-цели, однако хрупкие предметы могут разбиться, упав на землю. В случае успеха маневра предмет падает на землю на расстоянии от разоруженного, не превышающем МСл или МЛв разоружающего. Разоружающий выбирает точку, в которую упадет предмет. Уменьшите расстояние в 2 раза, если выбитый из рук предмет Громоздкий или Длинный. Если в результате получается 0, предмет падает под ноги разоруженного.
\newline
Если хотя бы одна рука героя свободна, он может выхватить предметы и оружие из рук противника, используя при Разоружении Рукопашный бой! Если герой использует обе руки, он получает +2 к проверке. В остальном он действует по обычным правилам Разоружения. При успехе Маневра оружие или предмет оказывается в его руках. Если герой выбирает целью Маневра щит, то Щит не учитывается при проверке. Однако зачастую щиты основательно закреплены на руке, и, чтобы сорвать с нее щит, понадобится дополнительная проверка Сл, Лв или Атлетики (Сл, Лв) против |10 + Щит|.
\subsection{Сбить с ног}
Чтобы сбить противника с ног, герой должен пройти проверку Дб против \textbf{|БАЗщ + Щит + Дб противника|}.
\newline
Герой получает штраф -2 за область поражения (ноги). Если существо стоит на 4 ногах, герой получает дополнительные -2. Если существо передвигается на брюхе, как люди-змеи или гигантские слизни, герой получает дополнительные -4 к проверке! Маневр может выполняться лишь Длинным оружием или при помощи Навыка «Рукопашный бой».
\subsection{Сломать снаряжение}
Чтобы нанести Повреждения предмету в руках или на теле противника, герой должен пройти проверку Дб или Мт против \textbf{|10 + Щит + Дб противника|}.
\newline
Герой получает штраф к Дб или Мт за размер предмета-цели (сверьтесь с таблицей ниже). За каждую 1, на которую герой прошел проверку, он наносит 1 Пв предмету. Если герой пытается разбить щит, Щит не учитывается в сложности проверки. 
\begin{center}
\begin{tabular}{|p{10cm}|p{4cm}|}
\hline
Оружие/предмет & Штраф к Доблести атакующего при разоружении/поломке \\ \hline
Доспехи, закрывающие большую часть тела (Броня 7+), башенный щит & 0 \\ \hline
Громоздкое и Длинное двуручное оружие, доспехи, закрывающие значительную часть тела (Броня 4+6) & -1 \\ \hline
Громоздкое или Длинное двуручное оружие, большой щит, легкие доспехи (Броня 1-3) & -2 \\ \hline
Двуручное оружие, Громоздкое или Длинное Универсальное оружие, щит, большой рюкзак & -3 \\ \hline
Универсальное оружие, Громоздкое или Длинное одноручное оружие, рюкзак & -4 \\ \hline
Одноручное оружие, баклер, широкий ремень & -5 \\ \hline
Легкое оружие, скрученный свиток, шапка, кошель на поясе & -6 \\ \hline
Кастет, перчатка, наруч, фиал с зельем, узкий ремень & -7 \\ \hline
Перстень, браслет, ключ & -8 \\ \hline
\end{tabular}
\end{center}

\subsection{Сокрушительная атака}
Герой атакует с Преимуществом. До начала его следующей Очереди все атаки по нему совершаются с Преимуществом. Сокрушительная атака не может совмещаться с Быстрой атакой, но может совмещаться с Атакой с разбега, Захватом, Поломкой оружия, Разоружением, Сбиванием с ног, Толчком или Финтом.
\subsection{Толчок}
Герой отталкивает противника на 1 метр прямо от себя. Чтобы сделать это, герой должен пройти проверку Дб против \textbf{|БАЗщ + Дб противника|}.
\newline
Если Маневр применяется одновременно с Атакой с разбега, герой отталкивает противника на 1 метр за каждую единицу, на которую прошел проверку. Герой не может оттолкнуть противника на большее число метров, чем преодолел сам в эту Очередь, но всегда толкает минимум на 1 метр в случае успеха проверки.
Увеличьте максимальное расстояние толчка в 2 раза за каждую категорию размера, на которую толкающий больше толкаемого, и уменьшите в 2 раза за каждую категорию, на которую толкающий меньше толкаемого.
\newline
Герой не может толкать существ и предметы, вес которых превышают его максимальную нагрузку.
\subsection{Финт}
Герой отвлекает противника. Герой должен пройти проверку Общения (Мд, Об) или Владения оружием/Рукопашного боя против \textbf{|10 + Владение оружием/Рукопашный бой противника (Ин, Мд)|}. Если проверка успешна, следующая атака (героя и любого его союзника) по этому противнику совершается с Преимуществом. Эффект длится до конца следующей Очереди героя, применившего Финт.
\subsection{Дистанционная атака}
Герой стреляет или бросает предмет в цель. Если цель находится на Дальней дистанции, атака совершается с Помехой. В Неподвижные цели герой стреляет с Преимуществом.
\newline
Оружие, не предназначенное для метания, может быть брошено с Помехой. Максимальная дистанция для броска такого оружия — 5. Подразумевается бросок, наносящий цели Повреждения, а не его максимальная дальность в принципе.
\newline
Если для вашей истории важно, на какую предельную дистанцию герой может метнуть предмет (например, при метании гранаты), то она равна \newline{|Сл героя × 3 — вес предмета в килограммах|} метров. Разделите результат на вес предмета в килограммах (минимум 1), если он не предназначен для метания. Например, герой с 10 Силой сможет метнуть двуручный топор весом в 5 кг на (10 × 3 — 5) ÷ 5 = 5 метров. Используйте Мт для попадания в некую конкретную область. В подобных случаях, пятачок земли 1 × 1 метр имеет БАЗщ 10. Разумеется, при броске на предельную дистанцию атака совершается с Помехой.
\paragraph{Дистанционные атаки имеют 2 типа дистанций: Ближняя и Дальняя.} Эти дистанции указаны в статистиках дальнобойного оружия и описании заклинаний. Если цель находится за пределами Дальней дистанции, герой не может поразить ее. Атаки на Дальней дистанции совершаются с Помехой. Совершая Дистанционную атаку, герой должен сделать проверку Меткости против \textbf{|Зщ цели|}. Герой может понизить свою Мт на любое число (вплоть до 0), если желает поразить цель вскользь.
\paragraph{Дистанционные атаки в ближнем бою:} герой совершает Дистанционные атаки с Помехой, если в 1 метре от него находится противник. Герой совершает Дистанционные атаки с Помехой, если его цель находится в ближнем бою, в котором он не принимает участия.
\paragraph{Перемещение и Дистанционные атаки:} если в свою Очередь герой перемещается на расстояние, превышающее 1 метр, и стреляет, проверка Меткости совершается с Помехой. Обратите внимание, что герой может выстрелить без Помехи до Перемещения.
\subsection{Прицеливание}
Выбрав Маневр Дистанционной атаки, герой может объявить, что он Прицеливается. В боевой сцене герой может сместить свою Очередь на 1—3 Очереди вниз, то есть действовать после менее быстрого героя или статиста. Если в текущем Круге герой действовал последним, то следующий прокуск Очереди перемещает его ход на следующий Круг в Боевой Сцене. Герой, сместивший свою Очередь на 1 вниз, получает Преимущество при проверках Меткости, а враги получают Преимущество, атакуя его. Герой, сместивший свою Очередь на 2 вниз, получает 2 Преимущества, а враги получают 2 Преимущества, атакуя его. Если герой сместил свою Очередь на 3 вниз, он получает 2 Преимущества и игнорирует Помеху при стрельбе на Дальнюю дистанцию. Враги получают 2 Преимущества, атакуя его.
\newline
Если во время Прицеливания герой одномоментно получает Пв, равные или превышающие его Вл, он теряет все бонусы Прицеливания и должен начинать заново. Прицеливание не может сочетаться с Быстрой атакой, Беглым Огнем и Огнем на подавление. Герой может прицеливаться только в цель, которую видит.
\paragraph{Укрытия делятся на 2 типа:} мягкое (кусты, высокая трава, куча хвороста) и твердое (камни, стены, башенные щиты). При стрельбе по противнику в мягком укрытии герой получает 1 Помеху, при стрельбе по противнику в твердом укрытии — 2 Помехи. В случае промаха герой поражает укрытие. Укрытия могут быть Повреждены и уничтожены.
\subsection{Беглый огонь}
Герой может выбрать несколько целей для стрельбы из оружия со Скорострельностю выше 1. Количество целей не должно превышать \textbf{|ММд+1|(минимум 2)}. Герой выбирает точное количество выстрелов, произведенных оружием в каждую цель. Это число не должно превышать Ск оружия, но может быть меньше. Например, если герой с ММд 2 стреляет из оружия с Ск3, он может выбрать 4 цели, но Ск оружия не позволяет распределить пули по всем целям, поэтому он может поразить только троих.
\newline
Совершите одну проверку Мт для всех выстрелов (отдельные проверки для каждого выстрела возможны по предварительной договоренности игроков и мастера) с количеством помех равным \textbf{|количеству целей-1(минимум 0)|}. Когда герой атакует несколько целей и выбирает различные Зоны поражения, для всех бросков используется наибольший штраф, если только отдельные проверки Мт по каждой из целей не оговорены заранее.
\newline
Эффекты КУ применяются только к одному из противников по выбору стрелка.
\subsection{Огонь на подавление}
Этот маневр может использовать герой с любым оружием со Скорострельностью 5 и выше. Герой поливает свинцом небольшой пятачок земли, не заботясь о точности. Герой может выбрать число смежных областей площадью 1х1 метр, не превышающее Скорострельность оружия, и совершить атаку с БПв оружия, модифицированным за дальность в случае необходимости. Атака совершается с 2 Помехами и Осечкой 5. Если оружие уже имеет параметр Осечки, используйте наибольший. Все существа, находящиеся в выбранных областях, получают Повреждения, если герой поразил их Защиту. В этом режиме оружие расходует 10 зарядов (даже если его фактическая Скорострельность ниже). Ведущий Огонь на подавление герой не может выбирать Зоны поражения.
\newline
Эффекты КУ не применяются при использовании этого маневра.
\section{Наездники}
\paragraph{Атаки наездника:} герой получает Преимущество в ближнем бою, если атакует существ, меньших по размеру, чем его скакун.
\paragraph{Атаки по наезднику:} в ближнем бою при атаке по наезднику на скакуне, превышающем размер атакующего, атакующий получает Помеху, если не атакует Длинным оружием.
\paragraph{Атаки по скакуну:} совершаются по обычным правилам.
\paragraph{Атаки скакуна:} во время своего Действия скакун может совершать любые Маневры, доступные в соответствии его Навыкам и снаряжению. Скакун не обязан заявлять такой же Маневр, как и наездник. Исключение составляет Атака с разбега.
\paragraph{Действие и Перемещение скакуна:} скакун Действует и Перемещается в ту же Очередь, что и его наездник.
\paragraph{Проверки скакуна:} Проверки Дб, Мт и Вл скакуна совершаются с использованием Навыка <<Обращение с животными>> наездника.
\newline
Например, если скакун атакует, его Рукопашный бой заменяется Обращением с животными наездника. Скакун может использовать собственный Рукопашный бой, если он больше Обращения с животными наездника.
\paragraph{Реакция наездника} равна \textbf{|(Рц скакуна + Рц героя) ÷ 2|}.

\section{Дорожные войны}
\paragraph{Не дрова везешь!:} из-за тряскипассажиры транспортного средства совершают любые активные проверки с Помехой.
\paragraph{Отвлекать водителя воспрещается!:} водителю довольно проблематично заниматься чем-то еще, кроме управления транспортным средством. Все атаки водителя совершаются с 2 Помехами. Водитель не может использовать Двуручное оружие.
\paragraph{Под откос:} если машина переворачивается или врезается в препятствие, все пассажиры (включая водителя) получают Дробящие Повреждения, равные \textbf{|30 - Управление транспортом Водилы - Бонус доспеха|}. Если машина не была оснащена ремнями безопасности (или пассажиры не сочли нужным воспользоваться ими), удвойте успешно нанесенные Повреждения.
\newline
Те из пассажиров, кто не был пристегнут, могут избежать Повреждений, совершив проверку Атлетики (Лв) против \textbf{|10 + нанесенные Повреждения|}.
\paragraph{Борт к борту:} водитель может использовать транспортное средство, как оружие. Его Доблесть равна \textbf{|Эксплуатация + модификатор Ловкости + Прочность транспортного средства|}. При провале маневра транспортное средство получает Повреждения, равные величине провала. Обратите внимание, что это не всегда (хоть и зачастую) означает тараны и удары бортами — в случае гироскутеров, мотоциклов и небольших машин водитель заманивает соперника на обочину или в кювет!
\paragraph{Наперегонки!:} герой может попытаться оторваться от преследования, или наоборот, догнать кого-то. Если герой превышает скорость, указанную в разделе <<по дороге всегда быстрее>>, то максимальная скорость в км/ч, которой он может достигнуть, равна \textbf{|Эксплуатация + Воля + Реакция — Опасность местности| х 10}. Разумеется, она все еще ограничена соответствующим параметром из таблицы. Совершите проверку Эксплуатации против \textbf{|10 + Эксплуатация(Лв) противника + Опасность местности|}. В случае успеха герой добивается желаемого, а машине потребуется проверка Износа в конце сцены. В случае провала смотрите раздел <<Под откос>>.
\paragraph{Маневровая скорость:} в течение Очереди водителя транспортное средство может преодолеть расстояние в метрах, равное \textbf{|Эксплуатация(Лв) + Воля водителя + Реакция водителя — Опасность местности|}. Это расстояние включает в себя маневры любой сложности.
\chapter{Социальные взаимодействия}
Здесь вы узнаете, как герои и статисты общаются друг с другом, и чего они могут достичь, не прибегая к физическому насилию.

\section{Эмоциональный фон Сцены}
Статисты реагируют на героев, опираясь не только на их внешность и красноречие. Куда важнее обстоятельства встречи. 
\newline Эмоциональный фон Сцены подскажет, будет ли у статистов желание оценить героев по одежке, и получат ли они шанс блеснуть красотой, обаянием и прочими талантами. Максимальное/минимальное Впечатление, которое получают герои в Сцене, указано в скобках рядом с названием Эмоционального фона. 
\newline Эмоциональный фон может различаться для разных статистов в одной группе - например, если герои хорошо знакомы кому-то из них.
\begin{tcolorbox}
  Чтобы завоевать доверие статистов, потребуется правильно выбранная линия поведения (и, возможно, больше одной Сцены). Договориться с разъяренными врагами очень сложно или даже невозможно, в то время как близкие друзья, родственники и поклонники благосклонно примут самые безумные идеи героев.
\end{tcolorbox}

\subsection{Явная опасность}
\paragraph{Возможные первые Впечатления:} Настороженность или хуже.
\newline Герои не являются личными врагами статистов, но принадлежат к угрожающей им силе. Такой, как вражеская армия, бандитский отряд или толпа вооруженных до зубов чужаков.

\subsection{Потенциальная угроза}
\paragraph{Возможные первые Впечатления:} Нейтралитет или хуже.
\newline Опасность, исходящая от героев, вполне осязаема. Как правило, ее источник - боевое снаряжение, которое можно быстро пустить в ход. Пистолеты в кобурах, незаряженные арбалеты, холодное оружие в ножнах демонстрируют, что герой готов к конфликту, но не планирует его развязывать (наверное). В этом случае статисты видят в герое не только угрозу, и надеются, что все обойдется.

\subsection{Отсутствие угрозы}
\paragraph{Возможные первые Впечатления:} от Остракизма до Доброжелательности.
\newline Статисты не ожидают агрессии и чувствуют себя в безопасности. Они не прочь поболтать и отпустить шутку-другую. В эту категорию попадают встречи героев со знакомыми, публикой и обслугой в увеселительных заведениях.

\subsection{Неформальная обстановка}
\paragraph{Возможные первые Впечатления:} от Настороженности до Восторга.
\newline Статисты расслаблены, готовы к пространным беседам, тесному взаимодействию и в целом настроены дружелюбно. Скорее всего, герой находится среди друзей, родственников или сослуживцев.

\subsection{Интимная обстановка}
\paragraph{Возможные первые Впечатления:} от Нейтралитета до Очарования.
\newline Так обсуждаются выгодные сделки, хитрые планы, ужасные тайны, грязные секреты и прочая информация, не предназначенная для посторонних ушей. Герой пользуется абсолютным доверием статистов.

\section{Первое Впечатление}
\tbd [литературная вставка] Встречают по одежке и что-то про это.
Для того, чтобы определить Первое Впечатление, в зависимости от эмоционального фона, герой совершает проверку
\begin{itemize}
  \item Обаяния, Выступления(Об), Искусства(Об) или Знатока(Об), если Эмоциональный фон - "Неформальная обстановка" или лучше;
  \item Обаяния, Выступления(Об) или Искусства(Об), если Эмоциональный фон - "Отсутствие угрозы" или лучше;
  \item Обаяния или Выступления(Об), если Эмоциональный фон - "Потенциальная угроза" или лучше;
  \item Обаяния, если Эмоциональный фон - "Явная опасность".
\end{itemize}
Герой добавляет Бонусы свойств и способностей, которые влияют на Впечатления.
В зависимости от полученного значения Впечатление будет следующим:
\paragraph{0 и меньше - Отвращение:} статисты отказываются иметь с героем дело. Вышибалы гонят его из кабака, обыватели бранятся и швыряют камни и мусор. Нападение вооруженных статистов - вопрос пары фраз, иным хватит и недоброго взгляда.
\begin{itemize}
  \item Торговля невозможна. Лавочники отказываются торговать и угрожают вызвать охрану.
  \item Просьбы о помощи отвергаются с негодованием, расспросы приводят статистов в ярость.
  \item В потенциально боевой ситуации статисты атакуют героя и бьются насмерть. Если сражение по каким-то причинам невозможно или бессмысленно (герой очевидно сильнее, сражение приведет к неблагоприятным для статистов последствиям), статисты вредят герою иными доступными методами. 
\end{itemize}
\paragraph{1-3 - Враждебность:} статисты с трудом выносят присутствие героя. В кабаке он, скорее всего, станет причиной драки. Обыватели и не думают скрывать своего отношения, но воздержатся от неприкрытой травли, если герой не даст повода. Впрочем, вооруженные статисты или городское ополчение легко найдут повод для стычки!
\begin{itemize}
  \item Торговля разорительна. Лавочники заламывают драконовские цены, а покупают за сущие гроши. СП покупки у статиста возрастает втрое, а СП, за которую статисты готовы что-то купить у героя, сокращается втрое.
  \item Просьбы о помощи отвергаются, расспросы встречаются в штыки.
  \item В потенциально боевой ситуации статисты атакуют героя и отступают лишь в случае очевидного перевеса на его стороне. Если сражение по каким-то причинам невозможно или бессмысленно, статисты вредят герою иными доступными методами.
\end{itemize}
\paragraph{4-6 - Остракизм:} статисты делают вид, что героя не существует. В кабаке ему откровенно не рады. Столик героя будет пустовать и ему придется самому забирать заказ у стойки. Если где-то неподалеку случится преступление, герой станет первым подозреваемым.
\begin{itemize}
  \item Торговля убыточна. Лавочники требуют внушительную наценку СП покупки у статиста возрастает вдвое, а СП, за которую статисты готовы покупать, сокращается вдвое.
  \item Просьбы о помощи и расспросы игнорируются. Обыватели смотрят угрюмо и, если герой окликает их, ускоряют шаг или не обращают на него внимания. Положение может изменить лишь внушительный для статиста подкуп. 
  \item В потенциально боевой ситуации статисты атакуют героя, если на их стороне перевес. В противном случае они отступят и вернутся с подмогой. Если сражение по каким-то причинам невозможно или бессмысленно, статисты действуют против героя иными доступными методами.
\end{itemize}
\paragraph{7-9 - Настороженность:} окружающие ведут себя высокомерно и подозрительно. Лавочники, кабатчики и ополченцы постараются обобрать героя до нитки. Что ж, таков незавидный удел чужаков.
\begin{itemize}
  \item Торговля невыгодна. СП покупки у статиста возрастает в полтора раза, а СП, за которую статисты готовы покупать, сокращается в полтора раза.
  \item Просьбы о помощи и расспросы игнорируются, хотя подкуп или униженные мольбы могут сработать.
  \item В потенциально боевой ситуации статисты атакуют героя, если его убийство не кажется затруднительным или может принести выгоду. В противном случае статисты ограничатся насмешками, грязной бранью и требованиями свалить в ужасе.
\end{itemize}
\paragraph{10-12 - Нейтралитет:} героя воспринимают в целом спокойно. Торговцы не будут обжуливать его слишком явно, а обыватели вряд ли обратят на него внимание.
\begin{itemize}
  \item Торговля идет без осложнений. Лавочники продают и покупают по СП из таблиц.
  \item Просьбы о помощи удовлетворяются, если они не слишком затруднительны. Статисты отвечают на расспросы, однако не стремятся дать полную и исчерпывающую информацию.
  \item В потенциально боевой ситуации статисты атакуют героя, только если спровоцированы.
\end{itemize}
\paragraph{13-15 - Доброжелательность:} герой вызывает сдержанное одобрение окружающих. Торговцы готовы продавать и покупать у него по честным ценам, а обыватели охотно идут с ним на контакт. В кабаке герой безусловно станет центром внимания.
\begin{itemize}
  \item Торговля идет неплохо. Лавочники продают и покупают по СП из таблиц, могут снабдить героя информацией или предоставить небольшую скидку.
  \item Просьбы о помощи удовлетворяются, если они не слишком затруднительны. Статисты отвечают на расспросы по возможности полно. Очевидно глупые вопросы и просьбы статисты встречают с юмором и даже могут дать полезный совет.
  \item В потенциально боевой ситуации статисты атакуют героя, только если их последовательно провоцируют. Они пытаются избегнуть боя и успокоить героя, если это возможно.
\end{itemize}
\paragraph{16-19 - Восторг:} герой кажется окружающим офигительным. Он без труда провернет выгодную торговую сделку и найдет союзников или подельников. Не исключено, что его даже пригласят на ведущую роль в местном празднике.
\begin{itemize}
  \item Торговля выгодна. СП покупки у статиста сокращается в полтора раза, а СП, за которую статисты готовы покупать, возрастает в полтора раза. Лавочники охотно снабдят героя информацией.
  \item Просьбы о помощи удовлетворяются, за исключением опасных или абсурдных. Статисты отвечают на расспросы максимально полно. Очевидно глупые вопросы и просьбы статисты встречают с юмором и могут дать полезный совет.
  \item В потенциально боевой ситуации статисты сдадутся на милость героя, если только он не демонстративно кровожаден. Тогда статисты отступят.
\end{itemize}
\paragraph{20 и больше - Очарованы:} появление героя вызывает больше радости и интереса, чем прибытие разъездного борделя или бродячего цирка. Герой получит внушительные скидки при торговле, а местные шишки, не говоря об обывателях, охотно примут его в своих жилищах.
\begin{itemize}
  \item Торговля крайне прибыльна. СП покупки у статиста сокращается вдвое, а СП, за которую статисты готовы покупать, возрастает вдвое. Лавочники снабжают героя ценной информацией и раскладывают перед ним лучшие товары.
  \item Статисты делают все, чтобы помочь герою и отвечают на расспросы максимально полно. Очевидно глупые вопросы и просьбы удовлетворяются тоже, хотя позже это может выйти герою боком.
  \item Бой невозможен - разве что, бойня. Статисты готовы сдаться на милость героя или даже временно перейти на его сторону.
\end{itemize}
\begin{tcolorbox}
  Этот список может использоваться, чтобы случайным образом определить Впечатления статистов от идей, предложенных героем. Разумеется, на 0 и меньше друзья и знакомые не будут бросаться на героя с оружием или кидать в него грязью, но не постесняются высказать все, что думают о герое и его нелепой выдумке.
\end{tcolorbox}

\paragraph{Автоматический успех/провал Проверки Впечатлений} возможен при помощи Хода "Повезло". В этом случае герой получает максимально/ минимально возможный результат согласно Эмоциональному фону.
\paragraph{Критический успех/провал Проверки Впечатлений} достигается при помощи Хода "Повезло", либо при выпадении 20 и 1 на К20 соответственно. При этом герой достигает результата, следующего за максимально/ минимально возможным согласно Эмоциональному фону.

\paragraph{Общее впечатление} от группы героев определяется по самому непривлекательному индивиду: проверку совершает герой с наихудшим бонусом к проверке. Если герои разделятся, к каждому из них применяется отдельное Впечатление (требующее новой проверки), хотя некоторые статисты наверняка запомнят, что героя видели в компании грубияна или урода!

\paragraph{Вооруженные люди в броне:} статисты нервничают, когда видят вооруженных людей, облаченных в броню. Это не относится к военным объектам и анархическим вольницам, но обыватели не ждут ничего хорошего от вооруженного человека в бронежилете, пока он не служит в городском ополчении или армии.
\newline Если герои не выглядят носящими оружие и броню по праву службы или необходимости (городское ополчение, наемники в военное время, дровосек с топором, охотник с луком), они не могут получить Впечатление лучше Нейтралитета. В случае любого Впечатления хуже Нейтралитета статисты сообщат о героях властям, хотя в остальном вряд ли будут вести себя вызывающе.

\section{Изменение Впечатления и переговоры}
В дальнейшем герой имеет возможность изменить мнение окружающих о себе с помощью действий и проверок Общения (для этого у него должен быть хотя бы шанс поговорить). В случае раскрытых попыток обмана и манипуляций со стороны героя, мнение о нем может измениться и в худшую сторону.
\newline Пытаясь расположить статиста к себе (или возмутительно обмануть его), герой проверяет Общение против \textbf{|10 + Вл|} статиста. Результат и внешняя форма проверки зависит от Стиля общения. В некоторых ситуациях Воля может быть заменена Наблюдательностью (Мд), или даже подходящим НавыкомЭ, например, когда герой пытается отвлечь статиста досужей болтовней или подсунуть испорченный товар. 

\begin{center} \begin{tabular}{|p{6cm}|p{3cm}|p{2.5cm}|p{2.5cm}|} \hline
Впечатление цели от собеседника на момент начала разговора & Мирные переговоры & Переговоры с позиции силы & Боевая сцена \\ \hline
Отвращение & 30 + Вл цели & 25 + Вл цели & Время слов прошло, к оружию! \\ \hline
Враждебность & 25 + Вл цели & 20 + Вл цели & 30 + Вл цели \\ \hline
Остракизм & 20 + Вл цели & 15 + Вл цели & 25 + Вл цели \\ \hline
Настороженность & 15 + Вл цели & 10 + Вл цели & 20 + Вл цели \\ \hline
Нейтралитет & 10 + Вл цели & 10 + Вл цели & 15 + Вл цели \\ \hline
Доброжелательность & 10 + Вл цели & 10 + Вл цели & 10 + Вл цели \\ \hline
Восторг & 5 + Вл цели & 5 + Вл цели & 10 + Вл цели \\ \hline
Очарование & 0 + Вл цели & 0 + Вл цели & 5 + Вл цели \\ \hline
\end{tabular} \end{center}

Переговоры с позиции силы предполагают ситуации, в которых герои имеют (или убедительно делают вид, что имеют) средства заставить статиста сотрудничать и демонстрируют их. Это вовсе не значит, что они на самом деле могут или будут их применять, но у статистов останется пренеприятное ощущение, что их принудили к компромиссу. В зависимости от статуса, благосостояния и влияния статиста, после окончания переговоров это может откликнуться как бессильной истерикой, так и наймом лучших охотников за головами.
\newline При переговорах с позиции силы герой может заменить проверку Общения проверкой Доблести или Меткости. Если проверка успешна, а статист уцелел, герой легко добьется своего.
\paragraph{Переговоры с группами:} группы существ менее восприимчивы к угрозам, лести, лжи и доводам разума. Каждый боится, что если он поддастся на уговоры, то другие посмеются над ним или пустят пулю в затылок. Помимо этого, удерживать внимание и отслеживать настроение множества существ сразу не так просто. 
\newline При переговорах с группами использется Выступление и самое распространенное значение Воли статистов в группе. Если выступление (и аудитория) не подготовлены заранее, герой совершает проверку с Помехой, когда ведет переговоры с группой из 10 или более существ. Герой совершает проверку с 2 Помехами, когда ведет переговоры с группой из 100 или более существ. Тут герою точно не обойтись без мегафона или Луженой глотки!
\paragraph{Социальные взаимодействия в бою:} болтать в пылу битвы - рискованное занятие. Если у героя возникло такое желание, он применяет Навык Общения или Выступления, использовав Действие.

\subsection{Стиль Общения}
Он зависит от того, модификатор какой Характеристики используется при проверке, и какой линии поведения придерживается герой.
\paragraph{Сила:} прямолинейное запугивание и успех проверки приносит Восторг (безусловно, показной). Статисты сделают все, что требует герой, пока он смотрит в их сторону. Если герой имеет возможность разыскать статиста и наказать за неповиновение (или статист верит в это), статист выполнит указания и без непосредственного присутствия героя.
\newline Провал провоцирует яростную атаку, если статист считает, что у него есть шанс на победу, или бегство. Статист обязательно вернется с подмогой!
\paragraph{Ловкость:} порой секс является эффективным подспорьем в достижении самых разных целей.
\newline Для того чтобы проверить Общение(Лв), герой должен уйти в Интерлюдию. Успех проверки приносит ему ничем не замутненный Восторг статиста. В случае неудачи Впечатление от героя снижается на величину провала.
\paragraph{Выносливость:} на пирушках важно не только умение героя подать себя, но и способность много съесть и выпить, продолжая общаться. Успех приносит Доброжелательность окружающих. Провал означает, что герой осрамился, а его репутация пострадала. Окружающие подвергают героя Остракизму, хотя в будущем у него будут возможности обелиться. Наверное.
\paragraph{Интеллект:} король торговли и дипломатии. Доводы разума при успехе принесут герою Доброжелательность статистов. Провал приводит к Нейтралитету, если только Эмоциональный фон не подразумевает худший вариант.
\paragraph{Мудрость:} поможет подловить лжеца на на мелких деталях или интуитивно не поверить в самые правдоподобные посулы. Чтобы заподозрить обман или притворство, герой должен успешно проверить Общение(Мд) против \textbf{|10 + [Общение лжеца]|}. 
\paragraph{Обаяние:} для непрямых угроз и заговаривания зубов герою пригодится Обаяние - сочетание эффектной подачи и личностной притягательности. 
\newline Успех проверки приносит герою Восторг и все ему сопутствующее. Статисты уверены, что им выгоднее согласиться на предложения героя, чем поступить по-своему, пока кто-то не откроет им глаза. В этом случае, как и при провале проверки, герой сполна отведает Враждебности окружающих. Никто не любит оставаться в дураках!
\paragraph{Соблазнение:} использует Обаяние героя в сочетании с сексуально агрессивным поведением различной степени очевидности. 
\newline Успех проверки Очаровывает жертву, пока она уверена, что герой вскоре очутится в ее постели. Если эта уверенность не подкрепляется поведением героя, он становится Отвратителен статисту (хотя статист все еще может быть не прочь затащить героя в постель). То же происходит и при провале проверки - статист понимает, что им пытаются манипулировать. Отвращение может вылиться как в знатный синяк на смазливом личике, так и в более серьезные последствия, если статист облечен властью и богат. 
\paragraph{Публичные выступления:} как правило, бывают хорошо подготовленными, но даже тогда главную роль играют ВыступлениеЭ и Обаяние. Проваленный бросок для странствующего певца редко выливается во что-то большее, чем предложение убраться с помостков. Политики, военачальники и революционеры рискуют гораздо больше, особенно при экспромтах.
\paragraph{Пытки:} палач может использовать Общение(Ин, Об),  чтобы убедить статиста рассказать все до начала пытки. Обычно это сопровождается демонстрацией пыточных принадлежностей. Если жертва упорствует, они идут в дело. Пытки занимают Интерлюдию. Обычно палачи используют Медицину(Ин, Мд), Ловкость рук(Сл, Лв, Мд) или Дб/ Мт с применением оружия, наносящего Несмертельные Пв. 
\newline За каждую единицу успеха проверки против \textbf{|10 + [Вл жертвы] + [МВн жертвы]|}, она отвечает на один вопрос, исчерпывающе и полно. Если жертва не знает ответов, она соврет или постарается сказать палачу то, что тот хочет услышать.

\section{Влияние на героев}
Герои и статисты могут Влиять друг на друга с помощью проверок Общения и иными доступными способами, например, феноменами или Ходами. 
\newline Когда герой подвергается Влиянию статиста, разрешите ситуацию согласно Капризу Судьбы "Чуждое влияние".

\section{Конфликты героев}
Когда герой подвергается Влиянию или нападению \textit{другого героя}, игрок получает выбор. Он либо полностью принимает последствия, добавляет герою 1 Очко опыта и действует соответственно случившемуся, либо объявляет, что герой не поддался манипуляциям или не может быть атакован. 
\newline Если конфликты героев - важная часть вашей игры, разрешайте их по общим правилам.




\ifx\islight\undefined
\chapter{Досуг и путешествия}
Здесь вы узнаете, как герои отдыхают, чем занимаются в свободное от приключений время, и что за опасности подстерегают их в путешествиях.
\section{Скрытая угроза}
Опасности могут поджидать героев на каждом шагу. Пробираясь сквозь дремучую чащу, прогуливаясь по незнакомому городу, сделав привал на лесной поляне или же развлекаясь в большом казино, герои могут наткнуться на серьезные проблемы, если не проявят бдительность. В отличае от обычной проверки Неприятностей, проверка Скрытой угрозы обычно производится в начале Сцены и не приводит к немедленным и очевидным проблемам для героев. Эффект Скрытой угрозы может быть отложен на несколько сцен, а если ведущий не забудет, то и несколько сессий.
\trouble
{Неожиданная слава}%no sweat name
{Герои будут вознаграждены за смелость, отзывчивость и доброту, а нерешительность, равнодушие и жестокость не возымеют далеко идущих последствий.}%no sweat description
{Круги на воде}%tough day name
{Действия героев не приведут к значительным последствиям. Полученные знакомства мимолетны, враги незлопамятны, а хозяева вещей, которые герои прибрали к рукам нескоро заметят пропажу.}%tough day description
{Тень затмения}%we have trouble name
{Проблема, которую все же реально заметить, пока не станет слишком поздно. Сцена еще может обернуться сущим кошмаром, но герои выйдут сухими из воды, если не будут хлопать ушами.}%we have trouble description
{Петля на шее}%fiasco name
{Герои попали в переплет. Убитые разбойники имели влиятельных покровителей, найденные предметы были кем-то спрятаны, а спасенная красотка обокрала караван и сбежала!}%fiasco description
\section{Встречи и находки}
Никто заранее не знает, когда герои на своем пути встретят неожиданную компанию или занимательную ситуацию. Даже просто гуляя по крупному городу они могут попасть в гущу событий, не говоря уж об исследовании таинственных руин!
\newline Проверка Встреч и Находок является проверкой Неприятностей, в которой дополнительно учитывается четность выпавшего числа. В таблице указаны возможные наполнения сцены в соответствии с проверкой.
\begin{tcolorbox}
Эта проверка требует броска кубика, даже если игроки принимают Каприз Судьбы. Проверка в этом случае определяет, какое из двух зол встречают герои: конфликтную Встречу или опасную Находку.
\newline При использовании Хода \textit{Повезло!} игроки могут сами решить, какая сцена их ждет: приятная Встреча или полезная Находка.
\end{tcolorbox}
\begin{center}
\begin{tabular}{|p{3cm}|p{6.5cm}|p{6.5cm}|}
\hline
\textbf{Результат проверки Неприятностей} & \textbf{Встречи(нечетный результат)} & \textbf{Находки(нечетный результат)}
\\ \hline
\textbf{Легкая добыча}\newline\textit{(19-20)} & Встреченные люди практически беззащитны. Отряд героев может помочь им или отобрать то, что они имеют. Все равно они не смогут оказать сопротивления. & Находка сулит богатства. Недавно покинутый дом или оставленный транспорт. Скорее всего там есть, чем поживиться и, похоже, рядом нет никого, кто мог бы помешать отряду героев.
\\ \hline
\textbf{Место интереса}\newline\textit{(13-18)} & Встреча нейтральна. Караван хорошо вооружен, но не собирается мешать отряду героев. Возможно, они захотят поболтать и обменяться товарами и информацией. & Занятное местечко. Давно оставленный лагерь, полуразрушенный дом или полусгнившая телега. Если внимательно осмотреть место, возможно можно найти полезную вещицу. Но останется вопрос - а оно того стоило?
\\ \hline
\textbf{Напряженная ситуация}\newline\textit{(7-12)} & Встреча на грани конфликта. Отряд героев встретил напуганных или разъяренных чем-то людей, а возможно прервали интимный момент раздела добычи над поверженным соперником. В любом случае, если герои не проявят тактичность, дальше будет говорить оружие. & Находка кричит об опасности. Разграбленный караван, место битвы или разодранный животными труп. Возможно, среди останков можно найти что-нибудь ценное, но нет никаких гарантий, что отряд героев не разделит участь павших.
\\ \hline
\textbf{Заваруха}\newline\textit{(1-6)} & К оружию! Нападение, засада, прямая конфронтация. Для того, чтобы мирно урегулировать эту Встречу придется расстаться с имуществом, обладать невероятным красноречием или надеяться на вмешательство Судьбы! & Находка приносит лишь беды. Опасный предмет, разрушенный мост или заваленный камнями перевал, кровавый алтарь со свежим жертвоприношением. Героям не получится пройти мимо, закрыв на происходящее глаза и они это знают
\\ \hline
\end{tabular}
\end{center}

\section{Отдых и восстановление}
Герой может полноценно отдыхать только во время Антракта.
\paragraph{Восстановление.} Во время отдыха герой восстанавливает число ЕЗ и/или ЕХ, равное \textbf{МВн(мин.1)}, а так же выводит из организма Интоксикацию в размере \textbf{МВн(мин.1)}. ЕЗ и ЕХ восстанавливаются в любых комбинациях в пределах восстанавливаемого значения, а Интоксикация считается отдельно.
\newline Если при этом герой получает квалифицированный медицинский уход, эти значения удваиваются. У лекаря должно быть хотя бы 1 Очко опыта во Врачевании. Один лекарь может одновременно присматривать за числом раненых, равным его МИн (минимум 1).
\newline Сломанные конечности требуют особого внимания. Чтобы герой восстановил функционирование сломанной конечности, лекарь должен пройти проверку Врачевания против 15. Если этого не сделать и восстановить ЕЗ магией или подождать, пока они восстановятся сами, конечность срастется неправильно и герой получит недостаток Старая Рана.
\paragraph{Отдых и доспехи:} герой может полноценно отдыхать (и восстанавливать ЕЗ и/или ЕХ), пока одет в доспех с БЗщ +4 или меньше. Герой, по каким-то причинам отдыхавший в доспехе с большим БЗщ, Устает. Обратите внимание, что ему все же удается худо-бедно поспать, а потому шанс задремать на посту или за столом в таверне существенно ниже, чем без сна вообще.
\section{Изготовление оружия, доспехов и механизмов}
Ремесленники редко путешествуют и участвуют в авантюрах. Им просто некогда этим заниматься — на изготовление добротного оружия или доспеха может уйти куча времени! Но если в вашей истории герои имеют доступ в оборудованную мастерскую, то во время Антрактов они могут изготовлять и ремонтировать снаряжение. Для изготовления снаряжения используется навык Наука, а для ремонта - Ремонт.
\paragraph{Время изготовления:} за каждую еденицу сложности проверки выше 10 герой или статист должен провести 1 Антракт, занимаясь изготовлением желаемого предмета. Ускорение изготовления невозможно, потому что это дело не терпит спешки.
\paragraph{Отдых и изготовление.}
Во время Антракта герой восстанавливает ЕЗ, ЕХ и выводит Интоксикацию несмотря на то, что занимается в это время изготовлением предметов, однако в это время он не может получать медицинский уход.
\begin{center}
\begin{tabular}{|p{10cm}|c|}
\hline
Предмет & Сложность \\ \hline
Оружие/доспех/щит & 15 + БПв/БЗщ \\ \hline
Оружие наносит Колющие Повреждения & +5 \\ \hline
Доспех имеет БЗщ 5 и больше & +5 \\ \hline
Доспех/щит с шипами & +5 \\ \hline
Примитивное устройство (мельница, влагоуловитель) & 10 \\ \hline
Простое механическое устройство (замок, музыкальная шкатулка) & 20 \\ \hline
Механическое устройство средней сложности (механическое опахало, башенные часы) & 25 \\ \hline
Сложное механическое устройство (двигатель внутреннего сгорания) & 30 \\ \hline
Конструкция комплексная(например, автомобиль включает в себя ходовую часть, корпус и так далее) & +10 \\ \hline
Медицинские препараты & (10+СП)/2 \\ \hline
Взрывчатка и Гранаты & (10+СП)/2 \\ \hline
Боеприпасы & СП/2 \\ \hline
Затраченное время увеличивается в 5 раз & -5 \\ \hline
Затраченное время увеличивается в 10 раз & -10 \\ \hline
Мастер не имеет доступа к оборудованной мастерской и работает подручными инструментами & Помеха на проверку \\ \hline
\end{tabular}
\end{center}

\section{Досуг и развлечения}
Чем занимаются герои, когда выдается свободная минутка? Ответ на этот, казалось бы, незначительный вопрос может во многом определить развитие истории и наполнить ее событиями! Разумеется, вам не нужно применять эти таблицы, если ответ очевиден для всех участников игры. Эти правила призваны наполнить историю событиями как случайными, так и логически вытекающими из сюжетной канвы. Игроки и мастер могут использовать их, чтобы дать героям игромеханические преимущества или узнать что-то важное, а также черпать в них идеи для дальнейшего развития сюжета. Если у игрока нет конкретных идей, он может решить, как герой проведет свободное время, выбрав любой вариант из таблицы «Досуг и Развлечения».
\paragraph{Эффекты:} эффекты, описанные в таблице, входят в игру в следующей сцене и длятся до ее окончания, но мастер может сохранить их и на большее время, если это соответствует контексту. Разумеется, все материальные ценности, которые приобрел герой, останутся при нем. Суммарное число положительных эффектов Досуга и Развлечений не может превышать \textbf{|1 + МОб героя|}(минимум 1).
\paragraph{Проблемы:} возможные негативные последствия Досуга и Развлечений.
\paragraph{Риск:} вероятность того, что Досуг или Развлечение по самым разным причинам обернутся тоской и унынием. Чем выше Риск, тем больше шанс почувствовать в конце Досуга или Развлечения лишь усталость, пустоту и бессилие, даже если герой не пострадал физически и формально получил больше, чем потратил.
\paragraph{Сложность приобретения:} за развлечения приходится платить, да и самые обычные с виду занятия могут потребовать некоторых трат. Если герой совмещает несколько видов Досуга и Развлечений, сложите СП.
\paragraph{Скрытая угроза:} иногда самые невинные забавы могут закончиться сущим кошмаром! Некоторые виды Досуга и Развлечений предполагают проверки Скрытой угрозы. При этих проверках отнимите Риск Досуга и Развлечений от выпавшего значения для определения результата.
\paragraph{Всего да побольше:} виды Досуга и Развлечений могут совмещаться друг с другом. Проконсультируйтесь с мастером, чтобы выяснить, какие именно, хотя обычно это следует из логики ситуации. В любом случае герой может одновременно совмещать не более \textbf{|1 + ММд|} (минимум 2) видов Досуга и Развлечений. В случае необходимости используйте наибольший показатель Риска. Когда совмещенные виды Досуга и Развлечений следуют друг за другом, герой может потратить приобретенные эффекты на необходимые проверки Досуга и Развлечений. Если он не сделает этого, эффекты не считаются потерянными.
\paragraph{Затраченное время:} подразумевается, что герой уделяет Досугу или Развлечению не меньше 1 часа. Хотя, как правило, больше.
\paragraph{Сложность проверок:} сложность всех необходимых проверок равна \textbf{|10 + Риск|}. В некоторых случаях указанные проверки могут быть усложнены или заменены другими в соответствии с логикой ситуации.
\paragraph{Доступность:} совершите проверку Неприятностей, чтобы определить, доступен ли желаемый вид Развлечения. Разумеется, бросок совершается только в том случае, если у мастера и игроков есть какие-то сомнения на этот счет!
\trouble
{Массовая культура}%no sweat name
{Развлечение доступно и дешево. Используйте СП в таблице}%no sweat description
{Утеха для ценителей}%tough day name
{Развлечение широко распространено, но не так уж доступно. Используйте увоенную СП в таблице.}%tough day description
{VIP залы}%we have trouble name
{Развлечение доступно, но в силу неких причин довольно дорого. Используйте утроенную СП в таблице.}%we have trouble description
{Частные клубы}%fiasco name
{Развлечение недоступно, хотя некоторые Атрибуты могут помочь герою отыскать желаемое. Используйте утроенную СП в таблице.}%fiasco description
\subsection{Cтруктура сцены досуга}
\begin{enumerate}
\item Определите доступность Досуга или Развлечения, а также сколько видов Досуга и Развлечений одновременно совмещает герой.
\item Совершите проверки, связанные с эффектами.
\item Совершите проверки, связанные с проблемами, если требуется.
\item Совершите проверку Скрытой угрозы, если требуется.
\end{enumerate}
\section{Варианты досуга}
\genAndGet{leisure}{leisure}{Досуг}

\section{ПУТЕШЕСТВИЯ}
Путешествия — один из основополагающих элементов многих жанров. Какая бы причина ни вынудила героев двинуться в путь, в дороге их подстерегает немало трудностей. Если путешествия являются важной частью вашей истории и вы желаете выяснить, насколько сложен будет путь, следуйте пунктам ниже:
\begin{enumerate}
\item \textbf{Сделал дело — гуляй смело.} Игроки должны решить, кто из героев или статистов в караване станет:
\begin{itemize}
\item \textbf{Навигатором}, который ищет безопасный путь, определяет подходящие места для стоянки и пополнения припасов. Навык «Выживание» — важнейший для Навигатора.
\item \textbf{Механиком}, который следит за состоянием машин и прочей техники в группе. Ему потребуется Эксплуатация (Ин, Мд).
\item \textbf{Погонщиком}, который следит за тем, чтобы вьючные животные не провалились в яму и не наелись ядовитой травы, а техника. Он использует Обращение с животными.
\item \textbf{Разведчиком}, который идет впереди каравана, отслеживая все подозрительное и разыскивая места, представляющие интерес. Ему понадобятся Наблюдательность.
\item \textbf{Проводнком}, который проведет караван мимо нежелательных встреч и внезапных препятствий. Ему понадобятся Скрытность и Наблюдательность. Эта роль не применяется, если отряд передвигается Маршем. 
\end{itemize}
\begin{tcolorbox}
Роли Механика и Погонщика являются опциональными. Если в отряде нет техники и въючных животных, Механик и Погонщик будут слоняться без дела и эти роли можно не назначать.
\end{tcolorbox}
Одну роль могут выполнять несколько героев (используйте правила Взаимопомощи), но один герой не может выполнять несколько ролей.
\newline
Когда по тем или иным причинам роль остается невыполненной, последствия могут быть самыми плачевными.
\paragraph{Если Навигатора, Погонщика(при условии, что в отряде есть вьючные животные) или Техника(при условии, что в отряде есть техника)} нет в караване, считайте, что при соответствующих проверках выпало 5. Если разница между этим результатом и целевой сложностью проверки превышает 10, считайте проверку Критическим провалом.
\paragraph{Если в караване нет Разведчика,} герои могут стать жертвами случайного нападения. Они уязвимы для всех тех опасностей, которые легко предотвратить, заметив вовремя! Сцены Встреч и Находок начинаются сразу после того, как определены их тип и Скрытая Угроза. В Боевых сценах отряд действует, как будто подвергся внезапному нападению.
\paragraph{Если в караване нет Проводника,} Отряд не может избежать Встреч и Находок и обязан принять участие в сценах, с ними связанных. Даже если Разведчик сообщил о том, что сцена не сулит ничего хорошего.
\item \textbf{По дороге всегда быстрее.} Определите \textbf{Опасность местности(ОМ)}, по которой предстоит пройти героям – в пути их ждет немало сюрпризов! Если в пути герои преодолевают местность с разными уровнями Опасности, используйте наибольший.
\begin{tcolorbox}
\paragraph{Терра инкогнита} Если герои отправляются в путешествие без карты, Повысьте ОМ на 1(макс.5). Если герои являются пионерами и карт местности, которую они покоряют просто не существует, повысьте ОМ на 2(макс.5).
\end{tcolorbox}
\begin{center}
\begin{tabular}{|p{7cm}|p{7cm}|c|}
\hline
Тип местности & Глубина вод для водного транспорта & Опасность \\ \hline
Обжитые пригороды, фермерские угодья, торговые тракты, области, подробно и точно нанесенные на карты. & Открытый океан. & 0 \\ \hline
Прерии, равнины, области, не слишком подробно нанесенные на карты. & Архипелаг или прибрежная зона материков. & 1 \\ \hline
Лесистые и болотистые равнины, холмы. & Широкие реки с простым фарватером & 2 \\ \hline
Лесные дебри, топи, скалистые холмы, руины больших городов. & Широкие, но мелеющие реки & 3 \\ \hline
Горы и пустыни. & Узкие извилистые реки с непредсказуемым фарватером & 4 \\ \hline
Джунгли и заболоченная чаща. & Реки с быстрым течением. Острые камни и опасные пороги прилагаются. & 5 \\ \hline
\end{tabular}
\end{center}
\item \textbf{Долго ли, коротко ли…} Определите длительность пути в днях, ориентируясь на скорость самого медленного транспорта каравана. Длительность задает базовую Сложность пути. Прибавьте к ней Опасность местности. Получившееся число — финальная Сложность пути.
\begin{center}
\begin{tabular}{|c|c|}
\hline
Длительность & Сложность пути \\ \hline
1-15 дней & 10 + ОМ \\ \hline
16-40 дней & 15 + ОМ \\ \hline
41 день и больше & 20 + ОМ \\ \hline
\end{tabular}
\end{center}
\begin{tcolorbox}
Для простоты определения Длительности путешествия считайте, что рельеф местности не влияет на скорость путешествия.
\end{tcolorbox}
\textbf{Марш:} герои могут увеличить скорость передвижения вдвое. При этом они получают Помеху на проверки Наблюдательности и должны совершить проверку Вн или Атлетики (Вн) против 15. В случае провала герои измотаны и находятся в состоянии Усталости до тех пор, пока не отдохнут минимум 8 часов. Опасность местности на марше возрастает на 1. Если в караване есть транспортные средства, скакуны или вьючные животные, опасность местности возрастает на 2.
\newline \textbf{Тише едешь — дальше будешь:} герои могут ополовинить скорость передвижения и получить преимущество на проверки Скрытности Проводника для того, чтобы избежать любых Встреч и Находок.

\item \textbf{Да что вы, ребята, я сам здесь впервой!} Навигатор совершает проверку Выживания (Ин), Механик совершает проверку Эксплуатации (Ин, Мд), Погонщик совершает проверку Обращения с животными (Сл, Ин, Мд, Об) против финальной Сложности пути. Результат провкерки Путешествия равен сумме результатов этих проверок. Например, если Навигатор провалил проверку на 5, а Погонщик – на 8, суммарная величина провала составит 13. Если Навигатор преуспел на 7, а погонщик провалил проверку на 4, суммарная величина успеха составит 3.
\newline Если результат хотя бы одной проверки был Критическим Провалом, то вся проверка Путешествия считается Критическим Провалом.
\newline
Успех проверок означает, что путешествие проходит без осложнений и позволяет героям наслаждаться относительным комфортом:
\begin{center}
\begin{tabular}{|c|p{10cm}|}
\hline
Величина успеха & Эффект \\ \hline
1-5 & - \\ \hline
6-10 & Расход еды, воды и других ресурсов сокращается вдвое. \\ \hline
11-15 & Караван прибывает на 1 день раньше. \\ \hline
16-20 & Караван может избежать 1 Встречи или Находки или совершить 1 дополнительный бросок по таблице Встреч и Находок. Караван прибывает на 2 дня раньше. \\ \hline
21 и больше & Караван может избежать до 2 Встреч или Находок или совершить до 2 дополнительных бросков по таблице Встреч и Находок. Караван прибывает на 3 дня раньше. \\ \hline
Критический Успех & Преимущество на проверку Встреч и Находок. \\ \hline
\end{tabular}
\end{center}
Если проверка провалена, то путешественники получают Повреждения, а Опасность Встреч и Находок, изначально равная \textbf{нулю}, возрастает.
\newline \textbf{Цена провала.} При провале проверки, герои и транспорт получают повреждения, а Опасность Встреч и Находок возрастает. Сверьтесь с таблицей для того, чтобы определить последствия.
\newline \textbf{Загрязнение.} В некоторых ситуациях герои идут по местности, знаменитой своими токсичными испарениями, искажающими эманациями или дурманящей флорой. В этом случае потерянные ЕЗ при провале заменяются Интоксикацией.
\begin{center}
\begin{tabular}{|c|p{5cm}|p{5cm}|}
\hline
Величина провала & Потерянные героями ЕЗ & Опасность Встреч и Находок \\ \hline
1-5 & ОМ(мин 1) & 1 \\ \hline
6-10 & ОМ+2 & 2 \\ \hline
11-15 & ОМ+4 & 3 \\ \hline
16-20 & ОМ+7 & 4 \\ \hline
21 и больше & ОМ+10 & 5 \\ \hline
\end{tabular}
\end{center}
\item \textbf{Остановки в пути}. Приятные неожиданности редки в пути, зато других хоть отбавляй. Совершите проверку Неприятностей и определите число остановок, которые привели к Встречи или Находоке:
\begin{center}
\begin{tabular}{ |p{2.7cm}|p{12cm}| }
\hline
\textbf{Результат проверки Неприятностей} & \textbf{Количество Встреч и Находок}
\\ \hline
19-20 & \textbf{0}+Опасность Встреч и Находок
\\ \hline
13-18 & \textbf{1}+Опасность Встреч и Находок
\\ \hline
7-12 & \textbf{2}+Опасность Встреч и Находок
\\ \hline
1-6 & \textbf{3}+Опасность Встреч и Находок
\\ \hline
\end{tabular}
\end{center}
\textbf{Сложность Встречи.} Чтобы определить целевую сложность проверок, не связанных с Общением, сложите \textbf{|10 + ОМ + Опасность Встреч и Находок|}.
\begin{tcolorbox}
Остановки не обязательно будут распределены по всему пути равномерно. Мастер волен решать в соответствии с логикой мира и повествования, на каком отрезке пути были совершены значимые Остановки. Возможно, начало похода было насыщено событиями или же все самое интересное проихошло только под конец путешествия.
\end{tcolorbox}
\item \textbf{Все, приехали.} Если результат проверки Путешествий стал Критическим Провалом, то это означает, что во время путешествия с караваном случилось что-то, что помешало им продолжить путь. Караван прошел половину пути или меньше и для того, чтобы продолжить движение, нужно начать путешествие заново с той точки, в которой вынужденно остановились герои.
\newline Количество Остановок уменьшается вдвое(округляя в большую сторону). Последняя Встреча или находка обязательно начинается с \textbf{Заварухи}, которая и привела к прерыванию путешествия.

\item \textbf{Встречи и Находки} Совершив проверку Встреч и Находок, определите наполнение сцены. Обратите внимание, что проверка «Скрытой угрозы» все еще может серьезно изменить смысловое наполнение сцены.
\item \textbf{Скрытая угроза.} В начале сцены Встречи или Находки, совершите проверку Скрытой угрозы. Отнимите от результата проверки Опасность Встреч и Находок, определенную на 4 этапе. Скрытая угроза не обязательно проявится в начале сцены, но дает мастеру хорошее представление о том, чем она может закончится.
\item \textbf{А что это унас тут?} Для того, чтобы заранее заметить Встречу и не проморгать Находку, Разведчик отряда должен преуспеть в проверки Внимательности против Сложности Встречи.
\item \textbf{Я тут мимо проходил.} Если отряд желает избежать Встречи, Проводник отряда должен преуспеть в проверки Скрытности(Ин) против Сложности Встречи, чтобы провести союзников мимо, не привлекая внимания. В случае провала, сцена Встречи начинается и проверки Впечатления статистов на отряд совершаются с Помехой.
\newline
Если Отряд желает обойти Находку стороной, Проводник отряда должен преуспеть в проверки Наблюдательности(Мд) против Сложности Встречи, чтобы найти обходной путь и не попасть в возможную засаду.
\item \textbf{Поболтаем?} Если отряд не избежал встречи и проверки Впечатлений достаточно хороши и нет спешки, большинство статистов готовы общаться и торговать с героями. Проверки Впечатлений не принесут результатов лучше Доброжелательности, но действия героев — могут!
\begin{tcolorbox}
\textbf{Маршрут изменен.} Никто не знает заранее, что таит в себе очередная Встреча или Находка. После очередной Остановки герои могут решить пойти в другую сторону. В этом случае все ожидающие их Встречи и Находки так и останутся неразведанными и начинается новое Путешествие.
\end{tcolorbox}
\end{enumerate}


\section{Перенос тяжестей}
Веса в игре измеряются в килограммах. Комфортная нагрузка для героя равна значению его \textbf{|Сл × 3|}. Если герой несет больший вес, то его Ск падает вдвое. В дополнение к этому, если нагрузка героя превышает параметр его \textbf{|Сл × 5|}, все его активные проверки совершаются с Помехой, а все атаки по нему совершаются с Преимуществом. Максимальная нагрузка героя, с которой он может идти, равна его \textbf{|Сл × 10|}.
\newline
Герой может толкать, тянуть и отрывать от земли вес, вдвое превышающий его максимальную нагрузку. Если герой толкает или тянет вес, превышающий его максимальную нагрузку, его Ск падает до 1.
\paragraph{Большие} герои могут нести больший вес. Герой увеличивает вдвое все параметры, связанные с переносом тяжестей, за каждую категорию размера больше Среднего. Маленькие герои ополовинивают эти параметры за каждую категорию размера меньше Среднего.
\paragraph{Четвероногие существа}, такие, как кони, мулы и слоны, способны переносить больший вес. Увеличьте вдвое все параметры, связанные с переносом тяжестей после учета бонусов или штрафов за размер. Например, чтобы определить комфортную нагрузку для Большого коня, умножьте его Сл на 3, затем на 2 за размер и еще на 2 — за четвероногость. Маленькая лайка при этом будет способна нести такой же вес, как Средний человек с аналогичным параметром Силы. Гигантские пауки, улитки и змеи считаются четвероногими для определения нагрузки!
\section{Плавание}
Вплавь герой передвигается с половиной своей Ск. Если герой несет вес, превышающий значение его Силы, ему потребуются проверки Сл или Атлетики (Сл), чтобы держаться на воде. Если герой плывет с весом, превышающим значение его \textbf{|Сл × 3|}, он получает Помеху на эту проверку. Если герой плывет с весом, превышающим значение его \textbf{|Сл × 5|}, он получает 2 Помехи. Герой автоматически проваливает проверку, если несет максимальный вес. Проваливший проверку герой находится в состоянии Удушья, хотя провал не всегда означает, что герой тонет. Используйте стандартную таблицу сложностей для отображения быстрого течения или неблагоприятных погодных условий. Иногда герой может передвигаться вплавь с полной Ск или быстрее, например, если он плывет по течению, хотя в таких случаях ему точно потребуются проверки Атлетики (Сл).
\newline
Герой, вынужденный сражаться во время плавания, совершает все атаки с Помехой. Если герой провалил проверку Сл или Атлетики (Сл) во время плавания, он пропускает свою Очередь!
\section{Падение}
Падая или прыгая с большой высоты, герой может пострадать. Герой может спрыгнуть с высоты в 3 метра без всякого вреда для себя. За первый метр после 3 герой получает 2 Пв, за каждый следующий метр полученные героем повреждения увеличиваются вдвое, то есть, спрыгнув с высоты 7 метров, герой получит 8 Пв. Если при прыжке герой получает Пв, достаточные для нанесения Опасной раны, то он ломает конечность — случайно определите какую. Успешная проверка Атлетики (Лв) против \textbf{|7 + общая высота падения|} избавит героя от Перелома, но не избавит его от Пв.
\newline
Если герой падает, то есть не смог сгруппироваться и подготовиться к прыжку, то он дополнительно получает 2 Пв за каждый метр, который пролетел, а также число Пв, равное сумме БЗщ щита и доспеха, которые на нем надеты. Возможно, потребуется проверка Неприятностей, чтобы определить, отделался герой синяками или получил серьезные травмы.
\section{Ловушки}
Герой попадает под действие ловушки, если не заметил ее вовремя при помощи проверки Наблюдательности (Мд) или активировал случайно, неудачно применив Наблюдательность (Ин). Опасность ловушки (как абстрактную, так и соответствующий параметр) определяет мастер. Многообразие ловушек слишком велико, чтобы подробно перечислять их здесь. Тем не менее, их можно разделить на несколько основных типов:
\begin{itemize}
\item \textbf{Импровизированные ловушки:} ловушки из подручных материалов — бесхитростные, но все еще смертельно опасные. Если герой подвергается действию ловушки, то получает Пв, равные \textbf{|Величине проверки Выживания (Ин) установившего ловушку — БАЗщ — Наблюдательность (Мд) — БД*|}.
\item \textbf{Ловушки, наносящие фиксированные Повреждения:} если герой подвергается действию ловушки, то получает Повреждения, равные \textbf{|Опасность ловушки — БАЗщ — БД*|}. Например, ловушка, изрыгающая пламя, имеет Опасность 30 и заполняет огнем область 3 × 3 метра. Если герой в чешуйчатом доспехе подвергнется ее действию, то получит 30 — 10 — 4 = 16 Пв.
\item \textbf{Ловушки с фиксированной Доблестью или Меткостью:} как правило, это ловушки с подвижными частями — дротики, выстреливающие из стен, или стальные лезвия, вылетающие из потолка. Ловушка обладает параметром Дб или Мт и в случае активации совершает проверку против Зщ героя по обычным правилам.
\item \textbf{Ловушки с ядами:} если герой подвергается действию ловушки, то в дополнение к прочим эффектам на него сразу накладывается действие яда, как при КУ отравляющими повреждениями.
\item \textbf{Несмертельные ловушки:} ловушки, которые могут захватывать, погружать в сон, наносить Несмертельные Пв или еще как-то выводить героя из строя вместо того, чтобы убивать его.
\item \textbf{Смертельные ловушки} не наносят Повреждений — если герой подвергся их действию, его ЕЗ сразу падают до 0. Скорее всего, у спутников жертвы будет совсем немного времени, чтобы вытащить изувеченное тело или то, что от него осталось. Например, герой упавший в яму с концентрированной кислотой, понижает свои ЕЗ до 0 и совершает проверку Вн против 15. Если он преуспевает, у спутников будет Круг на то, чтобы попытаться вытащить героя и помочь ему. В противном случае герой растворяется в кислоте.
\end{itemize}
\paragraph{*Ловушки и Бонус доспеха:} разумеется, крепкий доспех поможет спастись во многих случаях... Хотя если герой попал в яму с зыбучим песком, эта груда железа станет серьезной проблемой! Так или иначе, лучшая защита от ловушки — высокая Наблюдательность (Мд).
% феномены и бестиарий будут переработаны после того, как будет определено, удастся ли перенормировка Характеристик.
\end{document}
\endinput
\chapter{Феномены и Исказители}
\textbf{Феномены} - собирательный термин для всевозможных сверхсил, пугающих и необъяснимых. Они имеют самую разную природу. Паронормальные способности, мутации, телепатия, психокинезия и даже всамделишная волшба - или технологии, от нее неотличимые. Общее у них одно - они поглощают Энергию, которую не так просто восполнить. Те, кто умеет воплощать эти феномены, зовутся \textbf{Исказителями}. Своей удивительной мощью они искажают и разрушают привычные законы мира - зачастую, в своих собственных интересах.
\paragraph{Получение феноменов.} Герой начинает игру без известных ему феноменов. Есть несколько способов их использовать:
\begin{itemize}
  \item Выбрав Феноменальный Атрибут или Трюк, герой сразу получает доступ к нескольким Феноменам.
  \item В описании Атрибутов Могущества указано, как обладатель этого Атрибута может получить дополнительные Феномены.
    \newline Каждый Атрибут имеет свой способ получения феноменов, что отражает выбранный исказителем путь постижения непостижимого.
  \item Некоторые предметы обладают Функциями, позволяющими творить Феномены.
\end{itemize}
\begin{tcolorbox}
  Без навыка Концентрация герою будет сложно эффективно применять многие Феномены, но большинство Снарядов не требуют этого навыка для применения.
\end{tcolorbox}
\paragraph{Феноменальная характеристика (Фх).} Основная Характеристика, с помощью которой исказитель активирует феномены. Она указана в описании Атрибута, Трюка или Предмета, который он использует для активации. Модификатор Феноменальной характеристики (МФх) используется в большинстве формул, описывающих феномены.
\paragraph{Усиление Феномена.} При активации феномена исказитель часто может потратить дополнительную Эн и усилить эффект. Возможные для феноменов Усиления перечислены в их описании. Герой вправе применять одно и то же Усиление несколько раз.
\paragraph{Стоимость} Определяет количество Энергии, которое исказитель тратит на активацию.
\paragraph{Время активации} Определяет время, необходимое для активации. Если в описании не указано Время активации, то она расходует Действие.
\paragraph{Маневры и феномены.} Если активация феномена расходует Действие или Быстрое действие, исказитель может активировать его, как маневр Атака, Дистанционная Атака, Комбинированная атака, Беглый огонь или Концентрированный огонь с применением этого феномена. 
\paragraph{Длительность} Феномена определяет время, в течение которого действуют эффекты феномена. Если в описании не указана Длительность, то эффекты применяются мгновенно.
\paragraph{Поддержание.} По окончании Длительности герой может потратить количество Эн, равное Стоимости Поддержания, продлив действие феномена на указанный срок.
\newline Одновременно исказитель способен поддерживать \textbf{|МФх|} феноменов.
\paragraph{Размер имеет значение (РИЗ).} Стоимость феномена возрастает на \textbf{|2*МРз|} цели, если присутствует пометка (РИЗ). Поддержание феномена возрастает на \textbf{|МРз|} цели. Исполинские цели не могут подвергнуться эффектам такого феномена. 
\paragraph{Прерывание активации.} Если в процессе активации или поддержания феномена исказитель одномоментно теряет ЕЗ, числом превышающие его Вл, активированные феномены немедленно прерываются, а активация феномена автоматически считается проваленной, однако исказитель не возвращает Энергию, затраченную на попытку активации феномена. В случае с феноменами, активируемыми мгновенно, это возможно, если противник предварительно выбрал маневр «Выжидание». 
\newline Когда герой стремится помешать активатору феномена, не нанося тому Пв, он должен пройти проверку Навыка, логически применимого к ситуации (Общение — для едких шпилек и отвлекающих восклицаний, Ловкость рук — для отрезвляющих оплеух и т. д.) против \textbf{|20 + Вл|} исказителя.
\begin{tcolorbox}
  Если вы хотите добавить в подобные ситуации остроты, Выжидающий и активатор феномена должны Состязаться в Рц. Победитель действует первым. 
\end{tcolorbox}
\paragraph{Сопротивление.} Сложность проверки Концентрации(МФх) исказителя для успешной актвивации Феномена.
\newline Если в описании Сопротивления указана Характеристика или Навык, то вместо статичного значения можно совершить проверку по правилам Состязания, где бонус к Проверке равен \textbf{|Сопротивление - 10|}.
\newline Если в описании не указано Сопротивление, эффекты не дают возможности противостоять им.
\begin{tcolorbox}
  Чтобы Сцены с использованием феноменов не затягивались, авторский коллектив рекомендует использовать Эффективные показатели Навыков и Характеристик для Сопротивления – по крайней мере, когда речь не идет о Персонах и героях.
\end{tcolorbox}

\paragraph{Проверка Наведения.} Требуется, если активатор феномена не видит цель. Наведение является проверкой Концентрации(Фх). Если Сопротивление Наведению не указано в описании способности, сложность Наведения равна \textbf{|15|}. При проверке герой может получить штрафы и бонусы, указанные ниже.
\newline \textbf{Бонусы:}
\begin{itemize}
  \item[--] Герой касался цели = +1.
  \item[--] Герой знает дополнительную информацию о цели, кроме ее описания = +1.
  \item[--] Герой хорошо разглядел цель и запомнил, как она выглядит = +1.
  \item[--] Герой хорошо представляет место, где цель скрывается от Наведения = +1.
  \item[--] Цель не подозревает о том, что на нее производится Наведение = +1.
  \item[--] Герой знает приблизительное направление на цель = +1.
  \item[--] Герой знает приблизительное расстояние до цели = +1.
  \item[--] Герой знает точное положение цели (включает бонусы от знания приблизительного направления и расстояния до цели) = +3.
  \item[--] На цель наложен следящий Феномен = +5(включает бонус от знания точного положения цели) и +2 за каждый дополнительный следящий феномен.
\end{itemize}
\textbf{Штрафы:}
\begin{itemize}
  \item[--] Герой никогда не видел цель, и вынужден довольствоваться абстрактным описанием = -5.
  \item[--] Герой никогда не видел цель, но видел ее точное изображение = -1.
  \item[--] Герой не знает ни расстояния до цели, ни направления на цель = -1.
  \item[--] Герой плохо запомнил цель визуально = -1.
  \item[--] Герой не имеет представления об окружении цели = -1.
  \item[--] Цель подозревает о том, что на нее Наводят феномен = -1.
  \item[--] Цель скрывается с помощью Феноменов = -1 за каждый скрывающий Феномен.
  \item[--] В предыдущий Круг цель совершала Перемещение или изменила свое местоположение благодаря другими существам или силам = -1.
\end{itemize}

\paragraph{Форма Феномена.} Определяет свойства и внешние проявления феномена. Полное описание феноменов сгруппировано по их формам.

\genAndGet{powerForms}{powerForms}{}

\section{Алфавитный перечень Феноменов}
\printindex[powers]

\chapter{Бестиарий}
В этом разделе описаны существа, которые вероятнее всего повстречаются героям во время приключений.
\section{Карточка Существа}
У каждого существа есть набор свойств, которые описаны в карточке существа.
\paragraph{Название} существа содержит опциональные метки, которые определяют, распространяются ли некоторые эффекты на это существо.
\paragraph{[Не живое]} существо - это не только нежить, но и големы, механизмы и оживленные магией объекты. На них не действуют некоторые эффекты и феномены. В описании эффекта должно быть явно указано, что он не действует на Не живых существ.
\paragraph{[Легендарное]} существа исключительны во всем. Это не просто сильные существа, их можно пересчитать по пальцам руки во всем мире. Не редко эти существа являются именованными Персонами, даже если это неразумное существо. Герои вряд ли встретят двух одинаковых Легендарных существ одновременно.
\paragraph{Описание} существа говорит о том, как оно выглядит, какие имеет повадки и где возможно оно обитает. Но это не игромеханическое, а, скорее, литературное описание.
\paragraph{Характеристи.} Существа имеют те же Основные и Второстепенные характеристики, что и герои. У существа, особенно у Легендарного, даже могут быть Нити на момент первого появления в игре. Если существо использует Феномены, то в этой секции указано название характеристики, которая у него является Феноменальной.
\paragraph{Атаки}, которыми существо может наносить повреждения, указаны со всеми бонусами и навыками, которыми владеет существо. Если у существа нет Атак, это означает, что его попытки напасть прямым образом обречены на провал и существо использует другие методы, чтобы победить своих врагов.
\newline Если в названии есть звездочка(*), то атака имеет дополнительное свойство, которое можно узнать в описании соответствующего оружия в разделе Богатство и снаряжение или в оружиях существ в бестиарии.
\newline Если в БПв есть звездочка(*), то существо не умеет пользоваться этой атакой и совершает проверку Доблести или Меткости с Помехой.

\paragraph{Боевые навыки} существа такие же, как и у героев: Владение оружием, Стрельба и Рукопашный бой. В скобках указано значение, на которое этот навык развит. Если боевого навыка нет в списке, то у существа он не прокачан.
\paragraph{Навыки}, которые есть у существа. Они выбираются из списка Основных и Экспетных навыков. В скобках указано значение, на которое этот навык развит. Если навыка нет в списке, то у существа он не прокачан.
\paragraph{Феномены}, которые доступны Существу. В скобках указана стоимость феномена, а полное описание можно узнать в описании феномена в раздели Феномены и Исказители или в феноменах существ в бестиарии.
\paragraph{Недостатки} существа, которые ограничивают его поведение и которыми могут воспользоваться герои. В карточке существа дано полное описание всех его Недостатков.
\paragraph{Трюки} существа, с помощью которых оно становится сильнее. В карточке существа дано полное описание всех его Трюков.
\paragraph{Функции} существа, которые могут за Энергию дать ему какое-либо преимущество. В карточке существа дано полное описание всех его Функций и их стоимость.
\paragraph{Ходы} существа, которыми Судьба может вмешиваться в происходящие события. В карточке существа дано полное описание всех его Ходов, а в скобках указана и их стоимость в Нитях. Помните, что герои тоже могут активировать Ходы союзных существ, если потратят свои Нити.
\paragraph{Классификация существ} является условным разделением существ на группы. Это разделение существует для того, чтобы проще было подобрать существо из Бестиария под возникшую ситуацию.

\paragraph{Шаблоны}
Шаблоны существ - это загатовки, которые можно добавлять к готовым существам, чтобы дать им какую-то специализацию. Обычно это профессия или поведенческий архетип.
\newline Шаблоны включают в себя изменение Характеристик и дополнительные Атаки, Навыки, Феномены, Недостатки, Трюки, Функции и Ходы, которое существо получит вместе с Шаблоном.
\paragraph{Характеристики и Навыки(в том числе и боевые)}, указанные в Шаблоне прямо складываются с характеристиками существа, на которое этот шаблон накладывается
\paragraph{Атаки} Шаблона добавляются к существующим атакам Существа, но нужно перерасчитать БПв всех атак в соответствии с измененными Характеристиками и Навыками. БПв Атак Шаблона указан без учета Характеристик и Навыков.
\paragraph{Феномены, Недостатки, Трюки, Функции и Ходы} шаблона добавляются к исходному существу и если они вступают в противоречие с Феноменами, Недостатками, Трюками, Функциями и Ходами существа, то следует дать предпочтение Шаблону.

\section{Атаки и Феномены существ}
Некоторые атаки и феномены являются особенностями Существ и недоступны героям. Описания этих атак находятся здесь.
\section{Атаки}
\genAndGet{weapons}{weapons}{Существа}
\section{Феномены}
\genAndGet{powers}{powers-monsters}{}

\section{Существа}
\subsection{Бродяги}
Эти разумные существа склонны к странствиям. Они селятся в любых местах, которые находят для себя хоть сколь-нибудь пригодыми для жизни и их представителей можно встретить во всех уголках мира.
\genAndGet{monsters}{monsters}{Бродяга}

\subsection{Аборигены}
Аборигены не любят покидать места в которых живут, поэтому практически невозможно их встетить за пределами их территорий.
\genAndGet{monsters}{monsters}{Абориген}

\subsection{Отшельники}
Затворники, которые не любят не только путешествия, но и других разумных существ. Они изредка появляются в деревнях рядом со своим убежищем, но чаще - живут в уединении в глубинах гор или лесов.
\genAndGet{monsters}{monsters}{Отшельник}

\subsection{Чужаки}
Эти разумные существа никогда не были частью мира - их пригласили или принудили появиться здесь. Появление Чужака в любом месте мира это выдающиеся событие, притягивающие любопытных.
\genAndGet{monsters}{monsters}{Чужак}

\subsection{Мутанты}
Эти существа значительно отличаются от Бродяг, Аборигенов и даже от известных Отшельников. Их разум разительно отличается от того, что принято считать нормой и их считали бы Чужаками, если бы не было достоверно известно, что Мутанты были рождены в этом мире.
\genAndGet{monsters}{monsters}{Мутант}

\subsection{Животные}
Хищная и травоядная фауна, которую можно повсеместно встретить в дикой природе и изредка - в цивилизованных областях.
\genAndGet{monsters}{monsters}{Животное}

\subsection{Чудища}
Выдающихся размеров звери, огромные хищные растения, мифические существа и жуткие, но неразумные монстры, пришедшие из других миров или возникшие в следствие ужасающих экспериментов.
\genAndGet{monsters}{monsters}{Чудище}

\subsection{Нежить}
Мертвецы, возвращенные к жизни нечестивой силой.
\genAndGet{monsters}{monsters}{Нежить}

\subsection{Механизмы}
Ожившие предметы, сложные автоматоны и даже механические люди - эти существа никогда не были рождены, но тем не мение они живут своей неорганической жизнью.
\genAndGet{monsters}{monsters}{Механизм}

\subsection{Лиходеи[Шаблоны]}
Это бравые ребята с непростой профессией - деньги отнимать. Они не всегда это будут делать с помощью силы, ведь хитрость зачастую - более безопасный и верный для них метод.
\genAndGet{monsters}{monster-templates}{Лиходей}

\subsection{Простолюдины[Шаблоны]}
Простые и честные люди, делающие свое дело и далекие от интриг и прочих политических изысков
\genAndGet{monsters}{monster-templates}{Простолюдин}

\subsection{Военные[Шаблоны]}
Служивые люди. Регулярная армия, организованные дружинники, стража, тайная полиция. Их присутствия означает манифестацию местной власти, от губернатора до императора.
\genAndGet{monsters}{monster-templates}{Военный}

\subsection{Наемники[Шаблоны]}
Свободные люди, которые за удовлетворительную плату встанут на сторону того, из чьего кошеля идут деньги. Однако это не значит, что они будут верны, если им предложат сумму еще больше.
\genAndGet{monsters}{monster-templates}{Наемник}

\subsection{Знать[Шаблоны]}
Правители, богатеи и купцы, правящие массами.
\genAndGet{monsters}{monster-templates}{Знать}

\subsection{Отщепенцы[Шаблоны]}
Изгои, юродивые, социопаты. Они сторонятся цивилизации, или же были изгнаны из общества.
\genAndGet{monsters}{monster-templates}{Отщепенец}

%\section{Алфавитный перечень Существ}
\printindex[monsters]
\fi
%\chapter{Советы мастеру}
\paragraph{}
Правила игры устроены таким образом, чтобы помочь игрокам и мастеру совместно создавать интересную им историю. О ее жанре и настроении лучше договориться заранее. Происходящее на игре должно так или иначе развлекать всех собравшихся. Если какое-то правило или его отсутствие мешает рассказывать интересную вам и соигрокам историю, упраздните или придумайте его. Разумеется, такие важные действия героев, как сражение и общение, достаточно четко регламентированы, чтобы поддерживать динамику событий. В остальном формулировки правил оставляют большой простор для интерпретации. Не стоит сбрасывать со счетов и субъективный взгляд мастера и игроков. Кому-то покажется вполне уместным Плут-купец, по большей части честный... и эпизодически промышляющий контрабандой, а кому-то будет ближе классический образ Плута в настольных ролевых играх — проныры-полурослика, срезающего кошельки на городском рынке. Одному мастеру будет достаточно Честной физиономии героя даже для оправдания в суде, другой же ограничится ситуационным бонусом к проверкам.
\paragraph{}
Ну и, конечно, \textbf{самое важное правило в настольной ролевой игре — Мастер Всегда Прав}. Соблюдение этого правила необходимо для сохранения динамики игры (которой очень мешают споры любого рода) и поддержания единого воображаемого пространства, границы которого задаются именно мастером. Все споры по правилам и логике происходящего лучше отложить до окончания игровой встречи.
\section{Подготовка к игре}
\paragraph{}
Игровой сюжет во многом похож на литературный — не считая того, что зачастую ни мастер, ни игроки не знают, как он будет развиваться. Тем не менее, существует несколько простых вопросов, обсуждение которых до начала игры значительно улучшит полученный вами опыт.
\begin{enumerate}
\item \textbf{Что происходит вокруг героев?}
\newline
Ответ — экспозиция вашей истории. С его помощью вы устанавливаете значимые факты игрового мира и формируете общее воображаемое пространство. Также на этом этапе стоит договориться о том, в каких декорациях будет разворачиваться сюжет, и какую жанровую окраску он будет иметь. Обсудить этот вопрос можно как до создания героев, так и после него. В первом случае подразумевается, что игроки создадут героев под ситуацию, во втором — заранее созданные герои во многом станут основой для нее!
\item \textbf{Что случилось с героями?}
\newline
Ответ — завязка вашей истории, событие, побуждающее героев к действию.
\item \textbf{Что заставляет героев работать сообща?}
\newline
Ответ позволит вам узнать, в каких отношениях находятся герои, почему помогают друг другу и работают в команде. Возможно, герои — сослуживцы, старые знакомые, подельники или даже родственники… но не исключено, что вцепиться друг другу в глотки им мешает лишь внешняя угроза или несметные богатства, добыть которые в одиночку им не по силам! Не забудьте обсудить возможность противостояния героев друг другу — хотя на нем построено немало сюжетов, далеко не все игровые группы готовы к этому!
\item \textbf{Какова общая цель героев?}
\newline
Иногда ответ на этот вопрос следует из ответа на предыдущий. Разумеется, у героев могут быть и свои собственные цели, не известные их товарищам. Если герои не связаны какими-то общими устремлениями и часто действуют в одиночку, а в центре повествования только их личные цели, история рискует сильно потерять в динамике. 
\end{enumerate}
\paragraph{}
Потратив немного времени на обсуждение, вы сформируете прочную основу для игрового сюжета, исключите неприятные
неожиданности, а также во многом зададите жанр и настроение
истории.
\paragraph{Одна игра для всех:} убедитесь, что все собравшиеся за столом (включая вас) хотят играть в одну и ту же игру. Если один из игроков пришел убивать орков, другой собирается продать оркам груду ржавых мечей, украденных из арсенала феода, а третий — уговорить орков показать затерянный в лесу золотоносный ручей, вам будет довольно сложно угодить всем. Особенно, если лесные орки — лишь незначительная деталь экспозиции вашего сюжета.
\newline
Во избежание этого заранее обсудите с игроками их ожидания
от игры и ее жанр. Не забудьте сообщить игрокам, во что планируете
поиграть лично вы, ведь если в попытке угодить игрокам вы утратите интерес к происходящему, хорошей игры совершенно точно не получится. Идеальный, хоть и не всегда возможный вариант — так называемая «нулевая встреча», во время которой участники игры синхронизируют ожидания. На такой встрече вы сможете поучаствовать в создании героев и включить в сюжетный замысел элементы, которые интересны игрокам.
\newline
Не играйте против игроков — играйте вместе с ними. Это
не значит, что вы должны щадить героев или оставлять без
последствий сделанные ими глупости. Однако зачастую смерть
героев не завершает историю, а обрывает ее на самом интересном
месте. Позаботьтесь о том, чтобы у героев оставались пути для
отступления.
\paragraph{Мастер всегда прав:} это правило — вовсе не оправдание всякого рода тирании. Главная задача мастера — организация общего воображаемого пространства, в котором все явления подчиняются одним и тем же правилам. Конечно, немалую долю этого труда принимает на себя игровая система. Тем не менее, некоторые моменты не регламентированы правилами и отдаются всецело на откуп мастеру, его логике и видению мира и его договоренностям с игроками. К таковым, например, относятся Впечатления статистов (даже в пределах списка Социальных взаимодействий они могут быть очень разными), трактовка спорных моментов правил (например, сохраняет ли доступ к Трюкам и Атрибутам герой, превращенный в бурого медведя, попугая или дракона), а также определение широты возможностей некоторых Атрибутов. Поскольку перед игрой просто невозможно обсудить все, в случае неразрешимого спорного момента мнение мастера имеет больший вес, чем мнение игрока.
\paragraph{Герои:} основа запоминающегося сюжета — герои, не похожие друг на друга. Позаботьтесь о том, чтобы игроки выбрали разные Атрибуты, Трюки, Недостатки и Навыки или хотя бы скомбинировали их отлично друг от друга. Предупредите игроков об Атрибутах и Недостатках, заведомо не подходящих для сюжета. Вряд ли Сплетик, Проповедник или Торговец смогут проявить себя, барахтаясь в бесконечном болоте, населенном гигантскими жабами. Хотя, если героям повстречается племя людей-ящеров или диких эльфов, эти атрибуты заиграют новыми красками!
\paragraph{Оптимизация:} некоторые игроки могут намеренно создать воина с Атрибутами, повышающими Доблесть и Меткость и максимальным Навыком «Владение оружием», или обаятельную соблазнительную Красавицу с максимальным Навыком «Общение». На первый взгляд, такие герои слишком могущественны и легко добиваются своего. Однако лишь на первый. Да, на своем поле они сильны, но… Не боритесь — используйте! Позвольте игроку получить то, ради чего он создал такого героя. Герою обязательно понадобится поддержка друзей в ситуациях, в которых он не силен. Однако не переборщите, создавая такие ситуации, — игрок не должен ощущать бесполезность своего героя и уж тем более не должен чувствовать, что на его героя ополчился весь мир.
\paragraph{Игра роли:} игра роли не имеет никакого отношения к актерской игре, хотя игрок, обладающий актерскими дарованиями, безусловно, придаст истории колорит. Игра роли — это выбор игроком линии поведения, сообразной Характеристикам, Атрибутам и Недостаткам героя, по сути, сознательное, добровольное и разумное самоограничение. Например, Кровожадный Неистовый Дикарь с Интеллектом 6, обстоятельно предлагающий соратникам заманить противника в болото с помощью хитроумного отвлекающего маневра, выглядит довольно странно. Даже если идея сработала, и герои достигли успеха, игрок сыграл самого себя… в теле Дикаря.
Но, если то же самое предложит Осторожный Охотник за головами с Интеллектом 14 и Очками опыта в Военном деле , это будет, безусловно, в рамках роли. Все вышесказанное не означает, что герои — заложники своего образа и не могут отступать от него. Большая часть произведений литературы и кинематографа показывают героя и его характер в развитии. В длительных игровых кампаниях герой не просто может, но обязан меняться! Однако эти изменения должны происходить не на пустом месте, а логически вытекать из пережитого героем. Это и есть игра роли в настольной ролевой игре.
\paragraph{}
С другой стороны, не стоит забывать — игромеханический блок служит надежной гарантией того, что чрезмерно экстравагантные идеи игрока будут реализованы совсем не так, как планировалось. Идея игрока заманить врагов в болото будет транслироваться в игру через Дикаря с Интеллектом 6. Это значит, что шансов на успех у задумки не так много, даже если Дикарь потратил Очки опыта на Военное дело . Вне зависимости от того, провалит ли Дикарь проверку, прибегнет к успеху с Неприятностями или получит поддержку Судьбы, выкупив успех за Нити, история получит интересное развитие и обрастет деталями. Иными словами, если сам игрок не желает пропускать свои идеи через призму образа героя, вам не стоит делать это за него.

\section{Построение сюжета}
\paragraph{Во что поиграть:} правила "Нитей Судьбы> ориентированы на жанр приключенческого боевика. В таких историях герои попадают (случайно или намеренно) в круговорот событий, вершащих судьбы мира и с честью выходят из всех испытаний. Характерными представителями этого жанра являются, например, серия А. Сапковского о ведьмаке, книги А. Дюма о приключениях мушкетеров или кинотрилогия по роману Дж. Р. Р. Толкина "Властелин Колец". Сражения, погони, путешествия, насыщенность действием - важнейшие составляющие сюжета. Однако система позволит вам рассказывать абсолютно любую историю, в которой герои принимают решения и совершают действия, последствия которых важны и способны изменить что-то, хотя бы в их собственной жизни. Решения, действия и последствия - самая важная часть ролевой игры. Не забывайте - герои избраны Судьбой. Позвольте игрокам ощутить это!
\paragraph{}
Вам не потребуется подробный план сюжета. Более того, он может навредить, так как передача повествовательных прав предполагает активное вмешательство игроков в сюжетную канву. Подготовьте завязку, ключевых статистов и персон, задействованных в сюжете, и позвольте событиям развиваться непредсказуемо для всех. Сюжет имеет определенные законы построения. Он состоит из экспозиции, завязки, развития, кульминации и развязки (к которой может примыкать эпилог). Рассмотрим историю, герои которой - жители небольшой деревеньки на границе леса, населенного гоблинами.
\paragraph{Экспозиция:} эту часть лучше вынести за пределы игровой встречи. Она позволяет игрокам задать все интересующие вопросы. Например, хорошо ли укреплена деревня, далеко ли замок сеньора, в каких отношениях селяне с племенем гоблинов. Постарайтесь дать как можно более исчерпывающую информацию на этом этапе, иначе игрокам придется задавать вопросы в дальнейшем (что повредит динамике) или выстраивать действия героев исходя из своих представлений, которые могут (и наверняка будут) отличаться от ваших.
\paragraph{Завязка:} событие, побуждающее героев к действию. Например, гоблины требуют у селян десятерых девушек для жертвоприношения на восходе луны и угрожают сжечь деревню, если селяне откажут. Завязка - отправная точка вашей истории, свершившийся факт.
\paragraph{Развитие:} именно экспозиция и завязка определяют дальнейшие действия героев - отправятся ли они в замок сеньора за гвардией феода, организуют оборону деревни, попробуют обмануть гоблинов, договорятся с ними или отдадут им желаемое. Эта часть сама по себе также дробится на мини-сюжеты - сцены и вехи, имеющие ту же структуру, что и основной сюжет.
\paragraph{Кульминация:} момент наивысшего напряжения в сюжете. Каким будет этот момент, зависит от результатов действий героев. Вот несколько вариантов кульминации этой истории:
\begin{itemize}
\item[--] Утомленные герои с замиранием сердца наблюдают, как горстка гвардейцев противостоит орде разъяренных гоблинов.
\item[--] Один из героев вызывает на бой лучшего воина гоблинского племени и сражается с ним.
\item[--] Переодевшись в женские платья, герои позволяют гоблинам отвести их на капище и нападают на верховного жреца.
\item[--] Герои пытаются убедить гоблинов, что мясо коров и овец понравится гоблинскому идолу гораздо больше человеческого.
\item[--] Герои решают, что лучше пожертвовать частью, чем погибнуть всем, и отдают девушек гоблинам.
\end{itemize}
\paragraph{Развязка:} финал истории, вытекающий из кульминации. Преуспели герои или потерпели неудачу? Остались ли они живы? Что потеряли и приобрели? Ответы на эти вопросы игроки получают в развязке. Вот как может завершиться история с гоблинами:
\begin{itemize}
\item[--] Селяне чествуют победоносных гвардейцев и выдают одну из девушек замуж за командира отряда. Об усилиях героев никто не вспоминает.
\item[--] Лучший воин племени повержен, и гоблины в страхе отступают.
\item[--] Герои гибнут на капище, но племя гоблинов лишается верховного жреца и теряет многих воинов. На какое-то время деревня спасена.
\item[--] Гоблины забирают всех коров и овец, что были в деревне. Кое- кто из селян считает, что это слишком большая цена за десяток девиц.
\item[--] Герои слышат отчаянные крики, доносящиеся из леса, и пытаются убедить себя, что это просто ветер.
\end{itemize}
\paragraph{Эпилог:} если ваша история рассчитана на одну-две встречи, в ней может и не быть эпилога. Но если вы планируете длительную историю, эпилог необходим - он содержит элементы экспозиции и завязки для следующего приключения!
\section{Во время игры}
\paragraph{Действие и еще раз действие:} если цели героев слишком глобальны, динамика событий может серьезно пострадать. Время от времени игроки чересчур уходят в построение громоздких планов вместо того, чтобы играть. Поэтому рекомендуется дробить глобальные цели на небольшие, промежуточные. Например, цель "завоевать соседнее королевство" сразу же распадается на "собрать информацию об армии противника", "найти средства для снаряжения армии", "снарядить войска". Обычно игроки сами неплохо справляются с поиском промежуточных целей. Но, если вы чувствуете, что игра превращается в обсуждение игры, не стесняйтесь прервать это событием, побуждающим героев к немедленным действиям. Черпайте вдохновение из идей игроков. Пускай из соседнего королевства прибудет посольство, безумный алхимик попросит ссуду на грандиозный эксперимент по превращению свинца в золото, окончание усобицы баронов оставит без работы крупный отряд наемников. Помните, игра хороша, пока никто не скучает.
\paragraph{Личное время:} игроки создают героев для действия. Конечно, есть такие, кому просто приятно посиживать в уголке и наблюдать, как играют остальные, но это скорее исключение. Позаботьтесь о том, чтобы каждый из героев получил свою минутку славы хотя бы раз за игровую встречу.
\paragraph{Давайте разделимся:} иногда логика ситуации вынуждает героев разделиться. В подобных случаях учитывайте, что часть игроков на время выпадет из действия, и общая динамика истории понизится. Если вы не так давно выступаете в роли мастера, сюжетов, в которых разделение команды героев неизбежно и предполагается заранее, лучше избегать.
\paragraph{Вызов:} возможность потерпеть неудачу - один из основных двигателей игры. Виды деятельности героев, где шанс провала ничтожно мал или отсутствует вовсе, вряд ли способны занять внимание игроков надолго.
\newline
Чем больше Очков опыта получают герои по ходу игры, тем шире становятся их возможности. Возрастает и влияние героев на окружающий мир. Это необходимо учитывать при построении приключения. Охота на матерого волка может быть смертельно опасна для горстки скверно вооруженных крестьян, но рыцарь прикончит его одним ударом меча. Организуйте сюжет таким образом, чтобы игроки могли ощутить силу и значимость своих героев. Предложите героям важную дипломатическую миссию, управление феодом или битву с огромным драконом.
\newline
В то же время самые могучие герои нередко пасуют в областях, не покрытых их Атрибутами, Трюками и Навыками. Это может стать хорошей основой сюжета и побудить героев к действию. Великий военачальник принудил к повиновению десятки народов, но получится ли у него добиться послушания от своенравной дочери? Богач привык к заискиванию и лицемерному обожанию, но что если он попадет туда, где деньги не дороже грязи под ногами? Искусный интриган возвысился при помощи лжи и коварства, но чем он ответит на открытый и честный вызов
на дуэль?
\paragraph{Ходы Судьбы:} именно Нити Судьбы, а не запредельные Навыки и Характеристики делают героя героем. Внимательно изучите раздел книги, посвященный Нитям, и попросите игроков сделать то же самое. Своевременно оборванная Нить может полностью изменить игру. Не лишайте игроков возможности вмешаться в сюжет! Если игрок накопил четыре Нити и прикончил Главного Злодея выстрелом в глаз, он в своем праве. А уж если игрок воспользовался Ходом и влюбил злодея в своего героя, убедил злодея покаяться или объявил, что злодей - отец его героя... Нет, игрок вовсе не испортил задуманную вами эпическую битву, а подарил сюжету великолепный поворот. Цените это и будьте к этому готовы - знайте Ходы, в том числе Ходы Атрибутов и Грани, выбранные игроками. Несмотря на право вето, которое имеет мастер, не стоит пользоваться им слишком часто - абсолютное большинство идей игроков стоит того, чтобы включить их в сюжет. Если вы все же сомневаетесь, уделите немного времени обсуждению Хода - скорее всего, все собравшиеся так или иначе придут к соглашению.
\paragraph{Ходы без обрыва Нитей:} разумеется, герои, использующие Атрибуты без обрыва Нитей, вполне дееспособны. Не задавайте слишком высокие уровни проверок, ведь то, что сложно для героя с Атрибутом, для героя без Атрибута - на грани возможного.
\paragraph{Недостатки, Темные стороны и Грани:} если ваши игроки впервые знакомятся с "Нитями Судьбы", вам придется настроить их на нужный лад. Подбрасывайте достаточно заметные и очевидные возможности для Капризов Судьбы. Допустим, герой скрытно наблюдает за переговорами контрабандистов на городском рынке. Если герой Пьяница, пускай в соседней таверне продают три кружки эля по цене двух. Если герой Любвеобилен, обратите его внимание на миловидную цветочницу в торговом ряду напротив. Если герой Болтлив, сообщите игроку, что рядом обсуждают свежие новости. Предложите Плуту возможность быть узнанным кем-то из контрабандистов, Красавцу - назойливое внимание богатой матроны, а Аристократу - приставания попрошайки. Обсудите с игроком последствия. В самом скором времени игроки втянутся в сюжетостроение!
\paragraph{Неприятности:} вводите Неприятности в тех случаях, когда нельзя логически определить возможность того или иного происшествия, или если игрок спорит с вами о его вероятности. Прибегайте к Неприятностям, когда герои ведут себя неосмотрительно или просто неразумно. Если они ищут проблем - дайте их сполна... но и оставьте героям шанс выпутаться, если они приложат усилия! В отличие от Недостатков и Темной стороны последствия Неприятностей не обязаны быть известны заранее, хотя лучше дать игроку намек, чтобы он мог принять Неприятность и протянуть к герою Нить или, наоборот, избежать проблем, оборвав Нить.
\newline
Неприятности можно разделить на два типа: создающие события и лишающие событий. К создающим события относятся случайные встречи, угрожающие героям, задерживающие их или расходующие их ресурсы. Нападение головорезов в трущобах, визит приставучего болтливого родственника, умоляющий о помощи статист - из их числа. Лишающие событий Неприятности - трактирщик, который ничего не знает об убийстве, труп гонца, при котором не оказалось послания герцога, сундук, вместо сокровищ наполненный истлевшим тряпьем. При этом и те, и другие оказывают прямое влияние на развитие сюжета. Более того, абсолютно нормально, если герои с помощью Ходов Судьбы или применения других ресурсов извлекли из Неприятностей выгоду!
\paragraph{Менеджмент Нитей:} в начале игровой встречи у героев по две Нити. Этого достаточно для совершения одного-двух Ходов Судьбы. Но, если сюжет насыщен событиями, Нитей может не хватить. В таких случаях рекомендуется обновлять Нити по завершении важных сюжетных вех или перед их кульминацией. Сюжетная веха - это маленькая история в большой. Например, выход хоббитов из Дольна - начало такой истории, а ее кульминация - битва со стражем озера. За воротами Мории хоббитов ожидает начало новой вехи. Ее кульминация - сражение Гэндальфа и барлога.
\paragraph{Узы:} применение Уз существенно повышает нагрузку на мастера. Вам придется следить за тем, остается ли герой в рамках избранных игроком Уз, и судить об этом по собственному усмотрению. Это важно - ведь в противном случае герой просто получит дополнительную Нить, и его жизнь никак не осложнится. Также учитывайте, что герой с 2 Узами с самого начала игры может получить Критический успех на любую проверку.
\paragraph{Темные Нити:} отличный способ повысить ставки. Используйте Темные Нити как в открытом противостоянии, так и за кадром. В любом случае держите в голове - этот инструмент в первую очередь нужен для развития истории, а не для убийства героев.
\paragraph{Влияния на героев:} если кто-то из персон или статистов пытается убедить героя изменить линию поведения и преуспевает в проверке, предложите игроку начислить герою 1 Очко опыта. Если игрок принимает влияние (и опыт), это значит, что статист смог обольстить, обмануть или переубедить героя. Точно так же разрешаются попытки манипуляций между двумя героями. По взаимной договоренности с игроками вы можете использовать обычные проверки Воли и Чародейства без начисления опыта, чтобы манипулировать героями, однако один герой не может навязать другому свою волю с помощью Навыков или заклинаний, пока игрок на это не согласится!
\paragraph{Проверки:} каждый бросок кубика в игре - небольшое событие, ведущее к чему-то. Не заставляйте игроков бросать кубик, если успех или неудача проверки никак не повлияют на развитие истории. Избегайте проверок, убивающих героев сразу. Например, если герой карабкается по скале и проваливает проверку Атлетики, позвольте ему применить Успех с Неприятностями, чтобы уцепиться за выступ парой метров ниже, или же позвольте другому герою совершить проверку Атлетики, чтобы вовремя подхватить падающего. Разумеется, не стоит доводить этот принцип до абсурда - последствия принятых игроком решений очень важны, даже если это неудача или смерть.
\paragraph{Статисты:} вам не нужен подробный блок характеристик для каждого статиста, которого герои встречают в игре, особенно с учетом того, что вы не имеете полного контроля над развитием сюжета. Вместо этого определите для себя значимые свойства статиста. Например, дикарь, который должен по вашей задумке напасть на героев, обладает Доблестью 15 (с учетом Ловкости, Силы, Бонуса к Повреждениям за двуручный топор и Навыка "Владение оружием") и Выносливостью 18 (что дает ему 54 Единицы Здоровья). Вы можете просто придумать эти цифры. В случае необходимости вы легко вычислите другие параметры статиста. Например, двуручный топор имеет Бонус к Повреждениям +5, на остальную Доблесть остается в сумме 10. Значит либо дикарь весьма ловок и силен (что очень вероятно), либо он великолепно владеет топором, а значит, довольно умен. Но если герои сохранят дикарю жизнь, запишите или запомните эти цифры на случай появления дикаря в дальнейшем - его Характеристики стали фактом истории!
\paragraph{Сложность боевых сцен:} если сражения занимают значимое место в вашей истории, при построении боевых сцен обращайте внимание на следующее:
\begin{enumerate}
\item \textbf{Число противников.} Это особенно важно, если вы используете правило "Все на одного". В таком случае даже самые слабые статисты получают возможность атаковать с Преимуществом, если имеют перевес в численности.
\item \textbf{Способность статистов наносить Нокаут героям.} Исходите из того, что если статист наносит Нокаут героям, специализирующимся на боевых столкновениях, при броске 16 и более на кубике, то он не представляет серьезной опасности.
Исключение из этого - численное превосходство.
\item \textbf{Число атак противников.} Обычно даже самые ужасные монстры ограничены тремя атаками за Очередь (хотя благодаря подбору Атрибутов и Трюков это число может возрасти до 8). Но уже несколько статистов, вооруженных легким оружием, могут создать для героев проблемы, даже если в большинстве случаев не в состоянии наносить Нокаут.
\item \textbf{Способность героев наносить Нокаут статистам.} Чем она выше, тем больше шанс на скорую победу героев. Не забывайте, что способность наносить Нокаут не всегда связана с высокими значениями Доблести и Меткости и во многом зависит от Выносливости и Защиты статистов.
\end{enumerate}
\section{После игры}
Игровая встреча завершается раздачей Очков опыта. За что его выдавать - часть предыгровой договоренности. Среднее количество Очков опыта, которые приобретают герои за время одной игровой встречи, варьируется от 1 до 5. В это число не входят Очки опыта, которые герой получил, принимая влияние других героев или статистов.
\newline
Поскольку "Нити Судьбы" - ролевая игра, при начислении опыта вы можете (хоть и не обязаны) учитывать как успехи героев в решении внутриигровых проблем, так и игру роли в исполнении игроков.
\newline
Разумеется, вы можете выделить особо удачные идеи кого-то из игроков и начислить их героям больше опыта, но лучше этим не злоупотреблять. В конце концов, главная награда в настольной ролевой игре - сам процесс и общение с единомышленниками.

\end{document}
\endinput
%This is   a simple document\footnote{with a footnote}.
%discoveries in \underline{science} 
%\hypertarget{d6}{Определение инвариантного пространства}
%\hyperlink{d6}{смотреть здесь!}
