\section{Подготовка к игре}
\paragraph{}
Игровой сюжет во многом похож на литературный — не считая того, что зачастую ни мастер, ни игроки не знают, как он будет развиваться. Тем не менее, существует несколько простых вопросов, обсуждение которых до начала игры значительно улучшит полученный вами опыт.
\begin{enumerate}
\item \textbf{Что происходит вокруг героев?}
\newline
Ответ — экспозиция вашей истории. С его помощью вы устанавливаете значимые факты игрового мира и формируете общее воображаемое пространство. Также на этом этапе стоит договориться о том, в каких декорациях будет разворачиваться сюжет, и какую жанровую окраску он будет иметь. Обсудить этот вопрос можно как до создания героев, так и после него. В первом случае подразумевается, что игроки создадут героев под ситуацию, во втором — заранее созданные герои во многом станут основой для нее!
\item \textbf{Что случилось с героями?}
\newline
Ответ — завязка вашей истории, событие, побуждающее героев к действию.
\item \textbf{Что заставляет героев работать сообща?}
\newline
Ответ позволит вам узнать, в каких отношениях находятся герои, почему помогают друг другу и работают в команде. Возможно, герои — сослуживцы, старые знакомые, подельники или даже родственники… но не исключено, что вцепиться друг другу в глотки им мешает лишь внешняя угроза или несметные богатства, добыть которые в одиночку им не по силам! Не забудьте обсудить возможность противостояния героев друг другу — хотя на нем построено немало сюжетов, далеко не все игровые группы готовы к этому!
\item \textbf{Какова общая цель героев?}
\newline
Иногда ответ на этот вопрос следует из ответа на предыдущий. Разумеется, у героев могут быть и свои собственные цели, не известные их товарищам. Если герои не связаны какими-то общими устремлениями и часто действуют в одиночку, а в центре повествования только их личные цели, история рискует сильно потерять в динамике. 
\end{enumerate}
\paragraph{}
Потратив немного времени на обсуждение, вы сформируете прочную основу для игрового сюжета, исключите неприятные
неожиданности, а также во многом зададите жанр и настроение
истории.
\paragraph{Одна игра для всех:} убедитесь, что все собравшиеся за столом (включая вас) хотят играть в одну и ту же игру. Если один из игроков пришел убивать орков, другой собирается продать оркам груду ржавых мечей, украденных из арсенала феода, а третий — уговорить орков показать затерянный в лесу золотоносный ручей, вам будет довольно сложно угодить всем. Особенно, если лесные орки — лишь незначительная деталь экспозиции вашего сюжета.
\newline
Во избежание этого заранее обсудите с игроками их ожидания
от игры и ее жанр. Не забудьте сообщить игрокам, во что планируете
поиграть лично вы, ведь если в попытке угодить игрокам вы утратите интерес к происходящему, хорошей игры совершенно точно не получится. Идеальный, хоть и не всегда возможный вариант — так называемая «нулевая встреча», во время которой участники игры синхронизируют ожидания. На такой встрече вы сможете поучаствовать в создании героев и включить в сюжетный замысел элементы, которые интересны игрокам.
\newline
Не играйте против игроков — играйте вместе с ними. Это
не значит, что вы должны щадить героев или оставлять без
последствий сделанные ими глупости. Однако зачастую смерть
героев не завершает историю, а обрывает ее на самом интересном
месте. Позаботьтесь о том, чтобы у героев оставались пути для
отступления.
\paragraph{Мастер всегда прав:} это правило — вовсе не оправдание всякого рода тирании. Главная задача мастера — организация общего воображаемого пространства, в котором все явления подчиняются одним и тем же правилам. Конечно, немалую долю этого труда принимает на себя игровая система. Тем не менее, некоторые моменты не регламентированы правилами и отдаются всецело на откуп мастеру, его логике и видению мира и его договоренностям с игроками. К таковым, например, относятся Впечатления статистов (даже в пределах списка Социальных взаимодействий они могут быть очень разными), трактовка спорных моментов правил (например, сохраняет ли доступ к Трюкам и Атрибутам герой, превращенный в бурого медведя, попугая или дракона), а также определение широты возможностей некоторых Атрибутов. Поскольку перед игрой просто невозможно обсудить все, в случае неразрешимого спорного момента мнение мастера имеет больший вес, чем мнение игрока.
\paragraph{Герои:} основа запоминающегося сюжета — герои, не похожие друг на друга. Позаботьтесь о том, чтобы игроки выбрали разные Атрибуты, Трюки, Недостатки и Навыки или хотя бы скомбинировали их отлично друг от друга. Предупредите игроков об Атрибутах и Недостатках, заведомо не подходящих для сюжета. Вряд ли Сплетик, Проповедник или Торговец смогут проявить себя, барахтаясь в бесконечном болоте, населенном гигантскими жабами. Хотя, если героям повстречается племя людей-ящеров или диких эльфов, эти атрибуты заиграют новыми красками!
\paragraph{Оптимизация:} некоторые игроки могут намеренно создать воина с Атрибутами, повышающими Доблесть и Меткость и максимальным Навыком «Владение оружием», или обаятельную соблазнительную Красавицу с максимальным Навыком «Общение». На первый взгляд, такие герои слишком могущественны и легко добиваются своего. Однако лишь на первый. Да, на своем поле они сильны, но… Не боритесь — используйте! Позвольте игроку получить то, ради чего он создал такого героя. Герою обязательно понадобится поддержка друзей в ситуациях, в которых он не силен. Однако не переборщите, создавая такие ситуации, — игрок не должен ощущать бесполезность своего героя и уж тем более не должен чувствовать, что на его героя ополчился весь мир.
\paragraph{Игра роли:} игра роли не имеет никакого отношения к актерской игре, хотя игрок, обладающий актерскими дарованиями, безусловно, придаст истории колорит. Игра роли — это выбор игроком линии поведения, сообразной Характеристикам, Атрибутам и Недостаткам героя, по сути, сознательное, добровольное и разумное самоограничение. Например, Кровожадный Неистовый Дикарь с Интеллектом 6, обстоятельно предлагающий соратникам заманить противника в болото с помощью хитроумного отвлекающего маневра, выглядит довольно странно. Даже если идея сработала, и герои достигли успеха, игрок сыграл самого себя… в теле Дикаря.
Но, если то же самое предложит Осторожный Охотник за головами с Интеллектом 14 и Очками опыта в Военном деле , это будет, безусловно, в рамках роли. Все вышесказанное не означает, что герои — заложники своего образа и не могут отступать от него. Большая часть произведений литературы и кинематографа показывают героя и его характер в развитии. В длительных игровых кампаниях герой не просто может, но обязан меняться! Однако эти изменения должны происходить не на пустом месте, а логически вытекать из пережитого героем. Это и есть игра роли в настольной ролевой игре.
\paragraph{}
С другой стороны, не стоит забывать — игромеханический блок служит надежной гарантией того, что чрезмерно экстравагантные идеи игрока будут реализованы совсем не так, как планировалось. Идея игрока заманить врагов в болото будет транслироваться в игру через Дикаря с Интеллектом 6. Это значит, что шансов на успех у задумки не так много, даже если Дикарь потратил Очки опыта на Военное дело . Вне зависимости от того, провалит ли Дикарь проверку, прибегнет к успеху с Неприятностями или получит поддержку Судьбы, выкупив успех за Нити, история получит интересное развитие и обрастет деталями. Иными словами, если сам игрок не желает пропускать свои идеи через призму образа героя, вам не стоит делать это за него.
