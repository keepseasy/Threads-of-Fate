\section{Во время игры}
\paragraph{Действие и еще раз действие:} если цели героев слишком глобальны, динамика событий может серьезно пострадать. Время от времени игроки чересчур уходят в построение громоздких планов вместо того, чтобы играть. Поэтому рекомендуется дробить глобальные цели на небольшие, промежуточные. Например, цель "завоевать соседнее королевство" сразу же распадается на "собрать информацию об армии противника", "найти средства для снаряжения армии", "снарядить войска". Обычно игроки сами неплохо справляются с поиском промежуточных целей. Но, если вы чувствуете, что игра превращается в обсуждение игры, не стесняйтесь прервать это событием, побуждающим героев к немедленным действиям. Черпайте вдохновение из идей игроков. Пускай из соседнего королевства прибудет посольство, безумный алхимик попросит ссуду на грандиозный эксперимент по превращению свинца в золото, окончание усобицы баронов оставит без работы крупный отряд наемников. Помните, игра хороша, пока никто не скучает.
\paragraph{Личное время:} игроки создают героев для действия. Конечно, есть такие, кому просто приятно посиживать в уголке и наблюдать, как играют остальные, но это скорее исключение. Позаботьтесь о том, чтобы каждый из героев получил свою минутку славы хотя бы раз за игровую встречу.
\paragraph{Давайте разделимся:} иногда логика ситуации вынуждает героев разделиться. В подобных случаях учитывайте, что часть игроков на время выпадет из действия, и общая динамика истории понизится. Если вы не так давно выступаете в роли мастера, сюжетов, в которых разделение команды героев неизбежно и предполагается заранее, лучше избегать.
\paragraph{Вызов:} возможность потерпеть неудачу — один из основных двигателей игры. Виды деятельности героев, где шанс провала ничтожно мал или отсутствует вовсе, вряд ли способны занять внимание игроков надолго.
\newline
Чем больше Очков опыта получают герои по ходу игры, тем шире становятся их возможности. Возрастает и влияние героев на окружающий мир. Это необходимо учитывать при построении приключения. Охота на матерого волка может быть смертельно опасна для горстки скверно вооруженных крестьян, но рыцарь прикончит его одним ударом меча. Организуйте сюжет таким образом, чтобы игроки могли ощутить силу и значимость своих героев. Предложите героям важную дипломатическую миссию, управление феодом или битву с огромным драконом.
\newline
В то же время самые могучие герои нередко пасуют в областях, не покрытых их Атрибутами, Трюками и Навыками. Это может стать хорошей основой сюжета и побудить героев к действию. Великий военачальник принудил к повиновению десятки народов, но получится ли у него добиться послушания от своенравной дочери? Богач привык к заискиванию и лицемерному обожанию, но что если он попадет туда, где деньги не дороже грязи под ногами? Искусный интриган возвысился при помощи лжи и коварства, но чем он ответит на открытый и честный вызов
на дуэль?
\paragraph{Ходы Судьбы:} именно Нити Судьбы, а не запредельные Навыки и Характеристики делают героя героем. Внимательно изучите раздел книги, посвященный Нитям, и попросите игроков сделать то же самое. Своевременно оборванная Нить может полностью изменить игру. Не лишайте игроков возможности вмешаться в сюжет! Если игрок накопил четыре Нити и прикончил Главного Злодея выстрелом в глаз, он в своем праве. А уж если игрок воспользовался Ходом и влюбил злодея в своего героя, убедил злодея покаяться или объявил, что злодей — отец его героя... Нет, игрок вовсе не испортил задуманную вами эпическую битву, а подарил сюжету великолепный поворот. Цените это и будьте к этому готовы — знайте Ходы, в том числе Ходы Атрибутов и Грани, выбранные игроками. Несмотря на право вето, которое имеет мастер, не стоит пользоваться им слишком часто — абсолютное большинство идей игроков стоит того, чтобы включить их в сюжет. Если вы все же сомневаетесь, уделите немного времени обсуждению Хода — скорее всего, все собравшиеся так или иначе придут к соглашению.
\paragraph{Ходы без обрыва Нитей:} разумеется, герои, использующие Атрибуты без обрыва Нитей, вполне дееспособны. Не задавайте слишком высокие уровни проверок, ведь то, что сложно для героя с Атрибутом, для героя без Атрибута — на грани возможного.
\paragraph{Недостатки, Темные стороны и Грани:} если ваши игроки впервые знакомятся с <<Нитями Судьбы>>, вам придется настроить их на нужный лад. Подбрасывайте достаточно заметные и очевидные возможности для Капризов Судьбы. Допустим, герой скрытно наблюдает за переговорами контрабандистов на городском рынке. Если герой Пьяница, пускай в соседней таверне продают три кружки эля по цене двух. Если герой Любвеобилен, обратите его внимание на миловидную цветочницу в торговом ряду напротив. Если герой Болтлив, сообщите игроку, что рядом обсуждают свежие новости. Предложите Плуту возможность быть узнанным кем-то из контрабандистов, Красавцу — назойливое внимание богатой матроны, а Аристократу — приставания попрошайки. Обсудите с игроком последствия. В самом скором времени игроки втянутся в сюжетостроение!
\paragraph{Неприятности:} вводите Неприятности в тех случаях, когда нельзя логически определить возможность того или иного происшествия, или если игрок спорит с вами о его вероятности. Прибегайте к Неприятностям, когда герои ведут себя неосмотрительно или просто неразумно. Если они ищут проблем — дайте их сполна... но и оставьте героям шанс выпутаться, если они приложат усилия! В отличие от Недостатков и Темной стороны последствия Неприятностей не обязаны быть известны заранее, хотя лучше дать игроку намек, чтобы он мог принять Неприятность и протянуть к герою Нить или, наоборот, избежать проблем, оборвав Нить.
\newline
Неприятности можно разделить на два типа: создающие события и лишающие событий. К создающим события относятся случайные встречи, угрожающие героям, задерживающие их или расходующие их ресурсы. Нападение головорезов в трущобах, визит приставучего болтливого родственника, умоляющий о помощи статист — из их числа. Лишающие событий Неприятности — трактирщик, который ничего не знает об убийстве, труп гонца, при котором не оказалось послания герцога, сундук, вместо сокровищ наполненный истлевшим тряпьем. При этом и те, и другие оказывают прямое влияние на развитие сюжета. Более того, абсолютно нормально, если герои с помощью Ходов Судьбы или применения других ресурсов извлекли из Неприятностей выгоду!
\paragraph{Менеджмент Нитей:} в начале игровой встречи у героев по две Нити. Этого достаточно для совершения одного-двух Ходов Судьбы. Но, если сюжет насыщен событиями, Нитей может не хватить. В таких случаях рекомендуется обновлять Нити по завершении важных сюжетных вех или перед их кульминацией. Сюжетная веха — это маленькая история в большой. Например, выход хоббитов из Дольна — начало такой истории, а ее кульминация — битва со стражем озера. За воротами Мории хоббитов ожидает начало новой вехи. Ее кульминация — сражение Гэндальфа и барлога.
\paragraph{Узы:} применение Уз существенно повышает нагрузку на мастера. Вам придется следить за тем, остается ли герой в рамках избранных игроком Уз, и судить об этом по собственному усмотрению. Это важно — ведь в противном случае герой просто получит дополнительную Нить, и его жизнь никак не осложнится. Также учитывайте, что герой с 2 Узами с самого начала игры может получить Критический успех на любую проверку.
\paragraph{Темные Нити:} отличный способ повысить ставки. Используйте Темные Нити как в открытом противостоянии, так и за кадром. В любом случае держите в голове — этот инструмент в первую очередь нужен для развития истории, а не для убийства героев.
\paragraph{Влияния на героев:} если кто-то из персон или статистов пытается убедить героя изменить линию поведения и преуспевает в проверке, предложите игроку начислить герою 1 Очко опыта. Если игрок принимает влияние (и опыт), это значит, что статист смог обольстить, обмануть или переубедить героя. Точно так же разрешаются попытки манипуляций между двумя героями. По взаимной договоренности с игроками вы можете использовать обычные проверки Воли и Чародейства без начисления опыта, чтобы манипулировать героями, однако один герой не может навязать другому свою волю с помощью Навыков или заклинаний, пока игрок на это не согласится!
\paragraph{Проверки:} каждый бросок кубика в игре — небольшое событие, ведущее к чему-то. Не заставляйте игроков бросать кубик, если успех или неудача проверки никак не повлияют на развитие истории. Избегайте проверок, убивающих героев сразу. Например, если герой карабкается по скале и проваливает проверку Атлетики, позвольте ему применить Успех с Неприятностями, чтобы уцепиться за выступ парой метров ниже, или же позвольте другому герою совершить проверку Атлетики, чтобы вовремя подхватить падающего. Разумеется, не стоит доводить этот принцип до абсурда — последствия принятых игроком решений очень важны, даже если это неудача или смерть.
\paragraph{Статисты:} вам не нужен подробный блок характеристик для каждого статиста, которого герои встречают в игре, особенно с учетом того, что вы не имеете полного контроля над развитием сюжета. Вместо этого определите для себя значимые свойства статиста. Например, дикарь, который должен по вашей задумке напасть на героев, обладает Доблестью 15 (с учетом Ловкости, Силы, Бонуса к Повреждениям за двуручный топор и Навыка <<Владение оружием>>) и Выносливостью 18 (что дает ему 54 Единицы Здоровья). Вы можете просто придумать эти цифры. В случае необходимости вы легко вычислите другие параметры статиста. Например, двуручный топор имеет Бонус к Повреждениям +5, на остальную Доблесть остается в сумме 10. Значит либо дикарь весьма ловок и силен (что очень вероятно), либо он великолепно владеет топором, а значит, довольно умен. Но если герои сохранят дикарю жизнь, запишите или запомните эти цифры на случай появления дикаря в дальнейшем — его Характеристики стали фактом истории!
\paragraph{Сложность боевых сцен:} если сражения занимают значимое место в вашей истории, при построении боевых сцен обращайте внимание на следующее:
\begin{enumerate}
\item \textbf{Число противников.} Это особенно важно, если вы используете правило <<Все на одного>>. В таком случае даже самые слабые статисты получают возможность атаковать с Преимуществом, если имеют перевес в численности.
\item \textbf{Способность статистов наносить Опасные раны героям.} Исходите из того, что если статист наносит Опасную рану героям, специализирующимся на боевых столкновениях, при броске 16 и более на кубике, то он не представляет серьезной опасности.
Исключение из этого — численное превосходство.
\item \textbf{Число атак противников.} Обычно даже самые ужасные монстры ограничены тремя атаками за Очередь (хотя благодаря подбору Атрибутов и Трюков это число может возрасти до 8). Но уже несколько статистов, вооруженных легким оружием, могут создать для героев проблемы, даже если в большинстве случаев не в состоянии наносить Опасные раны.
\item \textbf{Способность героев наносить Опасные раны статистам.} Чем она выше, тем больше шанс на скорую победу героев. Не забывайте, что способность наносить Опасные раны не всегда связана с высокими значениями Доблести и Меткости и во многом зависит от Выносливости и Защиты статистов.
\end{enumerate}