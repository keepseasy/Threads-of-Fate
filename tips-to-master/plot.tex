\section{Построение сюжета}
\paragraph{Во что поиграть:} правила "Нитей Судьбы" ориентированы на жанр приключенческого боевика. В таких историях герои попадают (случайно или намеренно) в круговорот событий, вершащих судьбы мира и с честью выходят из всех испытаний. Характерными представителями этого жанра являются, например, серия А. Сапковского о ведьмаке, книги А. Дюма о приключениях мушкетеров или кинотрилогия по роману Дж. Р. Р. Толкина "Властелин Колец". Сражения, погони, путешествия, насыщенность действием — важнейшие составляющие сюжета. Однако система позволит вам рассказывать абсолютно любую историю, в которой герои принимают решения и совершают действия, последствия которых важны и способны изменить что-то, хотя бы в их собственной жизни. Решения, действия и последствия — самая важная часть ролевой игры. Не забывайте — герои избраны Судьбой. Позвольте игрокам ощутить это!
\paragraph{}
Вам не потребуется подробный план сюжета. Более того, он может навредить, так как передача повествовательных прав предполагает активное вмешательство игроков в сюжетную канву. Подготовьте завязку, ключевых статистов и персон, задействованных в сюжете, и позвольте событиям развиваться непредсказуемо для всех. Сюжет имеет определенные законы построения. Он состоит из экспозиции, завязки, развития, кульминации и развязки (к которой может примыкать эпилог). Рассмотрим историю, герои которой — жители небольшой деревеньки на границе леса, населенного гоблинами.
\paragraph{Экспозиция:} эту часть лучше вынести за пределы игровой встречи. Она позволяет игрокам задать все интересующие вопросы. Например, хорошо ли укреплена деревня, далеко ли замок сеньора, в каких отношениях селяне с племенем гоблинов. Постарайтесь дать как можно более исчерпывающую информацию на этом этапе, иначе игрокам придется задавать вопросы в дальнейшем (что повредит динамике) или выстраивать действия героев исходя из своих представлений, которые могут (и наверняка будут) отличаться от ваших.
\paragraph{Завязка:} событие, побуждающее героев к действию. Например, гоблины требуют у селян десятерых девушек для жертвоприношения на восходе луны и угрожают сжечь деревню, если селяне откажут. Завязка — отправная точка вашей истории, свершившийся факт.
\paragraph{Развитие:} именно экспозиция и завязка определяют дальнейшие действия героев — отправятся ли они в замок сеньора за гвардией феода, организуют оборону деревни, попробуют обмануть гоблинов, договорятся с ними или отдадут им желаемое. Эта часть сама по себе также дробится на мини-сюжеты — сцены и вехи, имеющие ту же структуру, что и основной сюжет.
\paragraph{Кульминация:} момент наивысшего напряжения в сюжете. Каким будет этот момент, зависит от результатов действий героев. Вот несколько вариантов кульминации этой истории:
\begin{itemize}
\item[--] Утомленные герои с замиранием сердца наблюдают, как горстка гвардейцев противостоит орде разъяренных гоблинов.
\item[--] Один из героев вызывает на бой лучшего воина гоблинского племени и сражается с ним.
\item[--] Переодевшись в женские платья, герои позволяют гоблинам отвести их на капище и нападают на верховного жреца.
\item[--] Герои пытаются убедить гоблинов, что мясо коров и овец понравится гоблинскому идолу гораздо больше человеческого.
\item[--] Герои решают, что лучше пожертвовать частью, чем погибнуть всем, и отдают девушек гоблинам.
\end{itemize}
\paragraph{Развязка:} финал истории, вытекающий из кульминации. Преуспели герои или потерпели неудачу? Остались ли они живы? Что потеряли и приобрели? Ответы на эти вопросы игроки получают в развязке. Вот как может завершиться история с гоблинами:
\begin{itemize}
\item[--] Селяне чествуют победоносных гвардейцев и выдают одну из девушек замуж за командира отряда. Об усилиях героев никто не вспоминает.
\item[--] Лучший воин племени повержен, и гоблины в страхе отступают.
\item[--] Герои гибнут на капище, но племя гоблинов лишается верховного жреца и теряет многих воинов. На какое-то время деревня спасена.
\item[--] Гоблины забирают всех коров и овец, что были в деревне. Кое- кто из селян считает, что это слишком большая цена за десяток девиц.
\item[--] Герои слышат отчаянные крики, доносящиеся из леса, и пытаются убедить себя, что это просто ветер.
\end{itemize}
\paragraph{Эпилог:} если ваша история рассчитана на одну-две встречи, в ней может и не быть эпилога. Но если вы планируете длительную историю, эпилог необходим — он содержит элементы экспозиции и завязки для следующего приключения!