\section{Путешествия}
Путешествия — один из основополагающих элементов многих жанров. Какая бы причина ни вынудила героев двинуться в путь, в дороге их подстерегает немало трудностей. Если путешествия являются важной частью вашей истории и вы желаете выяснить, насколько сложен будет путь, следуйте пунктам ниже:
\begin{enumerate}
\item \textbf{Сделал дело — гуляй смело.} Игроки должны решить, кто из героев или статистов в караване станет:
\begin{itemize}
\item \textbf{Навигатором}, который ищет безопасный путь, определяет подходящие места для стоянки и пополнения припасов. Навык «Выживание» — важнейший для Навигатора.
\item \textbf{Механиком}, который следит за состоянием машин и прочей техники в группе. Ему потребуется Эксплуатация (Ин, Мд).
\item \textbf{Погонщиком}, который следит за тем, чтобы вьючные животные не провалились в яму и не наелись ядовитой травы, а техника. Он использует Обращение с животными.
\item \textbf{Разведчиком}, который идет впереди каравана, отслеживая все подозрительное и разыскивая места, представляющие интерес. Ему понадобятся Наблюдательность.
\item \textbf{Проводнком}, который проведет караван мимо нежелательных встреч и внезапных препятствий. Ему понадобятся Скрытность и Наблюдательность. Эта роль не применяется, если отряд передвигается Маршем. 
\end{itemize}
\begin{tcolorbox}
Роли Механика и Погонщика являются опциональными. Если в отряде нет техники и въючных животных, Механик и Погонщик будут слоняться без дела и эти роли можно не назначать.
\end{tcolorbox}
Одну роль могут выполнять несколько героев (используйте правила Взаимопомощи), но один герой не может выполнять несколько ролей.
\newline
Когда по тем или иным причинам роль остается невыполненной, последствия могут быть самыми плачевными.
\paragraph{Если Навигатора, Погонщика(при условии, что в отряде есть вьючные животные) или Техника(при условии, что в отряде есть техника)} нет в караване, считайте, что при соответствующих проверках выпало 5. Если разница между этим результатом и целевой сложностью проверки превышает 10, считайте проверку Критическим провалом.
\paragraph{Если в караване нет Разведчика,} герои могут стать жертвами случайного нападения. Они уязвимы для всех тех опасностей, которые легко предотвратить, заметив вовремя! Сцены Встреч и Находок начинаются сразу после того, как определены их тип и Скрытая Угроза. В Боевых сценах отряд действует, как будто подвергся внезапному нападению.
\paragraph{Если в караване нет Проводника,} Отряд не может избежать Встреч и Находок и обязан принять участие в сценах, с ними связанных. Даже если Разведчик сообщил о том, что сцена не сулит ничего хорошего.
\item \textbf{По дороге всегда быстрее.} Определите \textbf{Опасность местности(ОМ)}, по которой предстоит пройти героям – в пути их ждет немало сюрпризов! Если в пути герои преодолевают местность с разными уровнями Опасности, используйте наибольший.
\begin{tcolorbox}
\paragraph{Терра инкогнита} Если герои отправляются в путешествие без карты, Повысьте ОМ на 1(макс.5). Если герои являются пионерами и карт местности, которую они покоряют просто не существует, повысьте ОМ на 2(макс.5).
\end{tcolorbox}
\begin{center}
\begin{tabular}{|p{7cm}|p{7cm}|c|}
\hline
Тип местности & Глубина вод для водного транспорта & Опасность \\ \hline
Обжитые пригороды, фермерские угодья, торговые тракты, области, подробно и точно нанесенные на карты. & Открытый океан. & 0 \\ \hline
Прерии, равнины, области, не слишком подробно нанесенные на карты. & Архипелаг или прибрежная зона материков. & 1 \\ \hline
Лесистые и болотистые равнины, холмы. & Широкие реки с простым фарватером & 2 \\ \hline
Лесные дебри, топи, скалистые холмы, руины больших городов. & Широкие, но мелеющие реки & 3 \\ \hline
Горы и пустыни. & Узкие извилистые реки с непредсказуемым фарватером & 4 \\ \hline
Джунгли и заболоченная чаща. & Реки с быстрым течением. Острые камни и опасные пороги прилагаются. & 5 \\ \hline
\end{tabular}
\end{center}
\item \textbf{Долго ли, коротко ли…} Определите длительность пути в днях, ориентируясь на скорость самого медленного транспорта каравана. Длительность задает базовую Сложность пути. Прибавьте к ней Опасность местности. Получившееся число — финальная Сложность пути.
\begin{center}
\begin{tabular}{|c|c|}
\hline
Длительность & Сложность пути \\ \hline
1-15 дней & 10 + ОМ \\ \hline
16-40 дней & 15 + ОМ \\ \hline
41 день и больше & 20 + ОМ \\ \hline
\end{tabular}
\end{center}
\begin{tcolorbox}
Для простоты определения Длительности путешествия считайте, что рельеф местности не влияет на скорость путешествия.
\end{tcolorbox}
\textbf{Марш:} герои могут увеличить скорость передвижения вдвое. При этом они получают Помеху на проверки Наблюдательности и должны совершить проверку Вн или Атлетики (Вн) против 15. В случае провала герои измотаны и находятся в состоянии Усталости до тех пор, пока не отдохнут минимум 8 часов. Опасность местности на марше возрастает на 1. Если в караване есть транспортные средства, скакуны или вьючные животные, опасность местности возрастает на 2.
\newline \textbf{Тише едешь — дальше будешь:} герои могут ополовинить скорость передвижения и получить преимущество на проверки Скрытности Проводника для того, чтобы избежать любых Встреч и Находок.

\item \textbf{Да что вы, ребята, я сам здесь впервой!} Навигатор совершает проверку Выживания (Ин), Механик совершает проверку Эксплуатации (Ин, Мд), Погонщик совершает проверку Обращения с животными (Сл, Ин, Мд, Об) против финальной Сложности пути. Результат провкерки Путешествия равен сумме результатов этих проверок. Например, если Навигатор провалил проверку на 5, а Погонщик – на 8, суммарная величина провала составит 13. Если Навигатор преуспел на 7, а погонщик провалил проверку на 4, суммарная величина успеха составит 3.
\newline Если результат хотя бы одной проверки был Критическим Провалом, то вся проверка Путешествия считается Критическим Провалом.
\newline
Успех проверок означает, что путешествие проходит без осложнений и позволяет героям наслаждаться относительным комфортом:
\begin{center}
\begin{tabular}{|c|p{10cm}|}
\hline
Величина успеха & Эффект \\ \hline
1-5 & - \\ \hline
6-10 & Расход еды, воды и других ресурсов сокращается вдвое. \\ \hline
11-15 & Караван прибывает на 1 день раньше. \\ \hline
16-20 & Караван может избежать 1 Встречи или Находки или совершить 1 дополнительный бросок по таблице Встреч и Находок. Караван прибывает на 2 дня раньше. \\ \hline
21 и больше & Караван может избежать до 2 Встреч или Находок или совершить до 2 дополнительных бросков по таблице Встреч и Находок. Караван прибывает на 3 дня раньше. \\ \hline
Критический Успех & Преимущество на проверку Встреч и Находок. \\ \hline
\end{tabular}
\end{center}
Если проверка провалена, то путешественники получают Повреждения, а Опасность Встреч и Находок, изначально равная \textbf{нулю}, возрастает.
\newline \textbf{Цена провала.} При провале проверки, герои и транспорт получают повреждения, а Опасность Встреч и Находок возрастает. Сверьтесь с таблицей для того, чтобы определить последствия. Эти повреждения \textit{игнорируют} Прочность.
\newline \textbf{Загрязнение.} В некоторых ситуациях герои идут по местности, знаменитой своими токсичными испарениями, искажающими эманациями или дурманящей флорой. В этом случае потерянные ЕЗ при провале заменяются Интоксикацией.
\begin{center}
\begin{tabular}{|c|p{5cm}|p{5cm}|}
\hline
Величина провала & Потерянные героями ЕЗ & Опасность Встреч и Находок \\ \hline
1-5 & ОМ(мин 1) & 1 \\ \hline
6-10 & ОМ+2 & 2 \\ \hline
11-15 & ОМ+4 & 3 \\ \hline
16-20 & ОМ+7 & 4 \\ \hline
21 и больше & ОМ+10 & 5 \\ \hline
\end{tabular}
\end{center}
\item \textbf{Остановки в пути}. Приятные неожиданности редки в пути, зато других хоть отбавляй. Совершите проверку Неприятностей и определите число остановок, которые привели к Встречи или Находоке:
\begin{center}
\begin{tabular}{ |p{2.7cm}|p{12cm}| }
\hline
\textbf{Результат проверки Неприятностей} & \textbf{Остановок}
\\ \hline
19-20 & \textbf{0}+Опасность Встреч и Находок
\\ \hline
13-18 & \textbf{1}+Опасность Встреч и Находок
\\ \hline
7-12 & \textbf{2}+Опасность Встреч и Находок
\\ \hline
1-6 & \textbf{3}+Опасность Встреч и Находок
\\ \hline
\end{tabular}
\end{center}
\textbf{Сложность Встречи.} Чтобы определить целевую сложность проверок, не связанных с Общением, сложите \textbf{|10 + ОМ + Опасность Встреч и Находок|}.
\begin{tcolorbox}
Остановки не обязательно будут распределены по всему пути равномерно. Мастер волен решать в соответствии с логикой мира и повествования, на каком отрезке пути были совершены значимые Остановки. Возможно, начало похода было насыщено событиями или же все самое интересное проихошло только под конец путешествия.
\end{tcolorbox}
\item \textbf{Все, приехали.} Если результат проверки Путешествий стал Критическим Провалом, то это означает, что во время путешествия с караваном случилось что-то, что помешало им продолжить путь. Караван прошел половину пути или меньше и для того, чтобы продолжить движение, нужно начать путешествие заново с той точки, в которой вынужденно остановились герои.
\newline Количество Остановок уменьшается вдвое(округляя в большую сторону). Последняя Встреча или находка обязательно начинается с \textbf{Заварухи}, которая и привела к прерыванию путешествия.

\item \textbf{Встречи и Находки.} Совершив проверку Встреч и Находок, определите наполнение сцены. Обратите внимание, что проверка «Скрытой угрозы» все еще может серьезно изменить смысловое наполнение сцены.
\item \textbf{Скрытая угроза.} В начале сцены Встречи или Находки, совершите проверку Скрытой угрозы. Отнимите от результата проверки Опасность Встреч и Находок, определенную на 4 этапе. Скрытая угроза не обязательно проявится в начале сцены, но дает мастеру хорошее представление о том, чем она может закончится.
\item \textbf{А что это унас тут?} Для того, чтобы заранее заметить Встречу и не проморгать Находку, Разведчик отряда должен преуспеть в проверки Внимательности против Сложности Встречи.
\item \textbf{Я тут мимо проходил.} Если отряд желает избежать Встречи, Проводник отряда должен преуспеть в проверки Скрытности(Ин) против Сложности Встречи, чтобы провести союзников мимо, не привлекая внимания. В случае провала, сцена Встречи начинается и проверки Впечатления статистов на отряд совершаются с Помехой.
\newline
Если Отряд желает обойти Находку стороной, Проводник отряда должен преуспеть в проверки Наблюдательности(Мд) против Сложности Встречи, чтобы найти обходной путь и не попасть в возможную засаду.
\item \textbf{Поболтаем?} Если отряд не избежал встречи и проверки Впечатлений достаточно хороши и нет спешки, большинство статистов готовы общаться и торговать с героями. Проверки Впечатлений не принесут результатов лучше Доброжелательности, но действия героев — могут!
\item \textbf{Отдохнули и в путь}. После каждой Остановки идет Антракт, во время которого герои продвигаются по намеченному маршруту.
\begin{tcolorbox}
\textbf{Маршрут изменен.} Никто не знает заранее, что таит в себе очередная Встреча или Находка. После очередной Остановки герои могут решить пойти в другую сторону. В этом случае все ожидающие их Встречи и Находки так и останутся неразведанными и начинается новое Путешествие.
\end{tcolorbox}
\end{enumerate}
