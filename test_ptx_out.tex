%Last time of file creation:  1749754136.5852435


\section{Снаряд}Исказитель отправляет Снаряд в цель. Снаряд наносит Пв, как Дальнобойное оружие, а вместо МЛв в формуле Меткости используется МФх.
\begin{itemize}
\item Сопротивление к Снарядам не применяется.
\item Тип Повреждений и Бонус к Повреждениям указаны в описании.
\end{itemize}Если в описании феномена не указано обратного, то:
\begin{itemize}
\item Дистанции Снаряда составляют 20/40;
\item Критический Удар Снаряда равен 20;
\item Скорострельность Снаряда рана 1.
\end{itemize}
\subsection{Воздушный поток}
\index[powers]{Воздушный поток}
\paragraph{} 
\textit{Яростные порывы ветра сыплют пыль в глаза, рвут из рук оружие, пытаются опрокинуть навзничь.}
\paragraph{Стоимость: }1 Эн
\newline
\textbf{Тип Повреждений: }Дробящие
\newline
\textbf{Бонус к Повреждениям: }-1/-2
\paragraph{Эффект: }если цель получила Пв в голову или глаз и осталась в сознании, все Активные проверки против цели совершаются с Преимуществом до начала ее следующей Очереди. 
\newline Ветер может унести или отбросить небольшие незакрепленные предметы
\paragraph{Усиление:}
\begin{itemize}
\item+2 Эн -> +1 к БПв.
\item+1 Эн -> +10 к дистанциям.
\item+1 Эн -> цель может быть Разоружена, если получила Пв в руку*.
\item+1 Эн -> цель может быть Сбита с ног, если получила Пв в ногу*.
\end{itemize}
\paragraph{}*Маневр применяется в дополнение к наносимым цели Пв. Герой проверяет Меткость Снаряда вместо Дб.
\subsection{Излучение}
\index[powers]{Излучение}
\paragraph{} 
\textit{Чистая Энергия срывается с ладони исказителя и устремляется к цели.}
\paragraph{Стоимость: }1 Эн
\newline
\textbf{Тип Повреждений: }Нет
\newline
\textbf{Бонус к Повреждениям: }0/-1
\newline
\textbf{Скорострельность: }3
\paragraph{Эффект: }феномен не обладает типом Пв и КУ без Усиления.
\paragraph{Усиление:}
\begin{itemize}
\item+1 Эн -> +1 к скорострельности.
\item+1 Эн -> +10 к дистанциям.
\item+1 Эн -> феномен получает один (и только один) тип Пв по выбору героя.
\end{itemize}
\paragraph{}
\begin{tcolorbox} От феномена без типа Пв не спасет Сопротивление или Иммунитет. 
\end{tcolorbox}
\subsection{Огненный плевок}
\index[powers]{Огненный плевок}
\paragraph{} 
\textit{Исказитель поражает врага смачным плевком, который вспыхивает на лету. Этот феномен является врожденной способностью некоторых мутантов.}
\paragraph{Стоимость: }3 Эн
\newline
\textbf{Тип Повреждений: }Огненные
\newline
\textbf{Бонус к Повреждениям: }+3/+2
\newline
\textbf{Дистанции: }10/20.
\newline
\textbf{Критический Удар: }16+.
\paragraph{Эффект: }получив 5 и более Пв, цель Загорается.
\subsection{Отрыжка}
\index[powers]{Отрыжка}
\paragraph{} 
\textit{Это остаточные негативные эманации исказителя, разрушительный эмоциональный заряд, обретший физическую форму. Хотя встречаются твари, которые действительно обдают противника едким содержимым желудка.}
\paragraph{Стоимость: }3 Эн
\newline
\textbf{Тип Повреждений: }Едкие
\newline
\textbf{Бонус к Повреждениям: }+1/0
\newline
\textbf{Дистанции: }
\textbf{|2 + МРз|}/
\textbf{|5 + МРз|}.
\newline
\textbf{Критический Удар: }13+.
\paragraph{Эффект: }феномен игнорирует БЩ и БД цели (исключая БД герметичных доспехов).
\paragraph{Усиление:}
\begin{itemize}
\item+1 Эн -> +1 к БПв
\end{itemize}
\subsection{Разряд}
\index[powers]{Разряд}
\paragraph{} 
\textit{Электрический разряд взрывается веером змеистых молний.}
\paragraph{Стоимость: }2 Эн
\newline
\textbf{Тип Повреждений: }Электрические
\newline
\textbf{Бонус к Повреждениям: }+1/0
\paragraph{Эффект: }при подсчете Пв, полученных от действия феномена, БД и БЩ цели ополовиниваются. Цель Загорается, если потеряла 5 и более ЕЗ.
\paragraph{Усиление:}
\begin{itemize}
\item+1 Эн -> +1 к БПв.
\item+1 Эн -> +10 Дистанциям.
\item+1 Эн -> После проведения 
\textit{успешной} атаки, позволяет провести еще одну атаку Разрядом в другую цель не более чем в 5 метрах от предыдущей. Источником атаки является последняя цель, пораженная Разрядом в эту Очередь.
\end{itemize}
\section{Бомба}Исказитель метает Бомбу – сгусток своей Энергии, во врагов, и наслаждается эффектом. Чтобы поразить цель, герой проверяет Меткость по обычным правилам, но МФх и КонцентрацияЭ определяют, насколько мощной получится Бомба. Бомбу, как и гранату, можно метать в землю, или прямо существо. Эффекты Взрыва указаны в описании конкретных феноменов. Если Цель Бомбы получает КУ при ее активации, эффекты КУ распространяются на всех существ в Радиусе Взрыва.
\newline Бомба является Метательным оружием. Пока Бомба не брошена, она занимет руку, а Стоимость Поддержания Бомбы равна 0. 
\begin{itemize}
\item Если в описании не сказано иначе, то Сопротивление к Бомбам не применяется.
\end{itemize}Если в описании феномена не указано обратного, то:
\begin{itemize}
\item Радиус Взрыва Бомбы равен 
\textbf{|1+МФх|} (минимум 2);
\item Сила Взрыва Бомбы равна 
\textbf{|10+Концентрация(МФх)|};
\item Тип Повреждений Бомбы является Дробящим;
\item Бонус Повреждений Бомбы равен 0/-1;
\item Дистанции Бомбы составляют 5/20;
\item Критический Удар Бомбы равен 20;
\end{itemize}
\subsection{Дрожь земли}
\index[powers]{Дрожь земли}
\paragraph{} 
\textit{Исказитель сотрясает твердь и раскалывает ее, засыпая все вокруг обломками.}
\paragraph{Стоимость: }2 Эн
\newline
\textbf{Радиус Взрыва: }1
\paragraph{Эффект: }исказитель не вправе назначать целью объект, висящий в воздухе (но они могут попасть во Взрыв). Область Взрыва становится Трудным ландшафтом.
\paragraph{Усиление:}
\begin{itemize}
\item+1 Эн -> +1 к БПв и Силе Взрыва.
\item+1 Эн -> +5 к Дистанциям.
\item+1 Эн -> +1 к радиусу Взрыва.
\item+1 Эн -> попавшие во Взрыв и получившие Пв падают.
\end{itemize}
\subsection{Жгучая стрела}
\index[powers]{Жгучая стрела}
\paragraph{} 
\textit{
\tbd}
\paragraph{Стоимость: }2 Эн
\newline
\textbf{Радиус Взрыва: }1
\newline
\textbf{Сила Взрыва: }0
\newline
\textbf{Тип Повреждений: }Кислотные
\paragraph{Эффект: }если цель получает Пв, она (но не существа в области Взрыва) начинает Растворяться. 
\newline Все существа, попавшие в область Взрыва, получают 
\textbf{МФх} Пв. Если изначальная цель Жгучей стрелы получает КУ, все существа в области взрыва получают состояние Растворения в дополнение к Пв.
\paragraph{Усиление:}
\begin{itemize}
\item+1 Эн -> +1 к БПв и Пв от Взрыва
\item+1 Эн -> +1 к радиусу Взрыва.
\end{itemize}
\subsection{Колючий град}
\index[powers]{Колючий град}
\paragraph{} 
\textit{Исказитель создает температурную аномалию и вихрь острого, как бритва, града.}
\paragraph{Стоимость: }4 Эн
\newline
\textbf{Сила Взрыва: }5 + Концентрация(МФх)
\newline
\textbf{Тип Повреждений: }Колющие, Ледяные.
\paragraph{Эффект: }потерявшая ЕЗ цель (но не существа в области Взрыва) замерзает* и становится Неподвижными до конца своей следующей Очереди. 
\newline Если цель подвергается КУ при активации Колючего града, все существа в области Взрыва замерзают и становятся Неподвижными на 
\textbf{|МФх активатора|} Очередей.
\paragraph{Усиление:}
\begin{itemize}
\item+1 Эн -> +1 к БПв и Силе Взрыва.
\item+1 Эн -> +5 к Дистанциям.
\item+1 Эн -> +1 к радиусу Взрыва.
\item+1 Эн -> попавшие во Взрыв и потерявшие ЕЗ замерзают* и становятся Неподвижными до конца своей следующей Очереди.
\end{itemize}
\paragraph{}Пв, необходимые для нанесения Опасной раны замерзшему, составляют 1/6 от максимальных ЕЗ. Пв, необходимые для отрубания частей тела, составляют 1/7 от максимальных ЕЗ.
\subsection{Огненный шар}
\index[powers]{Огненный шар}
\paragraph{} 
\textit{В руке Исказителя появляется тускло светящийся шар, с которого стекает призрачное пламя. После попадания в цель, шар взрывается бурей настоящего пламени.}
\paragraph{Стоимость: }2 Эн
\newline
\textbf{Тип Повреждений: }Огненные
\paragraph{Эффект: }если цель подвергается КУ при активации Пирокинеза, все в области Взрыва Загораются. Хотя возгорание может произойти и без этого – бензин, сено, волосы и многое другое только и ждут искорки, чтобы вспыхнуть!
\paragraph{Усиление:}
\begin{itemize}
\item+1 Эн -> +1 к БПв и Силе Взрыва.
\item+1 Эн -> +5 к дистанциям.
\item+1 Эн -> +1 к радиусу Взрыва.
\item+1 Эн -> попавшие во Взрыв Загораются, если получили Пв.
\end{itemize}
\section{Метка}Исказитель накладывает Метку на предмет или существо. Если цель сопротивляется, герой должен преуспеть в маневре «Касание». При провале Метка остается в руке героя. Он может по-вторить маневр позднее. Пока Метка не наложена, она занимет руку, а Стоимость Поддержания Метки равна 0.
\begin{itemize}
\item Сопротивление к Метке указано в описании.
\end{itemize}
\subsection{Без следа}
\index[powers]{Без следа}
\paragraph{} 
\textit{Эманации исказителя путают мысли, сбивают с толку и мешают обнаружить следы присутствия цели.}
\paragraph{Стоимость/Поддержание: }1 Эн/1 Эн
\newline
\textbf{Сопротивление: }
\textbf{|10 + Вл|}, если цель не желает попасть под эффекты.
\newline 
\textbf{Длительность: }1 Сцена.
\paragraph{Эффект: }существа, чья Воля не превышает Концентрации(МФх) исказителя, проверяют с Помехой Наблюдательность и Выживание для обнаружения следов цели.
\paragraph{Усиление:}
\begin{itemize}
\item+1 Эн -> +1 Сцена к Длительности.
\end{itemize}
\subsection{Бесплотность}
\index[powers]{Бесплотность}
\paragraph{} 
\textit{Исказитель превращает цель и все, что она несет (включая других существ) в неосязаемую проекцию, которую можно увидеть, но нельзя осязать, пока действует Феномен.}
\paragraph{Стоимость/Поддержание: }4 Эн/2 Эн
\newline
\textbf{Сопротивление: }
\textbf{|10 + Вл|}, если цель не желает попасть под эффекты.
\newline 
\textbf{Длительность: }1 Сцена.
\paragraph{Эффект: }цель превращается в размытый мерцающий силуэт. При этом она: 
\begin{itemize} 
\item Получает Сопротивление к Пв; 
\item Совершает с Преимуществом проверки, использующие Лв и Вн. 
\item Передвигается на половину своей Ск в любой среде (да, она может летать); 
\item Совершает с Помехой проверки, использующие Сл; 
\item Не может совершать Базовые Маневры, говорить, и взаимодействовать с предметами; 
\item Не может активировать феномены, хотя способна Поддерживать уже сотворенные. Если поблизости находится Питомчик героя, не подверженный действию Атомизации, цель может  активировать феномены с его помощью; 
\item Может проникать сквозь объекты. Если активация феномена заканчивается, когда цель внутри скалы, почвы, дерева и т. п., то цель оказывается в ловушке и состоянии Удушья, если ей необходим воздух; 
\item Не может быть Захвачена или Сбита с ног. Маневры Разоружения и Поломки снаряжения совершаются против нее с Помехой; 
\item Уязвима к Пв так же, как и существа с [Бестелесностью]. 
\end{itemize} Если цель роняет предмет или существо, которое переносила под эффектом феномена, они тут же теряют Бесплотность. Если ЕЗ цели достигают 0, активация завершается.
\paragraph{Усиление:}
\begin{itemize}
\item+1 Эн -> +1 Сцена к Длительности.
\end{itemize}
\subsection{Вихрь-хранитель}
\index[powers]{Вихрь-хранитель}
\paragraph{} 
\textit{
\tbd}
\paragraph{Стоимость/Поддержание: }2 Эн/1 Эн
\newline
\textbf{Сопротивление: }
\textbf{|10 + Вл|}, если цель не желает попасть под эффекты.
\newline 
\textbf{Длительность: }1 Сцена
\paragraph{Эффект: }цель окружает вихрь радиусом 1 метр, защищающий пуль и стрел. 
\begin{itemize} 
\item Вихрь занимает то же пространство, что и цель и двигается вместе с ней. 
\item Вихрь дает +1 к БАЗщ против Дистанционных атак. 
\item Существа в Боевом контакте цели получают Помеху на Активные проверки. 
\item Вихрь не позволяет цели летать, но способен замедлить падение, сократив его дистанцию на 2 метра. 
\item При должном усилении Вихрь может перерасти в ураган! 
\end{itemize}
\paragraph{Усиление:}
\begin{itemize}
\item+1 Эн -> +1 к БАЗщ и -2 метра к дистанции падения.
\item+1 Эн -> +1 Сцена к Длительности.
\item+1 Эн -> +1 метр к радиусу вихря.
\item+2 Эн -> радиус вихря увеличивается на 1 метр.
\item+2 Эн -> попавшие в смерч предметы и существа (кроме активатора и цели феномена) проверяют Лв/ Атлетику(Лв) против 
\textbf{|15|}, или против 
\textbf{|20|}, если передвигаются по воздуху. При успехе существо остается на месте и может продолжать действовать. При провале существо получает Ошеломление и Пв, равные величине провала. Затем существо размещается исказителем с любой стороны смерча.
\end{itemize}
\subsection{Вне времени}
\index[powers]{Вне времени}
\paragraph{} 
\textit{Исказитель вырывает цель из потока времени.}
\paragraph{Стоимость/Поддержание: }3 Эн/1 Эн
\newline
\textbf{Сопротивление: }
\textbf{|Вн цели|}.
\newline 
\textbf{Длительность: }До следующего Антракта.
\paragraph{Эффект: }Цель становится Неподвижной. Она не стареет, ей не требуется дышать и отправлять естественные потребности. Она выглядит уснувшей, хотя ее сердце не бьется, а кровь не течет по жилам. Болезни не причинят цели вреда, пока активен феномен. Внешние влияния – жара, холод, физические воздействия, все еще могут нанести ей вред.
\paragraph{Усиление:}
\begin{itemize}
\item+1 Эн -> -1 к Сопротвилению цели.
\item+2 Эн -> продлевает активацию до еще одного Антракта.
\item+3 Эн -> активация феномена бессрочна, поддержание не требуется.
\end{itemize}
\subsection{Допинг}
\index[powers]{Допинг}
\paragraph{} 
\textit{Исказитель расширяет пределы физических или ментальных возможностей цели на время.}
\paragraph{Стоимость/Поддержание: }2 Эн/1 Эн
\newline
\textbf{Сопротивление: }
\textbf{|10 + Вл|}, если цель не желает попасть под эффекты.
\newline 
\textbf{Длительность: }1 Cцена.
\paragraph{Эффект: }исказитель увеличивает на 2 одну Основную характеристику цели – даже выше 20.
\paragraph{Усиление:}
\begin{itemize}
\item+1 Эн -> Характеристика цели дополнительно возрастает на 2.
\end{itemize}
\subsection{Живучесть}
\index[powers]{Живучесть}
\paragraph{} 
\textit{Исказитель мобилизует скрытые резервы организма цели.}
\paragraph{Стоимость/Поддержание: }2 Эн/1 Эн
\newline
\textbf{Сопротивление: }
\textbf{|10 + Вл|}, если цель не желает попасть под эффекты.
\newline 
\textbf{Длительность: }До следующей Интерлюдии.
\paragraph{Эффект: }1 раз в течение активации феномена, когда ЕЗ цели достигают 0, она восстанавливает число ЕЗ, равное 
\textbf{|Концентрации(МФх) исказителя|}. Если цель подвергается действию эффекта, убивающего независимо от количества ЕЗ (ей отрубают голову или бросают в мусоросжигательную печь), феномен не имеет силы.
\paragraph{Усиление:}
\begin{itemize}
\item+2 Эн -> Эффект может быть активирован еще 1 раз.
\end{itemize}
\subsection{Защита}
\index[powers]{Защита}
\paragraph{} 
\textit{Исказитель окружает цель полупрозрачным силовым полем.}
\paragraph{Стоимость/Поддержание: }3 Эн/1 Эн
\newline
\textbf{Сопротивление: }
\textbf{|10 + Вл|}, если цель не желает попасть под эффекты.
\newline 
\textbf{Длительность: }До следующей Интерлюдии.
\paragraph{Эффект: }исказитель выбирает 1 тип Пв. Цель получает Сопротивление к этому типу Пв.
\paragraph{Усиление:}
\begin{itemize}
\item+2 Эн -> цель получает Сопротивление к 1 дополнительному типу Пв.
\end{itemize}
\subsection{Иллюзорная внешность(Мимикрическое поле)}
\index[powers]{Иллюзорная внешность(Мимикрическое поле)}
\paragraph{} 
\textit{Хитрая манипуляция с гравитационными полями и световыми волнами.}
\paragraph{Стоимость/Поддержание: }2 Эн/1 Эн
\newline
\textbf{Сопротивление: }
\textbf{|10 + Вл|}, если цель не желает попасть под эффекты.
\newline 
\textbf{Длительность: }1 Сцена.
\newline 
\textbf{Время активации: }1 Сцена.
\paragraph{Эффект: }цель выглядит, пахнет и ощущается как другое существо. Мастер вправе потребовать проверки Ин исказителя, когда воспроизводятся детали внешности, элементы одежды, характерные запахи. Если у исказителя есть Трюк «Хорошая память», и он видел того, чью внешность он копирует, проверка не требуется. 
\newline Исказитель может придать цели иллюзорный размер, отличный от ее собственного, однако это легко заметить при контакте.
\paragraph{Усиление:}
\begin{itemize}
\item+2 Эн -> время активации становится Действием.
\item+1 Эн -> цель изменяет вес на число килограммов, не превышающее 
\textbf{|[Концентрации(Фх) исказителя]*10|}.
\end{itemize}
\subsection{Исцеление}
\index[powers]{Исцеление}
\paragraph{} 
\textit{
\tbd}
\paragraph{Стоимость: }2 Эн
\newline
\textbf{Сопротивление: }
\textbf{|10 + Вл|}, если цель не желает попасть под эффекты.
\paragraph{Эффект: }цель восстанавливает 3 ЕЗ.
\paragraph{Усиление:}
\begin{itemize}
\item+1 Эн -> +3 к числу восстановленных ЕЗ.
\item+1 Эн -> после заживления ран не остается шрамов и рубцов.
\end{itemize}
\subsection{Клеймо}
\index[powers]{Клеймо}
\paragraph{} 
\textit{Исказитель оставляет на поверхности предмета или теле существа слабо светящийся знак.}
\paragraph{Стоимость: }1 Эн
\newline
\textbf{Сопротивление: }
\textbf{|10 + Вл|}, eсли цель не желает попасть под эффекты. Не имеющие значения Вл не могут сопротивляться.
\newline 
\textbf{Длительность: }до конца следующей Сцены.
\newline 
\textbf{Время активации: }1 Сцена
\paragraph{Эффект: }Клеймо может быть удалено только с помощью другого Феномена или инструметов могущества, пока Фномен активен, но может быть визуально скрыто. Исказитель всегда знает, где находится Клеймо (и цель) и может легко его отыскать при помощи других феноменов. Цель может снять снаряжение с Клеймом.
\paragraph{Усиление:}
\begin{itemize}
\item+1 Эн -> +1 Сцена к Длительности.
\item+2 Эн -> Время активации становится Действием.
\item+2 Эн -> Клеймо видит только активатор.
\item+1 Эн -> Во время активации Клейма исказитель может прикрепить к нему один Наговор или Область так, чтобы для определения дальности прикрепленного феномена центром стало Клеймо. Исказителю нужно заплатить полную стоимость прикрепленного феномена и концентрироваться для того, чтобы поддерживать активацию. После завершения активации прикрепленного феномена, активация Клейма также Прерывается.
\end{itemize}
\subsection{Локатор}
\index[powers]{Локатор}
\paragraph{} 
\textit{Весь спектр невизуальных средств восприятия мира.}
\paragraph{Стоимость/Поддержание: }3 Эн/1 Эн
\newline
\textbf{Сопротивление: }
\textbf{|10 + Вл|}, если цель не желает попасть под эффекты.
\newline 
\textbf{Длительность: }1 Сцена.
\paragraph{Эффект: }цель видит в темноте, но не различает цвета.
\paragraph{Усиление:}
\begin{itemize}
\item+1 Эн -> +1 Сцена к Длительности.
\item+1 Эн -> цель различает цвета.
\item+3 Эн -> цель видит сквозь объекты на расстояние равное 
\textbf{|[Концентрации(Фх) исказителя]|}.
\end{itemize}
\subsection{Мутация}
\index[powers]{Мутация}
\paragraph{} 
\textit{Исказитель меняет строение и биохимию жертвы на клеточном уровне.}
\paragraph{Стоимость: }4 Эн
\newline
\textbf{Сопротивление: }
\textbf{|10 + Вл|}, если цель не желает попасть под эффекты.
\newline 
\textbf{Длительность: }постоянно.
\newline 
\textbf{Время активации: }1 динамическая Сцена.
\paragraph{Эффект: }исказитель превращает цель в другое существо. Цель сохраняет свои Ин, Мд и Об, получая Сл, Лв и Вн своего нового тела. Вещи цели не превращаются вместе с ней. Цель может случайно разорвать их, увеличившись (а также застрять в цепочках, браслетах или доспехах), или оказаться ими придавленной. 
\begin{tcolorbox} Шаману будет не так-то просто вернуть изначальный облик, если пасть существа, в которое он превратился, не подходит для песни, а лапки – для танца. 
\end{tcolorbox} Мастер вправе требовать проверки Ин исказителя, если воспроизводятся детали облика, или если исказитель только слышал или читал о существе, в которое превращает цель.
\paragraph{Усиление:}
\begin{itemize}
\item+1 Эн -> время активации становится Действием.
\item+1 Эн -> новый облик может быть на 1 МРз больше или меньше начального размера цели.
\item+1 Эн -> цель приобретает часть свойств другого существа. Например, жабры, подвижные обезьяньи ступни или теплую шерсть.
\end{itemize}
\subsection{Окаменение}
\index[powers]{Окаменение}
\paragraph{} 
\textit{
\tbd}
\paragraph{Стоимость: }4 Эн
\newline
\textbf{Сопротивление: }Выносливость.
\newline 
\textbf{Длительность: }Постоянно.
\paragraph{Эффект: }жертва и все ее снаряжение превращаются в статую из сланца. Если феномен дезактивирован, жертва оживает. Однако время и непогода не щадят даже камни.
\paragraph{Усиление:}
\begin{itemize}
\item+1 Эн -> -1 к Сопротивлению.
\item+1 Эн -> жертва может окаменеть частично. Если окаменеет грудная клетка, жертва начнет задыхаться.
\item+1 Эн -> исказитель поддерживает жизнь в частично окаменевшей жертве. Ей не страшны удушье, голод и жажда, но она продолжает стареть.
\item+1 Эн -> окаменевшие глаза продолжают видеть.
\item+1 Эн -> окаменевшая жертва чувствует и мыслит.
\item+1 Эн -> окаменевшая жертва чувствует и мыслит. Чувствительность могут сохранить и отдельные окаменевшие части тела.
\end{itemize}
\subsection{Пиявка}
\index[powers]{Пиявка}
\paragraph{} 
\textit{Исказитель поглощает кровь или жизненную энергию жертвы для восстановления своих сил.}
\paragraph{Стоимость: }2 Эн
\newline
\textbf{Сопротивление: }нет.
\paragraph{Эффект: }исказитель должен преуспеть в Касании и дотронуться до тела цели. Он восстанавливает число своих ЕЗ или Эн в любых комбинациях, равное величине успеха. Цель теряет число ЕЗ, равное величине успеха. Исказитель не может восстановить больше ЕЗ и Эн, чем есть ЕЗ у жертвы. 
\newline Исказитель может использовать любые маневры с участием РДб для активации феномена, если касается тела цели в процессе.
\subsection{Рука смерти}
\index[powers]{Рука смерти}
\paragraph{} 
\textit{
\tbd}
\paragraph{Стоимость: }2 Эн
\newline
\textbf{Сопротивление: }Выносливость.
\paragraph{Эффект: }исказитель должен преуспеть в Касании и дотронуться до тела цели. Цель проверяет Вн против 
\textbf{|5 + величина успеха исказителя при Касании|} и умирает при провале. Исказитель может использовать любые маневры с участием РДб для активации феномена, если касается тела цели в процессе.
\subsection{Удар Ци}
\index[powers]{Удар Ци}
\paragraph{} 
\textit{герой наделяет оружие силой разить духов и призраков. Призрачный клинок не поможет против существ, чьи тела состоят из рассеянных, но вполне естественных субстанций - огня, воды, песка, слизи или сонма копошащихся мелких тварей!}
\paragraph{Стоимость/Поддержание: }2 Эн/1 Эн
\newline
\textbf{Сопротивление: }нет.
\newline 
\textbf{Длительность: }1 Круг.
\newline 
\textbf{Время активации: }Быстрое действие.
\paragraph{Эффект: }исказитель игнорирует бонусы к Зщ цели и прочие эффекты, которые дают силовые барьеры и поля. Барьеры получают Пв от атак под действием этого феномена. Существа, имеющие Сопротивление или Иммунитет к Пв из-за своей разреженности, а так же [Бестелесные] и [Рои], Уязвимы к атакам Субатомного снаряда.
\paragraph{Усиление:}
\begin{itemize}
\item+2 Эн -> +1 Круг к Длительности.
\end{itemize}
\subsection{Управление размером}
\index[powers]{Управление размером}
\paragraph{} 
\textit{
\tbd}
\paragraph{Стоимость/Поддержание: }3 Эн/1 Эн
\newline
\textbf{Сопротивление: }
\textbf{|10 + Вл|}, если цель не желает попасть под эффекты.
\newline 
\textbf{Длительность: }1 Сцена.
\newline 
\textbf{Время активации: }1 Сцена.
\paragraph{Эффект: }исказитель меняет МРз (но не форму) существа или предмета на 1. Он может сделать Средний предмет или существо Большим или Маленьким, а Маленький предмет или существо – Крошечным или Средним. Увеличиваясь, существо повышает свою Сл и Вн на 2 за каждую единицу МРз, уменьшаясь – понижает Сл и Вн на 2 за каждую единицу МРз.
\paragraph{Усиление:}
\begin{itemize}
\item+1 Эн -> МРз цели может быть дополнительно изменен на 1.
\item+2 Эн -> Время активации становится Действием.
\item+1 Эн -> исказитель может изменять размеры отдельных частей тела. Скелет вряд ли выдержит чрезмерно увеличенную руку или ногу!
\end{itemize}
\subsection{Фугас}
\index[powers]{Фугас}
\paragraph{} 
\textit{Мощь исказителя заставляет предметы взрываться при касании.}
\paragraph{Стоимость/Поддержание: }3 Эн/1 Эн
\newline
\textbf{Сопротивление: }
\textbf{|[Воля носителя предмета или оружия]|}, если носитель не желает, чтобы его оружие попало под эффекты.
\newline 
\textbf{Длительность: }До следующего Антракта.
\paragraph{Эффект: }Сила Взрыва 15, Радиус Взрыва 1. Взрыв имеет Дробящие Пв. 
\newline Фугас можно наложить: 
\begin{itemize} 
\item на боеприпасы или метательное оружие. В этом случае боеприпас или метательное оружие взорвется после проведения Дистанционной атаки с его использованием. 
\item на оружие ближнего боя. В этом случае оно взорвется после проведения успешной атаки. 
\item на предмет или снаряжение размером меньше, чем исказитель. В этом случае предмет становится ловушкой, срабатывающаей при касании. В этом случае Опасность ловушки равна Силе Взрыва. 
\end{itemize} При взрыве цель феномена получает повреждения по обычным правилам. 
\newline Исказитель может потратить Быстрое Действие и подорвать наложенный им Фугас самостоятельно. Он может это сделать даже если метка находится вне его поля зрения.
\paragraph{Усиление:}
\begin{itemize}
\item+1 Эн -> +1 к Силе Взрыва.
\item+1 Эн -> +1 к Радиусу Взрыва.
\end{itemize}
\paragraph{}
\begin{tcolorbox} Для того, чтобы использовать Фугас при стрельбе из огнестрельного оружия, исказителю следует выбрать целью боеприпас. Выбрав целью само оружие, исказитель должен метнуть его или ударить им, чтобы оно взорвалось. 
\end{tcolorbox}
\subsection{Хамелеон}
\index[powers]{Хамелеон}
\paragraph{} 
\textit{Говорят, что окружающий мир – грандиозная иллюзия, световые волны разной длины. Правда или нет, но скрывать цели Хамелеоном и впрямь несложно.}
\paragraph{Стоимость/Поддержание: }3 Эн/1 Эн
\newline
\textbf{Сопротивление: }
\textbf{|10 + Вл|}, если цель не желает попасть под эффекты.
\newline 
\textbf{Длительность: }1 Сцена.
\paragraph{Эффект: }цель и все, что она несет (в том числе другие существа), становится невидимым. Ее могут услышать или обнаружить на ощупь.
\paragraph{Усиление:}
\begin{itemize}
\item+1 Эн -> +1 Сцена к Длительности.
\end{itemize}
\section{Область}Герой создает Область, которая налагает эффекты на попавших в нее. Активатор Области может выбрать, влияют на него эффекты, или нет.
\begin{itemize}
\itemДистанция Области определяет максимальное удаление от героя, на котором накладываются эффекты области во время ее активации.
\item Сопротивление к Области указано в описании. Используйте Коллективную проверку, чтобы быстрее определить влияние Области на группу существ.
\end{itemize}Если в описании не указано иначе, Область является кругом вокруг героя и не перемещается вместе с ним.
\newline Возможные формы Области:
\begin{itemize}
\item[--] 
\textbf{Круг} вокруг героя. 
\textbf{Дистанция} определяет радиус круга.
\item[--] 
\textbf{Периметр} очерченный героем. Это должна быть замкнутая линия без самопересечений. 
\textbf{Дистанция} определяет 2 максимально удаленные друг от друга точки периметра.
\item[--] 
\textbf{Конус} перед героем. При сотворении Феномена герой может определить, насколько широким будет конус, но он не может быть больше, чем полукруг. Размах конуса не влияет на силу эффектов и затраты Энергии на сотворение Феномена. 
\textbf{Дистанция} определяет радиус конуса.
\end{itemize}
\subsection{Адреналин}
\index[powers]{Адреналин}
\paragraph{} 
\textit{Эманации исказителя приводят сознание целей в возбужденное состояние.}
\paragraph{Стоимость/Поддержание: }3 Эн/1 Эн
\newline
\textbf{Дистанция: }5
\newline
\textbf{Сопротивление: }
\textbf{|10 + Вл|}, если цель не желает попасть под эффекты. или 
\textbf{|10 + число потенциальных целей|}.
\tbd
\newline 
\textbf{Длительность: }1 Круг
\paragraph{Эффект: }цели получают Преимущество на Активные проверки. Исказитель выбирает, кто именно подвергается действию феномена.
\paragraph{Усиление:}
\begin{itemize}
\item+1 Эн -> +5 к дистанции.
\end{itemize}
\subsection{Встряска}
\index[powers]{Встряска}
\paragraph{} 
\textit{Земля дрожит и разверзается под ногами врагов, а вокруг оседают и рушатся здания!}
\paragraph{Стоимость: }3 Эн
\newline
\textbf{Дистанция: }5
\newline
\textbf{Сопротивление: }
\textbf{|10 + МЛв/Атлетика (Лв)|} для живых существ или 
\textbf{|10 + Прч|} для зданий.
\paragraph{Эффект: }Цели получают Пв, равные 
\textbf{|величине успеха Концентрации(Фх) исказителя|}. Величина успеха может разниться для целей в зависимости от их Сопротивления. 
\newline Живые цели падают, если получили Пв. 
\newline В случае КУ при проверке Концентрации, нанесенные Пв удваиваются.
\paragraph{Усиление:}
\begin{itemize}
\item+1 Эн -> +5 метров к дистанции.
\item+1 Эн -> -1 к Сопротивлению.
\item+1 Эн -> область становится Трудным ландшафтом.
\end{itemize}
\subsection{Гул[Конус]}
\index[powers]{Гул}
\paragraph{} 
\textit{Исказитель производит низкочастотные звуковые вибрации. Они оглушают, вызывают безотчетный животный ужас и разрушают хрупкие предметы.}
\paragraph{Стоимость: }2 Эн
\newline
\textbf{Дистанция: }5
\newline
\textbf{Сопротивление: }
\textbf{|Вн цели|}.
\paragraph{Эффект: }Попавшие под воздействие Оглушены и в Ужасе. В случае КУ на Концентрацию, жертвы получают Пв, равные 
\textbf{|МФх исказителя|}. 
\newline Хрупкие элементы обстановки, вроде бокалов для шампанского, оконных стекол и кинескопов, в зоне действия разлетаются веером осколков.
\paragraph{Усиление:}
\begin{itemize}
\item+1 Эн -> +5 Дистанции.
\item+1 Эн -> -1 к Сопротивлению.
\end{itemize}
\subsection{Депрессия}
\index[powers]{Депрессия}
\paragraph{} 
\textit{Эманации исказителя провоцируют угнетенное состояние рассудка.}
\paragraph{Стоимость/Поддержание: }3 Эн/1 Эн
\newline
\textbf{Дистанция: }5
\newline
\textbf{Сопротивление: }
\textbf{|10 + Вл|} или 
\textbf{|10 + число потенциальных целей|}.
\tbd
\newline 
\textbf{Длительность: }1 Круг
\paragraph{Эффект: }число жертв, попавших под воздействие, равно 
\textbf{|величине успеха Концентрации(Фх) исказителя|}. Жертвы получают Помеху на Активные проверки. Исказитель выбирает, кто именно подвергается действию феномена.
\paragraph{Усиление:}
\begin{itemize}
\item+1 Эн -> +5 к дистанции.
\end{itemize}
\subsection{Дыхание[Конус]}
\index[powers]{Дыхание}
\paragraph{} 
\textit{Исказитель изрыгает поток некоей опасной субстанции. Чудовища, мутанты и прочие подобные твари не стесняются использовать для изрыгания собственный рот.}
\paragraph{Стоимость: }3 Эн
\newline
\textbf{Дистанция: }
\textbf{|МРз + МВн|}
\newline
\textbf{Сопротивление: }
\textbf{|10 + Атлетика (Лв)|}
\paragraph{Эффект: }исказитель выбирает тип Пв, наносимых Дыханием, приобретая феномен. Все цели в Обасти получают Пв, равные 
\textbf{|величина успеха Концентрации(Фх) исказителя|}. В случае выпадения КУ при активации Дыхания, все жертвы получают КУ, соответствующий типу Пв Дыхания.
\paragraph{Усиление:}
\begin{itemize}
\item+1 Эн -> +5 к дистанции.
\end{itemize}
\subsection{Зыбун}
\index[powers]{Зыбун}
\paragraph{} 
\textit{Исказитель делает область почвы временно сыпучей и подвижной, из-за чего жертвы начинают стремительно погружаться под землю.}
\paragraph{Стоимость: }4 Эн
\newline
\textbf{Дистанция: }5
\newline
\textbf{Сопротивление: }
\textbf{|10 + МЛв/Атлетика(Лв)|}.
\paragraph{Эффект: }Феномен не работает в воздухе и воде, на бетонных и асфальтовых мостовых и в помещениях (если, конечно, это не халупа с земляным полом). 
\newline Попавшие под воздействие погружаются в грязь по колено; затем почва застывает. Чтобы освободиться, жертва должна израсходовать Действие и Перемещение. 
\newline Жертвы вычитают МЛв из своей Зщ, пока не освободятся.
\paragraph{Усиление:}
\begin{itemize}
\item+1 Эн -> +5 к дистанции.
\item+1 Эн -> -1 к Сопротивлению.
\item+2 Эн -> Цели проваливаются по грудь: 
\begin{itemize} 
\item Они вычитают МЛв и БЩ из своей Зщ. 
\item Все атаки по ним в Боевом контакте совершаются с Преимуществом. 
\item Все Дистанционные атаки по ним совершаются с Помехой. 
\item Чтобы освободиться, жертва должна преуспеть в проверке Сл или Атлетики (Сл) против 
\textbf{|20|} и пропустить свою Очередь целиком. 
\begin{itemize}
\item+3 Эн -> Цели проваливается по шею или глубже. Они становятся Неподвижны и не могут выбраться самостоятельно.
\item+4 Эн -> Цели проваливается целиком. Они становятся Неподвижны, страдают от Удушья и не могут выбраться самостоятельно.
\end{itemize}
\paragraph{}
\begin{tcolorbox} Стоимости последних двух усилений не являются кумулятивными. Для того, чтобы погрузить цель по шею достаточно потратить только 3 Эн на усиление, а для того, чтобы похоронить ее целиком достаточно 4 Эн. 
\end{tcolorbox}
\subsection{Ноктюрн сирены}
\index[powers]{Ноктюрн сирены}
\paragraph{} 
\textit{Гипнотический речетатив успокаивает и усыпляет бдительность окружающих.}
\paragraph{Стоимость/Поддержание: }4 Эн/2 Эн
\newline
\textbf{Дистанция: }5
\newline
\textbf{Сопротивление: }
\textbf{|10 + Вл|}.
\newline 
\textbf{Длительность: }1 Сцена.
\paragraph{Эффект: }Цели должны быть способны слышать, чтобы попасть под Эффект. Они становятся Нейтральны ко всем, пока феномен Поддерживается, даже если до того они были настроены враждебно. Атаки по ним совершаются по правилам Внезапного нападения. Если цель подвергается атаке и в результате теряет ЕЗ, она освобождается от эффекта. После окончания активации Феномена цель остается Нейтральной (и практически недееспособной) до конца Сцены. 
\newline Исказитель не может выбирать цели этого феномена - все, кто находится в пределах Дистанции становятся целями. Если под эффект попали друзья, исказителю будет непросто достучаться до них без пары затрещин!
\paragraph{Усиление:}
\begin{itemize}
\item+1 Эн -> +5 к Дистанции.
\item+1 Эн -> -1 к Сопротивлению.
\item+2 Эн -> +1 Сцена, в течение которой жертвы Нейтральны после завершения активации.
\item+6 Эн -> все, кто может слышать феномен, попадают под воздействие, вне зависимости от расстояния, на котором находятся от активатора. Жертвы стремятся сблизиться с исказителем. Попав в Область, они замирают.
\end{itemize}
\subsection{Поток времени(Власть над временем 
\tbd)}
\index[powers]{Поток времени(Власть над временем 
\tbd)}
\paragraph{} 
\textit{Исказитель меняет скорость течения времени для целей. Даже не пытайтесь разобраться, как он это делает. После Катастрофы от знаний - одни печали.}
\paragraph{Стоимость/Поддержание: }4 Эн/1 Эн
\newline
\textbf{Дистанция: }3
\newline
\textbf{Сопротивление: }
\textbf{|10 + Вл|}, если цель не желает попасть под эффекты. 
\textbf{|15 + число потенциальных целей|}.
\tbd
\newline 
\textbf{Длительность: }1 Круг.
\paragraph{Эффект: }перед активацией феномена исказитель выбирает, ускоряет он или замедляет течение времени. 
\begin{itemize} 
\item Ускоренная цель повышает свои Ск и БАЗщ на 1 и атакует с Преимуществом. 
\item Замедленная цель понижает свои Ск и БАЗщ на 1 и атакует с Помехой. 
\end{itemize} Исказитель выбирает, кто именно подвергается действию феномена.
\paragraph{Усиление:}
\begin{itemize}
\item+1 Эн -> +1 к Дистанции.
\item+1 Эн -> +1 к БАЗщ и Ск ускоренных существ или -1 к БАЗщ и Ск замедленных существ.
\end{itemize}
\subsection{Присутствие}
\index[powers]{Присутствие}
\paragraph{} 
\textit{Исказитель создает зону, где он присутствует везде и нигде одновременно. В ее пределах игра ведется по его правилам.}
\paragraph{Стоимость: }4 Эн
\newline
\textbf{Дистанция: }5
\newline
\textbf{Сопротивление: }15
\newline 
\textbf{Длительность: }1 Сцена.
\paragraph{Эффект: }Исказитель может: 
\begin{itemize} 
\item Иметь Боевй контакт с любым существом в пределах Дистанции. 
\item Совершать маневры Дальнего боя, как если бы он не находился в Боевом контакте ни с кем и никто не находился в Боевом контакте с ним. 
\item Переместиться в любую точку Области, израсходовав Быстрое действие – даже вне своей Очереди. Исказитель может избежать атаки, Переместившись до проверки Дб или Мт противника. 
\end{itemize}
\paragraph{Усиление:}
\begin{itemize}
\item+1 Эн -> +1 к Дистанции.
\end{itemize}
\section{Призыв}Герой перемещает или призывает откуда-то существа, предметы и субстанции. Никто, даже сам исказаитель не скажет, откуда именно они прибывают.
\newline Призванные существа исполняют все приказы призывателя. По истечении времени Призыва они обычно возвращаются восвояси, но порой могут и задержаться. В этом случае контроль исказителя над их действиями прекращается, а существа редко бывают довольны фактом призыва.
\newline Если не указана 
\textbf{Дистанция}, то призванный предмет оказывается в руке призывателя, а призванное существо на свободном пятачке не дальше метра от исказителя. Призванное существо не может по своей воле отойти от призывателя дальше, чем указано в Дистанции. Если призванное существо или предмет случайно или намеренно удаляются дальше, чем указано в Дистанции, Призыв немедленно Прерывается.
\newline 
\textbf{Сопротивление} к призыву указано в описании.
\subsection{Каменная стена}
\index[powers]{Каменная стена}
\paragraph{} 
\textit{
\tbd Здесь и сейчас этот камень нужнее.}
\paragraph{Стоимость/Поддержание: }2 Эн/1 Эн
\newline
\textbf{Дистанция: }5
\newline
\textbf{Сопротивление: }15+5 за каждую дополнительную секцию стены.
\newline 
\textbf{Длительность: }1 Сцена.
\paragraph{Эффект: }исказитель призывает 1 секцию каменной стены. Секция имеет объем 1 кубический метр, 20 ЕЗ, Прч 10 и Сопротивление к физическим атакам. Исказитель не может создать стену в воздухе, но может создать ее там, где уже стоит другое существо. В этом случае существо оказывается на верхушке секции стены.
\paragraph{Усиление:}
\begin{itemize}
\item+1 Эн -> исказитель добавляет к стене 1 секцию.
\item+1 Эн -> +5 к дистанции.
\end{itemize}
\subsection{Киянка}
\index[powers]{Киянка}
\paragraph{} 
\textit{Откуда бы не явилась эта колотушка, битые ею точно по ней не заскучают.}
\paragraph{Стоимость/Поддержание: }2 Эн/1 Эн
\newline
\textbf{Сопротивление: }15
\newline 
\textbf{Длительность: }1 Сцена.
\paragraph{Эффект: }герой создает Громоздкую каменную дубину с БПв +0, КУ 20, свойством "Кувалда" и Дробящим типом Пв. ТСл для использования Киянки составляет 10. Киянка игнорирует Прч цели. 
\newline Если при атаке Киянкой выпадает КП, исказитель падает Оглушенным, и активация феномена завершается. Если исказитель выпускает Киянку из рук, активация завершается.
\paragraph{Усиление:}
\begin{itemize}
\item+1 Эн -> +1 к БПв Киянки.
\item+1 Эн -> Кувалда становится Длинной, Двуручной, и получает Упредительный удар.
\item+1 Эн -> Киянка теряет свойство "Громоздкое".
\item+2 Эн -> +1 Сцена к Длительности.
\end{itemize}
\subsection{Ледяная секира}
\index[powers]{Ледяная секира}
\paragraph{} 
\textit{
\tbd Моржам и пингвинам она точно ни к чему.}
\paragraph{Стоимость/Поддержание: }2 Эн/1 Эн
\newline
\textbf{Сопротивление: }15
\newline 
\textbf{Длительность: }1 Сцена.
\paragraph{Эффект: }исказитель призывет ледяную секиру с БПв +0, КУ 18-20 и Рубящим и Ледяным типом Пв. ТСл для использования секиры составляет 6. Пв, необходимые для нанесения секирой Нокаута, составляют 1/6 от максимальных ЕЗ. Пв, необходимые для отрубания секирой частей тела, составляют 1/7 от максимальных ЕЗ. 
\newline Если при атаке секирой выпадает КП, она взрывается, и активация завершается. Все существа в радиусе 1 метра (включая исказителя) получают Колющие и Ледяные Пв, равные БПв секиры (минимум 1). То же происходит, если Секира намеренно уничтожена. 
\newline Если исказитель выпускает секиру из рук, активация завершается.
\paragraph{Усиление:}
\begin{itemize}
\item+1 Эн -> +1 к БПв Секиры.
\item+1 Эн -> Секира становится Длинной, Двуручной и получает Упредительный удар.
\item+2 Эн -> +1 Сцена к Длительности.
\end{itemize}
\subsection{Ледяная стена}
\index[powers]{Ледяная стена}
\paragraph{} 
\textit{Кубики льда для самой большой порции виски в мире.}
\paragraph{Стоимость/Поддержание: }2 Эн/1 Эн
\newline
\textbf{Дистанция: }5
\newline
\textbf{Сопротивление: }15
\newline 
\textbf{Длительность: }1 Сцена.
\paragraph{Эффект: }исказитель призывает 1 секцию обжигающе холодного льда. Секция имеет объем 1 кубический метр, 20 ЕЗ, Прч 5 и Уязвимость к Огненным Пв. Любое существо, касающееся стены, получает Ледяные Пв, равные 
\textbf{|20 – БД|}. Если существо теряет ЕЗ, оно замерзает* и становится Неподвижным на число Кругов, равное числу потерянных ЕЗ. Исказитель может создать стену там, где уже стоит другое существо. В этом случае существо оказывается на верхушке секции стены. 
\newline *Пв, необходимые для нанесения Нокаута замерзшему, составляют 1/6 от максимальных ЕЗ. Пв, необходимые для отрубания частей тела, составляют 1/7 от максимальных ЕЗ.
\paragraph{Усиление:}
\begin{itemize}
\item+1 Эн -> +1 к Пв, получаемым при касании стены.
\item+2 Эн -> Исказитель добавляет к стене 1 секцию.
\item+1 Эн -> +5 к дистанции.
\end{itemize}
\subsection{Ледяной щит}
\index[powers]{Ледяной щит}
\paragraph{} 
\textit{
\tbd Баловство, иногда все же способное принести пользу. Интересно, кто-то умеет призывать вольфрамовый щит?}
\paragraph{Стоимость/Поддержание: }1 Эн/1 Эн
\newline
\textbf{Сопротивление: }15
\newline 
\textbf{Длительность: }1 Сцена.
\paragraph{Эффект: }исказитель вооружается ледяным щитом с БЩ +1. ТВн щита составляет 8. Если при атаке по исказителю противник наносит КУ, Щит разлетается вдребезги. Нападающий получает Колющие и Ледяные Пв, равные БЩ Криогенного щита. То же происходит, если щит намеренно уничтожен. 
\newline Если исказитель выпускает щит из рук, активация завершается.
\paragraph{Усиление:}
\begin{itemize}
\item+1 Эн -> +1 к БЩ. Максимальный БЩ равен 
\textbf{|[МРз исказителя] + 5|}.
\item+1 Эн -> +1 Сцена к Длительности
\end{itemize}
\subsection{Огненная стена}
\index[powers]{Огненная стена}
\paragraph{} 
\textit{
\tbd Чем-чем, а горящим огнем нынче мало кого напугаешь. На то и расчет.}
\paragraph{Стоимость/Поддержание: }2 Эн/1 Эн
\newline
\textbf{Дистанция: }5
\newline
\textbf{Сопротивление: }15 +5 за каждую дополнительную секцию стены.
\newline 
\textbf{Длительность: }1 Сцена.
\newline 
\textbf{Время активации: }
\paragraph{Эффект: }исказитель призывает 1 секцию стены жаркого пламени. Секция имеет объем 1 кубический метр. Любое существо, входящее в пламя, получает Огненные Пв = 
\textbf{|20 – Зщ|} за каждую секцию стены, которую преодолевает. Если существо получает Пв, оно Загорается. Исказитель не может создать стену в воздухе, но может создать ее там, где уже стоит другое существо. В этом случае оно получит Пв.
\paragraph{Усиление:}
\begin{itemize}
\item+1 Эн -> +1 к Пв, которые наносит пламя.
\item+2 Эн -> герой добавляет к стене 1 секцию.
\item+1 Эн -> +5 к дистанции.
\end{itemize}
\subsection{Огненный меч}
\index[powers]{Огненный меч}
\paragraph{} 
\textit{К легкости клинка нужно привыкнуть, но, несмотря на это, он неожиданно удобен.}
\paragraph{Стоимость/Поддержание: }2 Эн/1 Эн
\newline
\textbf{Сопротивление: }15
\newline 
\textbf{Длительность: }1 Сцена.
\paragraph{Эффект: }исказитель призывает огненный меч с БПв +1, КУ 15+, Огненным и Рубящим типом Пв. Меч является Легким, не может использоваться для парирования и не может быть уничтожен ударами. Если при атаке мечом выпадает КП, активация завершается, а исказитель Загорается. 
\newline Если исказитель выпускает меч из рук, активация завершается.
\paragraph{Усиление:}
\begin{itemize}
\item+1 Эн -> +1 к БПв меча.
\item+1 Эн -> меч становится Длинным и получает Упредительный удар.
\item+2 Эн -> +1 Сцена к Длительности.
\end{itemize}
\subsection{Смрад}
\index[powers]{Смрад}
\paragraph{} 
\textit{
\tbd}
\paragraph{Стоимость/Поддержание: }3 Эн/1 Эн
\newline
\textbf{Дистанция: }10
\newline
\textbf{Сопротивление: }15
\newline 
\textbf{Длительность: }для определения Длительности исказитель проверяет Неприятности через число Кругов, равное своему 
\textbf{|МФх|}. 
\tbd
\paragraph{Эффект: }исказитель призывает облако смрадного тумана. Оно должно иметь радиус, не превышающий 
\textbf{|МФх|} исказителя, и возникает в любом месте полностью в пределах Дистанции. Все существа, находящиеся в облаке, Отравлены на число своих Очередей, равное 
\textbf{|5 + [Концентрация(Фх) исказителя] – [Вн существа]|}. В дополнение существа, попавшие под действие феномена,  Оглушены на число Очередей, равное 
\textbf{|[Концентрация(Фх) исказителя] – [Вн существа]|} – их одолевает неудержимая рвота! 
\newline Центр облака может передвигаться со Ск 1. Передвижение облака расходует Быстрое действие исказителя. 
\newline Существа, обладающие Иммунитетом к Заразе, и существа, которым не требуется дышать, невосприимчивы к Смраду.
\paragraph{Усиление:}
\begin{itemize}
\item+1 Эн -> +5 к дистанции.
\item+1 Эн -> +1 к Ск облака.
\item+1 Эн -> +1 к радиусу облака.
\end{itemize}
\subsection{Стена ветра}
\index[powers]{Стена ветра}
\paragraph{} 
\textit{
\tbd Трудно поверить, что эти чудовищные вихри где-то смиренно ждут призыва.}
\paragraph{Стоимость/Поддержание: }2 Эн/1 Эн
\newline
\textbf{Дистанция: }5
\newline
\textbf{Сопротивление: }10, +5 за каждую дополнительную секцию.
\newline 
\textbf{Длительность: }1 Круг.
\paragraph{Эффект: }исказитель призывает 1 секцию вихря. Секция имеет площадь основания 1 квадратный метр и неограниченную высоту. Пространство, заполненное вихрем, не могут преодолеть существа с Силой 12 или меньше, а также существа, использующие полет для передвижения. 
\newline Исказитель может создать Стену бури там, где уже находится другое существо. Попавшее в вихрь существо должно проверить Лв или Атлетику (Лв) против 
\textbf{|15|}, или против 
\textbf{|20|}, если оно летит. В случае успеха существо покидает вихрь. Оно располагается с любой его стороны по своему выбору. При провале проверки существо получает Пв, равные величине провала, становится Ошеломленным, и размещается с любой стороны по выбору исказителя.
\paragraph{Усиление:}
\begin{itemize}
\item+1 Эн -> +1 к Силе, необходимой для преодоления вихря.
\item+1 Эн -> герой добавляет к вихрю 1 секцию.
\item+1 Эн -> +5 к Дистанции.
\item+1 Эн -> +1 Круг к Длительности.
\end{itemize}
\subsection{Стена молний}
\index[powers]{Стена молний}
\paragraph{} 
\textit{
\tbd Неужели в мире осталось столько неизрасходованного электричества?}
\paragraph{Стоимость/Поддержание: }2 Эн/1 Эн
\newline
\textbf{Дистанция: }5
\newline
\textbf{Сопротивление: }10. +5 за каждый дополнительный метр площади.
\newline 
\textbf{Длительность: }1 Круг.
\paragraph{Эффект: }исказитель призывает 1 секцию электрического шторма. Секция имеет площадь основания 1 квадратный метр и неограниченную высоту. Любое существо, входящее в шторм, получает Электрические Пв = 
\textbf{|20 – Зщ|} за каждую секцию стены, которую преодолевает. Существа под воздействием феномена ополовинивают свои БД и БЩ. Если существо теряет 5 и более ЕЗ, оно Загорается. 
\newline Исказитель может создать Стену молний там, где уже находится другое существо, в этом случае оно получит Пв.
\paragraph{Усиление:}
\begin{itemize}
\item+1 Эн -> +1 к Пв, которые наносят молнии.
\item+2 Эн -> Исказитель добавляет к стене 1 секцию.
\item+1 Эн -> +5 к Дистанции.
\item+1 Эн -> +1 Круг к Длительности.
\end{itemize}
\subsection{Телекинез}
\index[powers]{Телекинез}
\paragraph{} 
\textit{
\tbd Жутковатое фрисби существами и предметами, совершенно не подходящими для игры. В команде исказителя – темная материя, гравитационные червоточины и магнитные поля.}
\paragraph{Стоимость/Поддержание: }1 Эн/1 Эн
\newline
\textbf{Дистанция: }5. Дистанция учитывается только в момент выбора цели. Если цель попала под действие феномена, и была перемещена за пределы Дистанции, Телекинез не прерывается, пока исказитель видит цель. 
\newline 
\tbd 5 для захвата цели, 15 максимальная Дистанция перемещения цели.
\newline
\textbf{Сопротивление: }15
\newline 
\textbf{Длительность: }1 Сцена.
\paragraph{Эффект: }Сила Телекинеза составляет 10. 
\newline Для определения Дб и Мт Телекинеза вместо боевых Навыков и Лв используется Ловкость рук(Фх). Сл Телекинеза участвует в формулах, как обычно. БПв Телекинеза равен 0. Каждый феномен Телекинеза считается одной конечностью во всех ситуациях, где это важно. 
\newline Во время действия Телекинеза исказитель может: 
\begin{itemize} 
\item Прибавить 1/2 Сл Телекинеза к своей Сл. 
\item Поднять в воздух одну цель, вес которой не превышает Максимальной нагрузки Телекинеза, в том числе и себя. Чтобы удерживать в воздухе несколько целей, исказитель должен активировать и поддерживать феномен (расходуя Эн, как обычно) несколько раз. 
\item Перемещать захваченных/ не сопротивляющихся существ и предметы на расстояние, равное Ловкости рук (Фх) (минимум на 1), израсходовав Быстрое действие. 
\item Перемещаться на расстояние, равное Атлетике(Фх) (минимум на 1), израсходовав Перемещение. 
\item Применять маневры, израсходовав Действие. С помощью Телекинеза герой не может совершать Разбег и Прицеливание, но прочие маневры допустимы – если Телекинез предварительно захватил подходящее оружие. 
\item Метать Захваченные цели. Они считаются Метательным оружием с БПв = МРз и Дистанцией 5/10; 
\item Сменить цель или метод применения Телекинеза; 
\item Активировать и поддерживать несколько феноменов Телекинеза одновременно, в том числе для взаимодействия с несколькими целями. 
\end{itemize}
\paragraph{Усиление:}
\begin{itemize}
\item+1 Эн -> +4 к Сл Телекинеза.
\item+1 Эн -> +1 к Дистанциям броска Телекинеза.
\item+1 Эн -> +5 к Дистанции.
\item+1 Эн -> +1 к скорости передвижения Телекинезом предметов и существ.
\end{itemize}

