
\subsection{Вторичный цикл}
\index[powers_ritual]{Вторичный цикл}
\paragraph{} 
\textit{Можно сколько угодно твердить, что феномен всего лишь ненадолго реанимирует роботов и аугментику давно погибших солдат. Но как это объясняет оживление свежих трупов?}
\paragraph{Стоимость/Поддержание: }20 Эн/10 Эн за каждые 10 умертвий под контролем исказителя. Неполный десяток считается за целый.
\leavevmode
\newline 
\textbf{СП: }
\begin{itemize} 
\item 15 за каждые 10 умертвий, которых исказитель планирует «оживить». 
\item 10 для захвата контроля или перехвата контроля над любым числом уже оживленных умертвий. 
\end{itemize}
\leavevmode
\newline 
\textbf{Сложность Подготовки: }20
\leavevmode
\newline 
\textbf{Сопротивление: }исказитель должен преодолеть: 
\begin{itemize} 
\item 
\textbf{|10 + Вл умертвия|} для подчинения разумных или блуждающих умертвий; 
\item 
\textbf{|10 + [Концентрация(Фх) соперника|]} для перехвата контроля над умертвиями у другого исказителя; 
\end{itemize}
\leavevmode
\newline 
\textbf{Время активации: }исказителю потребуется: 
\begin{itemize} 
\item 1 Сцена для «оживления» мертвецов вокруг; 
\item Израсходовать Действие для подчинения блуждающих умертвий или перехвата контроля над ними у другого исказителя. 
\end{itemize}
\leavevmode
\newline 
\textbf{Длительность: }До следующего Антракта.
\leavevmode
\newline 
\textbf{Дистанция: }5
\paragraph{Эффект: }мощь исказителя возвращает останки к отвратительной пародии на жизнь. Умертвия, пробужденные заклинанием, собираются на 40 Очков характеристик и имеют 3 Очка опыта. Ни одна Характеристика умертвия не может превышать 18. Все мертвяки имеют одинаковые Характеристики (хотя вы можете обсудить с мастером и соигроками возможность уникального создания каждого, или даже использование блоков Характеристик из части «Монстры и статисты»). Силу, ловкость и выносливость мертвецам придает феномен, поэтому скелет кота с 18 Силой и 18 Ловкостью – вполне приемлемый вариант. Останки боевого робота ничем не будут отличаться от трупа курицы, кроме размера и угрожающего вида. 
\newline Одновременно под контролем исказителя может находиться Х мертвяков, где Х = 
\textbf{|Концентрация(Фх)|}. Мертвяки не могут отходить от сотворившего их дальше, чем на 
\textbf{|Дистанция*5|} метров. 
\newline Управление мертвяками (всеми сразу) расходует Действие. Исказитель может: 
\begin{itemize} 
\item Видеть глазами любого из умертвий под его контролем и слышать тем, что заменяет им органы слуха. 
\item Отдавать умертвиям приказы, проверив Концентрацию(Ин) против 
\textbf{|15|}. Он получает бонус или штраф, равный 
\textbf{|МИн|} умертвия. Умертвие действует автономно до получения нового приказа. 
\item Задавать умертвиям вопросы. Если у умертвия достаточно высокий Ин, оно может связно ответить на них. Чтобы отвечать на вопросы о своей прошлой жизни, умертвие должно преуспеть в проверке Ин против 
\textbf{|10 + [количество дней с момента смерти]|}. Таким образом, имеет смысл расспрашивать мертвецов посвежее. 
\end{itemize} 
\paragraph{Цена ошибки: }проверка Неприятностей: 
\trouble {Их сон}{Умертвия падают замертво, лишенные энергетической подпитки хозяина. Они вполне пригодны для новой активации Вторичного цикла.} {Их покой}{Умертвия рассыпаются в прах, который больше не потревожит даже самый могущественный исказитель.} {Их воля}{Умертвия вырываются из-под контроля исказителя и обретают свободу воли, но пока лишь разбредаются по окрестностям. Возможно, у кого-то из них остались незавершенные дела.} {Их гнев}{Сбросив ярмо, умертвия жаждут отомстить исказителю, нарушившему их покой. Они преследуют и атакуют его всеми доступными им способами. 
\newline При истечении Длительности или Прерывании по желанию активатора, используйте результат «Их сон».}
\paragraph{Усиление:}
\begin{itemize}
\item+1 СП -> +5 к Дистанции
\item+2 СП -> продлевает Длительность до еще одного Антракта.
\item+1 СП -> максимальное число умертвий под контролем исказителя возрастает на 5.
\item+1 СП -> -1 к сложности проверки Ин для отвечающего на вопросы мертвеца.
\item+1 СП -> мертвяки используют Навыки контролирующего их исказителя. Разделите Навык исказителя на 4, чтобы определить значение Навыка мертвяка.
\item+1 СП -> 1 умертвие получает свободу воли. Оно может удаляться от исказителя на любое расстояние и использоваться для отсчета Дистанции Вторичного цикла. Исказителю все еще нужно контролировать «свободного» мертвяка, чтобы направлять его действия.
\item+1 СП -> умертвия получают +5 Очков характеристик в момент призыва.
\item+1 СП -> умертвия получают +2 Очка опыта в момент призыва.
\item+ СП -> 
\end{itemize}
\subsection{Ноосфера}
\index[powers_ritual]{Ноосфера}
\paragraph{} 
\textit{Из ниоткуда приходят не только странные предметы и существа. Информацией там тоже можно разжиться – если знать, где искать.}
\paragraph{Стоимость: }20 Эн
\leavevmode
\newline 
\textbf{СП: }15
\leavevmode
\newline 
\textbf{Сложность Подготовки: }10
\leavevmode
\newline 
\textbf{Сопротивление: }15
\leavevmode
\newline 
\textbf{Время активации: }1 Антракт.
\paragraph{Эффект: }исказитель задает вопрос – и получает ответ. Вопрос может быть любым (в том числе комплексным), но мастер вправе не давать исчерпывающий ответ – видения и знаки зачастую туманны и неоднозначны. Ведь, как известно, прошлое – история, а будущее – теория!
\paragraph{Цена ошибки: }Исказитель получает ложный, потенциально опасный ответ. Чтобы определить это, он должен преуспеть в Наблюдательности(Мд) или Науке(Мд).
\paragraph{Усиление:}
\begin{itemize}
\item+1 СП -> исказитель видит прошлое места – включая многое из того, что случилось в нем. Для этого он должен преуспеть в Концентрации против 
\textbf{|15 + [количество лет, прошедших с момента, который хочет увидеть исказитель]|}.
\item+2 СП -> исказитель может задать вопрос о местонахождении какого-либо объекта или существа. В течение 1 Сцены исказителя влечет в ту сторону, где расположен объект или существо. Для того чтобы правильно сориентироваться, исказитель должен преуспеть в проверке Выживания(МФх) против 
\textbf{|15|}. За каждые 10 метров, разделяющие его и цель поиска, сложность возрастает на 1.
\end{itemize}
\subsection{Оплот}
\index[powers_ritual]{Оплот}
\textbf{Форма: }Область(Периметр)
\paragraph{} 
\textit{Немного уюта и безопасности в любом нужном месте.}
\paragraph{Стоимость/Поддержание: }30 Эн/10 Эн
\leavevmode
\newline 
\textbf{СП: }10
\leavevmode
\newline 
\textbf{Сложность Подготовки: }10
\leavevmode
\newline 
\textbf{Время подготовки: }10
\leavevmode
\newline 
\textbf{Время активации: }1 Сцена.
\leavevmode
\newline 
\textbf{Длительность: }1 Сцена (которая может быть Антрактом или Интерлюдией).
\leavevmode
\newline 
\textbf{Дистанция: }3
\paragraph{Эффект: }исказитель укрывает выбранную область мутным силовым куполом. Проникнуть сквозь него затруднительно. Статист не может сделать этого, если его  Наука, Эксплуатация или Концентрация меньше его Науки, Эксплуатации, или Концентрации активатора Оплота. Персона должна победить активатора в Состязании Науки, Эксплуатации, или Концентрации. 
\newline Активатор чувствует, когда существо или предмет Крошечного и более размера проникает в Оплот, и знает, где именно это произошло.
\paragraph{Цена ошибки: }Атональный гул и искристые вспышки на внутренней стороне купола мешают расслабиться и заснуть. Чтобы получить эффекты Отдыха, герой должен преуспеть в проверке Вл против 
\textbf{|15|}.
\paragraph{Усиление:}
\begin{itemize}
\item+1 СП -> +1 к сложности проверки для тех, кто пытается проникнуть в Оплот.
\item+1 СП -> +1 к Дистанции.
\item+1 СП -> сверхъестественное перемещение из Оплота возможно только для активатора.
\item+1 СП -> Крошечные существа стараются покинуть периметр.
\item+1 СП -> Оплот остается прозрачным изнутри.
\end{itemize}
\subsection{Регенерация}
\index[powers_ritual]{Регенерация}
\textbf{Форма: }()
\paragraph{} 
\textit{Исказитель вызывает стремительный, но полностью контролируемый процесс регенерации тканей. Активируя клеточную память, он способен восстановить даже утраченные части тела.}
\paragraph{Стоимость[РИЗ]: }60 Эн
\leavevmode
\newline 
\textbf{СП: }10
\leavevmode
\newline 
\textbf{Сложность Подготовки: }10
\leavevmode
\newline 
\textbf{Сопротивление: }
\textbf{|10 + Вл|}, если цель не желает попасть под эффекты.
\leavevmode
\newline 
\textbf{Время активации: }1 Антракт
\paragraph{Эффект: }цель восстанавливает утраченные части тела, без лишаев, родинок, шрамов и татуировок, с молодой и здоровой кожей. 
\newline Также с помощью Регенерации исказитель может избавить цель от шрамов и злокачественных опухолей.
\paragraph{Цена ошибки: }проверка Неприятностей: 
\trouble {Оттиск}{Цель Ритуала обнаруживает на своей новой конечности или в глубине зрачка замысловатый, не лишенный изящества символ.} {Сбой}{Цель получает скрюченную лапку вместо руки или ноги, бесформенный нарост вместо носа или уха, сморщенный кусок желе вместо глаза и т.д. 
\newline В дальнейшем Регенерация может быть вновь применена на той же области.} {Искажение}{Цель получает недостаток «Урод» и скрюченную лапку вместо руки или ноги, бесформенный нарост вместо носа или уха, сморщенный кусок желе вместо глаза и т.д. 
\newline В дальнейшем Регенерация может быть вновь применена на той же области, но от Уродства не избавит. В этом случае лучше ампутировать Уродливую часть тела и применять Регенерацию "с чистого листа".} {Неконтролируемый мутагенез}{Герой понижает на 3 максимальные ЕЗ, отнимает 1 от любой из основных Характеристик и 1 от любой из Вторичных Характеристик. Затем игрок может милосердно позволить герою сдохнуть, либо повысить/понизить его МРз на 1 и заменить все Атрибуты на Измененного. Герой сохраняет Очки опыта в НавыкахЭ, но не может повышать их, пока не получит к ним доступ благодаря приобретению новых Атрибутов. Если у героя уже есть Атрибут «Измененный», он теряет все Атрибуты, кроме этого, и получает 1 новый феномен за каждый потерянный Атрибут.}
\paragraph{Усиление:}
\begin{itemize}
\item+ СП -> 
\item+ СП -> 
\end{itemize}
\subsection{Телеметрия}
\index[powers_ritual]{Телеметрия}
\paragraph{} 
\textit{В сознании исказителя возникает яркая и четкая картина происходящего вдали.}
\paragraph{Стоимость/Поддержание: }10 Эн/5 Эн
\leavevmode
\newline 
\textbf{СП: }15
\leavevmode
\newline 
\textbf{Сложность Подготовки: }15
\leavevmode
\newline 
\textbf{Сопротивление Наведению: }
\textbf{|10 + Вл|}
\leavevmode
\newline 
\textbf{Время активации: }1 Сцена
\leavevmode
\newline 
\textbf{Длительность: }до следующей Интерлюдии.
\leavevmode
\newline 
\textbf{Дистанция: }1 километр.
\paragraph{Эффект: }исказитель наблюдает за целью. Он видит все, происходящее в радиусе метра от нее. Если феномен не срабатывает, исказитель узнает, почему это произошло: или цель успешно сопротивлялась чарам, или он искал не там, или получил недостаточно информации о цели. Провал Концентрации не позволяет исказителю точно установить местонахождение цели, только примерную область, где она была на момент активации Телеметрии.
\paragraph{Цена ошибки: }Исказитель видит цель, но и цель видит его и знает, что за ней наблюдают сверхъестественным образом.
\paragraph{Усиление:}
\begin{itemize}
\item+1 СП -> +100 метров к дистанции или +1 метр к радиусу видимой области.
\item+1 СП -> -1 к Сопротивлению Наведению.
\item+2 СП -> если Телеметрия успешно активирована, оборвав 1 Нить, игрок позволит исказителю активировать другие феномены в области, которую тот видит. При этом исказитель игнорирует Дистанцию. Стоимость активации в Эн оплачивается отдельно от стоимости Телеметрии.	
\end{itemize}
\subsection{Телепортация}
\index[powers_ritual]{Телепортация}
\paragraph{} 
\textit{
\tbd}
\paragraph{Стоимость: }30 Эн
\leavevmode
\newline 
\textbf{СП: }20
\leavevmode
\newline 
\textbf{Сложность Подготовки: }15
\leavevmode
\newline 
\textbf{Сопротивление: }
\textbf{|10 + Вл|}, если цель не желает попасть под эффекты.
\leavevmode
\newline 
\textbf{Сопротивление Наведению: }20
\leavevmode
\newline 
\textbf{Дистанция: }1
\paragraph{Эффект: }исказитель перемещает цель в точку в пределах 1 километра от точки ее текушего положения. Для выбора точки, которую исказитель не видит, потребуется проверка Наведения.
\paragraph{Цена ошибки: }проверка Неприятностей: 
\trouble {Насыщение атмосферы электричеством}{В воздухе повис запах близкой грозы.} {Самое необходимое}{Цель прибывают в точку выхода полностью обнаженной. Все предметы, являющиеся частью тела цели (в том числе, имплантаты) остаются при ней.} {Отклонение}{Цели получают Расплату «Невыгодная позиция».} {Сбой координат}{Цели получают Расплату «Герои под ударом».}
\paragraph{Усиление:}
\begin{itemize}
\item+2 СП -> исказитель оставляет за собой червоточину, соединяющую точку отправки и точку прибытия. 
\newline Червоточина функционирует 1 Сцену за каждые 2 СП. В течение 1 Круга в червоточину может проникнуть 1 существо размером не больше Среднего.
\item+1 СП -> +1 к МРз существ, способных пройти сквозь червоточину.
\item+2 СП -> точка прибытия движется вместе с предметом, на котором находится. Например, она может прикрепиться к мешку в едущей телеге или стене кабины летящего конвертокрыла.
\item+1 СП -> +1 километр к максимально возможному расстоянию перемещения.
\item+1 СП -> +1 к Дистанции.
\item+1 СП -> +1 существо которое телепортируется вместе с целью.
\end{itemize}

