\chapter{Богатство, торговля и имущество}
Здесь подробно говорится о клинках, битах, пушках, броне, тачках, ездовых тварях, лекарствах, услугах и развлечениях. А еще о том, как обменять всякую рухлядь на полезные штуки, о том, как быстро они выходят из строя и, конечно же, о том, как использовать их на полную.
\begin{tcolorbox}
    Правила предполагают, что герой имеет при себе необходимые ему предметы туалета и личной гигиены, важные безделушки и снаряжение, необходимое для путешествия. Если в вашей игровой команде играет важную роль подбор и детальный учет предметов, герои приобретают их по обычным правилам, используя СП из раздела "Предметы: Всякая всячина".
\end{tcolorbox}

\section{Богатство и торговля}
Звонкие монеты, ценные материалы и сверкающие каменья... Ради них герои отправляются в опасные походы, ими платят ремесленнику и брадобрею, их бросают к ногам красавиц и могучих правителей. Хотя некоторые герои недальновидно прячут Богатство в сундук и закапывают в землю. Но речь, конечно же, пойдет не о них.
\newline Богатство отражает финансовое благосостояние героя. В ходе игры Богатство может возрастать или уменьшаться.
\paragraph{Начальный уровень Богатства героя:} 5.
\paragraph{Минимальный уровень Богатства:} 0.
\paragraph{Будешь должен.} Если в результате снижения Богатство героя должно опуститься ниже минимального уровня, оно становится равным 0, а герой получает Недостаток "Долг", "Враги" или "Преступник". Если у него уже был один из этих недостатков, он немедленно входит в игру.
\paragraph{Пополнение Богатства:} игрок вправе увеличить Богатство героя с помощью траты Очков опыта (даже в начале игры), однако оно может вырасти благодаря продаже ценностей. Некоторые Атрибуты и Трюки влияют на начальный уровень Богатства или дают временные бонусы к нему.
% \paragraph{Уровни Богатства:}
% \paragraph{0 - Нищий.} Герою нечем заплатить ни за черствый хлеб, ни за самую дешевую ночлежку. Герой не может приобретать услуги и предметы с СП 11 или больше.
% \paragraph{1-4 - В долгах, как в шелках.} Герой живет в режиме жесткой экономии. Мясо на его столе - редкий гость.
% \paragraph{5-10 - Средний класс.} Герой может побаловать себя время от времени... Впрочем, не слишком часто. Он все еще латает свою одежду вместо того, чтобы купить новую.
% \paragraph{11-15 - Пошел в гору.} Герой твердо стоит на ногах. Он может без проблем позволить себе излишества в еде и одежде.
% \paragraph{16-20 - Толстосум.} Герой способен оплатить работу лучших ремесленников, приобретать породистых лошадей, борзых и произведения искусства. Также его не сильно обременит наем хорошенькой горничной или привратника-ветерана. за состоянием счетов - это делают многочисленные приказчики. Особняк, карета, надомный лекарь, личная стража и красивые любовницы прилагаются.
% \paragraph{31 и более - Купается в золоте.} Тот самый момент, когда герой подумывает о покупке маленькой уютной страны или найме армии для ее захвата!
\paragraph{Проверка Богатства:} товары и услуги имеют \textbf{Сложность приобретения (СП)}. СП отражает абстрактную ценность предмета или услуги.
\paragraph{Провал проверки Богатства} означает, что герою не удалось договориться о сделке. Он не приобретает желаемый предмет, но и не понижает свое Богатство.
\paragraph{Понижение Богатства при покупке.} При успехе проверки герой получает желаемое, а его Богатство понижается в соответствии с таблицей.
\begin{center}\begin{tabular}{ |c|c| }\hline
    \textbf{СП покупки} & \textbf{Снижение Богатства} \\ \hline
    15 или больше & дополнительно -1 \\ \hline
    на 1-5 больше, чем Богатство героя & -1 \\ \hline
    на 6-10 больше, чем Богатство героя & -2 \\ \hline
    на 11-15 больше, чем Богатство героя & -4 \\ \hline
    на 16 и больше, чем Богатство героя & -8 \\ \hline
\end{tabular}\end{center}
\paragraph{Неслыханная щедрость:} герой получает Преимущество на проверку Богатства, но при успехе дополнительно теряет 1 Богатство.
Если СП товара или услуги меньше или равен Богатству героя, бросок не требуется - герой просто получает желаемое. Он все еще теряет 1 Богатство, если СП покупки 15 и больше.
\paragraph{Быстрая покупка} является Эффективным показателем Богатства, т.е. проверка не совершается, а ее результатом считается 10.  Предполагается, что используя Быструю покупку, герой взвешивает все «за» и «против», и платит сразу, не торгуясь. Герой с Богатством 0 не может сделать этого.
\paragraph{Покупка без проверки Богатства:} когда СП товара или услуги равна Богатству героя или меньше его, бросок не требуется — герой просто получает желаемое. Он все еще теряет 1 Богатство, если СП покупки 15 и больше.
\paragraph{Карманные расходы и их учет:} если герой не потерял Богатство при покупке товара или услуги, это значит, что он использовал рухлядь из своего кармана. Но рухлядь в караманах порой заканчивается. 
\newline Величина счетчика Карманных расходов равна значению Богатства героя. Если герой совершил больше покупок на Карманные расходы, чем его начение Богатства, он тут же теряет 1 Богатство, а счетчик Карманных расходов обнуляется. Так же счетчик обнуляется, если герой уходит в Антракт.
\paragraph{Покупка в складчину:} герои могут подкинуть друг другу барахлишка по правилам Взаимопомощи. Помощники теряют Богатство по обычным правилам. Не забывайте, что для них СП покупки меньше на 5.

\subsection{Покупки в начале игры:}
В начале игры герои покупают все необходимые предметы по отдельности - игнорируйте правила Комплексной покупки. Приобретение предметов происходит после распределения Очков опыта. В начале игры все снаряжение герой приобретает по правилам Быстрой покупки.

\subsection{Комплексные покупки}
Иногда герой вынужден внушительно раскошеливаться в течение Сцены – например, когда снаряжает охотничий отряд, готовит экспедицию на древнюю базу или подкупает толпу жадных бюрократов. 
\newline Комплексная покупка является Испытанием. Для ее совершения вам понадобится:
\begin{itemize}
    \item[--] Определить участников Испытания.
    \item[--] Определите суммарную СП товаров и услуг.
    \item[--] Игроки сами выбирают Счетчик победы – любое число (мастер вправе ограничить величину этого числа, ведь обычно время поджимает).
    \item[--] Разделите суммарную СП товаров и услуг на Счетчик победы
    \item[--] Полученное при делении число – Сложность каждой из Развивающих проверок. Счетчик поражения устанавливает мастер.
    \item[--] Приступайте к совершению проверок, как при обычном Испытании:
        \begin{itemize}
            \item[$\bullet$] \textbf{Для Допускающих} действий потребуются Общение, Торговля или Богатство. Это изображает подготовку сделки и ее «сопровождение» – устройство встречи, посулы, мелкие услуги, подарки, флирт, лесть, ложь, угрозы, слезные мольбы и т.д.
            \item[$\bullet$] \textbf{При Развивающих} действиях герои проверяют Богатство. Каждая проверка Богатства вызывает немедленное понижение его значения в соответствии с правилами. Если Испытание будет провалено, потери Богатства не возмещаются. Возможно, героев ловко обжулили, либо они поиздержались, умасливая продавца, либо задаток не возвращается по условиям сделки.
            \item[$\bullet$] \textbf{Сберегающие} действия представляют собой способы дополнительно сбить цену в процессе сделки и уговоры ненадолго придержать товар. Скорее всего, для этого героям понадобится Общение, Торговля и уместные в ситуации Атрибуты и Трюки.
        \end{itemize}
\end{itemize}

\begin{tcolorbox}
    Даже одиночные предметы с высокой СП могут приобретаться, как Комплексная покупка – до тех пор, пока это насыщает историю действием и деталями.
\end{tcolorbox}

\paragraph{Цена провала:} в случае неудачи Комплексной покупки герои  все еще могут получить жклаемое - если выберут Расплату за каждую единицу, на которую счетчик Победы меньше Счетчика поражения.
\paragraph{Варианты Расплаты:}
\begin{itemize}
    \item[--] Некомплект. В покупках недостает важных мелочей, но в суматохе никто этого не замечает.
    \item[--] Некачественные товары. Техника рассыпается в руках, а от припасов пованивает. Странно, почему на это не обратили внимания при оплате?
    \item[--] Происки недругов. Информация о том, что группа готовится к чему-то масштабному, расползается по округе. Это может привести к появлению конкурентов в погоне за сокровищами, попытке ограбления, и другим подобным проблемам.
    \item[--] Недостаток важного оборудования. Критически важные для героев предметы отсутствуют или неисправны. И ведь герои точно помнят, что все это паковали и проверяли!
    \item[--] Задержки поставок. Товар в наличии, но надо немного подождать. Достаточно впрочем, долго, чтобы это создало героям проблемы. Нет-нет, выставочные образцы не продаются!
\end{itemize}

\subsection{Продажа предметов и услуг}
Cовершается следующим образом: 
\begin{enumerate}
    \item Определите СП предмета или услуги.
    \item Повысьте Богатство продающего героя на столько, на сколько он понизил бы его значение, успешно купив предмет с получившимся СП.
\end{enumerate}
\begin{tcolorbox}
    Обычно конекст достаточно четко говорит о том, может ли статист купить предмет или услугу. В случае сомнений, воспользуйтесь проверкой Неприятностей. Общение и Выступление пригодятся любителям агрессивной рекламы. Сама продажа не требует каких-либо проверок.
\end{tcolorbox}
\paragraph{Комплексные продажи:} если герою нужно продать большую партию товаров сразу, их СП складывается. При этом СП набора снижается согласно таблице:
\begin{center}\begin{tabular}{ |c|c| }\hline
    \textbf{Суммарное количество предметов} & \textbf{Общее снижение СП} \\ \hline
    2 предмета & -3 \\ \hline
    3 предмета & -6 \\ \hline
    4 предмета & -10 \\ \hline
    5 предметов & -15 \\ \hline
    6 предметов & -21 \\ \hline
    7 предметов & -28 \\ \hline
    8 предметов & -36 \\ \hline
    9 предметов & -45 \\ \hline
    10 и больше предметов & -55 \\ \hline
\end{tabular}\end{center}
Затем следуйте обычным правилам продажи. Продавать предметы большими партиями редко бывает выгодно, если у героя уже достаточно большое значение Богатства.
    
\paragraph{Делим на всех!} СП предмета/предметов при продаже может быть разделен на любое количество долей, если сразу несколько героев претендуют на процент с продажи. Раздел СП происходит после понижения СП за факт продажи и износ.

\subsection{Бартер}
В мире есть места, где Богатство героев не имеет значения. Аборигенов не интересуют стеклянные бусы, галстуки, кружевные чулочки и непристойные картинки (вообще-то, очень даже интересуют, но статисты не готовы отдать за них полезные вещи). О таких придумках, как векселя и валюта, они и слышать не хотят. Здесь место торговли занимает старый добрый Бартер.
\paragraph{Проверка} Бартера совершается следующим образом:
\begin{enumerate}
    \item Определите суммарную СП предлагаемых героем товаров и услуг(СП продажи).
    \item СП продажи на число предметов (но не услуг) в предложении.
    \item Определите суммарную СП приобретаемых у статиста товаров и услуг(СП покупки).
    \item Совершите проверку \textbf{|СП продажи|} против \textbf{|10+СП покупки|}
\end{enumerate}
Провал проверки Бартера означает, что герой и статист не договорились, и сделка не состоится. Повторные проверки с участием тех же предметов невозможны (в этой Сцене так уж точно), но герой вправе попробовать обменять на желаемое другие вещи из своих закромов.
\begin{tcolorbox}
    Проверка Бартера является проверкой Богатства во всех отношениях. За одним исключением - при ее успехе или провале герои не повышают и не понижают значение своего Богатства.
\end{tcolorbox}

\paragraph{}Герой может использовать навык Торговли при Бартере, но тогда сложность всех проверок возрастает на 5. Торговцу, привыкшему говорить языком денег будет труднее убедить тех, кто концепцию денег не понимает и не принимает.

\subsection{Торг уместен}
Успешная проверка Торговли позволяет сделать одно из следующего (по выбору игрока):
\begin{itemize}
\item[--] Получить Преимущество на проверку Богатства.
\item[--] Дать Преимущество на проверку Богатства другому герою, если герой с Торговлей помогает торговаться.
\item[--] Повысить Богатство на дополнительный 1 в случае успеха проверки, если герой продает или помогает продавать.
\item[--] Уменьшить потерю Богатства на 1 в случае успеха проверки, если герой покупает или помогает покупать.
\end{itemize}
Провал проверки Торговли приводит к одному из следующего (по выбору игрока):
\begin{itemize}
\item[--] Герой получает Помеху на проверку Богатства.
\item[--] Другой герой получает Помеху на проверку Богатства, если герой с Торговлей помогает торговаться.
\item[--] Богатство повышается на 1 меньше, чем должно в случае успеха проверки, если герой продает или помогает продавать. В худшем случае герой потеряет предмет и не повысит свое Богатство.
\item[--] Потеря Богатства увеличивается на 1 в случае успеха проверки, если герой с Торговлей покупает или помогает покупать.
\end{itemize}

\subsection{Доступность товаров и услуг}
Далеко не всегда герои могут приобрести необходимое, даже если есть чем расплатиться. Если требуется выяснить, доступны ли нужные героям товары или услуги, совершите проверку Неприятностей.
\trouble
{Избыток}{Статист охотно продаст товар или возьмется за услугу – он не испытывает в товаре недостатка, а услуга не слишком его обременяет. Используйте СП из таблицы.}
{Наличие}{Несколько экземпляров товара есть в продаже. Статист расстанется с товаром или возьмется за услугу за честную цену. Повысьте СП на 2.}
{Дефицит}{Статист владеет товаром или необходимыми для услуги навыками, но не желает продавать товар или оказывать услугу – разве что за крупный барыш. Повысьте СП в 2 раза.}
{Нет}{У статиста нет необходимого товара или он не может оказать услугу.}

\section{Имущество}
\section{Имущество}
В этом разделе будут раскрыты характеристики и механики, присущие любому имуществу, которым могут владеть герои, будть то автомобиль, кинжал или скакун.

\paragraph{Сложность приобретения (СП):} Большинство имущества можно приобрести или продать за хорошую цену. Сложность приобретения определяет абстрактную стоимость снаряжения и аммуниции. Если СП не указана, то снаряжение нельзя приобрести в магазинах - либо оно слишком дешево, чтобы торговцы озаботились иметь эти вещи на складе, либо это уникальные образцы, приобретение которых может быть отдельным приключением!

\paragraph{Осечка} определяет, насколько ненадежным является снаряжение из-за его конструктивных особенностей или же из-за ненадлежащей эксплуатации. При любых проверках с участием снаряженияжения с Осечкой, если на кубике выпало число меньше или равное значению Осечки, проверка автоматически считается проваленной. Если герой использует несколько устройств с Осечкой, то общая Осечка для проверки равна максимальной Осечке среди устройств.
\newline Если герой носит броню с Осечкой, то она действует на все Активные Проверки героя.

\paragraph{Изъяны} Вещи указывают на любое негативное влияние Вещи на его владельца. Это может быть как незначительное неудобство, так и серьезное проклятие. Они делятся на категории по частоте появления:
\begin{enumerate}
\item \textbf{Ноша}. Эффекты Изъяна проявляются даже если герой носит свою Ношу в рюкзаке.
\item \textbf{Издержка}. Эффекты Изъяна проявляются каждый раз, когда герой совершает проверку с использованием этого снаряжения.
\item \textbf{Осечка}. Эффекты Изъяна проявляются только когда при проверке с использованием снаряжения выпала Осечка или Критический Провал.
\item \textbf{Глюк}. Эффекты Изъяна проявляются только при определенных условиях, которые указаны в описании снаряжения.
\end{enumerate}

\paragraph{Вес снаряжения для больших и маленьких существ}
Во всех таблицах указан вес снаряжения для существ Среднего размера. Чтобы определить во сколько раз изменится вес предмета, изготовленного для существа другого размера, сверьтесь с таблицей.
\begin{center}
\begin{tabular}{|c|c|}
\hline
Размер Существа & Множетитель веса \\ \hline
Крошечный & 1/3 \\ \hline
Маленький & 2/3 \\ \hline
Средний & 1 \\ \hline
Большой & 4/3 \\ \hline
Огромный & 5/3 \\ \hline
Громадный & 2 \\ \hline
\end{tabular}
\end{center}

\subsection{Износ и ремонт снаряжения}
Даже самое надежное устройство при ненадлежащей эксплуатации и недостаточном уходе начнет сбоить.
\paragraph{Проверка Износа} является Проверкой неприятностей, которая определяет, насколько ухудшилось состояние снаряжения во время эксплуатации.
\newline Снаряжение со свойством \textbf{Надежное} позволяет совершать эту проверку с Преимуществом.
\newline Снаряжение со свойством \textbf{Хрупкое} добавляет Помеху на проверку Износа.
\begin{tcolorbox}
Проверки Износа происходит каждый раз, когда герой совершает Критический Промах, используя снаряжение или оружие, а так же проверки Износа могут быть явно указаны в описании ситуаций, требующих их. Однако ведущий может потребовать проверку Износа в конце Сцены, если по его мнению герой слишком уж сильно издевается над своей экипировкой.
\end{tcolorbox}
\trouble
{Надежная штука}%no sweat name
{Устройство счастливо избежало неполадок и может быть использовано в дальнейшем.}%no sweat description
{Заело}%tough day name
{Небольшая проблема в механизме. Любой герой разберется с ней за 10 минут. Для того чтобы вернуть работоспособность устройства, совершите проверку Эксплуатации против 15. Устройство может быть использовано после проверки в любом случае, но в случае провала проверки его Осечка возрастает на 1.}%tough day description
{Заклинило}%we have trouble name
{Серьезная проблема в механизме. Для того чтобы вернуть работоспособность устройства, совершите проверку Эксплуатации против 20. Устройство может быть использовано после проверки в любом случае, но в случае провала проверки его Осечка возрастает на 2, а в случае успеха - на 1.}%we have trouble description
{Капитальная поломка.}%fiasco name
{Устройство не может быть использовано до ремонта, а его Осечка возрастает на 5.}%fiasco description
\paragraph{Ремонт Осечек} снаряжения доступен только в том случае, если Осечка была получена из-за Износа. Если Осечка является свойством нового снаряжения, избавиться от нее может только Изобретатель с помощью хода Эврика.
\newline
При ремонте осечек снаряжения герой должен совершить проверку Ремонта или Эксплуатация против \textbf{|15+Осечка|} и потратить материалы с СП, равныой величине Осечки. В случае провала проверки материалы не возвращаются. В случае критического провала снаряжение приходит в негодность и больше не полежит ремонту.
\begin{tcolorbox}
Если герой ремонтирует снаряжение, которое имело свойство Осечки до того, как она возрасла во время эксплуатации, герою все равно нужно учитывать в формулах полную Осечку снаряжения. Это значит, что механизм устройства настолько сложный, что его непросто ремонтировать.
\end{tcolorbox}
\paragraph{Ремонт Потери ЕЗ} снаряжения требует совершения проверки Ремонта против \textbf{|15+Потерянные ЕЗ|} и потратить материалы с СП, равныой величине количеству потерянных ЕЗ. В случае провала проверки материалы не возвращаются. В случае критического провала снаряжение приходит в негодность и больше не полежит ремонту.
\paragraph{Время ремонта:} герой ремонтирует одну вещь в течении Антракта. Если сложность ремонта выше 15, то за каждую дополнительную еденицу сложности ему придется провести один дополнительный Антракт за ремонтом.

\subsection{Энергия и топливо}

Энергия - это кровь цивилизации, в какой бы форме она не проявлялась. Запасы топлива в цистернах, магические кристаллы, излучающие энергию, емкие батареи - эти источники использовались и будут использоваться для того, чтобы разнообразные устройства, средства передвижения и вооружение продолжали работать.

\paragraph{СП Топлива:} относительно невелика - СП 5 за 100 зарядов. Топливо может применяться для заправки транспорта. Чтобы преобразовать Топливо в Энергию для оружия, устройств, или восполнения Эн героя, потребуется специальный генератор.

\paragraph{Экзотическое топливо:} некоторые предметы требуют особого топлива, которое не распространено широко в мире. Правила расхода и приобритения топлива в этих случаях будут указаны отдельно.

\paragraph{Заряды Топлива и Энергии} транспорт, устройства и оружие потребляют Заряды (Топлива и Энергии соответственно).

\paragraph{Энергия и Вторичная характеристика Энергии} Эн героев и статистов является гораздо более крупной единицей измерения, чем Заряды Топлива и Заряды Энергии. В 1 Эн содержится 100 Зарядов Энергии (Зр). Для передачи Эн в устройства и обратно требуется специальный адаптер питания.
\paragraph{КПД передачи.} Адаптеры питания не идеальны и пре конвертации Энергии в Заряды и обратно возникают потери. КПД определяет, сколько действительно Зарядов или Эн получит герой после конвертации.
\paragraph{Потребление Х:} устройство потребляет \textbf{|Х|} Зр за указанный период работы.

\paragraph{Энергетические ячейки:} энергетическое оружие и устройства питаются универсальными батареями, емкость которых зависит от размера. После использования всех Зарядов батарею нельзя перезарядить.
\begin{center}
\begin{tabular}{|c|c|c|c|}
\hline
Количество Зарядов & Размер батареи & Вес батареи & СП \\ \hline
10 & Пластинка (1 см3) & 0.1 & 3 \\ \hline
30 & Миниатюрная (5 см3) & 1 & 6 \\ \hline
100 & Крошечная (15 см3) & 3 & 9 \\ \hline
300 & Маленькая (40 см3) & 7 & 12 \\ \hline
500 & Средняя (70 cм3) & 15 & 15 \\ \hline
\end{tabular}
\end{center}

\subsection{Оружие}
От выбора оружия зависит жизнь или смерть героя. Впрочем, иногда оружие - всего лишь элемент социального статуса или стильное дополнение к костюму.
\paragraph{Бонус к Повреждениям (БПв):} опытный боец опасен сам по себе. Однако большинство воинов предпочитают иметь при себе оружие. Оружие дает Бонус к Повреждениям, который прибавляется к Доблести и Меткости. Бонус постоянен и зависит от выбранного оружия. Чем он больше, тем смертоноснее оружие в умелых (и даже не очень умелых) руках.

\paragraph{Дистанция поражения:} оружие ближнего боя применимо, только если цель находится в Боевом контакте героя  (обычно в 1 метре от него, или в соседней клетке, если используется масштабная карта).
\newline У дальнобойного оружия две дистанции - Ближняя, на которой оружие наиболее опасно, и Дальняя, предельная дистанция, на которой возможно поражение цели. Как правило, БПв на Дальней дистанции ниже, чем на Ближней.
\newline Если у дальнобойного оружия нет Дальней дистанции, это значит, что оно не способно нанести существенные повреждения за пределами Ближней дистанции.

\paragraph{Требуемая Сила (тСл):} для успешного использования оружия нужно быть сильным. Хилый герой не натянет лук и не рубанет двуручным топором. тСл для использования оружия указана в его параметрах. Если герой не имеет достаточной Силы для использования оружия, то все атаки им он совершает с Помехой. Герой, не имеющий достаточной Силы, не может сражаться Громоздким оружием вообще. Конечно, он способен размахивать им и даже выглядеть при этом угрожающе, но нанести вред может разве что случайно.

\paragraph{Единицы Здоровья (ЕЗ) оружия = |1/2 тСл|.} Например, большая дубина имеет тСл 13, значит, у нее 6 ЕЗ. Если у оружия несколько параметров тСл, для подсчета ЕЗ используется наибольший.

\paragraph{Прочность (Прч) оружия = |1/2 ЕЗ оружия|.} Прочность вычитается из Пв, которые получает оружие.

\paragraph{Тип Повреждений:} в зависимости от типа, повреждения могут иметь разные эффекты КУ и могут быть увеличены или уменьшены в зависимости от того, какие есть сопротивления или уязвимости у цели. Атаки имеют один из следующих типов Повреждений:
\newline \textbf{(Д)}робящие, \textbf{(Е)}дкие, \textbf{(К)}олющие, \textbf{(Л)}едяные, \textbf{(О)}гненные, \textbf{(П)}роникающие, \textbf{(Р)}убящие, \textbf{(Э)}лектрические, \textbf{(Я)}довитые.
\newline Если тип Повреждений отмечен символом "*", то он зависит от используемых боеприпасов.
\newline Если оружие имеет несколько типов Пв, перечисленных через черту (К/Р или Р/Д), герой может выбирать, какой тип Повреждений наносить при атаке. 
\newline Если типы Пв перечислены слитно (ОЭ или РЯ), то оружие наносит одновременно несколько типов Пв. 
\newline Подробнее о типах Повреждений читайте в главе "Боевые столкновения".

\paragraph{Критический Удар (КУ):} минимальное число на К20, при выпадении которого цель подвергается эффектам КУ. 

\paragraph{Халтура} - это дешевое, кое-как изготовленное оружие, которое имеет 1/2 ЕЗ, 1/2 Прч, -1 к БПв и +2 к Осечке. Такое оружие выходит из строя при Критическом провале Дб или Мт. Понизьте СП на 5. Если СП падают до 0, значит, оружие можно вытащить из соседней мусорной кучи, потратив на это Интерлюдию. Что большинство статистов и делает.
\paragraph{Работа мастера} - штучное изделие, изготовленное на заказ. Имеет +1 к БПв и свойство "Новое". Параметр тСл уменьшается на 1 (это не влияет на ЕЗ предмета). Повысьте СП на 10. 
\paragraph{Шедевр} - больше произведение искусства, чем средство убийства.  Шедевр получает все преимущества Работы мастера. Увеличьте ЕЗ предмета в 1.5 раза. Шанс КУ Шедевра возрастает на 1. Повысьте СП на 15.

\begin{tcolorbox}
    Халтура - распространенное оружие статистов, для которых убиство и грабеж - хобби, а не средство заработка. Это не значит, что они не умеют или боятся им пользоваться. Просто весь их заработок уходит на что-то более практичное и необходимое, вроде инструментов для обработки земли или сырья для самогоноварения.
\end{tcolorbox}

\paragraph{Импровизированное оружие(Импро):} может обладать любыми свойствами и типом Пв. Например, вилы Длинные и Колющие, а оглобля Длинная, Громоздкая и Дробящая. Как правило, Импровизированное оружие напоминает боевое. Из разбитой пивной кружки получится превосходный кастет, а из солдатского ремня - цеп. 
\newline Обычно БПв, Прч и ЕЗ импровизированного оружия ниже, чем у его боевого аналога. 
\newline Если Импровизированное оружие использовалось для нанесения Пв, проверьте его Износ в конце Сцены.	

\paragraph{Удары плашмя:} большинство видов рубящего и колющего оружия можно использовать для ударов плашмя. Это редко имеет смысл в бою насмерть, но неплохо работает, когда противник защищен от колющих и режущих атак, или если надо захватить его относительно невредимым. 
\newline Удары плашмя наносятся с Помехой и имеют Дробящие Пв. Дальнобойное оружие также может использовать это правило - стрелок пускает стрелу или пулю вскользь. Удары плашмя могут (но не обязаны) следовать правилам Несмертельных Пв.

\paragraph{Несмертельные Повреждения (НПв):} иногда героям требуется захватить противника живьем. Для этих ситуаций идеально подходит оружие, наносящее Дробящие Пв. 
\newline При атаке оружием с Дробящими Пв герой может нанести число Несмертельных Пв не превышающее \textbf{|1 + ММд / Наблюдательность(Мд) / Медицина(Мд)|}. Если проверка Дб приводит к получению большего числа Пв, остальные Пв наносятся обычным образом. Игрок должен сообщить о применении НПв до проверки героя.
\newline Когда при нападении цель получает и НПв, и Пв, это может привести к Смерти или Перелому, если Наблюдательность или  МедицинаЭ нападающего недостаточно высоки.
\newline Медицина может быть заменена Ремонтом или Обращением с животными, если цель - механизм или является животным/растением.
\newline Цель получает НПв так же, как обычные Пв. Они могут привести к Опасной ране и прочим эффектам. Сломанные и отрубленные благодаря НПв конечности считаются вывихнутыми, если нападающий того пожелает. Все КУ, нанесенные героем, вызывают Оглушение. Если проверка героя понижает ЕЗ цели до 0, игрок вправе объявить, что она потеряла сознание и осталась жива. Потерянные от НПв ЕЗ восстанавливаются в первой же Интерлюдии (даже если жертва пролежала ее на холодной земле со связанными руками и ногами).

\paragraph{Оружие для больших и маленьких существ:} в таблицах представлено оружие для существ Среднего размера (МРз 0). Для существ иного размера оружие может изготовляться под заказ. 
\newline В этом случает, прибавьте МРз существа к тСл, БПв и СП оружия. Стоимость оружия для маленьких существ не меняется - сокращается стоимость материалов, но возрастает сложность работы. Увеличение СП за размер складывается с изменением СП за Халтуру, Работу мастера или Шедевр.
\newline Если существо ипользует оружие, изготовленное для существа иного размера, вычтите МРз существа из тСл оружия. Это не влияет на ЕЗ оружия.
\begin{tcolorbox}
    И опять Маленькие и Крошечные существа фактически прибавят свой МРз к тСл оружия, ведь минус на минус дает плюс. Странная штука эта математика.
\end{tcolorbox}

\paragraph{Модификатор размера и Боевой контакт:} чем больше существо, тем проще ему атаковать цели, не стоящие к нему вплотную. Совершая атаки в ближнем бою, существо прибавляет свой МРз к Боевому контакту и Дистанции поражения, а его оружие получает свойство "Длинное" и "Упредительный удар". Например, оружие ближнего боя в руках Большого существа с МРз 1 имеет Дистанцию поражения 2, а копье и подобные ему виды вооружения получат Дистанцию поражения 3. МРз не может понизить Дистанцию поражения до 0. То есть даже удар Крошечного существа будет иметь Дистанцию поражения 1.

\paragraph{Тип/Магазин/Скорострельность (ТМС):} объединенный параметр дальнобойных видов оружия, в котором через черту указаны тип, размер магазина и Скорострельность оружия.
\begin{itemize}
    \item Тип определяет, какие боеприпасы использует оружие.
    \item Магазин определяет, сколько выстрелов оружие сделает, прежде чем потребуется Перезарядка.
    \item Скоросрельность указывает, сколько зарядов оружие может выпустить в течение Действия героя. Скорострельность оружия ограничена числом снарядов в Магазине оружия. Например, оружие со Скорострельностью 10 не может выпустить 10 снарядов, если в магазине осталось только 3. В этом случае Скорострельность оружия составит 3, затем ему потребуется Перезарядка.
\end{itemize}
\begin{tcolorbox}
    Если у оружия есть свойство Магазин или Потребление, значит, у оружия есть и свойство Перезарядка, даже если это не указано.
\end{tcolorbox}

\subsection{Cвойства оружия}
\paragraph{Бронебойное:} оружие ополовинивает Бонус доспеха и щита цели. Обратите внимание, что это свойство не влияет на Прочность цели.
\paragraph{Винтовка:} за каждый пропуск Очереди во время Прицеливания
шанс Критического удара возрастает на 1. Максимальный бонус к КУ
не может превышать МИн героя (минимум +1).
\paragraph{Возврат Х:} при выпадении Х или большего числа во время проверки Мт оружие не только наносит Пв, но и возвращается в руку к владельцу.
\paragraph{Граната:} Оружие или боеприпас взрывается сразу или через некоторое время после попадания по цели. Эффект взрыва зависит от начинки и будет описан в особых свойствах оружия.
\paragraph{Громоздкое:} Изъян, из-за которого оружие используется с Помехой в тесных помещениях и густых зарослях. Если оружие одновременно и Громоздкое, и Длинное, герой получает 2 Помехи. Герой не может использовать Громоздкое оружие лежа для атак в ближнем бою.
\paragraph{Двуручное:} требует 2 рук для использования.
\paragraph{Дистанция X/Y:} Оружие с этим свойством является дальнобойным и не предназначено для совершения атак в ближнем бою. \textbf{X} - Ближняя Дистанция оружия, \textbf{Y} - Дальняя Дистанция оружия. В графе БПв для этго оружия через черту написан урон для Ближней и Дальней дистанции соответственно.
\paragraph{Длинное:} позволяет атаковать врага в 2 метрах от героя. Цели, находящиеся ближе, герой атакует с Помехой. Длинное оружие используется с Помехой в тесных помещениях, густых зарослях и других подобных условиях.
\paragraph{Естественное:} оружие, которое является продолжением тела героя. Естественное оружие использует Рукопашную Доблесть для атак в ближнем бою и не может быть выронено(в том числе за счет маневров Разоружение), но все еще может быть повреждено.
\paragraph{Кавалерийское:} удвойте успешно нанесенные Пв, если геройверхом и применяет Атаку с разбега.
\paragraph{Кастет:} позволяет использовать как Навык Рукопашного боя, так и Навык Владения оружием в формуле Дб.
\paragraph{Кувалда:} если нанесенные Пв превышают МСл атакованного существа, оно падает. 
\newline
Маленькие существа падают, если нанесенные Пв превышают 1/2 их МСл, Крошечные — падают, если нанесенные Пв превышают 1/4 их МСл.
\newline
Большие существа падают, если нанесенные Пв превышают их МСл более чем в 2 раза, Огромные — если нанесенные Пв превышают их МСл более чем в 3 раза, Громадные — если нанесенные Пв превышают их МСл более чем в 4 раза.
\newline
Если оружие ближнего боя с этим свойством используется в одной руке, увеличьте необходимые для падения Пв в 2 раза. То есть для того, чтобы сбить с ног существо Среднего размера, понадобится превысить его МСл более чем в 2 раза.
\paragraph{Легкое:} позволяет совершать серии молниеносных выпадов и эффективно атаковать с оружием в каждой руке.
 оружие устойчиво к внешним воздействиям - все проверки Износа для этого оружия совершаются с Преимуществом.
 \paragraph{Накопление заряда:} совершая маневр Прицеливание, герой повышает бПв оружия на 2 за каждую пропущенную очередь (макс +6). Если герой теряет бонусы Прицеливания, бонус Накопления заряда на атаку все равно остается. Так же герой может объявить Прицеливание из этого оружия не заявляя цель маневра. В этом случае он получает только бонус Накопления заряда.
\paragraph{Тип оружия:} Любое дальнобойное оружие нуждается в боеприпасах. Но благодаря унификации герою не нужно носить разные боеприпасы для \textit{каждой} еденицы оружия, которая есть у него в арсенале. Стоимость и свойства боеприпасов описаны в следующем разделе. Тип оружия определяет тип боеприпасов, который использует это оружие.
\paragraph{Магазин/Скорострельность(МСк) Х/Y:} дальнобойное оружие с этим свойством способно выпускать множество снарядов сразу или один за другим и не требовать Перезарядки сразу после первого выстрела. \textbf{X} количество выстрелов, которое может сделать оружие без Перезарядки, а \textbf{Y} — число выстрелов, которое герой может произвести за время своего Действия. Скорострельное оружие может использоваться двумя способами — для поражения одной цели или нескольких. При использовании скорострельного оружия герой может одновременно поразить число целей, не превышающее его \textbf{|ММд+1|} (минимум 2 цели).
\newline
Если Скорострельное оружие используется против одной цели, повысьте БПв оружия на 1 за каждый снаряд после первого, выпущенный по ней. Например, скорострельный арбалет имеет БПв +1 и Скорострельность 3. Если герой трижды стреляет из скорострельного арбалета в одну цель, БПв скорострельного арбалета возрастает до +3 (то есть на 2).
Герой не может совершать Быструю атаку скорострельным оружием, но может использовать место этого Беглый огонь. В разделе Маневры есть полное описание этого маневра.
\begin{tcolorbox}
Если у оружие есть свойство Магазин или Потребление, то это значит, что у оружия есть свойство Перезарядка, даже если это явно не указано. Оружие с МСк надо Перезаряжать, когда закончатся заряды в магазине, а оружие с Потреблением - когда опустошена Энергоячейка.
\end{tcolorbox}
\paragraph{Тип/Магазин/Скорострельность(ТМС):} объедененный столбец, в котором через черту отражены тип оружия, размер магазина и его скорострельность.
\paragraph{Метательное:} может быть использовано и в ближнем и в дальнем бою. При метании оружие обычно падает в область, в которой находилась цель. Оно может быть поднято и использовано повторно. При Дистанционной атаке Метательным оружием герой добавляет свой МСл к БПв оружия. У метательного оружия \textbf{нет} свойства Перезарядка - количество выстрелов в раунд ограничена только Скорострельностью оружия. Если герой держит в каждой руке два одинаковых Метательных оружия, он может метнуть оба в рамках маневра Беглый огонь или Атака, повысив скорострельность оружия на 1, но получив при этом Помеху на этом маневр и дополнительно вторую Помеху, если оружие Громоздкое. Если у героя есть трюк Амбидекстер, он получает на одну Помеху меньше при совершении этого маневра. В таблицах свойство Метательное указано в столбце ТМС как тип оружия.
\begin{tcolorbox}
Используя метательное оружие с дополнительным свойством \textbf{Снаряды} герой не метает оружие целиком, а использует подготовленные для него снаряды. Им нельзя производить атаку в Ближнего боя, однако стоимость выстрела гораздо ниже, когда во врага летит только часть оружия. СП 10 зарядов для такого вида оружия равна \textbf{|СП оружия — 10|}. Стоимость 1 заряда для этого оружия равна \textbf{|СП оружия — 12|} (минимум 1).
\end{tcolorbox}
\paragraph{Огнемет Х}: при атаке огнемет поражает все объекты, находящиеся на в радиусе Х от точки прицеливания. Совершите только одну проверку Меткости и определите повреждения для всех пораженных объектов, исходя из нее. Обратите внимание, что в случае выпадения Критического удара его эффекты применяются ко всем объектам, пораженным огнеметом.
\paragraph{Отдача:} герой должен отказаться от Перемещения, если желает выпустить из оружия с этим свойством 5 или больше зарядов.
\paragraph{Перезарядка:} для перезарядки оружия с этим свойством герой должен отказаться от Действия или Перемещения. Оружие не может использоваться при Быстрой атаке. Арбалеты и пороховое оружие требуют 2 свободных рук при перезарядке.
\paragraph{Потайное:} герой получает Преимущество на проверки Скрытности и Ловкости рук при попытках спрятать оружие.
\paragraph{Потребление Х:} Оружие с этим свойством вместо боеприпасов использует энергию и тратит Х Зарядов за 1 выстрел или удар.
\paragraph{Противотанковое:} оружие предназначено для поражения тяжелобронированных целей. Это оружие игнорирует Прочность цели при попадании. Если у цели нет Прочности, она совершает проверку Внезапной Смерти со штрафом, равным величине успеха при попадании. Для статистов эта проверка считается автоматически проваленной.
\paragraph{Пулемет:} оружие не приспособлено для одиночных выстрелов. Если герой использует оружие для поражения нескольких целей, то должен выпустить минимум 3 пули в каждую.
\paragraph{Снайперское:} при стрельбе на дальнюю дистанцию оружие игнорирует штрафы Зон поражения, если герой проводит Прицеливаясь хотя бы один полный Круг. При стрельбе на ближней дистанции герой получает Помеху.
\paragraph{Сошки(с):} оружие оснащено сошками для стрельбы с упора. Если носитель имеет возможность установить сошки на какую-либо поверхность, используйте значение Минимальной силы с пометкой (с), в противном случае используйте значение после черты. Если у оружия нет значения Минимальной силы без символа (с), значит, стрельба с рук невозможна.
\paragraph{Тип Повреждений:}
в зависимости от типа, повреждения могут иметь разные эффекты КУ и могут быть увеличены или уменьшены в зависимости от того, какие есть сопротивления или уязвимости у цели. Атаки имеют один из следующих типов Повреждений:
\newline \textbf{(Д)}робящие, \textbf{(Е)}дкие, \textbf{(К)}олющие, \textbf{(Л)}едяные, \textbf{(О)}гненные, \textbf{(П)}роникающие, \textbf{(Р)}убящие, \textbf{(Э)}лектрические, \textbf{(Я)}довитые.
\newline
Если тип повреждений отмечен *, то он зависит от заряженных в оружие боеприпасов. См. раздел боеприпасов для определения типа повреждений.
\newline
Если оружие имеет несколько типов повреждений через черту(К/Р или Р/Д), герой может выбирать, какой тип повреждений наносить при атаке. Если типы повреждений перечислены слитно(ОЭ или РЯ), то оружие наносит одновременно несколько типов повреждений. Подробнее об этом написано в главе Боевые Столкновения.
\paragraph{Удавка:} этим свойством обладает любое оружие, достаточно гибкое, чтобы захлестнуть конечности противника. Для применения свойства необходимо две руки. Оружие может использоваться для Захватов. Удвойте Пв, успешно нанесенные Удавкой в шею. Существа, чьи ЕЗ опустились до 0 в результате Пв, нанесенных Удавкой в шею, теряют сознание, а не умирают, если атакующий того пожелает. Существа, у которых БД (но не БЩ) составляет 6 или больше, не могут быть атакованы Удавкой в шею — она надежно защищена.
\paragraph{Универсальное:} может использоваться в 1 или 2 руках. Смена хвата является Быстрым действием. В БПв и тСл указаны для одной и двух рук через черту.
\paragraph{Упредительный удар:} если противник входит в зону поражения оружия, герой может потратить Быстрое действие и провести внеочередную атаку по нападающему или его скакуну. Выход из зоны действия оружия и передвижение в ней не провоцируют Упредительный удар.
\paragraph{Фехтовальное:} герой может заменить МСл на ММд при подсчете Дб.
\paragraph{Хрупкое:} Изъян, из-за которого оружие уязвимо к самым незначительным повреждениям и не предназначено для парирования и блокирования. Ополовиньте ЕЗ оружия(минимум 1 ХП). Так же все проверки Износа этого оружия совершаются с Помехой.
\paragraph{Цеп:} игнорирует БЩ, хотя может заявлять щит как область поражения.
\paragraph{Символ "*"} обозначает качества оружия, не указанные в таблице, не входящие в унифицированный перечень Свойств и перечисленные в описательном блоке. \textbf{Символ "Ф"} обозначает оружие, уместное лишь в фантастическом антураже.


\subsection{Боеприпасы дальнобойного оружия}
Разные оружия применяют разные боеприпасы. Однако со временем производители начали вводить унифицированные боеприпасы для разных калибров, что позволило использовать одни и те же боеприпасы для не совсем одинакового вооружения.
\paragraph{Тип} боеприпаса определяет, в какие пушки можно его заряжать.
\paragraph{Количество} определяет, сколько зарядов можно приобрести, купив стандартную коробку боеприпасов. Если в этой графе стоит прочерк, боеприпас нельзя приобрести в стандартизированном варианте.
\begin{center}\begin{tabular}{|c|p{3cm}|p{10cm}|c|}
\hline
Тип & Оружие, использующее боеприпасы & Описание боеприпасов & Кол-во\\ \hline
П & Пистолеты, пистолеты-пулеметы & Малый калибр, низкая цена & 30 \\ \hline
Р & Револьверы, крупнокалиберные пистолеты & Тяжелые и дорогие пули & 15 \\ \hline
В & Винтовки & Быстрота, высокая пробивающая способность & 20 \\ \hline
Д & Дробовики & Дробь, картечь, реже - пули & 4 \\ \hline
О & Пулеметы и крупнокалиберные орудия & Действительно большие калибры. Настолько, что в них можно поместить взрывчатку & 10 \\ \hline
Г & Гранатометы и ракетные установки & Унифицированные гранаты и ракеты. Поражающий эффект зависит от начинки & 1 \\ \hline
Э & Энергетическое оружие & Батареи всех мастей. Если у оружия класса "Э" нет свойства "Потребление", оно тратит 1 Заряд за выстрел & - \\ \hline
Б\textsuperscript{ф} & Кинетческие ускорители & Металлические болты со сверхпрочным сердечником. Дешевы в изготовлении, смертоносны на больших скоростях. & 40 \\ \hline
М & Метательное оружие & Использует метательный боеприпас и приводится в действие силой стрелка & - \\ \hline
У & Уникальный боеприпас & Оружие использует не стандартизированный боеприпас, который подойдет только для него. Если СП не указана в описании оружия, то СП 10 зарядов для такого вида оружия равна \textbf{|СП оружия — 10|}. Стоимость 1 заряда для этого оружия равна \textbf{|СП оружия — 12|} (минимум 1). & * \\ \hline
% Ф & Феномены & Оружие использует Энергию героя для работы. & - \\ \hline
\end{tabular}\end{center}
* если у метательного оружия есть свойство Снаряды. Иначе каждую еденицу оружия надо приобретать отдельно.

\paragraph{} Кроме стандартных боеприпасов можно разжиться и специализированными. Они дают дополнительные свойства или улучшают свойства оружия. В таблице указаны наиболее распространенные типы боеприпасов и стоимость их стандартных коробок. Значения Дистанции, БПв, ТПв, КУ в таблице изменяют стандартные параметры оружия на указанное значение. 
\newline В столбце \textbf{Тип} указаны типы боеприпасов, для которых возможна модификация.
\newline В столбце \textbf{СП} указана стоимость стандартной коробки боеприпасов. 
\newline В столбце \textbf{КУ} отрицательные значения означают расширение диапазона КУ, а положительные - его сужение. Итоговое значение КУ оружия не может превышать 20 и быть меньше 1. Если боеприпас выводит КУ за пределы этих значений, то они становятся равными 20 и 1 соответственно. 

\begin{longtable}{|p{3cm}|p{2.5cm}|c||c|c|c|c||c|}\hline
Название & Особые свойства & Тип & Дистанция & БПв & ТПв & КУ & СП\\ \hline
Стандартные & - & - & - & - & - & - & 8\\ \hline
Утяжеленные & Оружие получает свойство "Кувалда" & РВОГ & -10/-20 & - & +Д & +1 & 10\\ \hline
Зажигательные & - & РВДО & - & +1/+1 & +О & -3 & 11\\ \hline
Бронебойные & - & РВДОГБ & - & +2/+2 & - & +2 & 11\\ \hline
Пустотелые & - & ПРВОБ & - & -1/-1 & - & -5 & 9\\ \hline
Подкалиберные & - & ПРВДО & +10/+20 & -1/-1 & - & - & 9\\ \hline
Разрывные & Полученные целью Пв удваиваются & ПРВДО & -10/-20 & -2/-2 & +Р & +1 & 12\\ \hline
Дозвуковые & При использовании глушителя патроны не производят хлопка. Стрелок не обнаруживает себя при промахе & ПРВО & -5/-10 & -1/-1 & - & - & 9\\ \hline
\end{longtable}


\newpage
\subsection{Перечень оружия ближнего боя}
Если в этой таблице у оружия есть Дистанции, это значит, что его можно как использовать в ближнем бою, так и метать в противника.
\genAndGet{weapons}{weapons}{Рукопашное}

% \newpage
% \subsection{Вспомогательное и импровизированное оружие}
% Все, что не является оружием, можно использовать как оружие. Но некоторые вещи просто напрашиваются на то, чтобы ими кого-нибудь ударили, порезали или проткнули.
% \genAndGet{weapons}{weapons}{Вспомогательное}

\newpage
\subsection{Перечень метательного оружия}
\genAndGet{weapons}{weapons}{Метательное}

\newpage
\subsection{Перечень огнестрельного оружия}
\genAndGet{weapons}{weapons}{Огнестрельное}

\newpage
\subsection{Перечень энергетического дальнобойного оружия}
\genAndGet{weapons}{weapons}{Энергетическое}

\newpage
\subsection{Бластеры\textsuperscript{ф}}
Очередная ипостась электричества на службе цивилизации и прогресса. Теперь бластеры готовы служить любому, кто врубается в их конструкцию.
\newline Бластер является фантастической модификацией дальнобойного оружия, благодаря которой вместо боеприпасов оно потребляет Зр. СП бластеров на 2 выше, чем в таблице оружия, а тип повреждений меняется на ОЭ.
\newline Любой бластер может использоваться в тазер-режиме. Цель не получает Пв, но должна преуспеть в проверке Вн против \textbf{|5 + [Величина успеха проверки Мт]|} или потерять сознание.\newline Потребление бластеров зависит от типа оружия:
\begin{itemize}
\item \textbf{Тип П} -- потребление 1
\item \textbf{Тип Р} -- потребление 3
\item \textbf{Тип В} -- потребление 2
\item \textbf{Тип Д} -- потребление 15
\item \textbf{Тип О} -- потребление 5
\end{itemize}
Бластер нельзя сделать из оружия типа Г,Б,М,У и, очевидно, Э.% и Ф.

\printindex[weapons]

\subsection{Гранаты и взрывы}
\paragraph{}
В историях всегда найдется место бутылкам с горючей смесью, емкостями с едкими летучими веществами и старым добрым гранатам. И это, и многое другое можно метнуть во врага. А потом наблюдать с безопасного расстояния, как он мечется в бессильных попытках погасить пламя, истекает кровью, изрешеченный осколками... или как бутыль предательски падает в траву, а отсыревший запал гаснет.
\paragraph{}
Гранаты — Метательное оружие с БПв = \textbf{|-2|} на Ближней и \textbf{|-4|} на Дальней дистанции. Непосредственно удар гранаты имеет Дробящие Пв и наносит КУ при выпадении 20. Используйте Мт, чтобы определить, поразил ли герой цель. Не забывайте, что гранаты можно метать в землю, поражая Зщ 10, если только герой не собирается нанести Повреждения самим броском!
\newline
Все бомбы и гранаты считаются одним видом оружия.
\paragraph{}
В случае промаха граната все равно взорвется (если только не выпала Осечка). Граната отклоняется на Х метров, где Х равен промаху героя по Зщ цели — ловкий противник может успеть отбросить гранату, а бронированный — отбить щитом или латным рукавом! Определите направление, в котором отклонилась граната, при помощи броска К20. В этом случае граната может пролететь большее расстояние, чем максимальная дистанция броска!
\begin{center}
\begin{tabular}{ |p{2.7cm}|p{12cm}| }
\hline
\textbf{К20} & \textbf{Направление}
\\ \hline
1-5 & Граната перелетела за цель
\\ \hline
6-10 & Граната отклонилась вправо от цели
\\ \hline
11-15 & Граната недолетела до цели
\\ \hline
16-20 & Граната отклонилась влево от цели
\\ \hline
\end{tabular}
\end{center}
\begin{tcolorbox}
Если нужно больше детализации по направлению, можно использовать принцип дартс: область вокруг цели делится на 20 равных секторов и каждому из них сопоставляется число, которое определяет направление отклонения.
\newline
\begin{center}
\includegraphics[width=0.5\textwidth]{darts}
\end{center}
\end{tcolorbox}
\paragraph{Вес} представленных гранат равен 0.5.
\paragraph{Дистанция} у гранат обычно Ближняя 5, Дальняя 20.
\paragraph{Центр Взрыва:} точка, из которой распространяется Взрыв.
\paragraph{Радиус Взрыва(РВ):} число метров, на которое распространяется Взрыв из центра Взрыва во все стороны. Если атака обладает свойством <<Взрыв>>, то цель, против которой совершается проверка Доблести или Меткости, также считается находящейся в области Взрыва и получает Повреждения и прочие эффекты и от него тоже!
\paragraph{Сила Взрыва(СВ):} все существа, попавшие в радиус Взрыва, получают Пв, равные \textbf{|Сила Взрыва — БАЗщ — БД|}.
\paragraph{Газ:} газ распространяется в радиусе Взрыва и остается там какое-то время. Для определения длительности действия газа совершите проверку Неприятностей. Существа, вдохнувшие газ, страдают от эффектов яда, громко кашляют и чихают, если они еще в состоянии кашлять и чихать. Существа, имеющие Иммунитет к Ядовитым Пв, не подпадают под действие газа. В помещении увеличьте временные промежутки вдвое.
\trouble
{Штиль}%no sweat name
{Газ рассеивается чере 10 минут}%no sweat description
{Бриз}%tough day name
{Газ рассеивается через 5 минут}%tough day description
{Порыв ветра}%we have trouble name
{Газ рассеивается через 1 минуту}%we have trouble description
{Шторм}%fiasco name
{Газ рассеивается по истечении полного Круга}%fiasco description

\paragraph{Другие опасности Взрывов:} Взрывы опасны не только Повреждениями. Попавшие во Взрыв доспехи, щиты и прочие предметы, закрепленные на теле попавшего во Взрыв, получают \textbf{|Пв = Сила Взрыва — Прч|} и могут быть уничтожены. Если Взрыв достаточно силен, попавшие в него существа с воплями разлетаются в разные стороны! Если существо получает Повреждения от взрыва, его отбрасывает от центра Взрыва на \textbf{|Сила Взрыва — МСл отброшенного — МЛв отброшенного|} метров. Отброшенный получает столько Пв, сколько метров пролетел. Когда отброшенный сталкивается с другим существом, оно должно совершить проверку Атлетики(Сл) против \textbf{|Зщ + мВн отброшенного|}, чтобы поймать отброшенного или Атлетики(Лв), чтобы увернуться. В случае провала, оба существа получают Пв, равные величине провала. Если другое существо увернулось, отброшенный продолжает полет и получает Пв согласно расстоянию, которое он пролетел. Существо не может быть отброшено на большее число метров, чем получило Пв от Взрыва. Увеличьте расстояние броска в 2 раза за каждую категорию размера меньше Среднего и уменьшите в 2 раза за каждую категорию больше Среднего. Падения и столкновения наносят Дробящие Пв.
\newline
Если при подсчете дистанции отбрасывания получилось 0 или меньше, существо остается на месте.



\subsection{Защита}
Весь перечень того, что в состоянии защитить тело от ударов и выстрелов - от вонючей шкуры дикаря до тактического силового доспеха. 
\paragraph{Бонус к Защите:} доспехи дают герою Бонус доспеха к Защите (БД), а щиты дают герою Бонус щита к Защите (БЩ).
\newline Герой не может надеть 2 комплекта доспехов одновременно, но может одновременно использовать 2 щита. В этом случае он получает БЩ обоих щитов.
\paragraph{Ограничитель модификатора Ловкости (оМЛв):} в большинстве своем доспехи и щиты - тяжелые конструкции, сковывающие движения. Многие доспехи и щиты ограничивают МЛв, доступный герою при Защите и совершении проверок. Так, герой с 20 Лв (МЛв +5), надевший кольчужную рубаху, будет добавлять к Зщ и проверкам Навыков, основанных на Ловкости, лишь +3, потому что ограничитель МЛв кольчужной рубахи составляет +3. Герой с высокой Лв, надевший доспех с низким оМЛв, столкнется с понижением Дб и Мт.
\paragraph{Требуемая Выносливость (тВн):} герой не может носить доспех или щит, если не обладает достаточной для этого Выносливостью. Сила не играет здесь важной роли - герой может поднять доспех и выдержать его вес, но выдохнется после нескольких минут активных действий. Разумеется, он может облачиться в доспех, чтобы повыпендриваться перед девчонками или попозировать фотографу, но в бою выглядит жалко. Все активные проверки такого героя совершаются с Помехой, а все атаки по нему совершаются с Преимуществом.
\paragraph{ЕЗ доспеха или щита = |тВн|.} Если доспех имеет несколько значений тВн, для подсчета ЕЗ используется максимальное.	
\paragraph{Прочность доспехов и щитов = |1/2 ЕЗ доспеха или щита|.} 
\paragraph{БПв щита.} Большинство щитов можно использовать, как оружие с КУ 20 и ТПв Д. Если в этом столбце указан прочерк, то бить щитом нельзя.
\paragraph{Доспехи и Осечка:} доспехи и щиты так же, как и оружие, подвержены износу и могут иметь свойство \textbf{"Осечка Х"}. Если герой, снаряженный доспехом или щитом, совершает Активную проверку и выбрасывает на К20 число, равное Х или меньше, проверка проваливается. 
\paragraph{Количество помех для проверок Скрытности (ПС):} защита может шуметь при движении - звон сочленений, скрежет пластин, шипение сервоприводов и гул силовой установки - все это мешает герою  скрывать свое присутствие.

\paragraph{Халтура:} дешевый, кое-как изготовленный щит или доспех тяжел и неудобен. Его тВн увеличивается на 1 (это не влияет на ЕЗ предмета), а оМЛв понижается на 1. Доспех или щит имеет 1/2 ЕЗ и 1/2 Прч и Осечку 5. Если щит или доспех не имел оМЛв, то он получает оМЛв +4. Когда герой получает КП на Активную проверку, доспех разрушается. Понизьте СП на 5. Если СП достигает 0, это означает, что изготовление доспеха требует Интерлюдии, которую герой тратит на сбор материалов, и Антракта на соединение их друг с другом.
\newline Халтурные доспехи и щиты - отличный выбор для тех, кто живет одним днем, но не слишком торопится умирать. То есть для абсолютного большинства жителей разрушенного мира, вынужденных иногда принимать участие в боевых действиях. Профессиональные воины пользуются Халтурной защитой лишь в крайних случаях.
\paragraph{Работа мастера.} Доспех или щит, изготовленный мастером, безупречно подходит заказчику. ТВн уменьшается на 1 (это не влияет на ЕЗ предмета), а оМЛв повышается на 1. СП повышается на 5.
\paragraph{Шедевр} представляет собой безупречный экземпляр мастерской работы и получает все преимущества работы мастера. ЕЗ предмета увеличиваются в 1.5 раза. В дополнение БД или БЩ возрастает на 1, а ПС понижается на 1. СП повышается на 10.
\newline Работа мастера и шедевр - штучные изделия. Все, кроме хозяина, при ношении такого доспеха получают все логически возможные эффекты халтуры, пока доспех не будет подогнан сведущим технарем под нового владельца.
\paragraph{Надевание и снятие доспехов и щитов:} время, необходимое для этого, зависит от оМЛв доспеха или щита:
\begin{itemize}
    \item[--] Доспех или щит не имеет оМЛв, время надевания и снятия составляет 1 минуту.
    \item[--] ОМЛв доспеха +2 и выше, время надевания составляет 5 минут, а время снятия составляет 1 минуту.
    \item[--] ОМЛв +1 и ниже, время надевания составляет 10 минут, а время снятия составляет 5 минут.
    \item[--] Кто-то помогает герою, то время снятия и надевания сокращается вдвое.
\end{itemize}
\paragraph{Доспехи для больших и маленьких существ:} в таблицах представлены доспехи и щиты для существ Среднего размера. Для существ иного размера доспехи изготовлются под заказ. 
\newline В этом случает, прибавьте МРз существа к тВн, БД, БЩ и СП доспехов и щитов. Стоимость доспехов для маленьких существ не меняется - сокращается стоимость материалов, но возрастает сложность работы. Увеличение СП за размер складывается с изменением СП за Халтуру, Работу мастера или Шедевр.
\paragraph{Атаки по доспехам и щитам:} доспех или щит могут быть выбраны зоной поражения (подробнее смотрите маневр "Сломать снаряжение" в разделе "Маневры"). Носитель получает Пв только в том случае, если доспех или щит получают Пв, которые не смогли полностью поглотить их Прч и ЕЗ.
\newline Если доспех или щит уничтожены, носитель теряет их БД и БЩ, однако его МЛв все еще ограничен оМЛв доспеха или щита.
\paragraph{Иммунитет, Сопротивление и Уязвимость к Повреждениям:} доспехи и щиты обладают иммунитетом к Ядовитым Пв, Сопротивлением к Колющим, Проникающим и Ледяным Пв и Уязвимы к Едким Пв. 
\begin{tcolorbox}
    При встрече с тяжелобронированным противником будет хорошей идеей сначала уничтожить его доспех и щит, вместо того чтобы пытаться поразить заоблачную Зщ. Хотя, тогда снаряжение не выйдет продать. 
\end{tcolorbox}

\subsection{Перечень Доспехов}
\genAndGet{protection}{protection}{Доспех}

\subsection{Перечень Щитов}
\genAndGet{protection}{protection}{Щит}



\subsection{Медикаменты и яды}
\paragraph{Интоксикация.} получение (Я)довитых Пв не приводит к потере ЕЗ напрямую и подсчитывается отдельно от остальных Пв. Интоксикация является суммой всех Ядовитых Пв, полученных героем, и отражает то, сколько яда попало в организм и как долго он может сопротивляться яду.
\paragraph{Максимальное значение Интоксикации} составляет \textbf{|Максимальные ЕЗ героя х 1.5|}. 
\paragraph{Интоксикация и смерть:} как только значение Интоксикации героя превышает \textbf{|Максимальные ЕЗ героя х 1.5|}, он должен совершить проверку Вн против \textbf{|15|}. При провале герой погибает (как правило, в страшных мучениях). При успехе он впадает в кому и становится Неподвижным до тех пор, пока его Интоксикация не уменьшится.
\begin{tcolorbox}
    Да, проверять Выносливость придется с Помехой за Отравление.
\end{tcolorbox}
\paragraph{Токсичность (Токс):} количество Ядовитых повреждений, которое наносит Лекарство или Яд при применении.
\paragraph{Первичный эффект:} Разовый эффект Лекарства или Яда. Для сопротивления эффекту герой должен преуспеть в проверке Вн против \textbf{|10+Токс|}.
\paragraph{Отравление:} пока значение Интоксикации превышает текущие ЕЗ героя, он получает состояние «Отравлен». 
\paragraph{Снятие состояния Отравления:} герой должен добиться того, чтобы значение Интоксикации не превышало его текущие ЕЗ. Он может достичь этого, как повысив текущие ЕЗ, так и понизив Интоксикацию.
\paragraph{Побочки:} целебные зелья, таинственные эликсиры и мощные стимуляторы зачастую являются не очень-то полезными. Особенно при частом применении. 
\newline Как только герой становится Отравлен, на него немедленно накладываются эффекты Побочек Лекарств и Ядов, которые наносили ему Ядовитые Пв в этой Сцене. 
\newline Если на героя подействовало лекарство или яд, пока герой Отравлен, Побочки начинают действовать \textit{сразу}.
\newline Эффекты Побочек прекращаются, как только герой перестает быть Отравлен (если в описании Побочек не сказано иного).
\begin{tcolorbox}
    Игрок может отказаться от совершения проверки Вн, если считает, что Первичный эффект пойдет герою на пользу. Судьбе виднее! Учтите, что иногда решение придется принимать наугад, до того, как эффекты войдут в игру. Например, если у пузырька с таблетками стерлась этикетка. Или когда единственный источник информации об эффектах зелья – приготовивший его шаман. Разумеется, у Судьбы есть свои способы добиться нужного результата – о них вы уже читали в главе «Нити, Ходы и Капризы».
\end{tcolorbox}
\paragraph{Антракт и побочки.} Антракт завершает действие всех Побочек, если в описании Яда или Лекарства не сказано обратного. Если при этом Интоксикация героя выше, чем текущие ЕЗ, он все еще в состоянии «Отравлен».
\paragraph{Интоксикация и Отдых:} во время Интерлюдий и Антракта герой понижает Интоксикацию на столько же единиц, сколько ЕЗ восстанавливает. Если герой применяет стимулятор или зелье, то он не понижает Интоксикацию, если это прямо не указано в описании препарата. 
\newline Если герой обращается за помощью к врачу или целителю, то снижение Интоксикации является отдельной Услугой. 

\subsection{Медикаменты}
\paragraph{Время приема (ВП):} время, необходимое для употребления одной порции Лекарства. У Ядов отсутствует ВП, т.к. оно сильно зависит от формы отравляющего вещества. 
\genAndGet{drugs}{drugs}{Лекарство}
\printindex[potions]

\subsection{Яды}
\paragraph{Отравленное оружие:} Яд на оружии действует в течение колличества атак, равных его Токсичности. Это правило распространяется и на оружие ближнего боя, и на дальнобойное оружие. Для дальнобойного оружия это значит, что порция Яда была распределена равномерно между боеприпасами. 
\newline Получив хотя бы 1 Пв, цель дополнительно получает Ядовитые Пв в размере Токсичности Яда и должна сопротивляться его Первичному эффекту.
\paragraph{Условия приема} имеют значение для ядов. их четыре – вдох, контакт, порез, проглатывание.
\paragraph{КУ Ядовитых повреждений:} если Ядовитая атака наносит КУ, Побочки Яда сразу входят в игру и действуют (если эффект можно растянуть во времени), пока герой не уйдет в Антракт, получит антидот или значение Интоксикации не упадет до 0.
\genAndGet{drugs}{drugs}{Яд}
\printindex[poisons]
\begin{tcolorbox}
    Как правило, те, кто использует Яды для всяких злодейских дел, не скупятся, и сразу скармливают жертве несколько доз.
    \newline Для тех же, кто ищет в Ядах силу – например, исказителей с Трюком «Третий глаз», галлюциноген – самый удачный выбор.
\end{tcolorbox}


\subsection{Предметы: всякая всячина}
Без одних жизнь действительно становится труднее, без других вполне можно обойтись. Хотя грань между первыми и вторыми исчезающе тонка – ведь и те, и другие отлично идут на Бартер.
\begin{tcolorbox}
    Вес предметов не указан намеренно – в зависимости от развития погибшей цивилизации, он может значительно колебаться. Иногда уровень технологий позволяет совместить несколько предметов в одном.
    \newline Разумеется, этот список – всего лишь пример. В разрушенном мире полно предметов, не попавших в него, но способных сыграть значимую роль в вашей истории.
\end{tcolorbox}
\genAndGet{items}{items}{Предмет}

\subsection{Припасы}
Если вы желаете добавить деталей в историю, то героям не стоит забывать о регулярных приемах пищи.
Припасы – один из лучших товаров для Бартера. Обратите внимание, что указанная в таблице СП может дополнительно возрасти, если Припас Новый или Редкий!
\paragraph{Сытность:} показывает, насколько насыщает Припас. В течение суток герой может употребить Припасы с суммарной Сытностью, не превышающей \textbf{|Вн|} героя.
\paragraph{Эффекты:} герой может использовать Припасы, чтобы поднять себе настроение (и получить эффекты) перед каким-нибудь важным событием. Эффекты входят в игру в следующей за приемом пищи Сцене, но могут сохраняться и дольше, если это соответствует логике ситуации.
\begin{tcolorbox}
    Если учет и расход ресурсов играет в вашей истории значимую роль, можно ввести в игру следующие правила:
    \paragraph{Неообходимое количество Припасов в сутки:} Если герой в течение суток не употребил Припасы с суммарной Сытностью, равной его \textbf{|Вн/2|}, он находится в Состяниях «Усталость» и «Ранен».
    \paragraph{Вода:} если герой в течение суток не употребил хотя бы 1 порцию воды или безалкогольных напиков с высоким ее содержанием, он находится в состоянии «Серьезно ранен».
\end{tcolorbox}
\paragraph{Перечень Припасов}
\genAndGet{food}{food}{Еда}

\section{Инструменты Могущества}
В этом разделе описаны могущественные артефакты, высокотехнологичные прототипы и технологические чудеса, которые позволяют герою достичь и преодолеть предел своих возможностей. Обычно Инструменты Могущества встречаются только в фантастическом антураже.

\paragraph{Название} Инструмента отражает его дух, историю или возможности. Редко названия Инструментов бывают обыденными.
\paragraph{Базовый предмет} Зачастую Инструмент построен на основе уже существующего, а иногда и весьма распространенного в мире, предмета, оружия или элемента экипировки. Обычно Инструмент сохраняет все характеристики Базового предмета и лишь усиливает их своими способностями.
\paragraph{Запас энергии: } количество Зарядов Инструмента которые можно тратить на активацию его Функций. Способ восстановления этой энергии может сильно отличаться в зависимости от антуража.
\paragraph{Сложность Приобретения(СП): }некоторые Инструменты научились довольно ловко воспроизводить и их даже можно увидеть на полках магазинов! Но если в описании нет СП, то Инструмент нельзя просто купить. Придется искать его в древних руинах, секретных лабораториях, а иногда и выпрашивать у богов. Если герой захочет продать Инструмент без СП, то количество Богатства, которое он получит от продажи будет определено исключительно его красноречием и умением торговаться.
\paragraph{Описание }Инструмента говорит о его виде и о нарративе его использования. Механика его работы описана в Трюках, Функциях, Изъянах и Ходах Инструмента.
\paragraph{Трюки }Инструмента - это его свойства, которые герой может использовать когда захочет или же они работают постоянно.
\paragraph{Функции }Инструмента требуют трату Зарядов для активации. В описании каждой Функции Инструмента в скобках указана её \textbf{Стоимость}.
\paragraph{Ходы }Инструмента позволяют совершать невозможное, но требуют за это обрыв Нитей героя. В описании каждого Хода Инструмента обязательно указана его \textbf{Стоимость}.
\paragraph{Изъяны }Инструмента указывают на любое негативное влияние Инструмента на его владельца. Это может быть как незначительное неудобство, так и серьезное проклятие.
\genAndGet{power-tools}{power-tools}

\section{Транспорт}
В этом разделе описаны механизмы и существа, которых можно использовать, как транспортные средста (ТС).
\paragraph{Символ "Ф"} обозначает вид транспорта, уместный лишь в фантастическом антураже.
\paragraph{Символ «*»} обозначает качества ТС, перечисленные в описательном блоке.
\paragraph{Ограничение Модификатора Ловкости(оМЛв):} крупные ТС ворочаются неохотно, а разогнавшись, не спешат тормозить. оМЛв определяет максимальный МЛв, который водитель может добавить к Эксплуатации при управлении ТС.
\paragraph{Ограничение Модификатора Ловкости(оМЛв) для животных:} не вся животина охотно подчиняется командам погонщика или наездника. Максимальный МЛв, который может герой добавить к проверкам Обращения с животными равен МЛв самого животного.
\paragraph{Проходимость транспортного средства(П):} ТС не может Путешествовать по местности с Опасностью, превышающей его Проходимость. 
\newline Проходимость животных увеличивается на 1, если они не несут груза или всадников.

\paragraph{Защита транспортного средства:} предполагается, что транспортные средства из таблицы уже оснащены защитными приспособлениями в пределах разумной целесообразности. 
\newline МЛв водителя прибавляется (или отнимается, если он отрицательный) к Зщ ТС.
\paragraph{Маневренность (М):} во многих ситуациях важны не только Проходимость и Скорость ТС, но и его способность избегать препятствий. Маневренность ТС = \textbf{|Проходимость — МРз|}. Маневренность добавляется к навыку водителя (или отнимается, если она отрицательная) при проверке ЭксплуатацииЭ для совершения сложных поворотов и перестроений.
\paragraph{Cкорость ТС во время Путешествий (Ск):} она значительно ниже максимума, который можно выжать из транспорта и измеряется в км/ч. 
\newline Если в графе присутствует \textbf{символ М}, это значит, что обычная скорость ТС равна максимальной и оно не может двигаться Маршем (если только скорость самого медленного участника каравана не в два раза меньше, чем Ск этого ТС).
\paragraph{Ск ТС во время Боевых сцен:} она численно равна Ск ТС во время Путешествий, но измеряется в метрах/клетках на тактической карте, которые ТС может преодолеть за свою Очередь. Это отражает необходимость маневрировать и лавировать между другими участниками Сцены (или давить их). Механическое ТС может совершать маневры ближнего боя, если оснащение и контекст позволяют это. Проверки Дб в этом случае заменяются ЭксплуатациейЭ водителя.
Животные используют в Боевых сценах карточки из раздела «монстры и статисты».
\paragraph{Расход топлива (Р)} определяет сколько Зарядов Топлива (или килограммов фуража для животных) тратит тот или иной вид транспорта на каждые 10 км пути.
\paragraph{Грузоподъемность/Вес (Г/В)} определяет количество полезной нагрузки, которую может везти ТС, и вес самого ТС для определения возможностей буксировки.
\paragraph{Буксировка:} предполагается, что ТС может буксировать вес, не превышающий его грузоподъемность в 10 раз.
\paragraph{Перегрузка:} если нагрузка ТС превышает Грузоподъемность не более, чем в 2 раза, либо буксируемый вес превышает вес ТС не более, чем в 2 раза, водитель получает Помеху на Эксплуатацию.
\paragraph{Сильная перегрузка:} если нагрузка ТС превышает Грузоподъемность не более, чем в 3 раза, либо буксируемый вес превышает вес ТС не более, чем в 3 раза, водитель получает 2 Помехи на Эксплуатацию.
Повреждение ТС:} после потери 1/3 ЕЗ ТС теряет половину Ск. После потери 2/3 ЕЗ его Эксплуатация проверяется с Помехой.
\newline Если транспортное средство одномоментно теряет 1/5 или более ЕЗ, водитель должен проверить Эксплуатацию против \textbf{|15|}. При провале транспортное средство глохнет.
\newline Когда зона поражения одномоментно получает Пв, равные 1/4 или более от максимальных ЕЗ ТС, она уничтожается. Все проверки водителя, связанные с повреждением зоны, считаются Критически проваленными. 
\paragraph{Зоны поражения ТС:} при атаке ТС могут быть выбраны Зоны поражения, критичные для его работы. Нападающий должен указать зону и проверить Дб или Мт с -2.
\newline Если Зона поражения одномоментно получает Пв, составляющие 1/5 или более от максимальных ЕЗ ТС, она временно выходит из строя. Лишние Пв теряются, но водитель должен проверить Эксплуатацию против \textbf{|15|}. При провале транспортное средство глохнет. Помимо этого:
\begin{itemize}
    \item[--] Повреждение Двигателя и трансмиссии приводят к тому, что ТС глохнет. Требуется успешно проверить Ремонт против \textbf{|15|}, чтобы снова завести его;
    \item[--] Повреждение Ходовой части приводит к потере управления. Водитель проверяет Эксплуатацию против \textbf{|15|}. При провале ТС терпит крушение;
    \item[--] Повреждение Орудия или Манипулятора выводит их из строя. Для того, чтобы восстановить функционал, требуется успешно проверить Ремонт против \textbf{|15|};
    \item[--] Повреждение Кабины или Пассажирского отсека сбрасывает водителя или пассажиров. Они должны успешно проверить Лв или Атлетику(Лв) против \textbf{|15|}. При провале герои выпадают из транспорта и получают Дробящие Пв, равные \textbf{|[величине провала]*[Опасность местности]|}. Если водитель и пассажиры пристегнуты ремнями безопасности, они совершают проверки с Помехой.
\end{itemize}
\paragraph{Атаки по водителю и пассажирам:} элементы конструкции ТС служат сносным укрытием для находящихся внутри. Поэтому водитель и пассажиры:
\begin{itemize}
    \item[--] Находятся в Мягком укрытии, если ТС имеет Прч 5 – 10.
    \item[--] Находятся в Твердом укрытии, если ТС имеет Прч 11 и больше.
\end{itemize}
Некоторые ТС обладают полностью закрытыми корпусами, и управляются изнутри при помощи телеметрических приборов. В этом случае находящиеся внутри не могут быть выбраны целью.
\paragraph{Инос ТС:} техника может выдержать многое, но многое может и не выдержать. Износ ТС проверяется в конце Сцены, если:
\begin{itemize}
  \item[--] ТС перегружено;
  \item[--] ТС Сильно перегружено;
  \item[--] ТС передвигалось по местности с Опасностью, превышающей его Проходимость;
  \item[--] ТС используется очевидно опасным или нецелевым образом. Для Техники размером Б и меньше это включает намеренный наезд на существо величиной с человека или крупного пса.
\end{itemize}
Животные не проверяют Износа, но получают Пв в зависимости от контекста Сцены.

\subsection{Гужевой транспорт}
Животные способны преодолевать водные преграды, а некоторые - еще и летать!
\genAndGet{transport}{transport}{Животное}

\subsection{Наземный транспорт}
\genAndGet{transport}{transport}{Наземный}

\subsection{Водный транспорт}
Проходимость этого транспорта означает то, насколько глубокая у него посадка. Чем выше проходимость, в тем более мелких реках и ручьях может передвигаться транспорт.
% \paragraph{Буксировка водного транспорта} Так как на воде обычно нет значительных перепадов высот, а вода не сильно сопротивляется движению, любые плавсредства можно буксировать, если их вес (вместе с барахлом) не превышает вес Буксира в ??? раз.
\paragraph{Перегрузка водного транспорта} В отличае от наземного транспорта, перегруженный водный транспорт не ломается, а начинает \textbf{тонуть}.
\tbd проверка неприятностей
\paragraph{Сильная перегрузка водного транспорта} приводит к немедленному затоплению ТС.

\genAndGet{transport}{transport}{Водный}

\subsection{Воздушный транспорт}
Проходимость летающих транспортных средств учитывается только при взлете и посадке. В эту категорию так же входит транспорт, способный находиться в открытом косомсе, но не предназначенный для межпланетных и межзвездных перелетов.
\paragraph{Буксировка воздушного транспорта} возможна только на земле.
\paragraph{Перегрузка воздушного транспорта.} Воздушный транспорт не терпит перегрузки. Все проверки Эксплуатации получают Осечку 9, а в случае провала ТС падает на землю.
\paragraph{Сильная перегрузка воздушного транспорта} не позволяет ему взлететь в принципе, хотя он может продолжать катиться по дорогам, если это позволяет Наземная Проходимость.
\genAndGet{transport}{transport}{Воздушный}

% \subsection{Космический транспорт}
% Космолеты имеет настолько большую скорость перемещения, что погони становятся бессмысленными, а перемещение между любыми двумя точками планеты они совершают меньше, чем за сутки, а скорость межпланетного и межзвездного перемещения очень сильно зависит от выбранного сеттинга, поэтому в таблице скорость космического транспорта не указана. Однако космический транспорт все еще имеет Проходимость, которая указвает на то, в насколько сложных условиях этот транспорт может совершать взлет и посадку.
% \newline космические корабли с проходимостью 0 не способны совершать посадку на поверхность планеты - вместо этого они используют средства орбитальной транспортировки, такие, как телепорты или орбитальные челноки.
% \newline космические корабли с проходимостью 1 способны совершать взлет и посадку с поверхности планеты только со специально подготовленных для них космодромов.
% \genAndGet{transport}{transport}{Космический}

\subsection{Исполинский транспорт}
Некоторые ТС достигают невероятных размеров. Обслуживать такой транспорт, а тем более владеть им, не по карману даже самому обеспеченному герою. Но герои могут взять этот транспорт в аренду или получить в пользование от организации - покровителя.
\newline Героям не нужно знать, сколько стоит, как тяжело обслуживается и насколько грузоподъемен транспорт, на котором они отправляются в путешествие, для приключения это - лишние детали. В описании Исполинского транспорта есть только его \textbf{Скорость} и \textbf{Проходимость} - этого достаточно, для того чтобы определить длительность и событийное наполнение пути. 
\newline Во время Остановок, Исполинский транспорт не участвует в Сценах целиком, а является элементом окружения героев. Сцена может развернуться и внутри транспорта.
\paragraph{Старинные Исполины}
\genAndGet{transport-gigantic}{transport-gigantic}{Старинный}
\paragraph{Современные Исполины}
\genAndGet{transport-gigantic}{transport-gigantic}{Современный}
% \paragraph{Фантастические Исполины}
% \genAndGet{transport-gigantic}{transport-gigantic}{Фантастический}
