\chapter{Советы мастеру}
\paragraph{}
Правила игры устроены таким образом, чтобы помочь игрокам и мастеру совместно создавать интересную им историю. О ее жанре и настроении лучше договориться заранее. Происходящее на игре должно так или иначе развлекать всех собравшихся. Если какое-то правило или его отсутствие мешает рассказывать интересную вам и соигрокам историю, упраздните или придумайте его. Разумеется, такие важные действия героев, как сражение и общение, достаточно четко регламентированы, чтобы поддерживать динамику событий. В остальном формулировки правил оставляют большой простор для интерпретации. Не стоит сбрасывать со счетов и субъективный взгляд мастера и игроков. Кому-то покажется вполне уместным Плут-купец, по большей части честный... и эпизодически промышляющий контрабандой, а кому-то будет ближе классический образ Плута в настольных ролевых играх - проныры-полурослика, срезающего кошельки на городском рынке. Одному мастеру будет достаточно Честной физиономии героя даже для оправдания в суде, другой же ограничится ситуационным бонусом к проверкам.
\paragraph{}
Ну и, конечно, \textbf{самое важное правило в настольной ролевой игре - Мастер Всегда Прав}. Соблюдение этого правила необходимо для сохранения динамики игры (которой очень мешают споры любого рода) и поддержания единого воображаемого пространства, границы которого задаются именно мастером. Все споры по правилам и логике происходящего лучше отложить до окончания игровой встречи.
\section{Подготовка к игре}
\paragraph{}
Игровой сюжет во многом похож на литературный — не считая того, что зачастую ни мастер, ни игроки не знают, как он будет развиваться. Тем не менее, существует несколько простых вопросов, обсуждение которых до начала игры значительно улучшит полученный вами опыт.
\begin{enumerate}
\item \textbf{Что происходит вокруг героев?}
\newline
Ответ — экспозиция вашей истории. С его помощью вы устанавливаете значимые факты игрового мира и формируете общее воображаемое пространство. Также на этом этапе стоит договориться о том, в каких декорациях будет разворачиваться сюжет, и какую жанровую окраску он будет иметь. Обсудить этот вопрос можно как до создания героев, так и после него. В первом случае подразумевается, что игроки создадут героев под ситуацию, во втором — заранее созданные герои во многом станут основой для нее!
\item \textbf{Что случилось с героями?}
\newline
Ответ — завязка вашей истории, событие, побуждающее героев к действию.
\item \textbf{Что заставляет героев работать сообща?}
\newline
Ответ позволит вам узнать, в каких отношениях находятся герои, почему помогают друг другу и работают в команде. Возможно, герои — сослуживцы, старые знакомые, подельники или даже родственники... но не исключено, что вцепиться друг другу в глотки им мешает лишь внешняя угроза или несметные богатства, добыть которые в одиночку им не по силам! Не забудьте обсудить возможность противостояния героев друг другу — хотя на нем построено немало сюжетов, далеко не все игровые группы готовы к этому!
\item \textbf{Какова общая цель героев?}
\newline
Иногда ответ на этот вопрос следует из ответа на предыдущий. Разумеется, у героев могут быть и свои собственные цели, не известные их товарищам. Если герои не связаны какими-то общими устремлениями и часто действуют в одиночку, а в центре повествования только их личные цели, история рискует сильно потерять в динамике. 
\end{enumerate}
\paragraph{}
Потратив немного времени на обсуждение, вы сформируете прочную основу для игрового сюжета, исключите неприятные
неожиданности, а также во многом зададите жанр и настроение
истории.
\paragraph{Одна игра для всех:} убедитесь, что все собравшиеся за столом (включая вас) хотят играть в одну и ту же игру. Если один из игроков пришел убивать орков, другой собирается продать оркам груду ржавых мечей, украденных из арсенала феода, а третий — уговорить орков показать затерянный в лесу золотоносный ручей, вам будет довольно сложно угодить всем. Особенно, если лесные орки — лишь незначительная деталь экспозиции вашего сюжета.
\newline
Во избежание этого заранее обсудите с игроками их ожидания
от игры и ее жанр. Не забудьте сообщить игрокам, во что планируете
поиграть лично вы, ведь если в попытке угодить игрокам вы утратите интерес к происходящему, хорошей игры совершенно точно не получится. Идеальный, хоть и не всегда возможный вариант — так называемая "нулевая встреча", во время которой участники игры синхронизируют ожидания. На такой встрече вы сможете поучаствовать в создании героев и включить в сюжетный замысел элементы, которые интересны игрокам.
\newline
Не играйте против игроков — играйте вместе с ними. Это
не значит, что вы должны щадить героев или оставлять без
последствий сделанные ими глупости. Однако зачастую смерть
героев не завершает историю, а обрывает ее на самом интересном
месте. Позаботьтесь о том, чтобы у героев оставались пути для
отступления.
\paragraph{Мастер всегда прав:} это правило — вовсе не оправдание всякого рода тирании. Главная задача мастера — организация общего воображаемого пространства, в котором все явления подчиняются одним и тем же правилам. Конечно, немалую долю этого труда принимает на себя игровая система. Тем не менее, некоторые моменты не регламентированы правилами и отдаются всецело на откуп мастеру, его логике и видению мира и его договоренностям с игроками. К таковым, например, относятся Впечатления статистов (даже в пределах списка Социальных взаимодействий они могут быть очень разными), трактовка спорных моментов правил (например, сохраняет ли доступ к Трюкам и Атрибутам герой, превращенный в бурого медведя, попугая или дракона), а также определение широты возможностей некоторых Атрибутов. Поскольку перед игрой просто невозможно обсудить все, в случае неразрешимого спорного момента мнение мастера имеет больший вес, чем мнение игрока.
\paragraph{Герои:} основа запоминающегося сюжета — герои, не похожие друг на друга. Позаботьтесь о том, чтобы игроки выбрали разные Атрибуты, Трюки, Недостатки и Навыки или хотя бы скомбинировали их отлично друг от друга. Предупредите игроков об Атрибутах и Недостатках, заведомо не подходящих для сюжета. Вряд ли Сплетик, Проповедник или Торговец смогут проявить себя, барахтаясь в бесконечном болоте, населенном гигантскими жабами. Хотя, если героям повстречается племя людей-ящеров или диких эльфов, эти атрибуты заиграют новыми красками!
\paragraph{Оптимизация:} некоторые игроки могут намеренно создать воина с Атрибутами, повышающими Доблесть и Меткость и максимальным Навыком "Владение оружием", или обаятельную соблазнительную Красавицу с максимальным Навыком "Общение". На первый взгляд, такие герои слишком могущественны и легко добиваются своего. Однако лишь на первый. Да, на своем поле они сильны, но... Не боритесь — используйте! Позвольте игроку получить то, ради чего он создал такого героя. Герою обязательно понадобится поддержка друзей в ситуациях, в которых он не силен. Однако не переборщите, создавая такие ситуации, — игрок не должен ощущать бесполезность своего героя и уж тем более не должен чувствовать, что на его героя ополчился весь мир.
\paragraph{Игра роли:} игра роли не имеет никакого отношения к актерской игре, хотя игрок, обладающий актерскими дарованиями, безусловно, придаст истории колорит. Игра роли — это выбор игроком линии поведения, сообразной Характеристикам, Атрибутам и Недостаткам героя, по сути, сознательное, добровольное и разумное самоограничение. Например, Кровожадный Неистовый Дикарь с Интеллектом 6, обстоятельно предлагающий соратникам заманить противника в болото с помощью хитроумного отвлекающего маневра, выглядит довольно странно. Даже если идея сработала, и герои достигли успеха, игрок сыграл самого себя... в теле Дикаря.
Но, если то же самое предложит Осторожный Охотник за головами с Интеллектом 14 и Очками опыта в Военном деле , это будет, безусловно, в рамках роли. Все вышесказанное не означает, что герои — заложники своего образа и не могут отступать от него. Большая часть произведений литературы и кинематографа показывают героя и его характер в развитии. В длительных игровых кампаниях герой не просто может, но обязан меняться! Однако эти изменения должны происходить не на пустом месте, а логически вытекать из пережитого героем. Это и есть игра роли в настольной ролевой игре.
\paragraph{}
С другой стороны, не стоит забывать — игромеханический блок служит надежной гарантией того, что чрезмерно экстравагантные идеи игрока будут реализованы совсем не так, как планировалось. Идея игрока заманить врагов в болото будет транслироваться в игру через Дикаря с Интеллектом 6. Это значит, что шансов на успех у задумки не так много, даже если Дикарь потратил Очки опыта на Военное дело . Вне зависимости от того, провалит ли Дикарь проверку, прибегнет к успеху с Неприятностями или получит поддержку Судьбы, выкупив успех за Нити, история получит интересное развитие и обрастет деталями. Иными словами, если сам игрок не желает пропускать свои идеи через призму образа героя, вам не стоит делать это за него.

\section{Построение сюжета}
\paragraph{Во что поиграть:} правила "Нитей Судьбы> ориентированы на жанр приключенческого боевика. В таких историях герои попадают (случайно или намеренно) в круговорот событий, вершащих судьбы мира и с честью выходят из всех испытаний. Характерными представителями этого жанра являются, например, серия А. Сапковского о ведьмаке, книги А. Дюма о приключениях мушкетеров или кинотрилогия по роману Дж. Р. Р. Толкина "Властелин Колец". Сражения, погони, путешествия, насыщенность действием — важнейшие составляющие сюжета. Однако система позволит вам рассказывать абсолютно любую историю, в которой герои принимают решения и совершают действия, последствия которых важны и способны изменить что-то, хотя бы в их собственной жизни. Решения, действия и последствия — самая важная часть ролевой игры. Не забывайте — герои избраны Судьбой. Позвольте игрокам ощутить это!
\paragraph{}
Вам не потребуется подробный план сюжета. Более того, он может навредить, так как передача повествовательных прав предполагает активное вмешательство игроков в сюжетную канву. Подготовьте завязку, ключевых статистов и персон, задействованных в сюжете, и позвольте событиям развиваться непредсказуемо для всех. Сюжет имеет определенные законы построения. Он состоит из экспозиции, завязки, развития, кульминации и развязки (к которой может примыкать эпилог). Рассмотрим историю, герои которой — жители небольшой деревеньки на границе леса, населенного гоблинами.
\paragraph{Экспозиция:} эту часть лучше вынести за пределы игровой встречи. Она позволяет игрокам задать все интересующие вопросы. Например, хорошо ли укреплена деревня, далеко ли замок сеньора, в каких отношениях селяне с племенем гоблинов. Постарайтесь дать как можно более исчерпывающую информацию на этом этапе, иначе игрокам придется задавать вопросы в дальнейшем (что повредит динамике) или выстраивать действия героев исходя из своих представлений, которые могут (и наверняка будут) отличаться от ваших.
\paragraph{Завязка:} событие, побуждающее героев к действию. Например, гоблины требуют у селян десятерых девушек для жертвоприношения на восходе луны и угрожают сжечь деревню, если селяне откажут. Завязка — отправная точка вашей истории, свершившийся факт.
\paragraph{Развитие:} именно экспозиция и завязка определяют дальнейшие действия героев — отправятся ли они в замок сеньора за гвардией феода, организуют оборону деревни, попробуют обмануть гоблинов, договорятся с ними или отдадут им желаемое. Эта часть сама по себе также дробится на мини-сюжеты — сцены и вехи, имеющие ту же структуру, что и основной сюжет.
\paragraph{Кульминация:} момент наивысшего напряжения в сюжете. Каким будет этот момент, зависит от результатов действий героев. Вот несколько вариантов кульминации этой истории:
\begin{itemize}
\item[--] Утомленные герои с замиранием сердца наблюдают, как горстка гвардейцев противостоит орде разъяренных гоблинов.
\item[--] Один из героев вызывает на бой лучшего воина гоблинского племени и сражается с ним.
\item[--] Переодевшись в женские платья, герои позволяют гоблинам отвести их на капище и нападают на верховного жреца.
\item[--] Герои пытаются убедить гоблинов, что мясо коров и овец понравится гоблинскому идолу гораздо больше человеческого.
\item[--] Герои решают, что лучше пожертвовать частью, чем погибнуть всем, и отдают девушек гоблинам.
\end{itemize}
\paragraph{Развязка:} финал истории, вытекающий из кульминации. Преуспели герои или потерпели неудачу? Остались ли они живы? Что потеряли и приобрели? Ответы на эти вопросы игроки получают в развязке. Вот как может завершиться история с гоблинами:
\begin{itemize}
\item[--] Селяне чествуют победоносных гвардейцев и выдают одну из девушек замуж за командира отряда. Об усилиях героев никто не вспоминает.
\item[--] Лучший воин племени повержен, и гоблины в страхе отступают.
\item[--] Герои гибнут на капище, но племя гоблинов лишается верховного жреца и теряет многих воинов. На какое-то время деревня спасена.
\item[--] Гоблины забирают всех коров и овец, что были в деревне. Кое- кто из селян считает, что это слишком большая цена за десяток девиц.
\item[--] Герои слышат отчаянные крики, доносящиеся из леса, и пытаются убедить себя, что это просто ветер.
\end{itemize}
\paragraph{Эпилог:} если ваша история рассчитана на одну-две встречи, в ней может и не быть эпилога. Но если вы планируете длительную историю, эпилог необходим — он содержит элементы экспозиции и завязки для следующего приключения!
\section{ВО ВРЕМЯ ИГРЫ}
\paragraph{Действие и еще раз действие:} если цели героев слишком глобальны, динамика событий может серьезно пострадать. Время от времени игроки чересчур уходят в построение громоздких планов вместо того, чтобы играть. Поэтому рекомендуется дробить глобальные цели на небольшие, промежуточные. Например, цель «завоевать соседнее королевство» сразу же распадается на «собрать информацию об армии противника», «найти средства для снаряжения армии», «снарядить войска». Обычно игроки сами неплохо справляются с поиском промежуточных целей. Но, если вы чувствуете, что игра превращается в обсуждение игры, не стесняйтесь прервать это событием, побуждающим героев к немедленным действиям. Черпайте вдохновение из идей игроков. Пускай из соседнего королевства прибудет посольство, безумный алхимик попросит ссуду на грандиозный эксперимент по превращению свинца в золото, окончание усобицы баронов оставит без работы крупный отряд наемников. Помните, игра хороша, пока никто не скучает.
\paragraph{Личное время:} игроки создают героев для действия. Конечно, есть такие, кому просто приятно посиживать в уголке и наблюдать, как играют остальные, но это скорее исключение. Позаботьтесь о том, чтобы каждый из героев получил свою минутку славы хотя бы раз за игровую встречу.
\paragraph{Давайте разделимся:} иногда логика ситуации вынуждает героев разделиться. В подобных случаях учитывайте, что часть игроков на время выпадет из действия, и общая динамика истории понизится. Если вы не так давно выступаете в роли мастера, сюжетов, в которых разделение команды героев неизбежно и предполагается заранее, лучше избегать.
\paragraph{Вызов:} возможность потерпеть неудачу — один из основных двигателей игры. Виды деятельности героев, где шанс провала ничтожно мал или отсутствует вовсе, вряд ли способны занять внимание игроков надолго.
\newline
Чем больше Очков опыта получают герои по ходу игры, тем шире становятся их возможности. Возрастает и влияние героев на окружающий мир. Это необходимо учитывать при построении приключения. Охота на матерого волка может быть смертельно опасна для горстки скверно вооруженных крестьян, но рыцарь прикончит его одним ударом меча. Организуйте сюжет таким образом, чтобы игроки могли ощутить силу и значимость своих героев. Предложите героям важную дипломатическую миссию, управление феодом или битву с огромным драконом.
\newline
В то же время самые могучие герои нередко пасуют в областях, не покрытых их Атрибутами, Трюками и Навыками. Это может стать хорошей основой сюжета и побудить героев к действию. Великий военачальник принудил к повиновению десятки народов, но получится ли у него добиться послушания от своенравной дочери? Богач привык к заискиванию и лицемерному обожанию, но что если он попадет туда, где деньги не дороже грязи под ногами? Искусный интриган возвысился при помощи лжи и коварства, но чем он ответит на открытый и честный вызов
на дуэль?
\paragraph{Ходы Судьбы:} именно Нити Судьбы, а не запредельные Навыки и Характеристики делают героя героем. Внимательно изучите раздел книги, посвященный Нитям, и попросите игроков сделать то же самое. Своевременно оборванная Нить может полностью изменить игру. Не лишайте игроков возможности вмешаться в сюжет! Если игрок накопил четыре Нити и прикончил Главного Злодея выстрелом в глаз, он в своем праве. А уж если игрок воспользовался Ходом и влюбил злодея в своего героя, убедил злодея покаяться или объявил, что злодей — отец его героя… Нет, игрок вовсе не испортил задуманную вами эпическую битву, а подарил сюжету великолепный поворот. Цените это и будьте к этому готовы — знайте Ходы, в том числе Ходы Атрибутов и Грани, выбранные игроками. Несмотря на право вето, которое имеет мастер, не стоит пользоваться им слишком часто — абсолютное большинство идей игроков стоит того, чтобы включить их в сюжет. Если вы все же сомневаетесь, уделите немного времени обсуждению Хода — скорее всего, все собравшиеся так или иначе придут к соглашению.
\paragraph{Ходы без обрыва Нитей:} разумеется, герои, использующие Атрибуты без обрыва Нитей, вполне дееспособны. Не задавайте слишком высокие уровни проверок, ведь то, что сложно для героя с Атрибутом, для героя без Атрибута — на грани возможного.
\paragraph{Недостатки, Темные стороны и Грани:} если ваши игроки впервые знакомятся с «Нитями Судьбы», вам придется настроить их на нужный лад. Подбрасывайте достаточно заметные и очевидные возможности для Капризов Судьбы. Допустим, герой скрытно наблюдает за переговорами контрабандистов на городском рынке. Если герой Пьяница, пускай в соседней таверне продают три кружки эля по цене двух. Если герой Любвеобилен, обратите его внимание на миловидную цветочницу в торговом ряду напротив. Если герой Болтлив, сообщите игроку, что рядом обсуждают свежие новости. Предложите Плуту возможность быть узнанным кем-то из контрабандистов, Красавцу — назойливое внимание богатой матроны, а Аристократу — приставания попрошайки. Обсудите с игроком последствия. В самом скором времени игроки втянутся в сюжетостроение!
\paragraph{Неприятности:} вводите Неприятности в тех случаях, когда нельзя логически определить возможность того или иного происшествия, или если игрок спорит с вами о его вероятности. Прибегайте к Неприятностям, когда герои ведут себя неосмотрительно или просто неразумно. Если они ищут проблем — дайте их сполна… но и оставьте героям шанс выпутаться, если они приложат усилия! В отличие от Недостатков и Темной стороны последствия Неприятностей не обязаны быть известны заранее, хотя лучше дать игроку намек, чтобы он мог принять Неприятность и протянуть к герою Нить или, наоборот, избежать проблем, оборвав Нить.
\newline
Неприятности можно разделить на два типа: создающие события и лишающие событий. К создающим события относятся случайные встречи, угрожающие героям, задерживающие их или расходующие их ресурсы. Нападение головорезов в трущобах, визит приставучего болтливого родственника, умоляющий о помощи статист — из их числа. Лишающие событий Неприятности — трактирщик, который ничего не знает об убийстве, труп гонца, при котором не оказалось послания герцога, сундук, вместо сокровищ наполненный истлевшим тряпьем. При этом и те, и другие оказывают прямое влияние на развитие сюжета. Более того, абсолютно нормально, если герои с помощью Ходов Судьбы или применения других ресурсов извлекли из Неприятностей выгоду!
\paragraph{Менеджмент Нитей:} в начале игровой встречи у героев по две Нити. Этого достаточно для совершения одного-двух Ходов Судьбы. Но, если сюжет насыщен событиями, Нитей может не хватить. В таких случаях рекомендуется обновлять Нити по завершении важных сюжетных вех или перед их кульминацией. Сюжетная веха — это маленькая история в большой. Например, выход хоббитов из Дольна — начало такой истории, а ее кульминация — битва со стражем озера. За воротами Мории хоббитов ожидает начало новой вехи. Ее кульминация — сражение Гэндальфа и барлога.
\paragraph{Узы:} применение Уз существенно повышает нагрузку на мастера. Вам придется следить за тем, остается ли герой в рамках избранных игроком Уз, и судить об этом по собственному усмотрению. Это важно — ведь в противном случае герой просто получит дополнительную Нить, и его жизнь никак не осложнится. Также учитывайте, что герой с 2 Узами с самого начала игры может получить Критический успех на любую проверку.
\paragraph{Темные Нити:} отличный способ повысить ставки. Используйте Темные Нити как в открытом противостоянии, так и за кадром. В любом случае держите в голове — этот инструмент в первую очередь нужен для развития истории, а не для убийства героев.
\paragraph{Влияния на героев:} если кто-то из персон или статистов пытается убедить героя изменить линию поведения и преуспевает в проверке, предложите игроку начислить герою 1 Очко опыта. Если игрок принимает влияние (и опыт), это значит, что статист смог обольстить, обмануть или переубедить героя. Точно так же разрешаются попытки манипуляций между двумя героями. По взаимной договоренности с игроками вы можете использовать обычные проверки Воли и Чародейства без начисления опыта, чтобы манипулировать героями, однако один герой не может навязать другому свою волю с помощью Навыков или заклинаний, пока игрок на это не согласится!
\paragraph{Проверки:} каждый бросок кубика в игре — небольшое событие, ведущее к чему-то. Не заставляйте игроков бросать кубик, если успех или неудача проверки никак не повлияют на развитие истории. Избегайте проверок, убивающих героев сразу. Например, если герой карабкается по скале и проваливает проверку Атлетики, позвольте ему применить Успех с Неприятностями, чтобы уцепиться за выступ парой метров ниже, или же позвольте другому герою совершить проверку Атлетики, чтобы вовремя подхватить падающего. Разумеется, не стоит доводить этот принцип до абсурда — последствия принятых игроком решений очень важны, даже если это неудача или смерть.
\paragraph{Статисты:} вам не нужен подробный блок характеристик для каждого статиста, которого герои встречают в игре, особенно с учетом того, что вы не имеете полного контроля над развитием сюжета. Вместо этого определите для себя значимые свойства статиста. Например, дикарь, который должен по вашей задумке напасть на героев, обладает Доблестью 15 (с учетом Ловкости, Силы, Бонуса к Повреждениям за двуручный топор и Навыка «Владение оружием») и Выносливостью 18 (что дает ему 54 Единицы Здоровья). Вы можете просто придумать эти цифры. В случае необходимости вы легко вычислите другие параметры статиста. Например, двуручный топор имеет Бонус к Повреждениям +5, на остальную Доблесть остается в сумме 10. Значит либо дикарь весьма ловок и силен (что очень вероятно), либо он великолепно владеет топором, а значит, довольно умен. Но если герои сохранят дикарю жизнь, запишите или запомните эти цифры на случай появления дикаря в дальнейшем — его Характеристики стали фактом истории!
\paragraph{Сложность боевых сцен:} если сражения занимают значимое место в вашей истории, при построении боевых сцен обращайте внимание на следующее:
\begin{enumerate}
\item \textbf{Число противников.} Это особенно важно, если вы используете правило «Все на одного». В таком случае даже самые слабые статисты получают возможность атаковать с Преимуществом, если имеют перевес в численности.
\item \textbf{Способность статистов наносить Опасные раны героям.} Исходите из того, что если статист наносит Опасную рану героям, специализирующимся на боевых столкновениях, при броске 16 и более на кубике, то он не представляет серьезной опасности.
Исключение из этого — численное превосходство.
\item \textbf{Число атак противников.} Обычно даже самые ужасные монстры ограничены тремя атаками за Очередь (хотя благодаря подбору Атрибутов и Трюков это число может возрасти до 8). Но уже несколько статистов, вооруженных легким оружием, могут создать для героев проблемы, даже если в большинстве случаев не в состоянии наносить Опасные раны.
\item \textbf{Способность героев наносить Опасные раны статистам.} Чем она выше, тем больше шанс на скорую победу героев. Не забывайте, что способность наносить Опасные раны не всегда связана с высокими значениями Доблести и Меткости и во многом зависит от Выносливости и Защиты статистов.
\end{enumerate}
\section{После игры}
Игровая встреча завершается раздачей Очков опыта. За что его выдавать — часть предыгровой договоренности. Среднее количество Очков опыта, которые приобретают герои за время одной игровой встречи, варьируется от 1 до 5. В это число не входят Очки опыта, которые герой получил, принимая влияние других героев или статистов.
\newline
Поскольку «Нити Судьбы» — ролевая игра, при начислении опыта вы можете (хоть и не обязаны) учитывать как успехи героев в решении внутриигровых проблем, так и игру роли в исполнении игроков.
\newline
Разумеется, вы можете выделить особо удачные идеи кого-то из игроков и начислить их героям больше опыта, но лучше этим не злоупотреблять. В конце концов, главная награда в настольной ролевой игре — сам процесс и общение с единомышленниками.
