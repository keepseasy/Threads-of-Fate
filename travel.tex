\chapter{Путешествия и опасности в пути}
\tbd литературная вставка

\section{Встречи и находки}
Никто заранее не знает, когда герои на своем пути встретят неожиданную компанию или занимательную ситуацию. Даже просто гуляя по крупному городу они могут попасть в гущу событий, не говоря уж об исследовании таинственных руин!
\newline Проверка Встреч и Находок является проверкой Неприятностей, в которой дополнительно учитывается четность выпавшего числа. В таблице указаны возможные наполнения сцены в соответствии с проверкой.

\begin{tcolorbox}
Эта проверка требует броска кубика, даже если игроки принимают Каприз Судьбы. Проверка в этом случае определяет, какое из двух зол встречают герои: конфликтную Встречу или опасную Находку.
\newline При использовании Хода \textit{"Повезло!"} игроки могут сами решить, какая сцена их ждет: приятная Встреча или полезная Находка.
\end{tcolorbox}

\begin{center} \begin{tabular}{|p{3cm}|p{6.5cm}|p{6.5cm}|} \hline
\textbf{Результат проверки Неприятностей} & \textbf{Встречи(Нечет)} & \textbf{Находки(Чет)} \\ \hline
\textbf{В добрый путь}\newline\textit{(19-20)} & Встреченные статисты практически беззащитны. Герои могут помочь им, а могут ограбить или убить - вряд ли сопротивление или последствия будут серьезными. \newline \textbf{Эмоциональный фон сцены:} Неформальная обстановка. & Находка сулит богатства. Недавно покинутый дом, оставленный транспорт - скорее всего, там есть, чем поживиться. А вокруг в кои-то веки нет никого, кто может или хочет помешать. \\ \hline
\textbf{Место интереса}\newline\textit{(13-18)} & Встреча не несет в себе угрозы. Статисты многочислены и/или хорошо вооружены, но в меру дружелюбны. Они скорее настроены на общение и торговлю, чем на резню. \newline \textbf{Эмоциональный фон сцены: Отсутствие угрозы.} & Занятное местечко. Давно оставленный лагерь, полуразрушенный дом или полусгнившая телега. Если внимательно осмотреть место, возможно можно найти полезную вещицу. Но останется вопрос - а оно того стоило? \\ \hline
\textbf{Напряженная ситуация}\newline\textit{(7-12)} & Встреча опасно близка к конфликту. Статисты напуганы или разъярены, либо их дело не терпит соглядатаев. Если герои не проявят осторожность и такт, заговорит оружие. \newline \textbf{Эмоциональный фон сцены: Потенциальная опасность.} & Находка кричит об опасности. Разграбленный караван, свежее побоище, растерзанный труп, сожженая машина. Здесь точно найдется, чем поживиться, но никто не знает, к чему это приведет. \\ \hline
\textbf{Заваруха}\newline\textit{(1-6)} & Герои влипли в передрягу. Чтобы решить дело миром, придется расстаться с богатством, имуществом и чувством собственного достоинства - или положиться на Судьбу.
\newline \textbf{Эмоциональный фон сцены:} Явная опасность. & От находки разит бедой, что бы это ни было - разрушенный взрывом мост, засыпанный камнями перевал, кровавый алтарь со свежим жертвоприношением. Просто пройти мимо не получится, и герои это знают. \\ \hline
\end{tabular} \end{center}

\section{Путешествия}
Путешествия - один из основополагающих элементов многих жанров. Какая бы причина ни вынудила героев двинуться в путь, в дороге их подстерегает немало трудностей. Если путешествия являются важной частью вашей истории и вы желаете выяснить, насколько сложен будет путь, следуйте пунктам ниже:
\begin{enumerate}
  \item \textbf{Сделал дело - гуляй смело.} Игроки должны решить, кто из героев или статистов в караване станет:
    \begin{itemize}
      \item \textbf{Навигатором}, который ищет безопасный путь, определяет места для стоянки и пополнения припасов. Навык Выживания - важнейший для Навигатора.
      \item \textbf{Механиком}, который следит за состоянием машин и прочей техники в караване. Ему потребуется Эксплуатация.
      \item \textbf{Погонщиком}, который следит за тем, чтобы глупые скоты не свалились в яму и не наелись Зараженной травы. Он использует Обращение с животными.
      \item \textbf{Разведчиком}, который идет впереди каравана, отслеживая подозрительное и разыскивая интересные места. Ему понадобится Наблюдательность.
      \item \textbf{Проводнком}, который проведет караван мимо препятствий и нежелательных встреч. Скрытность и Наблюдательность - главные орудия Проводника. Если караван передвигается Маршем, Проводник останется не у дел. Ему придется тряситсь в кузове, и ворчать под нос.
    \end{itemize}

    \begin{tcolorbox}
      Механик и Погонщик - ситуативные роли. Если в караване нет техники и/ или вьючных животных, они будут слонятся без дела и всем мешаться.
    \end{tcolorbox}
    Одну роль могут выполнять несколько героев (используйте правила Взаимопомощи), но один герой не может выполнять несколько ролей. 
    \newline Когда роль остается невыполненной, последствия могут быть самыми плачевными.
    \paragraph{Если в караване нет Навигатора, Механика или Погонщика,} считайте, что при соответствующих проверках выпало 5. Если разница между этим результатом и сложностью проверки превышает 10, проверка Критически провалена.
    \paragraph{Если в караване нет Разведчика,} герои крайне уязвимы для опасностей, которые важно заметив вовремя. Сцены Встреч и Находок начинаются сразу после того, как определен их тип. В Боевых сценах отряд подвергается Внезапному нападению.
    \paragraph{Если в караване нет Проводника,} отряд не может избежать Встреч и Находок и обязан принять участие в Сценах, с ними связанных. Даже если Разведчик сообщил о том, что они не сулят ничего хорошего.

    \item \textbf{По дороге всегда быстрее.} Определите Опасность местности (ОМ), по которой пойдут герои. Если они преодолевают местность с разной Опасностью, используется больший.
    \begin{center} \begin{tabular}{|p{7cm}|p{7cm}|c|} \hline
      \textbf{Путь по суше} & \textbf{Путь по воде} & \textbf{ОМ} \\ \hline
      Обжитые пригороды, фермерские угодья, сохранившиеся и вновь налаженные дорожные сети. & Открытый океан, грандиозный, суровый и величественный. & 0 \\ \hline
      Прерии и равнины. Вряд ли Катастрофа сильно изменила эти места.  & Архипелаги, прибрежная зона материков. & 1 \\ \hline
      Лесистые и болотистые равнины, холмы, руины городов. & Широкие и глубокие реки. & 2 \\ \hline
      Лесные дебри, топи, скалистые холмы, руины больших городов. & Широкие, но мелеющие реки. & 3 \\ \hline
      Горы, пустыни, руины мегаполисов, области, серьезно пострадавшие при Катастрофе. & Узкие извилистые реки с непредсказуемым фарватером. & 4 \\ \hline
      Джунгли, заболоченная чаща, руины крупных промышленных центров, области, наиболее пострадавшие при Катастрофе. & Извилистые реки с быстрым течением. Острые камни и пороги прилагаются. & 5 \\ \hline
    \end{tabular} \end{center}
    \begin{tcolorbox}
      \paragraph{Терра инкогнита} Если герои отправляются в путешествие без карты, Повысьте ОМ на 1(макс.5). Если герои являются пионерами и карт местности, которую они покоряют просто не существует, повысьте ОМ на 2(макс.5).
    \end{tcolorbox}

    \item \textbf{Долго ли, коротко ли...} Определите длительность пути в днях, ориентируясь на скорость самого медленного участника, и подсчитайте Сложность пути по таблице:
      \begin{center} \begin{tabular}{|c|c|} \hline
        \textbf{Длительность} & \textbf{Сложность пути} \\ \hline
        1-7 дней & 8 + 2*ОМ \\ \hline
        8-14 дней & 10 + 2*ОМ \\ \hline
        15-21 день & 12 + 2*ОМ \\ \hline
        22-28 дней & 14 + 2*ОМ \\ \hline
        29 дней и больше & 16 + 2*ОМ \\ \hline
      \end{tabular} \end{center}
      \begin{tcolorbox}
        Для простоты определения Длительности путешествия считайте, что рельеф местности не влияет на скорость путешествия.
      \end{tcolorbox}
      \textbf{Марш:} герои могут увеличить скорость движения вдвое. При этом они получают Помеху на Наблюдательность(Мд) и должны проверить Атлетику(Вн) против \textbf{|15|}. При провале герои  Устают, пока не уйдут в Антракт. 
      \newline ОM на марше возрастает на 1. Если в караване есть транспортные средства, скакуны или вьючные животные, ОМ на марше возрастает на 2.
      \newline \textbf{Тише едешь - дольше будешь:} герои могут ополовинить скорость движения и получить Преимущество на проверки Скрытности и Наблюдательности Проводника, чтобы избежать Встреч и Находок.

  \item \textbf{Да что вы, ребята, я сам здесь впервой!} Когда определены роли героев и Сложность пути, настает время проверки Путешествия против \textbf{|Сложности пути|}:
    \begin{itemize}
      \item Навигатор совершает проверку Выживания;
      \item Механик совершает проверку Эксплуатации;
      \item Погонщик совершает проверку Обращения с животными;
    \end{itemize}
    Результат проверки равен сумме успехов и провалов Навигатора, Механика и Погонщика. Например, если Навигатор провалил проверку на 5, а Погонщик - на 8, величина провала составит 13. Если Навигатор преуспел на 7, а Погонщик провалил проверку на 4, величина успеха составит 3.
    \newline КУ и КП создают дополнительные эффекты, но при этом все равно необходимо подсчитать общую сумму успехов и провалов. Если во время проверок возник и КУ, и КП, в игру войдут оба эффекта.
    \newline \textbf{Критический Провал} одной или нескольких проверок приведет к событию \textbf{"Все, приехали"}, прерывающему Путешествие на полдороге.  
    \newline \textbf{Критический Успех} одной или нескольких проверок дает каравану Преимущество на проверки Встреч и Находок, за \newline исключением события \textbf{"Все, приехали"}.
    \newline Успех проверки Путешествия означает, что путешествие проходит без осложнений и позволяет героям наслаждаться относительным комфортом. Положительные эффекты складываются. Например, если сумма успеха от 6 до 8, то караван получает и "Полноценный отдых" и "Чудесные деньки".
    \newline После отправления каравана все герои уходят в Антракт до следующей остановки.
    \begin{center} \begin{tabular}{|c|p{10cm}|} \hline
      \textbf{Величина успеха} & \textbf{Эффект} \\ \hline
      0-2 & Отсутствие провала - уже успех в своем роде. \\ \hline
      3-5 & \textbf{Полноценный отдых.} Герои восстанавливают удвоенное число ЕЗ во время Антрактов. \\ \hline
      6-8 & \textbf{Чудесные деньки.} Первая проверка героев во время Остановок совершается с Преимуществом. \\ \hline
      9-12 & \textbf{Свободное время.} Между двумя остановками герои могут уходить в Интерлюдии, доступные во время Антракта, заниматься ремеслом и ремонтом - если у них есть с собой все необходимое. \\ \hline
      13 и больше & \textbf{Спокойные места.} Герои не проверяют Скрытую угрозу во время Остановок. \\ \hline
    \end{tabular} \end{center}
    \textbf{Цена провала:} при провале проверки Путешествия путники теряют ЕЗ, а Опасность Встреч и Находок, изначально равная нулю, возрастает.  Сверьтесь с таблицей, чтобы определить последствия.
    \newline \textbf{Заражение.} Если путь лежит по землям, пораженным Заразой, при провале проверки Путешествия герои не только теряют ЕЗ, но и повышают Интоксикацию на число потерянных ЕЗ. Предполагается, что защитное снаряжение - если оно было, не сработало из-за человеческого фактора и стечения обстоятельств.

    \begin{center} \begin{tabular}{|c|p{5cm}|p{5cm}|} \hline
      \textbf{Величина провала} & \textbf{Повреждения} & \textbf{Опасность Встреч и Находок} \\ \hline
      0-2 & ОМ(мин 1) & 1 \\ \hline
      3-5 & ОМ+2 & 2 \\ \hline
      6-8 & ОМ+4 & 3 \\ \hline
      9-12 & ОМ+7 & 4 \\ \hline
      13 и больше & ОМ+10 & 5 \\ \hline
    \end{tabular} \end{center}

  \item \textbf{Остановки в пути.} Приятные неожиданности редки в пустошах, зато других хоть отбавляй. Проверьте Неприятности и определите число Остановок. Каждая из них принесет Встречу или Находку.
    \begin{center} \begin{tabular}{ |p{2.7cm}|p{12cm}| } \hline
      \textbf{Результат проверки Неприятностей} & \textbf{Количество Остановок} \\ \hline
      19-20 & 5 \\ \hline
      13-18 & 4 \\ \hline
      7-12 & 3 \\ \hline
      1-6 & 2 \\ \hline
    \end{tabular} \end{center}
    \textbf{Сложность Сцены:} Чтобы определить сложность проверок в Сцене, не связанных с Общением и боем, сложите \textbf{|10 + 2*ОМ + [Опасность Встреч и Находок]|}.
    \begin{tcolorbox}
      Остановки не обязательно будут распределены по всему пути равномерно. Мастер волен решать в соответствии с логикой мира и повествования, на каком отрезке пути были совершены значимые Остановки. Возможно, начало похода было насыщено событиями или же все самое интересное проихошло только под конец путешествия.
    \end{tcolorbox}

  \item \textbf{Все, приехали.} Если при проверке Путешествий выпал КП, случается событие, прерывающее путь. Караван проходит половину пути или меньше. Чтобы продолжить движение, придется начать Путешествие заново - с той точки, в которой забуксовали герои. 
  \newline Число Остановок в пути уменьшается вдвое, а последняя Встреча или Находка начнется с Заварухи. Той самой, которая и привела к досадной задержке.

  \item \textbf{Встречи и Находки.} Совершив проверку Встреч и Находок, определите наполнение сцены. Обратите внимание, что проверка Скрытой угрозы все еще может серьезно изменить смысловое наполнение сцены.
  
  \item \textbf{Скрытая угроза.} Во время Встречи или Находки (не обязательно в начале!) проверьте Скрытую угрозу. Отнимите от результата Опасность Встреч и Находок, определенную на 4 этапе. Скрытая угроза не обязательно проявится в начале Сцены, но дает мастеру хорошее представление о том, чем она может закончиться.
  
  \item \textbf{А что это унас тут?} Для того чтобы заранее заметить Встречу и не проморгать Находку, Разведчик каравана должен преуспеть в проверке Наблюдательности(Мд) против Сложности Сцены.
  
  \item \textbf{Я тут мимо проходил.} Если отряд желает избежать Встречи, Проводник должен преуспеть в Скрытности(Ин) против Сложности Сцены, чтобы караван не привлк внимания. При провала, Сцена Встречи начинается, а проверки Впечатления от героев совершаются с Помехой. 
    \newline Если караван желает обойти Находку стороной, Проводник отряда проверяет Наблюдательности(Мд) против Сложности Сцены, чтобы найти безопасный обходной путь. Герои теряют ЕЗ, равные величине провала, и повышают Токс на столько же единиц.

  \item \textbf{Поболтаем?} Если Эмоциональный фон Встречи и проверки Впечатлений достаточно хороши, и нет никакой спешки, большинство статистов готовы общаться, торговать, и всячески взаимодействовать с героями.
  
  \item \textbf{Отдохнули и в путь}. После каждой Остановки следует Антракт, во время которого герои продвигаются по намеченному маршруту. Они восстанавливают ЕЗ и Эн, но не могут использовать Антракт для других дел.
\end{enumerate}
\begin{tcolorbox}
  \textbf{Маршрут изменен.} Никто не знает заранее, что таит в себе очередная Встреча или Находка. После очередной Остановки герои могут решить пойти в другую сторону. В этом случае все ожидающие их Встречи и Находки так и останутся неразведанными и начинается новое Путешествие.
\end{tcolorbox}

\section{Короткие Путешествия}
Иногда повествование не распологает к полноценному Путешествию - фокус сконцентрирован на других аспектах или само путешествие длится меньше дня. В этом случае алгоритм Путешествия сокращается:
\begin{itemize}
  \item Сложность пути равна \textbf{|8+2*ОМ|};
  \item Остановки в пути отсутствуют;
  \item Проверки Встреч и находок и Скрытой угрозы совершаются один раз и определяют экспозицию и контекст первой Сцены в пункте назначения.
\end{itemize}
Остальные проверки совершаются по обычным правилам Путешествий.

\section{Перенос тяжестей}
Веса в игре измеряются в килограммах.
\paragraph{Комфортная нагрузка} героя равна значению его \textbf{|Сл*3|}. 
\paragraph{Герой Нагружен,} если несет нагрузку, превышающую Комфортную. Его Ск падает вдвое. 
\paragraph{Герой Перегружен,} если его нагрузка превышает параметр его \textbf{|Сл*5|}. Все его Активные проверки совершаются с Помехой, а все атаки по нему совершаются с Преимуществом. 
\paragraph{Максимальная нагрузка героя,} с которой он может идти, равна его \textbf{|Сл*10|}.
\newline Герой может толкать, тянуть и отрывать от земли вес, вдвое превышающий его Максимальную нагрузку. Если герой толкает или тянет вес, превышающий его Максимальную нагрузку, его Ск падает до 1. 
\paragraph{Большие герои могут нести больший вес.} Герой увеличивает вдвое все параметры, связанные с переносом тяжестей, за каждую категорию размера больше Среднего. Маленькие герои ополовинивают эти параметры за каждую категорию размера меньше Среднего.
\paragraph{Четвероногие существа,} такие, как кони и мулы, способны переносить больший вес. Увеличьте вдвое все параметры, связанные с переносом тяжестей после учета бонусов или штрафов за Размер. 
\newline Например, чтобы определить комфортную нагрузку для Большого коня, умножьте его Сл на 3, затем на 2 за размер, и еще на 2 - за четвероногость. Маленькая лайка при этом будет способна нести такой же вес, как Средний человек с аналогичным параметром Силы. Гигантские пауки, улитки и змеи считаются четвероногими для определения нагрузки.

\section{Плавание}
\paragraph{Перемещение вплавь:} в воде герой Перемещается по Трудному ландшафту. Иногда герой может Перемещаться вплавь с полной Ск или даже быстрее, например, если плывет по течению, хотя в таких случаях ему потребуются проверки Атлетики (Сл). Используйте стандартную таблицу сложностей для отображения быстрого течения или неблагоприятных погодных условий.
\paragraph{Бой в воде:} герой встрявший в бой в воде, совершает большую ошибку, а свои атаки - с Помехой. 
\newline Если герой несет вес, превышающий значение его Силы, потребуются проверки Сл или Атлетики (Сл), чтобы держаться на воде. Если герой Нагружен, он получает Помеху на проверку. Если герой Перегружен, он получает 2 Помехи. Герой автоматически проваливает проверку, если несет Максимальный вес. 
\newline Проваливший проверку герой находится в состоянии Удушья, хотя провал не всегда означает, что герой идет ко дну. 

\section{Падение с высоты}
Падая или прыгая, герой может пострадать. 
\paragraph{Безопасная высота:} высота, упав или спрыгнув с которой, герой не должен совершать проверку Падения. Безопасная высота понижается (до минимума в 1 метр) на:
\begin{itemize}
  \item 1, когда он Нагружен;
  \item 2, когда он Перегружен;
  \item 3, когда он несет Максимальный вес.
\end{itemize}
\paragraph{Быстрый спуск:} если герой прыгает и может смягчить падение при помощи окружения, то для него Безопасная высота равна \textbf{|МЛв+МВн+2|} метров.
\paragraph{Падение:} если герой падает, но смог сгруппироваться, подготовиться и зацепиться за что-нибудь, то для него Безопасная высота равна \textbf{|МЛв+2|} метров.
\paragraph{Свободное падение:} когда герою не за что зацепиться при падении, и процесс совершенно неконтролируем, то Безопасная высота для него составляет 1 метр.
\paragraph{Прыжок веры:} если герой прыгает, не имея возможности за что-то зацепиться, но в целом готов к приземлению, Безопасная высота для него равна \textbf{|МВн+1|} метров.
\paragraph{Проверка падения:} Когда герой прыгает или падает с высоты, превышающей Безопасную дистанцию, он проверяет Внезапную смерть под контролем Ловкости или Атлетики(Лв) против \textbf{|высоты прыжка или падения в метрах|}.

\section{Прыжки}
Максимальная дистанция прыжка нечасто попадает в фокус повествования. Но если вдруг такое случилось, герой может прыгнуть:
\begin{itemize}
  \item На свою \textbf{|Ск*10|} в сантиметрах при прыжке в высоту;
  \item На свою \textbf{|Ск/2|} в метрах при прыжке в длину, если не имел возможности разбежаться;
  \item На свою \textbf{|Ск/2+[расстояние разбега в метрах]|}. Максимальная дистанция прыжка с разбега не может превышать Ск героя.
\end{itemize}
В любой из формул Ск может быть заменена на МСл или Атлетику(Сл), если это даст герою лучший результат.
\newline Если герой прыгает, когда он Нагружен, он получает штраф -1 к проверке.
\newline Если герой прыгает, когда он Перегружен, он получает штраф -2 к проверке.
\newline Если герой прыгает с Максимальным весом, он получает штраф -3 к проверке.

\section{Ловушки}
Герой попадает под действие ловушки, если не заметил ее вовремя при помощи проверки Наблюдательности(Мд), активировал случайно, неудачно применив Наблюдательность(Ин), либо неудачно попытался преодолеть. 
\paragraph{Опасность ловушки} (как абстрактную, так и соответствующий параметр) определяет мастер. Многообразие ловушек слишком велико, чтобы подробно перечислять их здесь. Условно их можно поделить на 5 типов:
\begin{itemize}
  \item \textbf{Импровизированные ловушки:} ловушки из подручных материалов - бесхитростные, но смертельно опасные. Если герой подвергается действию ловушки, то получает Пв, равные \textbf{|[Величине проверки Выживания(Ин) установившего ловушку] - [Эффективный показатель Наблюдательности]|}*. 
    \newline *Если вам больше по душе случайности, подвергшийся действию ловушки герой может проверить Наблюдательность(Мд) против Опасности ловушки.
    \newline \textit{К примеру, охотник роет волчью яму и получает 18 на проверку Выживания. Герой с Наблюдательностью (Мд), равной 2, попавший в ловушку, получит 18 (Проверка Выживания охотника) - 1 (БД героя) - 12 (Эффективный показатель Наблюдательности) = 5 Пв.}
  \item \textbf{Ловушки, наносящие фиксированные Повреждения:} если герой подвергается действию ловушки, то получает Повреждения, равные \textbf{|[Опасность ловушки] - БАЗщ - БД|}*. 
    \newline *Если вам больше по душе случайности, подвергшийся действию ловушки герой может проверить Зщ против Опасности ловушки. 
    \newline \textit{Ловушка с лучами-бритвами, имеет Опасность 30 и заполняет лазерными лучами область 3 × 3 метра. Если герой с МРз 0 в кольчужной рубахе подвергнется ее действию, то получит 30 - 10 - 3 = 17 Пв.}
  \item \textbf{Ловушки с фиксированной Доблестью или Меткостью:} ловушки с подвижными частями - дротики, выстреливающие из стен, или стальные лезвия, вылетающие из потолка. Ловушка обладает значением Дб или Мт и при активации совершает проверку против Зщ героя по обычным правилам.
  \item \textbf{Ловушки с Ядами:} если герой подвергается действию ловушки, то находится в состоянии Отравления, получает Ядовитые Пв и Первичный эффект Яда.
    \newline Такие ловушки могут сочетаться с любым другим их типом, кроме Смертельных. Если у ловушки есть Дб или Мт, она способна наносить КУ и вводить в игру Побочки Ядов.
  \item \textbf{Несмертельные ловушки:} ловушки, которые могут Захватывать, погружать в сон, наносить Несмертельные Пв или еще как-то выводить героя из строя вместо того, чтобы убивать. Обычно представляют собой один из типов, перечисленных выше, но "гуманизированный".
  \item \textbf{Смертельные ловушки} не наносят Повреждений - если герой подвергся их действию, его ЕЗ сразу падают до 0. У спутников жертвы будет совсем немного времени, чтобы вытащить изувеченное тело или то, что от него осталось.
    \newline Например, герой упавший в яму с мутагенной Заразой, понижает свои ЕЗ до 0 и совершает проверку Вн против \textbf{|15|}. Если он преуспевает, у спутников будет Круг на то, чтобы попытаться вытащить героя и помочь ему. В противном случае герой растворяется в жиже, или чудовищно мутирует и нападает!
\end{itemize}
\paragraph{Ловушки и Бонус доспеха:} крепкий доспех поможет спастись во многих случаях. Совершите проверку Зщ против Опасности ловушки в дополнение к прочим проверкам (если только проверка Зщ уже не является единственным спасением), и используйте лучший результат, если герой может рассчитывать на броню и ловкость. Хотя если герой попал в яму с зыбучим песком, груда брони станет проблемой. 
\begin{tcolorbox}
  Так или иначе, лучшая из возможных защита от ловушки - высокая Наблюдательность (Мд).
\end{tcolorbox}