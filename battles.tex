\chapter{Боевые столкновения}
Полностью посвящена боевым сценам и их частным случаям, включая сражения верхом и на технике. А еще здесь рассказывается о \textit{последствиях} такого занятного времяпровождения.
\section{Общие термины}

\paragraph{Источник:} исходная точка воздействия - способности, эффекта, атаки и т.д. От источника определяется видимость цели, дистанция до нее, и совершаются прочие важные измерения. 
\paragraph{Круг:} в бою герои и статисты действуют по Очереди. Один полный Круг (то есть Очереди всех героев и статистов, принимающих участие в Сцене) занимает 5 секунд. Подразумевается, что герои и статисты действуют одновременно, однако для удобства Круг разделен на Очереди. 
\paragraph{Очередь:} часть Круга, посвященная отдельному герою или статисту. Продолжительность Очереди никак не регламентирована правилами, не считая длительности полного Круга, ограниченной 5 секундами.
\paragraph{Нападающий:} обозначает героя или статиста, агрессивно воздействующего на элементы Сцены. В описательных и игромеханических блоках в этом же контексте используются "атакующий", "поражающий", "стреляющий", "наносящий удар" и т.п.
\paragraph{Нападение:} комплекс агрессивных мер, посредством которого герой или статист воздействует на элементы Сцены. В описательных и игромеханических блоках в этом же контексте используются "атака", "нанесение", "удар", "поражение" и т.п.
\paragraph{Цель:} один или несколько элементов Сцены, являющихся точкой приложения усилий (не всегда агрессивных) героя или статиста. В описательных и игромеханических блоках в этом же контексте используются "жертва", "пораженный", "атакованный", "противник" и т.п. Герой должен видеть цель, чтобы взаимодействовать с ней, если не указано обратного.
\paragraph{Вредоносный эффект:} последствия воздействия нападающего или элемента Сцены на цель. Самый распространенный из таких эффектов - получение Повреждений. Некоторые вредоносные эффекты вступают в силу без получения Пв целью.
\paragraph{Зона действия:} область, в которой герой или статист может воздействовать на цель. Воздействия элементов Сцены так же обладают Зоной действия.
\paragraph{Боевой контакт:} зона воздействия оружия ближнего боя героя. Обычно Боевой контакт составляет 1 метр (соседняя с героем клетка на тактической карте), но некоторые Трюки, свойства и обстоятельства могут увеличить или уменьшить его.
\newline Возможна ситуация, когда герой находится в Боевом контакте у статиста, но статист не находится в Боевом контакте у героя и наоборот. Обычно это происходит при Внезапном нападении, а так же когда противники разного Размера, или один из них снаряжен Длинным оружием.
\paragraph{Полученные Повреждения:} Пв, полученные целью в результате отдельно взятого нападения или воздействия на нее. 
\paragraph{Предотвращенные Повреждения:} не учитываются при потере ЕЗ, но считаются полученными целью и нанесенными нападающим.
\paragraph{Игнорирование Повреждений:} если Повреждения проигнорированны, они не считаются полученными целью и нанесенными нападающим.
\begin{tcolorbox}
  Предотвращение Повреждений отражает ситуации, когда цель поражена, но не получила сколь-нибудь серьезного урона. Игнорирование Повреждений указывает, что цель увернулась от атаки, либо вообще не воспринимает ее, как вредоносный эффект.
\end{tcolorbox}

\section{Получение Повреждений}
Потеря ЕЗ целью в результате атак и вредоносных эффектов происходит в строгой последовательности.
\begin{enumerate}
  \item Определите, сколько Повреждений получила цель в результате успеха проверки нападающего или по каким-то иным причинам.
  \item Примените способности и свойства, активируемые получением Повреждений.
  \item Предотвратите Повреждения благодаря Прочности, затем примените Сопротивление, Иммунитет и прочие способности, сокращающие Пв. Примените Внезапную смерть и совершите сопутствующие проверки, если необходимо.
  \item Увеличьте Повреждения, если цель Уязвима к ним, и примените все прочие способности, увеличивающие полученные Пв.
  \item Подсчитайте общее число полученных целью Повреждений и понизьте на него ее текущие ЕЗ. Цель теряет это число ЕЗ*.
  \item Примените способности цели и все правила, активируемые потерей ЕЗ.
  \item Примените эффекты Критического Удара.
  \item Примените правило Болевого шока, если требуется.
  \item Проверьте, угрожает ли цели Нокаут и проверьте Вн, если требуется.
\end{enumerate}
*Если цель не обладает Иммунтитетом к ТПв воздействия, и правила предписывают ей потерять ЕЗ, то сразу переходите к пункту 5. 
\begin{tcolorbox}
  Цель может получить Пв, но не потерять ЕЗ. Это важно для срабатывания некоторых эффектов.
\end{tcolorbox}

\paragraph{Одномоментные получение Поврежений и потеря ЕЗ:} подразумевает получение Пв и потерю ЕЗ в результате одного воздействия или эффекта - одного маневра, одного феномена, одного Хода и т.п. Если Пв получены, а ЕЗ потеряны в течение одного Круга, но из-за разных воздействий - нескольких маневров, феноменов, Ходов, они не считаются нанесенными одномоментно.

\paragraph{Единицы Характеристик (ЕХ) и их потеря:} некоторые атаки, нанося Повреждения, отнимают не ЕЗ жертвы (или не только ЕЗ), а понижают ее Характеристики. Под потерей ЕХ понимаются все подобные случаи. 

\subsection{Сопротивление, Уязвимость, Иммунитет и Родная стихия}
Атаки имеют один или более разных типов Повреждений, но не все они одинаково опасны для целей.
\paragraph{Уязвимость} к типу Пв увеличивает полученные целью Повреждения в 2 раза. 
\paragraph{Сопротивление} к типу Пв уменьшает полученные целью Повреждения в 2 раза. 
\paragraph{Иммунитет} к типу Пв делает цель абсолютно невосприимчивой к Повреждениям. Она игнорирует Пв этого типа. Цель не теряет ЕЗ в результате воздействия, к ТПв которого она имеет Иммунтитет.
\paragraph{Родная стихия} позволяет цели восстанавливать ЕЗ! Она не получает Повреждений. Вместо этого цель восстанавливает столько ЕЗ, сколько она получила бы Повреждений без эффекта Родной стихии.
\paragraph{}Если цель имеет Уязвимость к нескольким типам Повреждений атаки сразу, полученные ею Пв удваиваются за каждую Уязвимость. 
\newline Уязвимость и Сопротивление сводят друг друга на нет. 

\subsection{Критический Удар и его эффекты}
\paragraph{Критический удар (КУ):} когда во время проверки Доблести или Меткости на К20 выпадает число, указанное в графе оружия "КУ" или превышающее его, атака получает дополнительные свойства, если цель потеряла хотя бы 1 ЕЗ. Свойства зависят от типа Повреждений, которое наносит оружие.
\paragraph{КУ и 20 на кубике:} если при атаке на кубике выпадает 20, цель теряет 1 ЕЗ даже если Дб или Мт недостаточны для получения Пв. Эффекты КУ применяются в полном объеме.

\paragraph{}Эффекты КУ в зависимости от типа наносимых повреждений:
\begin{itemize}
  \item \textbf{Дробящие:} получивший КУ Оглушен.
  \item \textbf{Едкие:} получивший КУ начинает Растворяться. Состояние наступает даже в том случае, если цель не потеряла ЕЗ при атаке.
  \item \textbf{Колющие:} получивший КУ страдает от Внутреннего кровотечения.
  \item \textbf{Рубящие:} получивший КУ страдает от Кровотечения.
  \item \textbf{Проникающие:} снаряд или осколок засел в теле. Получивший КУ страдает от Внутреннего кровотечения. После завершения боя он страдает от Агонии до тех пор, пока снаряд не будет извлечен (Медицина против \textbf{|15|}). 
    \newline Если герой подвергается лечению до того как снаряд извлечен, он не страдает от Агонии и Внутреннего кровотечения, но боль дает о себе знать. Все Активные проверки героя получают Помеху, пока снаряд не извлечен. Сложность Медицины для извлечения снаряда возрастает до \textbf{|20|}. 
  \item \textbf{Огненные:} получивший КУ Загорается. 
  \item \textbf{Электрические:} получивший КУ сбит с ног.
  \item \textbf{Ледяные:} получивший КУ Неподвижен до конца своей следующей Очереди.
  \item \textbf{Ядовитые:} получивший КУ Отравлен. Он подвергается Побочкам Яда или Лекарства.
\end{itemize}

\section{Состояния}
Физические  воздействия могут повлиять на состояние (и поведение) героя, как в бою, так и вне его. Эффекты двух
одинаковых состояний, происходящих из различных источников, не складываются, за исключением Внутреннего Кровотечения, Возгорания, Кровотечения, Отравления и Растворения. Эффекты различных состояний воздействуют на героя одновременно.
\paragraph{Агония:} герой охвачен мучительной болью. Герой может действовать только в случае успешной проверки Вл против 20. В боевых сценах совершайте проверку каждую Очередь героя, в остальных — один раз за сцену. При провале все, на что способен герой — лежать, вопить и поносить свою злую Судьбу. При успехе проверки герой держит себя в руках, но не может совершать Маневры и передвигается с половинной Ск. Состояние длится до следующего Антракта или пока герой не будет исцелен чарами (если Агония вызвана потерей ЕЗ), не получит квалифицированную врачебную помощь (если Агония вызвана Переломом) или сильное обезболивающее.
\paragraph{Внутреннее кровотечение:} жизненно важные органы героя повреждены. Каждую свою Очередь герой теряет |5 — МВн| ЕЗ (минимум 2 ЕЗ) до тех пор, пока не будет исцелен чарами или не получит квалифицированную врачебную помощь (Врачевание против 15).
\paragraph{Возгорание:} в начале своей Очереди горящее существо или предмет теряет 5 ЕЗ. Горящее существо может пропустить свою Очередь, чтобы потушиться. В этом случае до начала следующей Очереди существа атаки по нему совершаются по правилам Внезапного нападения. Состояние длится \textbf{|5 — МЛв цели|} Кругов (минимум 1 Круг).
\paragraph{Кровотечение:} герой истекает кровью. Каждую свою Очередь герой теряет число ЕЗ, равное БПв оружия, вызвавшего КУ(мин 1). Состояние длится до тех пор, пока герой не будет исцелен или перевязан (Врачевание против 10).
\paragraph{Неподвижность:} по каким-то причинам герой не может двигаться и защищать себя. Он может быть схвачен, заморожен, опутан сетью или просто спит. Герой теряет бонус защиты щита и бонус ловкости к защите. Все атаки по герою совершаются с Преимуществом.
\paragraph{Оглушение:} герой ненадолго теряет связь с реальностью (обычно из-за доброго удара по голове или под дых). Герой не может совершать Действие в свою следующую Очередь. В дальнейшем герой совершает все свои активные проверки с Помехой число Очередей, равное \textbf{|5 — МВн|} (минимум 1 Очередь).
\paragraph{Ослепление:} герой ничего не видит. Все атаки по нему совершаются по правилам Внезапного нападения. Герой атакует с 2 Помехами. Он не может атаковать цели, находящиеся от него на расстоянии (в метрах) большем, чем его Наблюдательность.
\paragraph{Отравление:} герой находится под действием яда. Смотрите описание яда для определения эффекта.
\paragraph{Ошеломление:} герой приходит в замешательство и не может совершать Действие в свою следующую Очередь. В дальнейшем герой совершает все свои активные проверки с Помехой число Очередей, равное \paragraph{|5 — МИн|} (минимум 1 Очередь). Также это состояние может быть причиной недосыпа, злоупотребления алкоголем или наркотическими зельями. Тогда избавиться от него поможет только полноценный отдых.
\paragraph{Ранен:} если ЕЗ героя достигают 2/3 от максимальных, то он Ранен. Его Ск ополовинивается. Обычно это состояние связано с потерей ЕЗ героем, но иногда оно возможно и в иных случаях — например, если герой страдает от вывиха или последствий болевого приема.
\paragraph{Растворение:} в начале своей Очереди существо или предмет получает 1 Пв вне зависимости от своей Прч. Если жертва была одета, она может остановить действие эффекта, пропустив 1 Очередь и сорвав с себя одежду, хотя снятие доспеха может занять куда больше времени! Если же одежды на жертве не было... Ей понадобится помощь квалифицированного медика и проверка Медицины против 15. Растворение вызывается множеством самых различных субстанций и веществ, поэтому способ нейтрализации, который сработал в прошлый раз, в следующий может сделать только хуже! Проверка Медицины необходима для каждого нового источника Растворения.
\newline
Если меры не были приняты, состояние заканчивается через \textbf{|10 + Пв, нанесенные вызвавшей состояние атакой|} Кругов. В конце сцены совершите проверку Неприятностей, чтобы узнать, пришло ли в негодность снаряжение жертвы (при желании можно совершить отдельную проверку для каждого предмета).
\trouble
{Отдушка}%no sweat name
{Никаких последствий, кроме специфического запаха, который исчезнет после хорошей стирки}%no sweat description
{Патина}%tough day name
{Незначительные повреждения. Герой может исправить их самостоятельно, совершив проверку соответствующего НавыкаЭ против 15 или заплатив ремесленнику 1/4 СП предмета (минимум 1 СП). До завершения ремонта предмет получает Осечку 6.}%tough day description
{Ржа}%we have trouble name
{Серьезные повреждения, все еще поддающиеся ремонту.Герой может исправить их самостоятельно, совершив проверку Ремонта против 20 или заплатив ремесленнику 1/2 СП предмета (минимум 1 СП). До завершения ремонта предмет получает Осечку 10.}%we have trouble description
{Труха}%fiasco name
{Снаряжение приведено в полнейшую негодность. Возможно, удастся всучить его подвыпившему старьевщику и выручить пару медяков.}%fiasco description
\paragraph{Серьезно ранен:} если ЕЗ героя достигают 1/3 от максимальных, то он Серьезно ранен. Все его активные проверки совершаются с Помехой. Обычно это состояние связано с потерей ЕЗ героем, но иногда оно возможно и в иных случаях — например, если герой страдает от вывиха или последствий болевого приема.
\paragraph{Сон:} герой спит. Все проверки Наблюдательности героя совершаются с Помехой. Пока герой спит, он Неподвижен. Если герой проснулся и вынужден сразу же действовать, он Ошеломлен. Героя автоматически разбудит достаточно громкий звук — выстрел из пистолета или звон медного колокольчика. Более тихие звуки потребуют проверки Наблюдательности (с Помехой)!
\paragraph{Удушье:} герой задыхается. Все его проверки совершаются с Помехой. Герой Потеряет сознание, если состояние продлится дольше, чем \textbf{|Вн × 10|} секунд, и умрет вне зависимости от величины его ЕЗ, если состояние продлится дольше, чем \textbf{|Вн × 20|} секунд.
\paragraph{Ужас:} герой дрожит от ужаса. Он совершает все активные проверки с Помехой, пока источник его ужаса находится поблизости (в зоне видимости или слышимости). В дальнейшем состояние длится число Очередей, равное \textbf{|10 — Вл героя|}.
\paragraph{Усталость:} все активные проверки совершаются героем с Помехой. Атаки по герою совершаются с Преимуществом.
\section{Боевые сцены}
Боевая сцена начинается, если один или несколько героев подверглись атаке, либо напали на кого-то сами. В начале боевой сцены:
\begin{enumerate}
\item Определите, подвергся ли кто-то из участников Внезапному нападению. Обычно это сопряжено с проваленными проверками Наблюдательности против Скрытности противника. Если никто из участников сражения не пытался передвигаться скрытно и ожидал нападения, фактор внезапности вряд ли стоит учитывать. С другой стороны, Стремительный герой, герой с трюком <<Гопля!>> или герой с высокой Реакцией вполне может застать противников врасплох, молниеносно выхватив оружие и атаковав!
\item Определите позиции участников сцены относительно друг друга и окружающих предметов. На обсуждение этого стоит потратить несколько минут (или даже больше), чтобы игроки четко представляли возможности героев и их противников.
\item Начало Круга. Все участники сцены действуют в порядке, определяемом их Реакцией. Возможно, для завершения битвы понадобится больше одного Круга.
\end{enumerate}
\paragraph{Круг:} в бою герои и статисты действуют по Очереди. Один полный \textbf{Круг} (то есть Очереди всех героев и статистов, принимающих участие в сцене) занимает \textbf{5 секунд}. Подразумевается, что герои и статисты действуют в бою одновременно, однако для удобства игры круг разделен на Очереди. Статисты под управлением мастера действуют в одну общую Очередь, однако мастер может разделить их Очереди в соответствии с параметрами Реакции. Общая Очередь ориентируется на статиста с наименьшей Реакцией.
\paragraph{Очередность в бою:} первым действует герой или статист с наибольшей Реакцией. Если у двух героев или статистов одинаковая Реакция, первым действует тот, у кого больше Ловкость. Если и они равны, первым действует выигравший Состязание в Реакции.
\paragraph{Очередь:} Очередь героя состоит из Перемещения, Действия и Быстрого действия. В течение своей Очереди герой может:
\begin{itemize}
\item[--] \textbf{Совершить Перемещение.} Для Перемещения используется значение Ск — столько метров/клеток может преодолеть герой. Герой может отказаться от Перемещения, чтобы получить дополнительное Быстрое действие.
\item[--] \textbf{Совершить Действие.} Во время Действия герой использует предметы, атакует, творит чары, убеждает окружающих прекратить кровопролитие или делает что-то еще, требующее концентрации внимания.
\item[--] \textbf{Совершить Быстрое действие.} Быстрое действие включает односложные фразы, быстрые жесты, шаги в пределах 1 метра и броски предметов без намерения кому-то повредить или поразить конкретную цель. Хрупкие предметы могут сломаться или разбиться! У героя есть лишь одно Быстрое действие за круг, но мастер может отступать от правила, если считает ситуацию располагающей к этому. Иногда выполнение Быстрого действия возможно и в чужую Очередь. Такие случаи указаны в описании соответствующих Трюков или Атрибутов.
\end{itemize}
\begin{tcolorbox}
Герой может отказатьсяот Действия, чтобы совершить дополнительное Перемещение или дополнительное Быстрое действие.
\newline
Так же герой может отказаться от Перемещения, чтобы совершить дополнительное Быстрое действие.
\end{tcolorbox}
Герой может отказаться от Перемещения или Действия, чтобы подняться с земли, или взять предмет, или аккуратно положить предмет, или привести оружие в боевую готовность, или убрать оружие в ножны, или перезарядить оружие со свойством <<Перезарядка>>, или вскочить в седло.
\newline
Перемещение, Действие и Быстрое действие используются в любых комбинациях. Например, совершая Быструю атаку, герой со Скоростью 6 может за одну Очередь пройти на 1, атаковать, потом пройти на 2, атаковать еще раз, затем пройти на 3 и уронить предмет.
\subsection{Детали боевой сцены}
\paragraph{Внезапное нападение.} Если герой не заметил врага или заметил в последний момент (то есть провалил проверку Наблюдательности против Скрытности врага), он захвачен врасплох. Герой не может Перемещаться и Действовать независимо от своей Реакции, а также вычитает МЛв и БЩЗщ из своей Зщ. Если герой дожил до своей следующей Очереди, он сражается по обычным правилам.
\paragraph{Все на одного:} сражение с несколькими противниками требует от воина высочайшего мастерства. Если герой сражается с несколькими противниками, в начале Очереди он должен выбрать, кому из них он уделяет больше внимания. Выберите число противников, равное \textbf{|1 + ММд героя|} (минимум 1). Они атакуют героя, как обычно. Все противники сверх этого числа атакуют героя с Преимуществом.
\paragraph{Ложись!:} герой может упасть, использовав Быстрое действие. Дистанционные атаки по лежащему герою совершаются с Помехой, атаки в ближнем бою по нему совершаются с Преимуществом. Лежащий герой может ползти с 1/2 Ск и атаковать в ближнем бою с Помехой. В остальном герой действует по обычным правилам.
\paragraph{Невидимки:} если герой атакует невидимого (из-за чар, укрытия или по иным причинам) противника, бросок совершается с Помехой. Если невидимого противника нет в атакуемой области, герой автоматически промахивается. Невидимые существа атакуют по правилам Внезапного нападения. Если существо невидимо благодаря Скрытности, а не волшебству, то, атаковав, оно обнаружит себя вне зависимости от успеха атаки.
\paragraph{Перемещение через занятые области:} если герой желает пройти через область, занятую враждебным существом, он должен пройти проверку Атлетики против \textbf{|БАЗщ + Дб противника|}. В случае успеха герой передвигается через занятую область, в случае провала падает рядом с противником. Перемещение героя на этом заканчивается.
\paragraph{Трудный ландшафт.} Иногда перемещение героя затруднено густым подлеском, глубоким снегом, скользким льдом, качающейся палубой. В такой ситуации Ск героя ополовинивается.
\subsection{Зоны поражения}
Герой может выбрать для атаки любую из перечисленных зон. Если при атаке не заявлена специфическая Зона поражения, удар наносится в торс.
\newline Сложность маневра Возрастает на указанную в таблице Зон поражения, но при успехе маневра возможен дополнительный эффект.
\begin{center}
\begin{tabular}{|c|c|}
\hline
Зона поражения & Сложность \\ \hline
Торс & 0 \\ \hline
Конечность & 2 \\ \hline
Пах & 3 \\ \hline
Шея & 5 \\ \hline
Голова & 5 \\ \hline
Глаз & 7 \\ \hline
\end{tabular}
\end{center}
\paragraph{Торс:} попадание по торсу не вызывает никаких эффектов, кроме синяков, шрамов, шишек и потери Единиц Здоровья.
\paragraph{Конечность:} попадание по руке или ноге может вызвать Перелом или Потерю конечности.
\paragraph{— Переломы:} попадание по руке или ноге может вызвать Перелом. Когда конечность одномоментно получает Пв, превышающие \textbf{|1/5 от максимальных ЕЗ|}, она считается сломанной. Переломы считаются Опасной раной.
\paragraph{— Потеря конечностей:} если конечность одномоментно получает Пв, превышающие \textbf{|2/5 от максимальных ЕЗ|}, то она отрублена, размолота в кашу или приведена в полнейшую негодность каким-то иным образом. Повреждения, превышающие требуемое для уничтожения конечности число, теряются. Герой, потерявший конечность, находится в состоянии Кровотечения. Само собой, Потеря конечности считается Опасной раной!
\paragraph{Пах:} мужчина, получивший удар в пах, должен пройти проверку Вн против \textbf{|10 + полученные Пв|}. При провале он не может действовать и перемещаться число Очередей, равное числу, на которое провалил проверку (хотя может орать, сквернословить и совершать Быстрые действия). Все атаки по нему совершаются с Преимуществом.
\paragraph{Шея:} если шея одномоментно получает Пв, превышающие \textbf{|1/4 от максимальных ЕЗ + МВн|}, то жертва умирает. Без проверок. В случае Колющего или Проникающего удара смерть наступает от обильного кровотечения, Дробящие атаки ломают позвоночник, Рубящие удары обезглавливают.
\newline
Разумеется, это относится к живым гуманоидным существам, у которых мозг находится в голове. Умертвия, разумные грибы, трехголовые драконы и тому подобные создания вполне могут существовать и без головы (или одной из них).
\paragraph{Голова:} получивший удар в голову должен пройти проверку Вн против \textbf{|10 + полученные Пв|}. При провале жертва Теряет сознание. Если в голову нанесена Опасная рана, жертва должна преуспеть в проверке Вн против 15 или умереть.
\paragraph{Глаз} не может быть выбран для поражения Громоздким оружием. 2 и более Пв приводят к потере глаза. Существо, потерявшее все свои глаза, Ослеплено. Крупных существ ослепить сложнее — Большое существо лишается глаза при получении глазом 3 Пв, огромное — 4 Пв, гигантское — 5 Пв. Маленькие и крошечные существа лишаются глаза при получении в глаз 1 Пв. Лишние Пв наносятся в голову.
\newline
Если атака имеет Колющие или Проникающие Пв, удвойте успешно нанесенные Пв — атака поражает не только глаз, но и мозг! В этом случае получивший удар должен пройти проверку Вн против \textbf{|10 + полученные Пв|} или Потерять сознание. Проверка совершается с Помехой.
\newline
Если Колющая или Проникающая атака наносит Опасную рану в глаз, жертва должна преуспеть в проверке Вн против 15 или умереть. Проверка совершается с Помехой.
\newline
Зачастую глаза — самое уязвимое место на теле огромных монстров!

\section{Маневры}
Во время Действия герой может совершить Маневр. Перечисленные ниже Маневры может выполнить любой герой, если он снаряжен соответствующим образом. Некоторые Маневры могут совмещаться друг с другом или дополнительно требовать траты Перемещения или Быстрого действия.
\paragraph{Совершение Маневра:}
\begin{enumerate}
\item Выберите цель. Герой не может поразить цель за пределами досягаемости оружия или заклинания.
\item Определите штрафы и бонусы к проверке. Как правило, это штраф зоны поражения или штрафы за размер предмета при Разоружении или Поломке оружия. Также перед совершением проверки определите, есть ли у героя Преимущество или Помеха на нее.
\item Совершите необходимую проверку и определите последствия. Получает цель Маневра Повреждения, или герой промахивается, нанесен ли Критический Удар и т. д. Например, успех при Разоружении или Поломке оружия лишит противника оружия или уничтожит (или повредит) ценный предмет на его теле, успех при Захвате даст герою возможность справиться с противником, не нанося Повреждений, или просто сломать ему хребет.
\end{enumerate}
\paragraph{Зоны Поражения}
\subsection{Атака}
Герой атакует в ближнем бою. Герой может атаковать противника в 1 метре от себя (или в 2 метрах, если его оружие Длинное) и должен совершить проверку Дб против \textbf{|Зщ цели|}. Количество нанесенных Пв равно величине успеха проверки. Неподвижные цели герой атакует с Преимуществом. Если в результате подсчета получается 0 или меньше, то удар соскользнул с доспеха или просто не достиг цели. Герой может понизить свою Дб на любое число (минимум до 0), если желает нанести удар не в полную силу.
\subsection{Атака с разбега}
Герой преодолевает по прямой расстояние в интервале от своей \textbf{|Ск +1|} до своей \textbf{|Ск × 2|} и атакует с Преимуществом. До начала его следующей Очереди все атаки по нему совершаются с Преимуществом. Атака с разбега не может совмещаться с Быстрой атакой, Выжиданием и Плетением чар, но может совмещаться с другими маневрами. Если герой совершает Сокрушительную атаку с разбега, он получает
2 Преимущества. Но так же и враги при атаках по нему до начала его следующей Очереди!
\newline
Перемещение героя уже учтено в этом Маневре, то есть при его использовании герой может преодолеть расстояние, не превышающее его \textbf{|Ск × 2|}. Герой не может совершить Перемещение и затем использовать Атаку с разбега!
\subsection{Быстрая атака}
Герой совершает 2 атаки с Помехой. Если герой имеет оружие в каждой руке и оба оружия Легкие, только одна атака совершается с Помехой. Каждой из этих атак герой может совершать Дистанционную атаку, Захватывать, Ломать снаряжение, Разоружать, Сбивать с ног, Толкать или выполнять Финт.
\subsection{Выжидание}
Герой ожидает совершения некоего действия кем-то из окружающих. Любой иной маневр может быть выполнен как Выжидающий.
\newline
Очередность событий при этом определяется Реакцией. Чтобы
опередить противников, герой должен преуспеть в проверке
Реакции против \textbf{|10 + Рц статиста|}.
\subsection{Захват}
Для маневра нужна как минимум 1 свободная рука. Герой проходит проверку Дб (Рукопашный бой) против \textbf{|БАЗщ + Дб(Рукопашный бой) противника|}. При успехе противник не получает Повреждений, но становится Захваченным. Маневр может сочетаться с Атакой с разбега, Быстрой атакой (это имеет смысл, если герой пытается захватить сразу двух
противников) и Сокрушительной атакой. Если Захват успешен, все атаки в ближнем бою против схваченного получают Преимущество, пока он не освободится. Некоторые Трюки и виды оружия позволяют проводить Захват при помощи Владения оружием. В этом случае замените во всех формулах Рукопашный бой на Владение оружием для инициатора Захвата. Если у героя есть Трюк «Знаток оружия», то он может использовать во всех формулах Владение оружием, даже являясь Захваченным!
\begin{tcolorbox}
Если герой использует для хватания или удерживания две руки, он получает +2 ко всем проверкам захвата. Если у героя больше двух рук, то он получает еще +2 к проверкам за каждую пару рук.
\end{tcolorbox}
\paragraph{Схвативший может} (в том числе в ту же Очередь, когда совершен Захват) cовершать любые действия со следующими дополнениями и исключениями:
\begin{itemize}
\item[--] Атаковать схваченного любым одноручным оружием (даже дистанционным). Схваченный не добавляет МЛв и Щит к своей Зщ.
\item[--] Атаковать другие цели по обычным правилам, если у него есть свободная рука с одноручным оружием ближнего боя или одноручным дистанционным оружием.
\item[--] Перемещаться на половину своей Ск. Схвативший передвигается без ограничений, если на 2 категории и более превышает Размером схваченного.
\item[--] Сменить зону, за которую удерживает противника. Герой должен пройти проверку Захвата по новой выбранной Зоне. В случае провала, он продолжает держать цель в Захвате, но зона, за которую он держит цель не меняется.
\item[--] Блокировать противника. До начала своей следующей Очереди схвативший не дает схваченному выполнять любые действия (даже говорить). Все что может делать схваченный — это пытаться вырваться. Схвативший не может перемещаться и совершать никаких действий, кроме Быстрых.
\item[--] Душить схваченного. Для этого противник должен быть схвачен за шею. Схвативший совершает проверку Дб(Рукопашный бой) против \textbf{|Дб(Рукопашный бой) + МВн схваченного|}. Величина успеха равна нанесенным Пв. Если в ходе удушения ЕЗ схваченного достигает 0, он Теряет сознание. Обратите внимание, что отрицательный МВн прибавляется к Пв от удушения.
\item[--] Бросить схваченного. Герой не может бросать существ и предметы, вес которых превышают его \textbf{|комфортную нагрузку × 2|}. Дальность броска не может быть дальше, чем \textbf{|МСл+МЛв|} бросающего. Увеличьте максимальное расстояние броска в 2 раза за каждую категорию размера, на которую бросающий больше бросаемого, и уменьшите в 2 раза за каждую категорию, на которую бросающий меньше бросаемого.
\newline Для того, чтобы бросить схваченного в конкретную точку, бросающей должен совершить проверку Меткости для Метательного оружия с БПв 0. При падении бросаемый получает столько Пв, сколько метров пролетел. Если точке, куда совершен бросок, находится другое существо, то бросаемый падает ему под ноги.
\newline Для того, чтобы попасть по другому существу, нужно совершить проверку Меткости с Помехой для метательного оружия с БПв равным \textbf{|МРз + Броня бросаемого|}.
\item[--] Отпустить схваченного.
\end{itemize}
\paragraph{Схваченный может} совершать перечисленные действия:
\begin{itemize}
\item[--] Проводить атакующие Маневры по схватившему с Помехой. Для атаки может использоваться только Легкое оружие, шипы на доспехе, а также кулаки (зубы, когти, клювы, щупальца и т. п.). В остальном атаки проводятся по обычным правилам.
\item[--] Творить Феномены, если может соблюсти необходимые для этого условия (что бывает затруднительно, если героя держат). Все проверки, необходимые для успеха Феномена, совершаются с Помехой.
\item[--] Пытаться вырваться. Чтобы вырваться, схваченный должен пройти проверку Дб (Рукопашный бой), Атлетики (Сл, Лв), Сл или Лв против \textbf{|БАЗщ + Дб (Рукопашный бой) схватившего|}. Попытка вырваться считается Действием.
\item[--] Достать Легкое оружие. Схваченный может отказаться от Перемещения, чтобы достать оружие, несмотря на то, что фактически обездвижен.
\item[--] Совершать Быстрые действия.
\end{itemize}
\paragraph{Схваченный не может} совершать следующие действия:
\begin{itemize}
\item[--] Передвигаться, если только не превышает схватившего размером на 2 или больше. Также существа, размером превышающие схватившего, могут атаковать и другие цели (без Помехи), кроме схватившего. 
\end{itemize}
Схвативший/схваченный автоматически получают Пв, если на доспехе схватившего/схваченного есть шипы. Пв равны разнице бонуса доспехов схватившего и схваченного. Например, если герой в кольчужной рубахе (Броня +3) схватил противника в шипованном кольчато-пластинчатом доспехе (Броня +6), то герой получит 3 Пв (противник, само собой, не получает Пв). Шипы наносят Пв в начале каждой Очереди схватившего/схваченного, пока схвативший не отпустит противника. Для того чтобы удерживать противника, покрытого шипами, схвативший должен пройти проверку Вл против \textbf{|10 + полученные Пв|}. При провале он отпускает его в начале своей следующей Очереди!
\subsection{Защитная стойка}
Герой сосредочен на обороне. Все атаки по нему до начала его следующей Очереди совершаются с Помехой. Лежащий на земле герой также может выбирать этот маневр!
\subsection{Провокация}
Этот Маневр может сочетаться с любым другим и требует Быстрого действия. При помощи оскорбительных фразочек и еще более оскорбительных жестов герой привлекает к себе внимание врага. Совершив проверку Общения (Об) против \textbf{|10 + Вл цели|}, герой провоцирует противника и становится целью его следующей атаки. В некоторых ситуациях герою точно не обойтись без Луженой глотки!
Если жертвы не слышат героя или не видят его, или не понимают язык, на котором он говорит, проверка совершается с Помехой. Если героя и не видят, и не слышат, Провокация не сработает! Маневр действует только на одну цель одновременно, но если у героя больше одного Быстрого действия, он может Провоцировать несколько раз за Очередь, привлекая внимание разных противников!
\begin{tcolorbox}
Провокация действует только в боевых сценах (то есть, когда бой уже начался) и имеет смысл в тех случаях, когда враги могут атаковать оскорбившего их героя. В противном случае они выберут другую цель. 
\end{tcolorbox}
\subsection{Разоружение}
Чтобы выбить предмет из рук противника, герой должен пройти проверку Дб против \textbf{|10 + Щит + Дб противника|}. Герой получает штраф к Дб за размер предмета-цели (сверьтесь с таблицей ниже). Если противник держит предмет в 2 руках, герой получает дополнительные -2 к проверке. Маневр не может повредить предмету-цели, однако хрупкие предметы могут разбиться, упав на землю. В случае успеха маневра предмет падает на землю на расстоянии от разоруженного, не превышающем МСл или МЛв разоружающего. Разоружающий выбирает точку, в которую упадет предмет. Уменьшите расстояние в 2 раза, если выбитый из рук предмет Громоздкий или Длинный. Если в результате получается 0, предмет падает под ноги разоруженного.
\newline
Если хотя бы одна рука героя свободна, он может выхватить предметы и оружие из рук противника, используя при Разоружении Рукопашный бой! Если герой использует обе руки, он получает +2 к проверке. В остальном он действует по обычным правилам Разоружения. При успехе Маневра оружие или предмет оказывается в его руках. Если герой выбирает целью Маневра щит, то Щит не учитывается при проверке. Однако зачастую щиты основательно закреплены на руке, и, чтобы сорвать с нее щит, понадобится дополнительная проверка Сл, Лв или Атлетики (Сл, Лв) против |10 + Щит|.
\subsection{Сбить с ног}
Чтобы сбить противника с ног, герой должен пройти проверку Дб против \textbf{|БАЗщ + Щит + Дб противника|}.
\newline
Герой получает штраф -2 за область поражения (ноги). Если существо стоит на 4 ногах, герой получает дополнительные -2. Если существо передвигается на брюхе, как люди-змеи или гигантские слизни, герой получает дополнительные -4 к проверке! Маневр может выполняться лишь Длинным оружием или при помощи Навыка «Рукопашный бой».
\subsection{Сломать снаряжение}
Чтобы нанести Повреждения предмету в руках или на теле противника, герой должен пройти проверку Дб или Мт против \textbf{|10 + Щит + Дб противника|}.
\newline
Герой получает штраф к Дб или Мт за размер предмета-цели (сверьтесь с таблицей ниже). За каждую 1, на которую герой прошел проверку, он наносит 1 Пв предмету. Если герой пытается разбить щит, Щит не учитывается в сложности проверки. 
\begin{center}
\begin{tabular}{|p{10cm}|p{4cm}|}
\hline
Оружие/предмет & Штраф к Доблести атакующего при разоружении/поломке \\ \hline
Доспехи, закрывающие большую часть тела (Броня 7+), башенный щит & 0 \\ \hline
Громоздкое и Длинное двуручное оружие, доспехи, закрывающие значительную часть тела (Броня 4+6) & -1 \\ \hline
Громоздкое или Длинное двуручное оружие, большой щит, легкие доспехи (Броня 1-3) & -2 \\ \hline
Двуручное оружие, Громоздкое или Длинное Универсальное оружие, щит, большой рюкзак & -3 \\ \hline
Универсальное оружие, Громоздкое или Длинное одноручное оружие, рюкзак & -4 \\ \hline
Одноручное оружие, баклер, широкий ремень & -5 \\ \hline
Легкое оружие, скрученный свиток, шапка, кошель на поясе & -6 \\ \hline
Кастет, перчатка, наруч, фиал с зельем, узкий ремень & -7 \\ \hline
Перстень, браслет, ключ & -8 \\ \hline
\end{tabular}
\end{center}

\subsection{Сокрушительная атака}
Герой атакует с Преимуществом. До начала его следующей Очереди все атаки по нему совершаются с Преимуществом. Сокрушительная атака не может совмещаться с Быстрой атакой, но может совмещаться с Атакой с разбега, Захватом, Поломкой оружия, Разоружением, Сбиванием с ног, Толчком или Финтом.
\subsection{Толчок}
Герой отталкивает противника на 1 метр прямо от себя. Чтобы сделать это, герой должен пройти проверку Дб против \textbf{|БАЗщ + Дб противника|}.
\newline
Если Маневр применяется одновременно с Атакой с разбега, герой отталкивает противника на 1 метр за каждую единицу, на которую прошел проверку. Герой не может оттолкнуть противника на большее число метров, чем преодолел сам в эту Очередь, но всегда толкает минимум на 1 метр в случае успеха проверки.
Увеличьте максимальное расстояние толчка в 2 раза за каждую категорию размера, на которую толкающий больше толкаемого, и уменьшите в 2 раза за каждую категорию, на которую толкающий меньше толкаемого.
\newline
Герой не может толкать существ и предметы, вес которых превышают его максимальную нагрузку.
\subsection{Финт}
Герой отвлекает противника. Герой должен пройти проверку Общения (Мд, Об) или Владения оружием/Рукопашного боя против \textbf{|10 + Владение оружием/Рукопашный бой противника (Ин, Мд)|}. Если проверка успешна, следующая атака (героя и любого его союзника) по этому противнику совершается с Преимуществом. Эффект длится до конца следующей Очереди героя, применившего Финт.
\subsection{Дистанционная атака}
Герой стреляет или бросает предмет в цель. Если цель находится на Дальней дистанции, атака совершается с Помехой. В Неподвижные цели герой стреляет с Преимуществом.
\newline
Оружие, не предназначенное для метания, может быть брошено с Помехой. Максимальная дистанция для броска такого оружия — 5. Подразумевается бросок, наносящий цели Повреждения, а не его максимальная дальность в принципе.
\newline
Если для вашей истории важно, на какую предельную дистанцию герой может метнуть предмет (например, при метании гранаты), то она равна \newline{|Сл героя × 3 — вес предмета в килограммах|} метров. Разделите результат на вес предмета в килограммах (минимум 1), если он не предназначен для метания. Например, герой с 10 Силой сможет метнуть двуручный топор весом в 5 кг на (10 × 3 — 5) ÷ 5 = 5 метров. Используйте Мт для попадания в некую конкретную область. В подобных случаях, пятачок земли 1 × 1 метр имеет БАЗщ 10. Разумеется, при броске на предельную дистанцию атака совершается с Помехой.
\paragraph{Дистанционные атаки имеют 2 типа дистанций: Ближняя и Дальняя.} Эти дистанции указаны в статистиках дальнобойного оружия и описании заклинаний. Если цель находится за пределами Дальней дистанции, герой не может поразить ее. Атаки на Дальней дистанции совершаются с Помехой. Совершая Дистанционную атаку, герой должен сделать проверку Меткости против \textbf{|Зщ цели|}. Герой может понизить свою Мт на любое число (вплоть до 0), если желает поразить цель вскользь.
\paragraph{Дистанционные атаки в ближнем бою:} герой совершает Дистанционные атаки с Помехой, если в 1 метре от него находится противник. Герой совершает Дистанционные атаки с Помехой, если его цель находится в ближнем бою, в котором он не принимает участия.
\paragraph{Перемещение и Дистанционные атаки:} если в свою Очередь герой перемещается на расстояние, превышающее 1 метр, и стреляет, проверка Меткости совершается с Помехой. Обратите внимание, что герой может выстрелить без Помехи до Перемещения.
\subsection{Прицеливание}
Выбрав Маневр Дистанционной атаки, герой может объявить, что он Прицеливается. В боевой сцене герой может сместить свою Очередь на 1—3 Очереди вниз, то есть действовать после менее быстрого героя или статиста. Если в текущем Круге герой действовал последним, то следующий прокуск Очереди перемещает его ход на следующий Круг в Боевой Сцене. Герой, сместивший свою Очередь на 1 вниз, получает Преимущество при проверках Меткости, а враги получают Преимущество, атакуя его. Герой, сместивший свою Очередь на 2 вниз, получает 2 Преимущества, а враги получают 2 Преимущества, атакуя его. Если герой сместил свою Очередь на 3 вниз, он получает 2 Преимущества и игнорирует Помеху при стрельбе на Дальнюю дистанцию. Враги получают 2 Преимущества, атакуя его.
\newline
Если во время Прицеливания герой одномоментно получает Пв, равные или превышающие его Вл, он теряет все бонусы Прицеливания и должен начинать заново. Прицеливание не может сочетаться с Быстрой атакой, Беглым Огнем и Огнем на подавление. Герой может прицеливаться только в цель, которую видит.
\paragraph{Укрытия делятся на 2 типа:} мягкое (кусты, высокая трава, куча хвороста) и твердое (камни, стены, башенные щиты). При стрельбе по противнику в мягком укрытии герой получает 1 Помеху, при стрельбе по противнику в твердом укрытии — 2 Помехи. В случае промаха герой поражает укрытие. Укрытия могут быть Повреждены и уничтожены.
\subsection{Беглый огонь}
Герой может выбрать несколько целей для стрельбы из оружия со Скорострельностю выше 1. Количество целей не должно превышать \textbf{|ММд+1|(минимум 2)}. Герой выбирает точное количество выстрелов, произведенных оружием в каждую цель. Это число не должно превышать Ск оружия, но может быть меньше. Например, если герой с ММд 2 стреляет из оружия с Ск3, он может выбрать 4 цели, но Ск оружия не позволяет распределить пули по всем целям, поэтому он может поразить только троих.
\newline
Совершите одну проверку Мт для всех выстрелов (отдельные проверки для каждого выстрела возможны по предварительной договоренности игроков и мастера) с количеством помех равным \textbf{|количеству целей-1(минимум 0)|}. Когда герой атакует несколько целей и выбирает различные Зоны поражения, для всех бросков используется наибольший штраф, если только отдельные проверки Мт по каждой из целей не оговорены заранее.
\newline
Эффекты КУ применяются только к одному из противников по выбору стрелка.
\subsection{Огонь на подавление}
Этот маневр может использовать герой с любым оружием со Скорострельностью 5 и выше. Герой поливает свинцом небольшой пятачок земли, не заботясь о точности. Герой может выбрать число смежных областей площадью 1х1 метр, не превышающее Скорострельность оружия, и совершить атаку с БПв оружия, модифицированным за дальность в случае необходимости. Атака совершается с 2 Помехами и Осечкой 5. Если оружие уже имеет параметр Осечки, используйте наибольший. Все существа, находящиеся в выбранных областях, получают Повреждения, если герой поразил их Защиту. В этом режиме оружие расходует 10 зарядов (даже если его фактическая Скорострельность ниже). Ведущий Огонь на подавление герой не может выбирать Зоны поражения.
\newline
Эффекты КУ не применяются при использовании этого маневра.

\section{Тяжелые травмы}
Не каждому удается пережить Боевую сцену. Еще меньше тех, кто пережил ее, сохранив в целости все части тела. Большинство после такого уходит на покой, но герой вполне может продолжить приключения. Конечно, без трудностей не обойдется, но героям не привыкать.
\paragraph{Однорукие герои:} однорукие герои не могут использовать Двуручное оружие. Если рука обрублена ниже локтя, герой может закрепить на ней щит или одноручное оружие.
\paragraph{Одноногие герои:} одноногий герой с костылем или протезом считается перемещающимся по Трудному ландшафту. Костыль может использоваться для атаки. Без костыля или протеза одноногий герой считается перемещающимся по Трудному ландшафту и ополовинивает свою Ск (т.е. суммарно его Ск сократится в 4 раза). Одноногие герои не могут использовать Громоздкое оружие.
\paragraph{Безногие герои:} безногие герои понижают свой МРз на 1. Ск безногих героев составляет |МЛв| метров. Если у безногого героя есть какое-то средство перемещения - например, тачка с колесиками - при Перемещении он двигается на число метров, равное |Атлетике (Лв) / 4| (минимум на 1 метр). Безногий герой считается сидящим. Поэтому он проверяет Дб с Помехой, а враги в своем Боевом контакте атакуют его с Преимуществом. Безногие герои не могут использовать Громоздкое и Длинное оружие.
\paragraph{Потеря глаз:} герой, лишившийся всех глаз, находится в состоянии Ослепления. Одноглазый герой проверяет Меткость с Помехой. 
\begin{tcolorbox}
  Одноногие герои с костылем или протезом и Чувством равновесия не испытывают никаких неудобств при движении и сражении, а безногие герои с этим Атрибутом все также опасны в ближнем бою.
\end{tcolorbox}

\section{Наездники}
\paragraph{Атаки наездника:} герой получает Преимущество в ближнем бою, если атакует существ, меньших по размеру, чем его скакун.
\paragraph{Атаки по наезднику:} в ближнем бою при атаке по наезднику на скакуне, превышающем размер атакующего, атакующий получает Помеху, если не атакует Длинным оружием.
\paragraph{Атаки по скакуну:} совершаются по обычным правилам.
\paragraph{Атаки скакуна:} во время своего Действия скакун может совершать любые Маневры, доступные в соответствии его Навыкам и снаряжению. Скакун не обязан заявлять такой же Маневр, как и наездник. Исключение составляет Атака с разбега.
\paragraph{Действие и Перемещение скакуна:} скакун Действует и Перемещается в ту же Очередь, что и его наездник.
\paragraph{Проверки скакуна:} Проверки Дб, Мт и Вл скакуна совершаются с использованием Навыка <<Обращение с животными>> наездника.
\newline
Например, если скакун атакует, его Рукопашный бой заменяется Обращением с животными наездника. Скакун может использовать собственный Рукопашный бой, если он больше Обращения с животными наездника.
\paragraph{Реакция наездника} равна \textbf{|(Рц скакуна + Рц героя) ÷ 2|}.

\section{Дорожные войны}
\paragraph{Не дрова везешь!:} из-за тряскипассажиры транспортного средства совершают любые активные проверки с Помехой.
\paragraph{Отвлекать водителя воспрещается!:} водителю довольно проблематично заниматься чем-то еще, кроме управления транспортным средством. Все атаки водителя совершаются с 2 Помехами. Водитель не может использовать Двуручное оружие.
\paragraph{Под откос:} если машина переворачивается или врезается в препятствие, все пассажиры (включая водителя) получают Дробящие Повреждения, равные \textbf{|30 - Управление транспортом Водилы - Бонус доспеха|}. Если машина не была оснащена ремнями безопасности (или пассажиры не сочли нужным воспользоваться ими), удвойте успешно нанесенные Повреждения.
\newline
Те из пассажиров, кто не был пристегнут, могут избежать Повреждений, совершив проверку Атлетики (Лв) против \textbf{|10 + нанесенные Повреждения|}.
\paragraph{Борт к борту:} водитель может использовать транспортное средство, как оружие. Его Доблесть равна \textbf{|Эксплуатация + модификатор Ловкости + Прочность транспортного средства|}. При провале маневра транспортное средство получает Повреждения, равные величине провала. Обратите внимание, что это не всегда (хоть и зачастую) означает тараны и удары бортами — в случае гироскутеров, мотоциклов и небольших машин водитель заманивает соперника на обочину или в кювет!
\paragraph{Наперегонки!:} герой может попытаться оторваться от преследования, или наоборот, догнать кого-то. Если герой превышает скорость, указанную в разделе <<по дороге всегда быстрее>>, то максимальная скорость в км/ч, которой он может достигнуть, равна \textbf{|Эксплуатация + Воля + Реакция — Опасность местности| х 10}. Разумеется, она все еще ограничена соответствующим параметром из таблицы. Совершите проверку Эксплуатации против \textbf{|10 + Эксплуатация(Лв) противника + Опасность местности|}. В случае успеха герой добивается желаемого, а машине потребуется проверка Износа в конце сцены. В случае провала смотрите раздел <<Под откос>>.
\paragraph{Маневровая скорость:} в течение Очереди водителя транспортное средство может преодолеть расстояние в метрах, равное \textbf{|Эксплуатация(Лв) + Воля водителя + Реакция водителя — Опасность местности|}. Это расстояние включает в себя маневры любой сложности.