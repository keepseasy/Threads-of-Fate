\chapter{Боевые столкновения}
Полностью посвящена боевым сценам и их частным случаям, включая сражения верхом и на технике. А еще здесь рассказывается о \textit{последствиях} такого занятного времяпровождения.
\section{Общие термины}

\paragraph{Источник:} исходная точка воздействия - способности, эффекта, атаки и т.д. От источника определяется видимость цели, дистанция до нее, и совершаются прочие важные измерения. 
\paragraph{Круг:} в бою герои и статисты действуют по Очереди. Один полный Круг (то есть Очереди всех героев и статистов, принимающих участие в Сцене) занимает 5 секунд. Подразумевается, что герои и статисты действуют одновременно, однако для удобства Круг разделен на Очереди. 
\paragraph{Очередь:} часть Круга, посвященная отдельному герою или статисту. Продолжительность Очереди никак не регламентирована правилами, не считая длительности полного Круга, ограниченной 5 секундами.
\paragraph{Нападающий:} обозначает героя или статиста, агрессивно воздействующего на элементы Сцены. В описательных и игромеханических блоках в этом же контексте используются "атакующий", "поражающий", "стреляющий", "наносящий удар" и т.п.
\paragraph{Нападение:} комплекс агрессивных мер, посредством которого герой или статист воздействует на элементы Сцены. В описательных и игромеханических блоках в этом же контексте используются "атака", "нанесение", "удар", "поражение" и т.п.
\paragraph{Цель:} один или несколько элементов Сцены, являющихся точкой приложения усилий (не всегда агрессивных) героя или статиста. В описательных и игромеханических блоках в этом же контексте используются "жертва", "пораженный", "атакованный", "противник" и т.п. Герой должен видеть цель, чтобы взаимодействовать с ней, если не указано обратного.
\paragraph{Вредоносный эффект:} последствия воздействия нападающего или элемента Сцены на цель. Самый распространенный из таких эффектов - получение Повреждений. Некоторые вредоносные эффекты вступают в силу без получения Пв целью.
\paragraph{Зона действия:} область, в которой герой или статист может воздействовать на цель. Воздействия элементов Сцены так же обладают Зоной действия.
\paragraph{Боевой контакт:} зона воздействия оружия ближнего боя героя. Обычно Боевой контакт составляет 1 метр (соседняя с героем клетка на тактической карте), но некоторые Трюки, свойства и обстоятельства могут увеличить или уменьшить его.
\newline Возможна ситуация, когда герой находится в Боевом контакте у статиста, но статист не находится в Боевом контакте у героя и наоборот. Обычно это происходит при Внезапном нападении, а так же когда противники разного Размера, или один из них снаряжен Длинным оружием.
\paragraph{Полученные Повреждения:} Пв, полученные целью в результате отдельно взятого нападения или воздействия на нее. 
\paragraph{Предотвращенные Повреждения:} не учитываются при потере ЕЗ, но считаются полученными целью и нанесенными нападающим.
\paragraph{Игнорирование Повреждений:} если Повреждения проигнорированны, они не считаются полученными целью и нанесенными нападающим.
\begin{tcolorbox}
  Предотвращение Повреждений отражает ситуации, когда цель поражена, но не получила сколь-нибудь серьезного урона. Игнорирование Повреждений указывает, что цель увернулась от атаки, либо вообще не воспринимает ее, как вредоносный эффект.
\end{tcolorbox}

\section{Получение Повреждений}
Потеря ЕЗ целью в результате атак и вредоносных эффектов происходит в строгой последовательности.
\begin{enumerate}
  \item Определите, сколько Повреждений получила цель в результате успеха проверки нападающего или по каким-то иным причинам.
  \item Примените способности и свойства, активируемые получением Повреждений.
  \item Предотвратите Повреждения благодаря Прочности, затем примените Сопротивление, Иммунитет и прочие способности, сокращающие Пв. Примените Внезапную смерть и совершите сопутствующие проверки, если необходимо.
  \item Увеличьте Повреждения, если цель Уязвима к ним, и примените все прочие способности, увеличивающие полученные Пв.
  \item Подсчитайте общее число полученных целью Повреждений и понизьте на него ее текущие ЕЗ. Цель теряет это число ЕЗ*.
  \item Примените способности цели и все правила, активируемые потерей ЕЗ.
  \item Примените эффекты Критического Удара.
  \item Примените правило Болевого шока, если требуется.
  \item Проверьте, получила ли цель Опасную рану и проверьте Вн, если требуется.
\end{enumerate}
*Если цель не обладает Иммунтитетом к ТПв воздействия, и правила предписывают ей потерять ЕЗ, то сразу переходите к пункту 5. 
\begin{tcolorbox}
  Цель может получить Пв, но не потерять ЕЗ. Это важно для срабатывания некоторых эффектов.
\end{tcolorbox}

\paragraph{Одномоментные получение Поврежений и потеря ЕЗ:} подразумевает получение Пв и потерю ЕЗ в результате одного воздействия или эффекта - одного маневра, одного феномена, одного Хода и т.п. Если Пв получены, а ЕЗ потеряны в течение одного Круга, но из-за разных воздействий - нескольких маневров, феноменов, Ходов, они не считаются нанесенными одномоментно.

\paragraph{Единицы Характеристик (ЕХ) и их потеря:} некоторые атаки, нанося Повреждения, отнимают не ЕЗ жертвы (или не только ЕЗ), а понижают ее Характеристики. Под потерей ЕХ понимаются все подобные случаи. 

\subsection{Сопротивление, Уязвимость, Иммунитет и Родная стихия}
Атаки имеют один или более разных типов Повреждений, но не все они одинаково опасны для целей.
\paragraph{Уязвимость} к типу Пв увеличивает полученные целью Повреждения в 2 раза. 
\paragraph{Сопротивление} к типу Пв уменьшает полученные целью Повреждения в 2 раза. 
\paragraph{Иммунитет} к типу Пв делает цель абсолютно невосприимчивой к Повреждениям. Она игнорирует Пв этого типа. Цель не теряет ЕЗ в результате воздействия, к ТПв которого она имеет Иммунтитет.
\paragraph{Родная стихия} позволяет цели восстанавливать ЕЗ! Она не получает Повреждений. Вместо этого цель восстанавливает столько ЕЗ, сколько она получила бы Повреждений без эффекта Родной стихии.
\paragraph{}Если цель имеет Уязвимость к нескольким типам Повреждений атаки сразу, полученные ею Пв удваиваются за каждую Уязвимость. 
\newline Уязвимость и Сопротивление сводят друг друга на нет. 

\subsection{Критический Удар и его эффекты}
\paragraph{Критический удар (КУ):} когда во время проверки Доблести или Меткости на К20 выпадает число, указанное в графе оружия "КУ" или превышающее его, атака получает дополнительные свойства, если цель потеряла хотя бы 1 ЕЗ. Свойства зависят от типа Повреждений, которое наносит оружие.
\paragraph{КУ и 20 на кубике:} если при атаке на кубике выпадает 20, цель теряет 1 ЕЗ даже если Дб или Мт недостаточны для получения Пв. Эффекты КУ применяются в полном объеме.

\paragraph{}Эффекты КУ в зависимости от типа наносимых повреждений:
\begin{itemize}
  \item \textbf{Дробящие:} получивший КУ Оглушен.
  \item \textbf{Едкие:} получивший КУ начинает Растворяться. Состояние наступает даже в том случае, если цель не потеряла ЕЗ при атаке.
  \item \textbf{Колющие:} получивший КУ страдает от Внутреннего кровотечения.
  \item \textbf{Рубящие:} получивший КУ страдает от Кровотечения.
  \item \textbf{Проникающие:} снаряд или осколок засел в теле. Получивший КУ страдает от Внутреннего кровотечения. После завершения боя он страдает от Агонии до тех пор, пока снаряд не будет извлечен (Медицина против \textbf{|15|}). 
    \newline Если герой подвергается лечению до того как снаряд извлечен, он не страдает от Агонии и Внутреннего кровотечения, но боль дает о себе знать. Все Активные проверки героя получают Помеху, пока снаряд не извлечен. Сложность Медицины для извлечения снаряда возрастает до \textbf{|20|}. 
  \item \textbf{Огненные:} получивший КУ Загорается. 
  \item \textbf{Электрические:} получивший КУ сбит с ног.
  \item \textbf{Ледяные:} получивший КУ Неподвижен до конца своей следующей Очереди.
  \item \textbf{Ядовитые:} получивший КУ Отравлен. Он подвергается Побочкам Яда или Лекарства.
\end{itemize}

\section{Состояния}
Физические и ментальные воздействия могут повлиять на состояние (и поведение) героя, как в бою, так и вне его. Эффекты двух одинаковых состояний, происходящих из различных источников, используют наихудший вариант.
\newline Исключениями являются Внутреннее Кровотечение и Растворение - их эффекты складываются.
\paragraph{Агония:} герой охвачен мучительной болью, и проверяет Вл против \textbf{|20|} в момент получения состояния, а затем - в начале каждой Сцены. 
\newline При провале все, на что способен герой - лежать, вопить и поносить Судьбу. При успехе он держит себя (и, возможно, части своего тела) в руках. Он может проверять Дб и Мт и передвигается с половинной Ск. 
\newline Состояние длится, пока герой не восстановит 1+ ЕЗ (если Агония вызвана потерей ЕЗ), или не получит врачебную помощь (если Агония вызвана иными причинами). 
\paragraph{Внутреннее кровотечение:} органы героя повреждены. В начале каждой своей Очереди он теряет \textbf{|5 - МВн|} ЕЗ (минимум 2 ЕЗ), пока не восстановит 1+ ЕЗ или не получит врачебную помощь (Медицина против \textbf{|15|}).
\paragraph{Возгорание:} в начале своей Очереди цель и ее снаряжение получают 5 Пв и теряют 1 ЕЗ. Горящий может израсходовать Действие и Перемещение, чтобы потушиться. В этом случае, до начала следующей Очереди цели, атаки по ней совершаются по правилам Внезапного нападения.
\newline Состояние длится \textbf{|5 - МЛв цели|} Кругов (минимум 1 Круг).
\paragraph{Кровотечение:} герой истекает кровью. В начале каждой своей Очереди он теряет число ЕЗ, равное \textbf{|ЕЗ, потерянным при КУ|}, пока не восстановит 1+ ЕЗ или будет перевязан (Медицина или другой уместный Навык против \textbf{|10|}). 
\paragraph{Невидимость:} герой невидим. Герой не может быть выбран целью, и нападает Внезапно.
\paragraph{Неподвижность:} герой не может двигаться и защищать себя. Возможно, он схвачен, заморожен, опутан сетью или просто мертвецки пьян. Зщ героя падает до \textbf{|БАЗщ + БД|}. Все атаки по герою в Боевом контакте совершаются с Преимуществом. 
\paragraph{Оглушение:} герой теряет связь с реальностью (обычно из-за доброго удара по голове или под дых). Герой не может Действовать в свою следующую Очередь. В дальнейшем герой совершает Активные проверки с Помехой число Очередей, равное \textbf{|5 - МВн|}.
\paragraph{Ослепление:} герой ничего не видит. Все атаки по нему совершаются по правилам Внезапного нападения. Герой проверяет Дб и Мт с 2 Помехами. Он не может нападать на цели, находящиеся дальше, чем его Наблюдательность(Мд) в метрах.
\paragraph{Отравление:} герой под действием Яда. В дополнение к Первичному эффекту Яда или Лекарства, герой совершает все проверки с Помехой.
\paragraph{Ошеломление:} герой в замешатешльстве. Как правило, из-за какой-то неожданности, вроде нападения врагов. Герой вычитает БЩ и МЛв из своей Зщ и ополовинивает Ск. Состояние длится число Очередей, равное \textbf{|5 - МИн|}.

\paragraph{При смерти:} герой находится между жизнью и смертью и не способен сколь-нибудь активно участвовать в происходящем до следующей Интерлюдии. Герой Неподвижен и без сознания. В Боевом контакте он может быть добит без каких-либо проверок, если никто из окружающих этому не помешает.
\newline Если герой дотянул до начала следующей Интерлюдии, в зависимости от жанра и настроения игры он:
\begin{itemize}
  \item Приходит в себя в 0 ЕЗ (и Агонии).
  \item Впадает в кому и покидает повествование до оказания квалифицированной медицинской помощи или выполнения иных обусловленных контекстом условий.
  \item Умирает, не приходя в сознание.
\end{itemize}
\begin{tcolorbox}
  Если герой впал в кому, авторский коллектив рекомендует передать игроку управление персоной или статистом, или даже ввести в повествование нового героя.
\end{tcolorbox}
\paragraph{Ранен:} если текущие ЕЗ не превышают 2/3 от максимальных, то герой Ранен. Его Ск ополовинивается. 
\newline Это состояние связано с потерей ЕЗ, но иногда наступает, если герой страдает от вывиха или болевого приема.
\paragraph{Растворение:} в начале своей Очереди цель и ее снаряжение теряют 1 ЕЗ. Жертва может избавиться от состояния, пропустив Очередь и сорвав одежду. Если одежды на жертве не было, ей понадобится помощь квалифицированного химика и проверка Науки против \textbf{|15|}. Провал проверки наделит жертву Недостатками "Урод" или "Старая рана" до завершения игровой встречи. Если жертва уже обладает этими Недостатками, мастер может немедленно ввести их в игру.
\newline Если меры не приняты, состояние завершится через \textbf{|10 + [Пв, полученные при вызвавшей состояние атаке]|} Кругов.
\newline Растворение вызывается множеством различных веществ. Способ нейтрализации, который сработал однажды, в следующий раз может навредить! Проверка Науки нужна для каждого нового Растворения.
\newline В конце Сцены проверьте Неприятности снаряжения. Не исключено, что также понадобится проверить Износ.
\trouble
{Отдушка}{Никаких последствий, кроме специфического запаха, который исчезнет после хорошей стирки.}
{Патина}{Незначительные повреждения. Герой может исправить их, проверив Ремонт против \textbf{|15|}, или возместив мастеру 1/4 СП предмета. До завершения ремонта Осечка предмета возрастает на 1.}
{Ржа}{Серьезные повреждения, все еще обратимые. Герой может исправить их, проверив Ремонт против \textbf{|20|}, или возместив мастеру 1/2 СП предмета. До завершения ремонта Осечка предмета возрастает на 5.}
{Труха}{Снаряжение приведено в полнейшую негодность. Возможно, удастся всучить его подвыпившему старьевщику.}
\paragraph{Серьезно ранен:} Серьезно ранен: если текущие ЕЗ не превышают 1/3 от максимальных, то герой Серьезно ранен. Все Активные проверки совершаются с Помехой. 
\newline Это состояние связано с потерей ЕЗ, но иногда наступает, если герой страдает от вывиха или болевого приема. 
\paragraph{Сон:} герой спит, его Наблюдательность проверяется с Помехой. Во сне он Неподвижен. Если герой проснулся и вынужден действовать, он Ошеломлен до начала своей следующей Очереди. 
\newline Героя разбудят громкие звуки - выстрел из пистолета, звон будильника или вопли гибнущих товарищей. Тихие звуки потребуют проверки Наблюдательности (с Помехой).
\paragraph{Удушье:} герой задыхается, его проверки совершаются с Помехой. Герой потеряет сознание, если состояние продлится дольше, чем \textbf{|Вн*2|} Кругов в Боевой сцене, и умрет, если состояние продлится дольше, чем \textbf{|Вн*4|} Кругов.  
\paragraph{Ужас:} герой дрожит от ужаса. Все его Активные проверки совершаются с Помехой, пока источник Ужаса находится в зоне видимости или слышимости. Жертва получает Узы "Я бегу со всех ног от источника Ужаса".
\newline Когда источник Ужаса остался вне зоны видимости и слышимости, состояние длится число Очередей, равное \textbf{|10-Вл|}.
\paragraph{Усталость:} герой изнурен, все его Активные проверки совершаются с Помехой. Атаки по герою совершаются с Преимуществом.
\begin{tcolorbox}
\paragraph{} В некоторых кампаниях требуется ограничить возможность героев восполнять свои внутренние ресурсы - применения способностей, ЕЗ и Эн. В этих случаях можно добавить в игру два дополнительных состояния.
\paragraph{Истощение:} Герой долгое время функционировал на пределе, почти без сна и отдыха и его способности к восстановлению снижены. Все Заряды способностей, а так же восстановление ЕЗ и Эн во время Антрактов и Интерлюдий снижены ввдвое.
\paragraph{Сильное истощение:} Герой долгое время подвергался тяжелейшим испытаниям, при постоянной нехватки еды, воды и сна. Восстановиться совершенно не получается - ЕЗ, Эн и Заряды способностей возможны только с помощью зелий, медикаментов и механизмов.
\paragraph{} Для снятия состояний Истощение и Сильное истощение герою требуется провести 1 Антракт в полном покое при достаточном питании.
\end{tcolorbox}
\section{Боевые сцены}
Боевая сцена начинается, если один или несколько героев подверглись нападению, либо напали на кого-то сами.

\subsection{Детали Боевой сцены}
\paragraph{Болевой шок:} если герой одномоментно теряет число ЕЗ, превышающее его \textbf{|Вл|}, то его следующая проверка получает Помеху. 
Разумеется, эффект относится к существам, которые способны испытывать боль. 
\paragraph{Нокаут:} если герой одномоментно теряет 1/5 и более от максимальных ЕЗ, он проверяет Вн против \textbf{|15|}. При успехе герой продолжает сражаться, при провале теряет сознание.
\paragraph{Потеря сознания:} герой, потерявший сознание, очнется в начале следующей Сцены. Если не замерзнет, не истечет кровью, не будет добит или съеден. Герой, пришедший в сознание в 0 ЕЗ, находится в Агонии!
\newline Герой немедленно приходит в себя, если восстанавливает 1+ ЕЗ.
\paragraph{Смерть:} потеряв последнюю ЕЗ, герой проверяет Вн против \textbf{|15|}. При успехе он остается жив, теряет сознание и обзаводится шрамом на память. При провале герой или персона находятся При смерти.
\newline Некоторые эффекты убивают вне зависимости от числа ЕЗ. Возможность проверки Вн при такой смерти определяется мастером и контекстом Сцены.
\paragraph{Внезапная Смерть.} Порой урон очевидно смертелен, и никакая Защита и Выносливость здесь не спасут. Это включает попадания из танковых орудий, воздействие фугасных бомб, массированные артобстрелы, горные обвалы, извержения вулканов, метеоритные дожди и старые недобрые ядерные взрывы.
\newline Если герой или Персона теоретически имеют шанс уцелеть, проверяется  Внезапная смерть. Иммунитет к Пв атаки, вызывающей Внезапную смерть, дает 1 Преимущество на эту проверку, Родная Стихия - 2 Преимущества:
\trouble
{Ни царапины}{Герой невредим. Как ему это удалось - загадка.}
{Задело}{Герой получает Нокаут и хороший повод для отборной брани - если успешно проверит Вн.}
{Потрепало}{Герой получает Нокаут и тяжелую травму. Определите по таблице Случайного поражения, какая часть тела героя выведена из строя до конца следующей Сцены с его участием. Насколько серьезна травма в принципе, подскажут контекст и опытный врач.}
{Смерть}{Герой теряет все свои ЕЗ, а затем следует обычным правилам Смерти. Он получает тяжелую травму. Определите по таблице Случайного поражения, какая часть тела героя выведена из строя до конца следующей Сцены с его участием. Насколько серьезна травма в принципе, подскажут контекст и опытный врач.}
\paragraph{Смерть и статисты:} мастер вправе считать проверки Вн статистов (но не Персон) при Нокауте и Смерти автоматически проваленными. Также статисты (но не Персоны) всегда получают "Смерть" при Внезапной смерти. Попав в Состояние "При смерти", они сразу же погибают, если только Состояние не было вызвано Ходом "Повезло".
\paragraph{Все на одного:} сражение с толпой требует высочайшего мастерства. Если герой находится в Боевом контакте у нескольких противников, в начале Очереди он выбирает \textbf{|1 + ММд героя|} (минимум 1) противников, на которых сконцентрирован. Они атакуют героя, как обычно. Остальные проверяют Дб с Преимуществом, выбрав его целью.
\paragraph{Внезапное нападение.} Если герой не заметил врага, провалив Наблюдательность против его Скрытности, либо вовсе не ждет нападения из-за Эмоционального фона Сцены, он захвачен врасплох. Это значит, что:
\begin{itemize}
  \item Герой не может Перемещаться и Действовать независимо от значения Рц;
  \item МЛв и БЩ вычитаются из Зщ героя;
  \item Атаки по герою получают Преимущество;
  \item Если герой дожил до своей Очереди, он страдает от Ошеломления;
  \item Герой не имеет Боевого контакта.
\end{itemize}
\paragraph{Лежащие и сидящие герои:} лечь на землю или опуститься на корточки может быть хорошей идеей в одной ситуации и губительной - в другой.
\begin{itemize}
  \item Мт по лежащему или сидящему герою проверяется с Помехой;
  \item Дб по лежащему или сидящему герою проверяется с Преимуществом;
  \item Лежащий или сидящий на корточках герой может Перемещаться на \textbf{|МЛв|} метров;
  \item Лежащий или сидящий на корточках герой проверяет Дб с Помехой;
  \item Боевой контакт лежащего героя всегда равен 1;
  \item В свою Очередь герой может упасть, израсходовав Быстрое действие.
\end{itemize}
\paragraph{Невидимки и бой с ними:} тем, кто невидим, в бою попроще. Но и им не всегда удается избежать нападения. Особенно, когда противник точно знает, что они где-то здесь.
\begin{itemize}
  \item Герой может определить примерное местонахождение Невидимого противника по косвенным признакам в \textbf{|Наблюдательности(Мд)|} метров от себя.
  \item Если герой атакует Невидимого противника, Дб или Мт проверяются с 2 Помехами;
  \item Герой может израсходовать Действие и обнаружить Невидимую цель, успешно проверив Наблюдательность(Ин). Затем он может израсходовать Быстрое действие и указать местоположение цели союзникам. В этом случае они проверяют Дб и Мт с 1 Помехой, пока цель не сменит местоположение;
  \item Если нападающий невидим благодаря Скрытности, то, атаковав, он обнаружит себя;
  \item Для проверки Скрытности герою требуется разорвать зрительный контакт со всеми, от которых он желает спрятаться. Затем герой расходует Действие и при успехе проверки становится Невидимым.
\end{itemize}
Перемещение через занятые области: если герой преодолевает область, занятую противником, он должен проверить Атлетику против \textbf{|10 + [Дб противника] + [МРз]|}. Преуспев, герой Перемещается через область, при провале падает на границе и завершает Перемещение.
\newline Герой без проверок преодолевает области, занятые союзниками.
\paragraph{Трудный ландшафт:} иногда Перемещение героя затруднено густым подлеском, глубоким снегом, скользкой грязью или тряской автомобиля на полном ходу. В такой ситуации Ск героя ополовинивается.
\paragraph{Укрытия:} мешают стрельбе и помогают прятаться. Делятся на: 
\begin{itemize}
  \item Мягкие укрытия (кусты, высокая трава, куча хвороста) дают Помеху на Мт;
  \item Твердые укрытия  (камни, стены, башенный щит) дают 2 Помехи на Мт. 
\end{itemize}
В случае промаха атакующего по цели, Пв получает укрытие. Оно может быть уничтожено.
\newline Помехи за Укрытие подразумевают, что герой виден в какой-то момент Круга. Если герой весь Круг лежит ничком за толстой бетонной плитой и не отсвечивает, его не вправе заявлять целью. Зато вправе накрывать зоной поражения эффектов, если всем известно, где залег герой.

\subsection{В начале Боевой сцены}
\begin{enumerate}
  \item \textbf{Определите, подвергся ли кто-то из участников Внезапному нападению.} Обычно это сопряжено с провалом Наблюдательности против Скрытности нападающих. Если никто из участников Боевой сцены не пытался передвигаться скрытно до ее начала и ожидал нападения, фактор внезапности вряд ли стоит учитывать. С другой стороны, герой с трюком "Гопля!" или высокой Реакцией может застать противников врасплох, молниеносно выхватив оружие и атаковав.
  \item \textbf{Определите позиции участников Сцены относительно друг друга и расположение окружающих предметов и/или элементов ландшафта.} На обсуждение стоит потратить достаточно времени, чтобы дать игрокам представление о возможностях героев и их противников. 
    \begin{tcolorbox}
      Для предельной ясности вы можете использовать тактическую карту, расчерченную на клетки. Авторский коллектив рекомендует установить размер клетки, как 1 кв.м. Обозначьте фишками героев и их противников, разместите или нарисуйте на карте предметы и ландшафт. Тактическая карта - превосходный вариант, если ваша игровая команда предпочитает детализировнные Боевые сцены.
    \end{tcolorbox}
  \item \textbf{Определите Очередность участников Сцены.} Обычно первым действует участник боя с наибольшей Реакцией. Если у двух участников одинаковая Рц, первым действует счастливчик с большей Ловкостью. Если Лв равна, первым действует выигравший Состязание в Рц. 
    \newline Когда мастер желает привнести в Сцену немного интриги, очередность действий может определяться проверкой Реакции. Первым действует участник, получивший большую величину успеха, и так далее.
  \item \textbf{Начало Круга.} Все участники Сцены действуют согласно определенной Очередности. Возможно, для завершения битвы понадобится больше одного Круга.
\end{enumerate}

\subsection{Структура Очереди}
В течение своей Очереди герой может совершить Перемещение, Действие и Быстрое действие. 
\paragraph{Во время Действия} герой использует предметы, совершает маневры, творит феномены, убеждает окружающих прекратить кровопролитие или делает что-то еще, требующее концентрации внимания. 
\paragraph{Для Перемещения} используется значение Скорости - столько метров/клеток может преодолеть герой.
\paragraph{Быстрое действие} включает односложные фразы, отрывистые жесты, Шаги в пределах 1 метра и броски предметов без проверок Мт. Хрупкие предметы могут сломаться или разбиться. 
\newline Герой может иметь несколько Быстрых действий. Любое применение способности, использующее Быстрое действие, расходует одно из них. Резерв Быстрых действий героя обновляется в начале каждой его Очереди. Если герой не выполнил в течение своей Очереди Действие или Перемещение, его резерв Быстрых действий возрастает на один за каждое из них.
\newline Иногда выполнение Быстрого действия возможно и в чужую Очередь. Такие случаи указаны в описании соответствующих Трюков, Атрибутов и заклинаний, либо продиктованы контекстом Сцены.

\paragraph{Отказ от Действия или Перемещения} позволяет сделать герою следующее: 
\begin{itemize}
  \item Совершить дополнительное Перемещение;
  \item Совершить дополнительное Быстрое действие;
  \item Подняться с земли;
  \item Взять предмет;
  \item Аккуратно положить предмет;
  \item Привести оружие в боевую готовность;
  \item Убрать оружие в ножны или кобуру;
  \item Презарядить оружие;
  \item Вскочить в седло скакуна, за руль тачки или в ее кузов.
\end{itemize}

\paragraph{Последовательность} Действия, Быстрого действия и Перемещения может быть произвольной в пределах Очереди. Например, совершая Быструю атаку, герой со Скоростью 6 может за одну свою Очередь пройти на 1, атаковать, потом пройти на 2, атаковать еще раз, затем пройти на 3 и уронить предмет.
\begin{tcolorbox}
  Быстрые действия могут показаться чем-то малозначительным, но на деле с их использованием сногое связано. Уникальные ходы и мощные Трюки часто расходуют Быстрое действие, и ориентированному на бой герою стоит иметь пару-тройку в запасе. Не забывайте, отказавшийся от Перемещения получает дополнительное Быстрое действие. Это значит, что герой, уже попавший в гущу событий, располагает минимум двумя Быстрыми действиями.
\end{tcolorbox}

\subsection{Атаки ближнего боя}
Насилие на расстоянии вытянутой руки, или чуточку дальше. Загодя обзавестись хоть каким-то оружием ближнего боя - надежный фундамент победы. 
\newline Чтобы поразить цель в ближнем бою, герой должен располагать такой целью (или целями) в своем Боевом контакте.
\begin{itemize}
  \item Боевой контакт героя Крошечного, Маленького или Среднего размера, не снаряженного Длинным оружием, составляет 1.
  \item Совершая Атаку ближнего боя, герой проверяет Доблесть.
\end{itemize}

\subsection{Дистанционные атаки}
Представляют всю широту эффективных методов убийства издалека. Гибель цивилизации -  не повод отказываться от ее плодов, даже если они слегка подгнили.
\newline Дистанционные атаки имеют 2 типа дистанций: Ближняя и Дальняя. Они указаны в описании оружия и феноменов.
\begin{itemize}
  \item Если цель находится за пределами Дальней дистанции, герой не может поразить ее.
  \item Атаки на Дальней дистанции совершаются с Помехой.
  \item Совершая Дистанционную атаку, герой проверяет Меткость.
\end{itemize}
\paragraph{Дистанционные атаки и Боевой контакт:} ближний бой - не лучшее место для демонстрации навыков стрельбы. Но выбор не всегда есть даже у самых лучших стрелков.
\begin{itemize}
  \item Герой совершает Дистанционные атаки с Помехой, если находится в Боевом контакте у одного или нескольких противников.
  \item Герой совершает Дистанционные атаки с Помехой, если цель находится в чьем-то Боевом контакте, кроме его собственного.
\end{itemize}
\paragraph{Перемещение и Дистанционные атаки:} если в свою Очередь герой перемещается на расстояние, превышающее 1 метр, он проверяет Меткость с Помехой. Конечно, герой может выстрелить без Помехи до Перемещения.
\subsection{Зоны поражения}
Зоны поражения позволяют закончить бой быстро и наверняка, особенно, если герой нападает Внезапно. Также они помогут вывести противника из строя, не убивая его.
\newline Герой вправе выбрать любую из Зон. Если при атаке не заявлена специфическая Зона, удар наносится в торс.
\begin{center} \begin{tabular}{|c|c|} \hline
  \textbf{Зона поражения} & \textbf{Штраф к Дб/Мт} \\ \hline
  Торс & 0 \\ \hline
  Конечность & -2 \\ \hline
  Пах & -3 \\ \hline
  Шея & -5 \\ \hline
  Голова & -5 \\ \hline
  Глаз & -7 \\ \hline
\end{tabular} \end{center}
\paragraph{Торс:} попадание не имеет никаких эффектов, кроме синяков, шрамов, шишек и потери уверенности в себе вкупе с некоторым числом ЕЗ. 
\paragraph{Конечность:} попадание по руке или ноге может вызвать Перелом или Потерю конечности. Когда
\begin{itemize}
  \item Если конечность одномоментно теряет 1/5 от максимальных ЕЗ, она сломана. Перелом приводит к Нокауту;
  \item Если конечность одномоментно теряет  1/4 от максимальных ЕЗ, она отрубается или разлетается в клочки. Пв, превышающие требуемое для уничтожения конечности число ЕЗ, теряются;
  \item Если конечность потеряна, герой страдает от Кровотечения (на внушительные 1/4 ЕЗ). Само собой, потеря конечности ведет к Нокауту.
\end{itemize}
\paragraph{Пах:} существование этой Зоны - веский повод носить бронегульфик.
\begin{itemize}
  \item Мужчина, отхвативший в пах, проверяет Вн против \textbf{|10 + [потерянные ЕЗ]|}. При провале он не может Действовать и Перемещаться число Очередей, на которое провалил проверку, хотя может орать, сквернословить и совершать Быстрые действия. Все атаки по нему получают Преимущество.
  \item Женщины тоже не в восторге от ударов в пах, но столь разрушительных игромеханических последствий не испытывают. 
\end{itemize}
\paragraph{Шея:} зомби, грибы, трехголовые мутанты и тому подобные твари могут какое-то время обходиться без шеи и головы.
\newline Остальным сложнее, так как если шея одномоментно теряет 1/4 и более от максимальных ЕЗ, герой эффектно расстается с головой и умирает.
\paragraph{Голова:} если и есть что-то, более хрупкое, чем пах, так это она. К тому же
\begin{itemize}
  \item Получивший удар проверяет Вн против \textbf{|10 + [потерянные ЕЗ]|}. При провале он теряет сознание.
  \item Если в голову нанесены Пв, достаточные для Нокаута, жертва, проверяет Вн против \textbf{|15|}, но умирает при провале.
\end{itemize}
\paragraph{Глаз:} самое уязвимое место на теле огромных монстров. А еще
\begin{itemize}
  \item Глаз не может быть поражен Громоздким оружием;
  \item Одномоментная потеря \textbf{|2 + МРз|} ЕЗ и более приводит к потере глаза;
  \item Маленькие и Крошечные существа лишаются глаза при потере им 1 ЕЗ;
  \item Жертва, потерявшая все глаза, Ослеплена;
  \item Все лишние Пв получает голова;
  \item Если атака имеет Колющие или Проникающие Пв, потерянные жертвой ЕЗ удваиваются - атака поражает не только глаз, но и мозг. Жертва проверяет Вн с Помехой против \textbf{|10 + потерянные ЕЗ|} и теряет сознание при провале;
  \item Если Колющая или Проникающая атака наносит в глаз Пв, достаточные для Нокаута, жертва проверяет Вн против \textbf{|15|} с Помехой, и умирает при провале.
\end{itemize}
\subsection{Таблица случайного поражения}
Если в контексте ситуации важно, куда именно пришелся удар или подействовал эффект, а специфические зоны не были заявлены, герой проверяет Случайное поражение, когда является нападающим или целью. Во всех прочих случаях проверку совершает мастер.
\begin{center} \begin{tabular}{|c|c|} \hline
  \textbf{Результат на К20} & \textbf{Зона поражения} \\ \hline
  1-6 & Торс \\ \hline
  7-8 & Левая нога \\ \hline
  9-10 & Правая нога \\ \hline
  11 & Пах \\ \hline
  12-13 & Левая рука \\ \hline
  14-15 & Правая рука \\ \hline
  16 & Шея \\ \hline
  17 & Голова \\ \hline
  18 & Левый глаз \\ \hline
  19 & Правый глаз \\ \hline
  20 & Проверяющий сам выбирает Зону поражения \\ \hline
\end{tabular} \end{center}

\section{Маневры}
Во время Действия герой может совершить маневр. Некоторые маневры могут составить цепочку-комбо, требовать особого снаряжения, а так же траты Перемещения и/или Быстрого действия.
\paragraph{Маневр расходует} Действие, если в его описании не указано обратного.
\paragraph{Комбо:} многие маневры сочетаются друг с другом, иногда в весьма неожиданных комбинациях. Комбо считается единым маневром.
Чтобы составить комбо:
\begin{enumerate}
  \item Выберите 1 Базовый маневр.
  \item Выберите 0-2 разных Модифицирующих маневра.
  \item Выберите 0-1 Специальный маневр.
\end{enumerate}
Итого, максимальное число маневров в комбо – 4.

\paragraph{Два оружия ближнего боя в двух руках} обычно не лучшая идея, если у героя нет Трюка «Амбидекстр». Оружие мешает друг другу и больше путает нападающего, чем вредит цели. Тем не менее, если оба оружия Легкие, герою будет проще управиться с ними при Быстрой атаке. 

\paragraph{Два дальнобойных оружия в двух руках:} способны чуток увеличить темп стрельбы. Да, герой с «Амбидекстром» стреляет с двух рук гораздо эффективнее. Но и герой без него
\begin{itemize}
  \item Выбирает основное оружие в одной из рук, и прибавляет к нему Скорострельность вспомогательного оружия в другой.
  \item Всегда выбирает основным оружие с меньшим БПв. Координировать такой поток стрельбы совсем не просто, потери в точности и ущербе неизбежны.
  \item Может повысить Скорострельность основного оружия на число, не превышающее своего \textbf{|МЛв|} и Скорострельности вспомогательного оружия (минимум на 1).
  \item Может повысить Скорострельность основного оружия на число, не превышающее своего \textbf{|МЛв*2|} и Скорострельности вспомогательного оружия (минимум на 2), если одно из оружий Легкое.
\end{itemize}

\subsection{Совершение маневра}
\begin{enumerate}
  \item Выберите нападающего.
  \item Выберите Базовый маневр или составьте комбо на его основе.
  \item Выберите цели нападающего.
  \item Определите штрафы и бонусы к проверке нападающего. Как правило, это:
    \begin{itemize}
      \item Штраф зоны поражения;
      \item Штраф за размер предмета;
      \item Штраф и Помеха при поражении на Дальней дистанции;
      \item Помехи и Преимущества за положение цели, в т.ч. за Укрытие;
      \item Помехи и Премущества, сопутствующие некоторым маневрам;
      \item Помехи и Преимущества, связанные со свойствами воздействия нападающего и защитных средств цели.
    \end{itemize}
  \item Совершите необходимые проверки и определите последствия. 
\end{enumerate}
\begin{tcolorbox}
  Внимательно изучите возможности, которые предлагает перечень маневров. Он включает абсолютное большинство действий, которые может совершить герой со своими противниками или окружением.
  \newline Маневры предполагают сопротивление врага. Захваченные врасплох противники также сопротивляются – просто недостаточно быстро. Если цель не способна или не желает сопротивляться, то используйте рекомендации из раздела «Когда бросать кубик».
\end{tcolorbox}

\subsection{Базовые маневры}
Та самая база. Выбрав маневр, решите, будет ли герой составлять комбо. Если нет, он ограничится простыми и действенными приемами. 
\paragraph{Атака.} Герой атакует в ближнем бою.
$\bullet$ Герой выбирает цель в своем Боевом контакте и проверяет Дб.

\paragraph{Дистанционная атака.} Герой стреляет или бросает предмет в цель. 
$\bullet$ Герой выбирает цель в досягаемости оружия и проверяет Мт. 

\paragraph{Комбинированная атака.} Герой использует оружие ближнего и дальнего боя одновремено. 
$$\bullet$$ Герой выбирает выбирает две цели (или одну и ту же цель дважды) в досягаемости оружия и в своем Боевом контакте. Он получает Помеху на проверки Дб и Мт до завершения Очереди, затем проверяет Дб или Мт в соответствии с выбранной целью. 

\subsection{Модифицирующие маневры ближнего боя}
Эффект Модифицирующих маневров в комбо замещает эффекты Базовых в случае противоречий. 

\subsubsection{Быстрая атака.}
Герой жертвует точностью ради быстроты.
$\bullet$ Герой выбирает две цели (или одну и ту же цель дважды). 
\newline $\bullet$ Он проверяет Дб с Помехой по каждой из целей. 
\newline $\bullet$ Если герой имеет оружие в каждой руке и оба оружия Легкие, только одна проверка совершается с Помехой. 

\subsubsection{Сокрушительная атака.}
Герой вкладывает всю злость в один мощный удар.
$\bullet$ Герой проверяет Дб с Преимуществом. До начала его следующей Очереди все атаки по нему совершаются с Преимуществом. 

\subsubsection{Точная атака.}
Герой атакует уязвимые места противника. По крайней мере, пытается. 
$\bullet$ Шанс выпадения КУ при проверке Дб возрастает на ММд или МЛв героя (минимум на 1).

\subsection{Модифицирующие маневры дистанционного боя}
Эффект Модифицирующих маневров в комбо замещает эффекты Базовых в случае противоречий.

\subsubsection{Беглый огонь.}
\tbd Автоматический режим стрельбы – вот истинное чудо цивилизации.
$\bullet$ Герой может выбрать несколько целей, стреляя из оружия (в том числе, в двух руках) с суммарной Скорострельностью 2 и выше. Число целей не должно превышать \textbf{|1 + ММд|} героя (минимум 2) и значения Скорострельности оружия. 
\newline $\bullet$ Маневр допускает выбор различных зон поражения у разных целей. Герой совершает одну проверку Мт с Помехой и наибольшим штрафом зон поражения. Эффекты КУ применяются только к одному из противников по выбору стрелка.

\subsubsection{Верный выстрел.}
Герой замирает на пару секунд, чтобы точно не промазать.
$\bullet$ Герой совершает маневр Дистанционной атаки с Преимуществом. До начала его следующей Очереди атаки по нему совершаются с Преимуществом. 

\subsubsection{Концентрированный огонь.}
Возможность расстрелять весь рожок в какого-нибудь засранца.
$\bullet$ Маневр применим, если герой использует оружие (в том числе, в двух руках) с суммарной Скорострельностью 2 и выше. Когда Скорострельное оружие используется против одной цели, БПв оружия возрастает на 1 за каждый заряд после первого, выпущенный по ней. 
\newline Например, пистолет имеет БПв +2 и Скорострельность 3. Если герой трижды стреляет из пистолета в одну цель, БПв пистолета возрастает до +4 (то есть на 2). Если герой применит пистолет-пулемет с БПв +2 и Скорострельностью 10, по одной цели, то его БПв возрастет до +11 (то есть на 9).
\newline $\bullet$ Маневр может совмещаться с Беглым огнем. Например, из пистолета-пулемета с БПв +2 герой может выпустить 4 пули в одну цель,  1 – в другую и 5 – в третью. Первый выстрел будет иметь БПв +5, второй – БПв +2, а третий – БПв +6.

\subsection{Специальные маневры}
Эффект Специальных маневров в комбо замещает эффекты Базовых и Модифицирующих в случае противоречий.

\subsubsection{Захват.}
Герой пытается схватить и удержать противника:
\begin{itemize}
  \item Герой проверяет РДб против \textbf{|БАЗщ + [РДб противника]|}. При успехе противник не получает Пв, но становится захваченным.
  \item Если герой использует для Захвата несколько конечностей, он получает к проверке бонус, равный числу конечностей. Например, Захватывая 2 руками, герой получит +2.
  \item Пока противник захвачен, все проверки по нему в Боевом контакте получают Преимущество.
\end{itemize}
Для Маневра нужна минимум 1 свободная рука. Некоторые Трюки и виды оружия позволяют проводить Захват. В этом случае замените для нападающего РДб на Дб во всех формулах.
\newline Если у героя есть Трюк «Знаток оружия», то он вправе использовать Дб, даже являясь захваченным.

\paragraph{Нападающий может} (в том числе в ту же Очередь, когда совершен Захват):
\begin{itemize}
  \item Атаковать захваченного одноручным оружием в свободной руке в соответствии с условиями Модифицирующих маневров. Захваченный вычитает МЛв и БЩ из своей Зщ;
  \item Атаковать другие цели одноручным оружием в свободной руке в соответствии с условиями Модифицирующих маневров;
  \item Перемещаться на половину Ск. Нападающий передвигается без ограничений, если его МРз на 2+ больше МРз Захваченного; Нападающий не может перемещаться, если его МРз на 2+ меньше МРз захваченного;
  \item Сменить Зону, за которую Захватил. Герой проверяет РДб против \textbf{|БАЗщ + РДб захваченного|}. Если герой использует для Захвата несколько конечностей, он получает бонус, равный числу конечностей. Например, Захватывая 2 руками, герой получит +2 к РДб;
  \item Блокировать захваченного. До начала своей следующей Очереди захваченный не в состоянии совершать любые действия (даже Быстрые). Все, что может блокированный захваченный – вырываться;
  \item Душить захваченного. Для этого жертва должна быть захвачена за шею. Захваченный получает Пв = |К20 + [РДб нападающего] – [РДб захваченного] – [МВн захваченного]|. Если в ходе удушения ЕЗ захваченного достигают 0, он теряет сознание. Отрицательный МВн прибавится к Пв от удушения;
  \item Метнуть захваченного. Герой не может метать цели, вес которых превышает его \textbf{|[Комфортную нагрузку]*2|}.
    \newline Нападающий считает захваченного Громоздким Импровизированным Метательным оружием с БПв = \textbf{|МРз|} захваченного, Ближней дистанцией = \textbf{|МСл + МЛв|} нападающего, КУ = \textbf{|20 – МРз|} захваченного и Дробящими Пв. Захваченного нельзя метнуть на Дальнюю дистанцию;
  \item Отпустить захваченного.
\end{itemize}
Смена Зоны захвата и метание захваченного требуют наличия в комбо атак, не израсходованных на непосредственно Захват, а удушение, блокирование и отпускание – нет.
\newline После того, как герой успешно совершил Захват, в свои последующие Очереди он вправе выбирать другие Специальные маневры.

\paragraph{Захваченный может}:
\begin{itemize}
  \item Атаковать нападающего с Помехой одноручным Легким оружием;
  \item Атаковать любые цели с Помехой одноручным Легким оружием, если МРз захваченного больше МРз нападающего;
  \item Творить феномены, если соблюдет необходимые условия. Проверки, необходимые для успеха феномена, совершаются с Помехой;
  \item Вырываться. Захваченный расходует Действие и проверяет РДб, Атлетику (Сл, Лв), Сл или Лв против \textbf{|БАЗщ + РДб + МРз|} нападающего. При успехе он освобождается;
  \item Достать Легкое оружие. Захваченный может отказаться от Перемещения, чтобы достать оружие, несмотря на то, что фактически обездвижен;
  \item Совершать Быстрые действия;
  \item Перемещаться, если его МРз на 2+ больше МРз нападающего. 
\end{itemize}
\begin{tcolorbox}
  Если герой планирует применять болевые приемы, обратите внимание на Трюк «Костолом».
\end{tcolorbox}

\subsubsection{Касание}
Легкое прикосновение, с трудом ощутимое сквозь одежду и совершенно незаметное для облаченных в броню.
\begin{itemize}
  \item Маневр не наносит Пв. Если герою требуется дотронуться до сопротивляющейся цели, он должен преуспеть в проверке РДб против \textbf{|БАЗщ + МЛв|} цели;
  \item \tbd Герой должен выбрать открытый участок тела в качестве зоны поражения, если касание предполагает контакт с телом. Чем выше БД, тем сложнее отыскать такой участок!
\end{itemize}

\subsubsection{Разоружение}
Герой пытается выбить предмет из рук противника или сбить предмет с его тела.
\begin{itemize}
  \item Маневр не наносит Пв. Герой должен преуспеть в проверке Дб против \textbf{|БАЗщ + Дб + БЩ|} цели;
  \item Герой получает штраф к Дб за размер предмета (сверьтесь с таблицей ниже);
  \item Если противник держит предмет в 2 руках, герой получает -2 к Дб;
  \item Маневр не может повредить предмету, однако хрупкие предметы могут разбиться при падении;
  \item При успехе предмет падает на расстоянии от цели, не превышающем МЛв нападающего. Нападающий выбирает точку, в которую упадет предмет;
  \item Если хотя бы одна рука героя свободна, он может выхватить предмет или оружие из рук противника, применив РДб. Если герой использует обе руки, он получает +2 к РДб. При успехе предмет оказывается в его руках;
  \item Когда герой выбирает целью маневра щит, то БЩ не учитывается при проверке. Щиты основательно закреплены на руке. Чтобы сорвать щит понадобится дополнительная проверка Сл, Лв или Атлетики против \textbf{|10 + БЩ|}.
\end{itemize}

\subsubsection{Сбить с ног}
Герой настоятельно предлагает противнику прилечь и отдохнуть.
\begin{itemize}
  \item Маневр не наносит Пв. Чтобы сбить противника с ног, герой должен успешно проверить Дб против \textbf{|10 + Дб + БЩ + МРз|} противника;
  \item Герой получает штраф -2 за зону поражения (ноги);
  \item Если существо четвероногое, герой получает дополнительные -2;
  \item Если существо передвигается на брюхе, как змеи или гигантские слизни, герой получает дополнительные -4;
  \item Маневр выполняется только Длинным оружием, или при помощи РДб.
\end{itemize}

\subsubsection{Сломать снаряжение}
Герой расточительно портит снарягу противника.
\begin{itemize}
  \item Чтобы нанести ущерб предмету в руках или на теле цели, герой должен проверить Дб или Мт против \textbf{|БАЗщ + Дб + БЩ|} противника.
  \item Предмет получает Пв, равные величине успеха героя. Если герой пытается разбить щит, БЩ не учитывается в сложности проверки.
  \item Герой получает штраф к Дб или Мт за размер предмета (сверьтесь с таблицей ниже).
\end{itemize}
\begin{tcolorbox}
  Большинство предметов, созданных для боевых действий, имеет высокую Прочность и весьма устойчиво к Повреждениям. Этого нельзя сказать обо всех остальных предметах – вряд ли их хватит хотя бы на пару ударов.
\end{tcolorbox}

\begin{center} \begin{tabular}{|p{7cm}|p{3cm}|} \hline
  \textbf{Оружие/предмет} & \textbf{Штраф к ДБ при разоружении/поломке} \\ \hline
  Доспехи, закрывающие большую часть тела (БД 7+), башенный щит. & 0 \\ \hline
  Громоздкое и Длинное двуручное оружие, доспехи, закрывающие значительную часть тела (БД 4-6). & -1 \\ \hline
  Громоздкое или Длинное двуручное оружие, большой щит, легкие доспехи (БД 1-3). & -2 \\ \hline
  Двуручное оружие, Громоздкое или Длинное Универсальное оружие, щит, большой рюкзак. & -3 \\ \hline
  Универсальное оружие, Громоздкое или Длинное одноручное оружие, рюкзак. & -4 \\ \hline
  Одноручное оружие, кулачный щит, широкий ремень. & -5 \\ \hline
  Легкое оружие, скрученная газета, шапка, барсетка. & -6 \\ \hline
  Кастет, перчатка, наруч, пузырек, узкий ремень. & -7 \\ \hline
  Перстень, браслет, ключ. & -8 \\ \hline
\end{tabular} \end{center}

\subsubsection{Толчок}
Иногда требуется лишь подтолкнуть, а гравитация сделает остальное.
\begin{itemize}
  \item Маневр не наносит Пв. Герой отталкивает цель на 1 метр прямо от себя, успешно проверив Дб, Сл или Атлетику (Сл) против \textbf{|БАЗщ + Дб - МРз|} цели;
  \item Если маневр совмещается с Разбегом, герой отталкивает цель на 1 метр прямо от себя за каждую единицу успеха;
  \item Герой не может толкать существ и предметы, вес которых превышают его максимальную нагрузку.
\end{itemize}

\subsubsection{Финт}
Герой отвлекает противника обманным выпадом или жестом.
\begin{itemize}
  \item Маневр не наносит Пв. Герой проверяет Общение (Мд, Об) или Владение оружием/Рукопашный бой против \textbf{|10 + [Владение оружием/Рукопашный бой(Ин, Мд)]|} цели;
  \item При успехе следующая атака (героя и любого его союзника) по цели Финта совершается с Преимуществом. Эффект длится до конца следующей Очереди героя, применившего Финт.
\end{itemize}

\subsection{Независимые маневры}
Условия выполнения Независимых маневров указаны в описании. Они дополняют эффекты прочих маневров.

\subsubsection{Выжидание}
Герой ожидает провоцирующего события или действия, прежде чем что-либо предпринимать.
\begin{itemize}
  \item Любое Действие может быть совмещено с этим маневром и выполнено, как Выжидающее. Например: Захватить, если враг потянется за оружием, Атаковать, если враг начнет активацию феномена, нажать кнопку, \textit{если} включится сирена воздушной тревоги и т.д.
  \item В этом случае герой не выполняет выбранные маневры, пока не произойдет провоцирующее событие или действие;
  \item Чтобы среагировать раньше, чем противник фактически совершит провоцирующее действие, герой должен успешно проверить Рц против \textbf{|10 + [Рц противника]|}. 
\end{itemize}

\subsubsection{Защитная стойка}
Герой полностью сосредочен на обороне.
\begin{itemize}
  \item Все атаки по герою до начала его следующей Очереди совершаются с Помехой;
  \item Лежащий или сидящий герой тможет выбрать этот маневр;
  \item Герой не может выбирать Модифицирующие и Специальные  маневры, если выбрал этот. 
\end{itemize}

\subsubsection{Огонь на подавление}
Герой упоенно поливает огнем пятачок земли, не заботясь о точности.
\begin{itemize}
  \item Он может использовать маневр, если снаряжен оружием (в том числе, оружием в двух руках) с суммарной Скорострельностью 5 и выше;
  \item Он не вправе совмещать маневр с каким-либо другим;
  \item Герой выбирает число смежных областей площадью 1х1 метр, не превышающее Скорострельности оружия. Он проверяет Мт с 2 Помехами и Осечкой 5. Если оружие уже имеет Осечку, используется большая;
  \item Он расходует Действие и Перемещение;
  \item Все существа, находящиеся в выбранных областях, получают Пв, если герой поразил их Зщ. Герой всегда расходует число зарядов, равное маскимальной Скорострельности своего вооружения. Он не может выбирать зоны поражения;
  \item Эффекты КУ не применяются при использовании этого маневра.
\end{itemize}

\subsubsection{Прицеливание}
Герой терпеливо ждет удачного момента для выстрела.
\begin{itemize}
  \item Маневр совмещается только с Дистанционной атакой.
  \item Герой может выбрать противника, Прицелится в него и сместить свою Очередь на 1-3 вниз, то есть действовать после менее быстрого героя или статиста. Если в текущем Круге герой действовал последним, то следующий пропуск Очереди смещает его ход на следующий Круг.
  \item Герой, сместивший свою Очередь на 1, проверяет Мт с Преимуществом. Враги атакуют его с Преимуществом, пока он Прицеливается.
  \item Герой, сместивший свою Очередь на 2, проверяет Мт с 2 Преимуществами. Враги атакуют его с 2 Преимуществами, пока он Прицеливается.
  \item Герой, сместивший свою Очередь на 3, проверяет Мт с 2 Преимуществами и игнорирует Помеху за Дальнюю дистанцию. Враги атакуют его с 2 Преимуществами, пока он Прицеливается.
  \item После совершения атаки герой снова может действовать в соответстви со значением своей Рц.
  \item Если во время Прицеливания герой одномоментно теряет ЕЗ, превышающие его \textit{|Вл|}, он  теряет все бонусы Прицеливания. 
\end{itemize}

\subsubsection{Провокация}
При помощи оскорбительных фразочек и еще более оскорбительных жестов герой привлекает внимание врага.
\begin{itemize}
  \item Цель должна видеть или слышать героя;
  \item Маневр может совмещаться с любым другим и расходует Быстрое действие;
  \item Успешно проверив Общение (Об) против \textbf{|10 + [Вл цели]|}, герой разъяряет противника и становится его следующей целью;
  \item Противник использует лучшие из своих способностей, чтобы напасть на героя. 
\end{itemize}
Если жертва не слышит героя, не видит его или не понимает его язык, проверка совершается с Помехой. Если героя и не видят, и не слышат, Провокация не работает.
\newline Если противник осознает, что неспособен навредить провокатору в течение своей Очереди, Провокация не работает.
\begin{tcolorbox}
  Провокация имеет смысл в случаях, когда враги могут атаковать или как-то еще наказать оскорбившего их героя без чрезмерных затруднений. Выманить противника из дота или танка Провокацией не получится, хотя отвлечь огонь пулеметчика или танкиста на себя – вполне.
\end{tcolorbox}

\subsubsection{Натиск}
Герой наседает на противника в ближнем бою и агрессивно теснит его, вынуждая отойти.
\begin{itemize}
  \item Маневр не наносит Пв, расходует Действие и Перемещение. Герой проверяет Дб против \textbf{|10 + [Дб цели]|}. При успехе цель отступает от героя на число метров, равное величине его успеха. Цель не может отступить на число метров, превышающее ее Ск;
  \item Герой двигается за целью, стремясь сохранить Боевой контакт, если это возможно и безопасно для него;
  \item Если Ск героя меньше, чем Ск цели, цель все равно должна отступить на число метров, равное величине успеха Дб героя. При этом цель может выйти из Боевого контакта героя;
  \item Герой выбирает, куда и по какой траектории отступит цель;
  \item Если цель не может отступить на требуемое число метров (ввиду недостаточно высокой Ск, либо из-за отсутствия места для отступления), она получает Помеху на Дб и Мт до конца своей следующей Очереди;
    \newline Цель не обязана отступать, если это приведет к заведомо гибельным последствиям – падению со скалы, погружению в зыбучий песок, глубокую воду и т.д. В этом случае она получает Помеху на Дб и Мт, как описано выше. Цель должна отступить, если это сделает ее положение потенциально опасным или просто менее выгодным. Она отойдет в неглубокий ручей с сильным течением, под медленно вращающиеся лопасти вентилятора, на крутую лестницу, гнилые помостки, подтаявший, но с виду достаточно прочный лед и т.д.
  \item Герой не может выбирать Базовые, Модифицирующие и Специальные  маневры в дополнение к этому.
\end{itemize}
\begin{tcolorbox}
  Цель может и должна отступать, даже если она двигалась в предшествующую Натиску очередь. 
\end{tcolorbox}

\subsubsection{Разбег}
Герой обрушивается на врага с воинственным криком и инерцией разгона.

\begin{itemize}
  \item Маневр расходует Перемещение;
  \item Герой должен преодолеть по прямой расстояние в интервале от своей \textbf{|Ск+1|} до своей \textbf{|Ск*2|}. Сделав это, он получает Преимущество на следующую проверку Дб, ММт, или проверку, использующую МСл;
  \item Все атаки по герою совершаются с Преимуществом до начала его следующей Очереди;
  \item Этот маневр полезен даже в небоевых ситуациях.
\end{itemize}


\section{Тяжелые травмы}
Не каждому удается пережить Боевую сцену. Еще меньше тех, кто пережил ее, сохранив в целости все части тела. Большинство после такого уходит на покой, но герой вполне может продолжить приключения. Конечно, без трудностей не обойдется, но героям не привыкать.
\paragraph{Однорукие герои:} однорукие герои не могут использовать Двуручное оружие. Если рука обрублена ниже локтя, герой может закрепить на ней щит или одноручное оружие.
\paragraph{Одноногие герои:} одноногий герой с костылем или протезом считается перемещающимся по Трудному ландшафту. Костыль может использоваться для атаки. Без костыля или протеза одноногий герой считается перемещающимся по Трудному ландшафту и ополовинивает свою Ск (т.е. суммарно его Ск сократится в 4 раза). Одноногие герои не могут использовать Громоздкое оружие.
\paragraph{Безногие герои:} безногие герои понижают свой МРз на 1. Ск безногих героев составляет |МЛв| метров. Если у безногого героя есть какое-то средство перемещения - например, тачка с колесиками - при Перемещении он двигается на число метров, равное |Атлетике (Лв) / 4| (минимум на 1 метр). Безногий герой считается сидящим. Поэтому он проверяет Дб с Помехой, а враги в своем Боевом контакте атакуют его с Преимуществом. Безногие герои не могут использовать Громоздкое и Длинное оружие.
\paragraph{Потеря глаз:} герой, лишившийся всех глаз, находится в состоянии Ослепления. Одноглазый герой проверяет Меткость с Помехой. 
\begin{tcolorbox}
  Одноногие герои с костылем или протезом и Чувством равновесия не испытывают никаких неудобств при движении и сражении, а безногие герои с этим Атрибутом все также опасны в ближнем бою.
\end{tcolorbox}

\section{Кавалерия в бою}
\tbd литературная вставка.
\paragraph{Атаки наездника:} он получает Преимущество на Дб, если атакует цели меньшего размера, чем его скакун.
\paragraph{Атаки по наезднику:} атакуя наездника в Боевом контакте, нападающий, не снаряженный Длинным оружием, получает Помеху на Дб, если МРз скакуна превышает его МРз. Возможна ситуация, когда из-за размеров скакуна наездник окажется вне его Боевого контакта.
\paragraph{Атаки по скакуну:} совершаются по обычным правилам.
\paragraph{Атаки скакуна:} во время Действия скакун может совершать любые маневры, доступные благодаря его Навыкам и снаряжению. Скакун не обязан заявлять такой же маневр, как и наездник. Исключение составляет Разбег.
\paragraph{Действие и Перемещение скакуна:} скакун Действует и Перемещается в ту же Очередь, что и наездник.
\paragraph{Проверки скакуна:} Проверки Дб, Мт и Вл скакуна могут быть заменены Обращением с животными наездника. Скакун использует собственные параметры, если они лучше.
\paragraph{Реакция наездника} равна \textbf{|([Рц скакуна]+[Рц героя])/2|}.
\paragraph{Лечение скакуна:} скакун Отдыхает в Антракте и Интерлюдиях по тем же правилам, что и герои, до тех пор, пока это не противоречит контексту (вряд ли конь или ездовое кобо смогут отдохнуть в кабаке или бане, хотя бывает всякое). Обращение с животнымиЭ заменяет МедицинуЭ для скакуна во всех отношениях.

\section{Дорожные войны}
\tbd литературная вставка.
\paragraph{Не дрова везешь!:} из-за тряски пассажиры ТС совершают Активные проверки с Помехой. 
\paragraph{Отвлекать водителя воспрещается:} водителю проблематично заниматься чем-то, кроме управления ТС. Все проверки водителя, не связанные с управлением ТС, совершаются с 2 Помехами. Водитель не может использовать Двуручное оружие.
\paragraph{Под откос:} если машина переворачивается или врезается в препятствие, все пассажиры (включая водителя) получают Дробящие Пв, равные \textbf{|30 - [Эксплуатация водителя] - БД|}. Если ТС не оснащено ремнями безопасности, или все их проигнорировали, удвойте полученные пассажирами Пв. 
\newline Те из пассажиров, кто не был пристегнут, могут избежать Повреждений, совершив проверку Атлетики (Лв) против \textbf{|Пв, которые герой должен получить|}.
\paragraph{Борт к борту:} водитель может использовать транспортное средство, как оружие ближнего боя. Дб водителя равна \textbf{|Эксплуатация(МЛв) + Прч ТС|}. При провале маневра транспортное средство получает Пв, равные величине провала. Это не всегда означает тараны и удары бортами - в случае мопедов, мотоциклов и малолитражек водитель заманивает противника на обочину или в кювет.
\paragraph{Погоняем?:} герой может оторваться от преследования, или наоборот, догнать кого-то. Он проверяет Эксплуатацию против \textbf{|10 + [Эксплуатация(Лв) противника] + [Опасность местности]|}. В случае успеха герой добивается желаемого. При любом исходе ТС потребуется проверка Износа в конце Сцены.
