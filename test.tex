% !TEX options=--shell-escape
\documentclass[a4paper]{book}
\setcounter{secnumdepth}{3}
\setcounter{tocdepth}{3}
\usepackage{cmap}
\usepackage[T2A]{fontenc}
\usepackage[utf8]{inputenc}
\usepackage[russian]{babel}
\usepackage[left=2cm,right=2cm,top=2cm,bottom=2cm,bindingoffset=0cm]{geometry}
\usepackage[poster]{tcolorbox}
\usepackage{longtable}
\usepackage{wrapfig}
\usepackage{multicol}
\usepackage{multirow}
\usepackage{vwcol}
\usepackage{imakeidx}
\indexsetup{level=\subsection,toclevel=section,noclearpage}
\makeindex[title=Алфавитный перечень Трюков, name=tricks]
\makeindex[title=Алфавитный перечень Оружия, name=weapons]
\makeindex[title=Алфавитный перечень Атрибутов, name=attributes]
\makeindex[title=, name=powers]
\makeindex[title=Алфавитный перечень Существ, name=monsters]
\makeindex[title=Алфавитный перечень Архетипов, name=monster-templates]
\usepackage{pythontex}
\usepackage[hypertexnames=true]{hyperref}
\hypersetup{
       colorlinks,
       citecolor=black,
       filecolor=black,
       linkcolor=black,
       urlcolor=black,
}
\tolerance=1
\emergencystretch=\maxdimen
\hyphenpenalty=10000
\hbadness=10000
\newcommand{\trouble}[8]{
\begin{center}
\begin{tabular}{ |p{2.7cm}|p{12cm}| }
\hline
\textbf{Результат проверки Неприятностей} & \textbf{Последствия}
\\ \hline
19-20 & \textbf{#1. }#2
\\ \hline
13-18 & \textbf{#3. }#4
\\ \hline
7-12 & \textbf{#5. }#6
\\ \hline
1-6 & \textbf{#7. }#8
\\ \hline
\end{tabular}
\end{center}
}

\newcommand{\troubleControl}{герой должен совершить проверку \textit{Неприятностей под Контролем} }
\endinput
\troubleControl
\trouble
{}{}%no sweat name/description
{}{}%tough day name/description
{}{}%we have trouble name/description
{}{}%fiasco name/description

\newcommand{\tbd}{\textcolor{orange}{\textbf{\textit{TBD}}}}
\newcommand{\err}{\textcolor{red}{\textbf{\textit{ERR}}}}
\newcommand{\genAndGet}[3]{
%\pyc{import sys}
%\pyc{sys.stdout.reconfigure(encoding='utf-8')}
\pyc{baseName = "#1"}
\pyc{dataName = "#2"}
\pyc{form = "#3"}
\pyc{str = main(baseName,dataName,form)}
\py{str}
}
\newcommand{\earlyEnd}{
----> current stop <----
\end{document}
\endinput
}
\begin{document}
\pyc{from scripts.genFromYaml import main}
%-------------------------------------------------------------------------------------
%--  begin  --------------------------------------------------------------------------
%-------------------------------------------------------------------------------------
\subsection{Гужевой транспорт}
Животные способны преодолевать водные преграды, а некоторые - еще и летать!
\genAndGet{transport}{transport}{Животное}

\subsection{Наземный транспорт}
\genAndGet{transport}{transport}{Наземный}

\subsection{Водный транспорт}
Проходимость этого транспорта означает то, насколько глубокая у него посадка. Чем выше проходимость, в тем более мелких реках и ручьях может передвигаться транспорт.
\genAndGet{transport}{transport}{Водный}

\subsection{Воздушный транспорт}
Проходимость летающих транспортных средств учитывается только при взлете и посадке. В эту категорию так же входит транспорт, способный находиться в открытом косомсе, но не предназначенный для межпланетных и межзвездных перелетов.
\genAndGet{transport}{transport}{Воздушный}

\subsection{Исполинский транспорт}
Транспортные средства могут достигать поистине невероятных размеров. Обслуживать такой транспорт, а тем более владеть им не по карману даже самому обеспеченному герою. Но герои могут взять этот транспорт в аренду или получить в пользование от организации, на которую они в данный момент работают.
\newline Героям не нужно знать, сколько стоит, как тяжело обслуживается и насколько грузоподъемен корабль, на котором они отправляются в путешествие, для приключения это - лишние детали. В описании Исполинского транспорта есть только его Скорость и Проходимость - этого достаточно, для того чтобы определить длительность и событийное наполнение похода.
\newline Во время Остановок, Исполинский транспорт не учавствует в Сценах целиком, а является частью окружения, в котором действуют герои. А иногда Сцена может быть целиком внутри транспорта, на котором путешествуют герои.
\paragraph{Старинные Исполины}
\genAndGet{transport-gigantic}{transport-gigantic}{Старинный}
\paragraph{Современные Исполины}
\genAndGet{transport-gigantic}{transport-gigantic}{Современный}
% \paragraph{Фантастические Исполины}
% \genAndGet{transport-gigantic}{transport-gigantic}{Фантастический}
%-------------------------------------------------------------------------------------
%--  end  ----------------------------------------------------------------------------
%-------------------------------------------------------------------------------------
\end{document}
\endinput
