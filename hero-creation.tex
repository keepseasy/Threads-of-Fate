\chapter{Создание героя}
Здесь вы узнаете о том, как герои и статисты устроены с точки зрения механики, что дается им легко, а что вызывает затруднения, и как именно Судьба может направить их или помочь им.
\newline Что вам понадобится для создания героя? Прежде всего, мысленно представить его. Откуда он? Каков его характер? Что он любит, и что ненавидит? Есть ли у него друзья, враги или семья? В чем он хорош, а в чем - не очень? Сколько ему лет? Чем он зарабатывает на жизнь? И, конечно же, как его зовут?
\newline Ответив на вопросы, вы получите примерно абзац текста. Это - ваша подсказка. Какие-то детали непременно останутся за ее пределами, но о них вы сможете вспомнить позже - если захотите. В конце концов, большую часть образа раскрывают действия героя во время игры. 

\paragraph{У героя есть}
\begin{itemize}
\item[--] Основные и Вторичные характеристики, задающе пределы физических и ментальных возможностей.
\item[--] Атрибуты - важнейшие детали образа и таланты, определяющие род занятий.
\item[--] Трюки - ловкие приемы или качества, способные облегчить жизнь.
\item[--] Недостатки, изображающие слабости и страсти.
\item[--] Амплуа и Грани, определяющие отношения героя с миром и место в нем.
\item[--] Узы, показывающие внутренние убеждения героя.
\item[--] Навыки, отвечающие за степень подготовки к всевозможным жизненным ситуациям.
\item[--] Богатство, отражающее благосостояние.
\item[--] Ячейки для установки Имплантов (если в мире умеют изготавливать и устанавливать их). Начальное число ячеек равно \textbf{|3 + Модификатор размера|} героя.
\end{itemize}

\section{Характеристики героя}
Очки характеристик: создавая героя, вы распределяете 70 Очков характеристик по Основным характеристикам. Очки распределяются в любых сочетаниях. Например, вы можете распределить 16 очков в любую одну Характеристику, получив значение 16. Если вы распределите 16 Очков по двум Характеристикам, то можете получить значения 9 и 7, 10 и 6 и т. д.

\subsection{Начальная и максимальная величина Характеристик}
При создании героя ни одна из Характеристик не может превышать 18. Игрок не вправе поднять Характеристики выше этого предела, даже если у героя есть свободные Очки Характеристик или Очки опыта. В дальнейшем, ни одна из Характеристик героя не может превысить 20 за счет траты Очков опыта, но может превысить 20 при помощи стимуляторов, аугментаций, специальных способностей и т.д.
\newline Базовые значения Основных Характеристик героев могут быть абсолютно любыми - в пределах от 1 до 20.
\newline Мастер может уменьшать или увеличивать количество Очков характеристик. Следует, однако, помнить, что герои, созданные меньше, чем на 60 очков, не вполне приспособлены к приключениям. 70 Очков характеристик - оптимальный вариант, так как в этом случае придется выбрать, в чем герой будет хорош, а в чем - не очень.
\newline Для антагонистов и прочих значимых фигур мастер вправе использовать любое количество Очков характеристик, даже превышающее то, на которое созданы герои. Для существ под управлением мастера не действует ограничение на максимальную величину Характеристик (хотя его стоит держать в голове).

\subsection{Основные характеристики}
\paragraph{Сила (Сл):} показывает, насколько герой развит физически. От этой Характеристики зависит, какой вес может нести герой, насколько далеко и высоко он прыгает, и ущерб, который он причиняет оружием, использующим мускульную силу. Также параметр Силы влияет на то, какое оружие герой сможет успешно применять.

\paragraph{Ловкость (Лв):} отвечает за быстроту и координацию движений. Эта Характеристика так же важна для воина, как и Сила. Еще она пригодится герою, который собирается стрелять, карабкаться по руинам и водить багги.

\paragraph{Выносливость (Вн):} пригодится любому герою. Высокая Выносливость означает, что герой крепок телом, редко болеет и легко переносит ранения. Также от этой Характеристики зависит, в каких доспехах герой сможет эффективно действовать.

\paragraph{Интеллект (Ин):} помогает запоминать информацию, делать выводы, а также учиться на своих и чужих ошибках. Интеллект необходим любому герою, который желает иметь высокие параметры Навыков - именно он устанавливает предел их роста.
\newline Герой с, например, 2 Интеллектом не сможет распределить в любой Навык больше 2 Очков опыта. Это не помешает ему относительно связно выражать мысли, найти работу, завести семью и вообще наслаждаться жизнью. Особенно, если удастся закорешиться с действительно умными ребятами, которые растолкуют, что к чему. 

\paragraph{Мудрость (Мд):} включает в себя находчивость, наблюдательность, бытовое здравомыслие и глубинные инстинкты. Именно Мудрость поможет герою вовремя заметить опасность… или просто избежать ее.
\newline Благодаря пассивным проверкам Наблюдательности, Мудрость - весьма важная Характеристика. Нужна она и героям, которые желают действовать раньше остальных, так как Мудрость входит в состав Вторичной Характеристики "Реакция".

\paragraph{Обаяние (Об):} позволит наладить контакт с окружающими и понравиться им, не особенно усердствуя. Этот параметр незаменим для того, кто предпочитает действовать исподволь и добиваться своего без применения физического насилия. Помимо этого, Обаяние определяет, в каком ключе герой воспринимает окружающий мир. Высокое Обаяние - залог оптимизма!
\newline Обаяние никак не связано с привлекательностью, хотя обаятельные герои часто кажутся окружающим симпатичными. 

\begin{tcolorbox}
    Игромеханически за внешнюю привлекательность отвечают Атрибут "Красивый", а так же Трюки "Соблазнительный" и "Стильный". 
    \newline Если ваш герой не планирует трепать языком, располагать к себе статистов, использовать Функции и феномены - Обаяние последняя в списке значимых для него Характеристик. Тем, кто не в восторге от героя из-за плохих проверок Впечатления, всегда можно отстрелить уши, отрубить тестикулы или просто от души наварить в торец.
\end{tcolorbox}

\subsection{Модификаторы Характеристик (М)} в большинстве формул используется не полное значение Основных характеристик, а \textbf{Модификатор = | (Основная характеристика - 10) ÷ 2|}.
\begin{center}
\begin{tabular}{ |c|c| }
\hline
\textbf{Основная характеристика} & \textbf{Модификатор} \\ \hline
  1 & -5 \\ \hline
  2-3 & -4 \\ \hline
  4-5 & -3 \\ \hline
  6-7 & -2 \\ \hline
  8-9 & -1 \\ \hline
  10-11 & 0 \\ \hline
  12-13 & 1 \\ \hline
  14-15 & 2 \\ \hline
  16-17 & 3 \\ \hline
  18-19 & 4 \\ \hline
  20 & 5 \\ \hline
\end{tabular}
\end{center}
\subsection{Проверки Основных характеристик}
Проверка Основной характеристики является Проверка с Бонусом, равным модификатору Основной характеристики.

\subsection{Размеры существ}
Правила берут за эталон существо ростом от 150 до 210 см. Оно имеет Средний размер. Но в сюжетах найдется немало отклонений от этого стандарта - мутанты, роботы, киборги и \tbd.
\newline Размер не влияет на Основные характеристики напрямую, однако большие существа - легкая мишень для атак, а маленькие существа вынуждены использовать маленькое оружие, наносящее меньше ущерба.
\newline Существа занимают область в зависимости от размеров. Это не означает, что существо занимает область целиком (хотя бывает и такое). Несомненно одно - в ней существо может и будет мешать передвижению недругов.
\newline Высота в холке четвероногих существ зачастую меньше, чем рост существ соответствующего размера, указанный в таблице.

\paragraph{Каков Размер моего героя?} Игрок вправе начать приключение Крошечным, Маленьким или Большим героем. 

\paragraph{Модификатор размера (МРз):} наравне с Выносливостью определяет, насколько существо или объект восприимчивы к урону. Чем больше цель атаки, тем сложнее нанести ей серьезный вред без специального снаряжения.
\newline Размер, отличный от Среднего, выгоден в одних ситуациях и мешает в других. Небольшому существу легче прятаться и избегать ударов, а крупному - атаковать противника, используя массу и габариты. 
\paragraph{На что влияет МРз?} МРз прибавляется к Скорости, результатам проверок Силы, Доблести, Меткости метания, и Навыков (Сл).
\newline МРз вычитается из БАЗщ, результатов проверок Меткости, Ловкости и Навыков (Лв).
\begin{tcolorbox}
    Не забывайте, минус на минус дает плюс. Это означает, что, например, Крошечное существо фактически повышает свою БАЗщ, Меткостьт, МЛв и навыки от Ловкости на 2.
    %Также обратите внимание, что Меткость Метания крупных существ не изменится, так как МРз одновременно и прибавляется к ней, и отнимается от нее.
\end{tcolorbox}

Некоторые способности учитывают размеры существ. В этом случае Большое существо считается как 2 Средних, Огромное - как 3 Средних, а Громадное - как 4!
\begin{center}
    \begin{tabular}{ |c|c|c|c|c| }
        \hline
        Размер & Модификатор размера & Возможный рост & Занимаемая область
        \\ \hline
        Крошечный(К) & -2 & 0.01-0.65 метра & 0.5 × 0.5 метра
        \\ \hline
        Маленький(М) & -1 & 0.66-1.49 метра & 1 × 1 метра
        \\ \hline
        Средний(С) & 0 & 1.5-2.1 метра & 2 × 2 метра
        \\ \hline
        Большой(Б) & +1 & 2.2-3 метра & 3 × 3 метра
        \\ \hline
        Огромный(О) & +2 & 3.01-9.99 метра & 5 × 5 метров
        \\ \hline
        Громадный(Г) & +3 & 10 метров и больше & 7 × 7 метров или даже больше
        \\ \hline
    \end{tabular}
\end{center}
\paragraph{Исполинские (И) существа и устройства.} Иногда существо или устройство немыслимо велики, а герои перед ними абсолютно незначительны. Исполинские существа не принимают участия в Сценах, они и есть Сцена, на которой разворачивается действие. Герои и существа, даже Громадные, могут взаимодействовать только с частью Исполина, а не с ним целиком.

\subsection{Вторичные характеристики}
\paragraph{Воля (Вл) = (Ин + Мд) ÷ 4.} Отвечает за самоконтроль, сопротивление всевозможным соблазнам и внешним влияниям.

\paragraph{Реакция (Рц) = (Лв + Мд) ÷ 4.} Определяет порядок действия в Боевых сценах и прочих ситуациях, в которых это важно.

\paragraph{Скорость (Ск) = (Лв + Вн) ÷ 4 + МРз.} За 5 секунд герой может преодолеть число метров (или клеток, если используется масштабная карта), равное своей Ск.
\paragraph{Ск и четвероногие существа:} четвероногие существа, такие, как кони, кошки и слоны, увеличивают свою Ск в 2 раза, перемещаясь по земле. Умножьте Ск после прибавления или вычитания МРз. Тараканы, пауки и многоножки считаются четвероногими для определения Ск, а гигантские слизни, улитки и змеи - нет!
\paragraph{Ск и [Полет]:} если существо способно летать, умножьте его Ск на 3 после прибавления или вычитания МРз. Если существо по каким-то причинам не может использовать [Полет], применяйте его обычную Ск.

\paragraph{Энергия (Эн) = (Вн + Об) ÷ 4.} Отражает абстрактный запас внутренних сил. Источником Энергии героя может служить оптимизм, ненависть, неукротимый дух, вездесущий эфир, миниатюрный ядерный реактор и множество других вещей и явлений, как философско-мистического, так и обыденного толка.
\newline Энергия расходуется, когда герой активирует феномен или Функцию Атрибута - невероятные способности, недоступные большинству окружающих. Текущее значение Энергии не может опуститься ниже 0, либо превысить ее максимальное значение.

\paragraph{Чтобы восстановить Энергию, герой должен:}
\begin{itemize}
    \item[--] Уйти в Антракт и восстановить Эн до максимального значения.
    \item[--] Воспользоваться Интерлюдией с пометкой "Отдых", чтобы восстановить \textbf{|МОб|} (минимум 1) Эн.
    \item[--] Употребить быстродействующие стимуляторы или зелья.
    \item[--] Преобразовать топливо в Энергию, если снаряжение или способности героя предоставляют такую возможность. Для того чтобы восстановить 1 Эн, следует потратить 100 Зарядов (Зр).
    \item[--] Применить Ход "Рука Судьбы", чтобы немедленно восстановить \textbf{|2 + МОб|} (минимум 1) Эн.
    \item[--] Трюки и Атрибуты также могут позволить герою восстанавливать Эн. 
\end{itemize}

\paragraph{Единицы Здоровья (ЕЗ) = |Вн*(МРз+3)|.} Показывают, сколько ущерба герой способен вынести, прежде чем потеряет сознание или умрет. Текущее значение ЕЗ не может опуститься ниже 0, либо превысить их максимальное значение.
\begin{tcolorbox}
  Единицы Здоровья являются условностью, позволяющей отслеживать боеспособность героя и его состояние в целом. Впрочем, эта условность достаточно правдоподобна, чтобы пользоваться ею в любых историях.
\end{tcolorbox}
Большинство существ величиной с человека имеют Средний размер. Их ЕЗ = \textbf{|Вн*3|}.
\newline Единицы Здоровья расходуются, когда герой получает Повреждения, или теряет Единицы Здоровья по каким-то причинам. 
\paragraph{Чтобы восстановить ЕЗ, герою придется:}
\begin{itemize}
    \item[--] Уйти в Антракт, чтобы восстановить \textbf{|МВн|} (Минимум 1) ЕЗ.
    \item[--] Использовать Интерлюдию "Посещение врача", "Расслабляющее ничегонеделание" или "Сон". Такая Интерлюдия может дополнять Антракт или сочетаться с ним.
    \item[--] Применить Навык Медицины во время Антракта.
    \item[--] Употребить быстродействующие стимуляторы или снадобья.
    \item[--] Использовать Атрибуты, Трюки, феномены и другие умения героя и его союзников.
\end{itemize}

У Вторичных Характеристик нет Модификаторов. Во всех формулах и при проверках используется их полное значение. В случае Энергии используется значение текущей Эн. Единицы Здоровья не используются для каких-либо проверок.
\newline Если при определении величины Вторичной характеристики результат деления получился меньше 1, то значение Вторичной характеристики все равно составит 1. 

\subsection{Проверки Вторичных Характеристик}
Проверка Вторичных характеристики является Проверка с Бонусом, равным значению Вторичной характеристики.

\subsection{Боевые характеристики}
\paragraph{Повреждения (Пв):} результатом успешных проверок Боевых Характеристик (за исключением Защиты) является получение Повреждений целью атаки.
\paragraph{Бонус к Повреждениям (БПв)} дается оружием и особыми способностями. Чем он выше, тем больше шансов Повредить цель. Бонус к Повреждениям является частью Боевых Характеристик Доблести и Меткости, и изменяется в зависимости от того, какое оружие или феномен использует герой.
\paragraph{Естественное оружие и Безоружный Бонус к Повреждениям (ББПв):} отражают ущерб, который герой наносит, используя свое тело - кулаки, пятки, хвост, клыки или рога. ББПв = \textbf{|МРз - 2|}, и может иметь отрицательное значение. Например, у героя Среднего размера он составит -2.
\paragraph{Доблесть (Дб) = |Нв Владения оружием + МСл + МЛв + БПв + МРз|.} Чем выше Доблесть героя, тем он опаснее в ближнем бою.
\paragraph{Рукопашная Доблесть (РДб) = |Нв Рукопашного боя + МСл + МЛв + ББПв + МРз|.} Эта Характеристика используется героем для нанесения ударов ногами, руками, клыками, когтями и другими частями тела, а также для совершения некоторых маневров.
\newline РДб является частным случаем проверки Доблести.
\newline Если правила предписывают проверить Доблесть, примените БПв оружия, которое использует герой.
\paragraph{Меткость (Мт) = |Нв Стрельбы + МЛв + БПв - МРз|.} Чем выше Меткость, тем опаснее герой в дистанционном бою. 
\paragraph{Меткость Метания (ММт) =|Нв Стрельбы + МСл + МЛв + БПв + МРз|.} Она используется при метании предметов, гранат и использовании Метательного оружия (луков и пращей в том числе). 
\paragraph{Меткость Снарядов (МтС) = |Нв Стрельбы + МФх (модификатор Феноменальной характеристики) + БПв - МРз|.} Она используется при активации феноменов - невероятных и сверхъестественных способностей.
\newline ММт и МтС являются частными случаями проверки Меткости. Если правила предписывают проверить Меткость, примените БПв оружия, которое использует герой.
\begin{tcolorbox}
	Обратите внимание, что для Меткости Метания Модификатор Размера добавляется так же, как и для Доблести, а не отнимается, как в обычной Меткости.
\end{tcolorbox}
\paragraph{Защита (Зщ) = |БАЗщ + МЛв + БД + БЩ|.} Чем выше Защита, тем сложнее поразить героя атаками. Помимо МЛв в Защиту входят:
\paragraph{Базовая защита (БАЗщ) = |10 - МРз|.} Она отражает Защиту неподвижного существа. Высокая БАЗщ обычно указывает на небольшой размер или иные факторы, затрудняющие попадание, но не связанные с подвижностью или броней. 
\begin{center}
    \begin{tabular}{ |c|c|c|c|c| }
        \hline
        Размер существа & Базовая защита
        \\ \hline
        Крошечный(К) & 12
        \\ \hline
        Маленький(М) & 11
        \\ \hline
        Средний(С) & 10
        \\ \hline
        Большой(Б) & 9
        \\ \hline
        Огромный(О) & 8
        \\ \hline
        Громадный(Г) & 7
        \\ \hline
    \end{tabular}
\end{center}

\paragraph{Бонус доспеха к Защите (БД):} защита, которую дают доспехи. Бонус доспеха изменяется в зависимости от того, какой доспех носит герой.
\paragraph{Бонус щита к Защите (БЩ):} защита, которую дают щиты. Бонус щита изменяется в зависимости от того, какой щит использует герой.
\newline Ношение некоторых доспехов и щитов ограничивает максимальный модификатор Ловкости, который можно использовать при подсчете Защиты и прочих проверках.
\paragraph{Прочность (Прч):} отражает особенности строения или конструкции существ и объектов, позволяющие противостоять ущербу. Прочность предотвращает полученные Повреждения и вычитается из них.
\newline Прочность не является Боевой Характеристикой, но тесно связана с ними.

\subsection{Проверки Боевых характеристик}
Проверка Боевой характеристики (кроме Защиты) является Проверка с Бонусом, равным значению Боевой характеристики.
\newline Герой может понизить значение боевой характеристики (но не повысить, если характеристика \textit{уже} отрицательная) вплоть до нуля во время атаки, если желает нанести удар не в полную силу.
\newline сложность проверки обычно равна Защите цели, которую выбрал герой.
\newline Подробнее о проверках Боевых Характеристик читайте в разделе "Маневры".

\paragraph{Проверки Защиты:} в Динамических сценах, не подразумевающих детализации - например, когда герой прорывается сквозь разъяренную толпу, бежит под обстрелом или мчится по коридору с ловушками, мастер вправе определить Повреждения героя при помощи проверки Защиты - Проверки с бонусом, равным \textbf{|Зщ - 10|}.

\subsection{Нулевой уровень характеристик}
В приключениях героев подстерегает немало опасностей, и далеко не все из них явные. Яды, болезни, зараженная вода и пища могут понизить значение любой Характеристики до нуля. Понижение Основной Характеристики, даже временное, приводит к понижению зависящей от нее Вторичной.
\newline Нулевой уровень Характеристики означает, что существо автоматически проваливает любые проверки, связанные с ней. Упавшие до нуля Сила, Ловкость и Скорость приводят к состоянию, близкому к параличу (хотя существо все слышит и может наблюдать за происходящим вокруг). Упавшие до нуля Интеллект, Мудрость и Обаяние приводят к коме. Если Реакция или Скорость существа падает до 0, оно впадает в ступор. Во всех этих случаях существо считается находящимся в состоянии Неподвижности. Существо с нулевой Волей покорно выполняет любые отданные ему приказы, даже очевидно самоубийственные. Если существо получит несколько приказов, противоречащих друг другу, то оно постарается выполнить их, удовлетворив максимальное число требований. Если Выносливость существа падает до нуля, то оно немедленно умирает или разваливается на части.
\newline Исключение - Боевые Характеристики. Даже с отрицательными значениями Боевых Характеристик герой может пытаться атаковать противников, каким бы безнадежным это не казалось.


\section{После определения основных и вторичных характеристик}
\begin{itemize}
\item[--] Выберите для героя 2 Атрибута. Герой может отказаться от 1 Атрибута и получить 5 дополнительных Очков опыта. Если герой отказывается от 2 Атрибутов сразу, он получает 10 дополнительных Очков опыта (всего!).
\item[--] Выберите для героя 2 Трюка.
\item[--] Выберите для героя 0-1 Недостатка (на старте не рекомендуется повторять их у героев разных игроков).
\item[--] Выберите для героя 0-1 Грань (не рекомендуется повторять их у разных героев).
\item[--] Выберите для героя 0-1 Уз (на старте не рекомендуется повторять их у героев разных игроков).
\item[--] Распределите 10 Очков опыта по Навыкам героя.
\end{itemize}
\begin{tcolorbox}
Атрибуты и Трюки наделяют героя огромным количеством способностей. Если вы хотите отразить в игре процесс становления героев, а не авантюру уже состоявшихся искателей приключений, можно начать игру с одним Трюком и одним Атрибутом.
\end{tcolorbox}

\section{Навыки}
Значения Навыков отображают глубину познаний героя в различных областях. В скобках указана Характеристика, Модификатор которой прибавляется к Навыку при проверках, - она означает часть Навыка, которую формируют талант и природные склонности. К некоторым Навыкам могут прибавляться Модификаторы различных Характеристик.
\paragraph{Значение} навыка равно числу Очков опыта, распределенных в Навык.
\paragraph{Максимальное значение Навыка} не может превышать значения Интеллекта героя.
\paragraph{Типы навыков:} Навыки делятся на две категории - Основные и Экспертные.
\paragraph{Основные Навыки} представляют собой виды деятельности, в которых можно достичь успеха, делая выводы из своих побед и неудач. Конечно, многие герои постигают их при поддержке наставника, но мастера-самоучки - не такая уж редкость.
\newline Проверка Основного навыка может быть совершена даже героем, совершенно не разбирающимся в предмете.
\paragraph{Экспертные Навыки} изображают те области знаний, в которых шанс добиться результата без планомерного обучения или исключительного дарования ничтожно мал. Доступ к Навыкам открывают Атрибуты, так как большинство их обладателей имеют специальную подготовку либо врожденный и осознанный ими дар.

\subsection{Проверки Навыков}
Когда герой проверяет Навык, используя Основную характеристику, он совершает Проверку с Бонусом равным \textbf{|[значение Навыка] + [Модификатор Основной Характеристики]|}
\newline Когда герой проверяет Навык, используя Вторичную характеристику, он совершает Проверку с Бонусом равным \textbf{|[значение Навыка] + [значение Вторичной Характеристики]|}
\newline Как правило, Характеристику, которую использует Навык, определяет мастер, но игрок может предложить заменить ее на другую, если у него появились идеи, соответствующие контексту.
\newline Если вы не используете правила Состязания, но герою кто-то активно противостоит сложность проверки равна \textbf{|[значение Навыка оппонента] + [значение Вторичной Характеристики оппонента / Модификатор Основной Характеристики оппонента] + 10|}.

\paragraph{Перед проверкой} навыка определите:
\begin{itemize}
    \item[--] Уместно ли применение Навыка в контексте Сцены?
    \item[--] Может ли применение Навыка принести желаемый результат?
    \item[--] Какая Характеристика будет участвовать в применении Навыка и чем это обусловлено?
    \item[--] Требуются ли для применения Навыка какие-то дополнительные средства (инструменты, снаряжение, участие окружающих)?
    \item[--] Могут ли какие-то дополнительные средства облегчить применение Навыка и понизить Сложность проверки?
\end{itemize}

\paragraph{Альтернативное применение Навыков:} использование Навыков - творческий процесс. Зачастую, добиться желаемого можно множеством разных способов, особенно, если к этому располагает контекст. 

\paragraph{Нулевой уровень Основные Навыков:} если игрок не распределил в Навык героя хотя бы 1 Очко опыта, этот Навык проверяется с Помехой.
\paragraph{Нулевой уровень Экспертных Навыков:} если герою доступен Экспертных навык и игрок не распределил в него хотя бы 1 Очко опыта, этот Навык проверяется с Помехой. 
\newline Если герою недоступен Экспертный навык, он не может совершать соответстующие проверки, но он может использовать статичное значение этого навыка с двумя Помехами.
\paragraph{Нулевой уровень боевых Навыков:} если игрок не распределил хотя бы 1 Очко Опыта во Владение оружием, Рукопашный бой или Стрельбу героя, соответствующие проверки Доблести и Меткости совершаются с Помехой.

\paragraph{Проверка Навыка или проверка Характеристики?} Поле деятельности многих Навыков дублируют профильные Характеристики. Как определить, что выбрать для проверки? Как и во многих других случаях, рекомендуется обратиться к контексту и настроению игры. Обычно проверки Навыков требуются в областях, предполагающих минимальное знакомство с процессом. Более тривиальные задачи допускают проверки Характеристик. Разумеется, если у героя есть Навык, он может применить его в любом случае, значительно упрощая себе задачу.
\newline Герою хватит проверки Характеристики, чтобы:
\begin{itemize}
    \item[--] Вскарабкаться по узловатому стволу старого дерева (Ловкость);
    \item[--] Договориться с приятелем (Обаяние);
    \item[--] Пробежать трусцой пару километров (Выносливость);
    \item[--] Запомнить четверостишье (Интеллект);
    \item[--] Выкопать небольшую яму (Сила).
\end{itemize}
Герою потребуется Навык, чтобы:
\begin{itemize}
    \item[--] Вскарабкаться по отвесной скале или гладкому столбу (Атлетика (Ловкость));
    \item[--] Успокоить вооруженного дебошира (Общение (Обаяние));
    \item[--] Пробежать пару километров в полной боевой выкладке (Атлетика (Выносливость));
    \item[--] Запомнить содержание небольшой книги (Наблюдательность (Интеллект));
    \item[--] Выкопать длинную траншею или глубокую могилу (Атлетика (Сила)).
\end{itemize}

\subsection{Перечень основных навыков}
\genAndGet{skills}{skills}{Базовый}

\subsection{Перечень экспертных навыков}
\genAndGet{skills}{skills}{Экспертный}

\section{Атрибуты}
Атрибуты - важнейшие детали образа героя и во многом отвечают за то, как его воспринимают окружающие. 
\newline Впрочем, и герои, и статисты могут обойтись (и зачастую обходятся) без Атрибутов. Не каждый проповедник - пламенный оратор, не каждая красавица способна очаровывать окружающих, не каждый офицер отдает толковые приказы. Герой вполне может быть проповедником и занимать формальное место в иерархии культа, родиться миловидным, или получить офицерский чин, но при этом не иметь соответствующего Атрибута. Его приобретение будет означать, что на Проповедника снизошла благодать (или он наконец-то научился ладно излагать догматы и полоскать мозги), Красавица расцвела (или поняла, как использовать красоту в достижении целей), а Офицер закалился в боях и заслужил уважение подчиненных (или заставил себя бояться).
Атрибут дает герою следующие возможности:
\begin{itemize}
    \item[--] Набор Экспертных навыков, к которым герой получает доступ при приобритении Атрибута;
    \item[--] Набор свойств, действующих постоянно или вводимых в игру без участия Нитей или Энергии. Условия применения свойств указаны в их описании;
    \item[--] Функции, действующие за счет траты Энергии героя. Стоимость Функции в Эн и прочие условия применения указаны в скобках после названия;
    \item[--] Cнаряжениe с указанными свойствами и СП. Начальное снаряжение имеет полный набор расходников, т.е. транспортные средства заправлены, оружие заряжено и т.д. Начальное снаряжение может быть продано, потеряно, украдено, уничтожено;
    \item[--] Темную Сторону - ситуативный Недостатка, сопутствующий Атрибуту;
    \item[--] Уникального Хода, который входит в игру при обрыве Нитей или с помощью проверок Неприятностей. Стоимость Хода в Нитях и условия его применения, в том числе в Боевых сценах, указаны в скобках после названия.
\end{itemize}

\begin{tcolorbox}
    У Атрибута всегда есть Ход и хотя бы одно Свойство, но далеко не у каждого - набор Экспертных навыков, Функции или Начальное снаряжение. Выбирайте тот, число способностей которого позволит вам комфортно с ним взаимодействовать.
\end{tcolorbox}

\paragraph{Стоимость Начального снаряжения:} некоторое снаряжение имеет фиксированную сложность приобретения (СП), например, Офицерский значок (СП 30) и Энциклопедия (СП 20). Если в описании указано "СП Х и меньше", то герой получает любой предмет указанной категории с СП, равной или меньшей Х. Если в описании указано "суммарно Х СП", то сумма значений СП всех предметов, выбранных героем, не должна превышать Х. Прочее снаряжение герой выбирает и покупает самостоятельно, расходуя Богатство.
\newline Подробнее о СП читайте в главе "Богатство и снаряжение".
\paragraph{Свойства, Функции, Ходы и течение времени:} время, которое занимает применение свойства, Функции или Хода, определяется мастером. Иногда вполне допустимо позволить герою уладить свои дела в Интерлюдии или Антракте. Даже когда временные рамки обозначены, мастер вправе отступить от них, если это уместно в контексте ситуации.
\paragraph{Свойства и Ходы, зависящие от Модификаторов Характеристик} всегда могут быть использованы минимум 1 раз за указанный период или воздействовать минимум на 1 цель, даже если применяемый Модификатор нулевой или отрицательный. Например, Дипломат с 9 Об (МОб -1) 1 раз активировать свойство "Парламентер".
\paragraph{Восполнение способностей героев:} способности некоторых Атрибутов могут быть возобновлены без ухода в Антракт или использования Интерлюдий. Условия восполнения указаны в описании таких Атрибутов. Если условия требуют наличия Услуги, СП восполнения может возрасти.
%\paragraph{Антракт и способности героев:} пребывание в Антракте возобновляет запас всех способностей, число применений которых за игровую встречу ограничено - если контекст не противоречит этому. Это никогда не требует расхода каких-либо ресурсов. Способности, действие которых ограничено рамками игровой встречи, прекращают работать при уходе героя в Антракт.
\paragraph{Уникальный Ход} позволяет герою преуспевать в задачах, сложных или невозможных для героев с другими Атрибутами. Добиться успеха при совершении Хода помогут как своенравная Судьба, так и способности самого героя. 
\newline Некоторые из Ходов ориентированы на применение в бою. Это никоим образом не должно останавливать от использования их в быту, если у игроков и мастера возникла идея, соответствующая настроению игры. И напротив - мирные ходы наверняка найдут применение в Боевых сценах.
\paragraph{1 или более Нитей} Некоторые Уникальные ходы отдают стоимость в Нитях на откуп мастеру. Определяйте стоимость Хода в соответствии с контекстом и логикой ситуации. Помните, что Ходы из перечня "Повезло", позволяющие выкупить успех и Критический успех на любую проверку за 2 и 4 Нити, доступны абсолютно любому герою. Там, где герой без Атрибута может преуспеть при помощи покупки обычного успеха за 2 Нити (или при помощи обычной проверки), герой с применимым к ситуации Атрибутом обойдется 1 Нитью. Если герой без Атрибута может преуспеть, только купив Критический успех за 4 Нити, герой с применимым к ситуации Атрибутом уложится в 2-3 Нити. Успех, за который мастер потребует 5 Нитей, должен изображать нечто умопомрачительное - один из тех случаев, о котором и герой, и свидетели события будут вспоминать всю оставшуюся жизнь. Без сомнений, такой успех быстро обрастет слухами и домыслами, а со временем - легендами. Разумеется, без применимого к ситуации Атрибута такой успех попросту невозможен.
\paragraph{0 Нитей.} Условия некоторых Ходов могут снизить стоимость до 0 Нитей и меньше. В этом случае для успеха все еще требуется обрыв 1 Нити.
\paragraph{Уникальный ход без обрыва Нитей.} Герои и персоны достаточно компетентны, чтобы добиться эффекта Уникальных ходов без вмешательства Судьбы, а у статистов и вовсе нет выбора большую часть времени. В этом случае активация Хода требует проверки Неприятностей под контролем Навыков или Характеристик. 
\begin{tcolorbox}
    Обратите внимание, что если игрок использует Ход героя без обрыва Нитей, он не может использовать любые Ходы Судьбы для влияния на результат, хотя вправе пользоваться Функциями, Трюками, Успехами с Расплатой и т.д.
\end{tcolorbox}
\paragraph{Сложность Уникального Хода без обрыва Нитей.} Если сложность контрольной проверки не указана в описании Хода, ее задает мастер, ориентируясь на таблицу сложности задач. Не рекомендуется устанавливать ее выше |20| - обладающий атрибутом герой неплохо разбирается в том, что делает.
\paragraph{Уникальный Ход без обрыва Нитей и Расплата:} некоторые из Ходов требуют выбрать Расплату даже при провале. Это означает, что герой пожертвовал временем, силами и ресурсами, но в итоге его все равно постигла неудача. Да, случается и такое. Разумеется, Успех с Расплатой может использоваться при проверках, связанных с Уникальным Ходом.
\paragraph{Темная сторона:} за все приходится платить, и возможности Атрибутов - не исключение. Фактически, Темная сторона - это Недостаток в Атрибуте. 
\begin{tcolorbox}
Герой может взять несколько одинаковых Атрибутов, получая все их преимущества согласно описанию. При этом активация Уникального Хода не будет требовать меньшего числа Нитей.
\end{tcolorbox}
\paragraph{Атрибуты, придуманные игроками и замена Уникальных ходов}
По договоренности с мастером игрок может:
\begin{itemize}
    \item[--] Начать игру с Атрибутом, который придумал сам. Возможности Атрибута и Уникального Хода должны быть определены до начала игры.
    \item[--] Заменить Уникальный Ход одного Атрибута на Уникальный Ход другого. Например, игрок может заменить Уникальный Ход Технаря на Уникальный Ход Мусорщика или Гражданина убежища. Замена производится до приобретения Атрибута.
\end{itemize}

\subsection{Атрибуты Наследия}
Некоторые атрибуты отражают не только способности героя, но и его происхождение. Их герой получил от предков - так или иначе. Герой вправе иметь лишь один Атрибут Наследия и обычно не может приобрести его по ходу игры - это то, чему нельзя научиться, а только получить при рождении (или сборке). Если герой - полукровка, и может претендовать на несколько Наследий, ему придется выбрать одно из них, или же отказаться от всех, ступив на путь, совершенно отличный от его предназначения.
\begin{tcolorbox}
	Иногда, впрочем, допустимо приобретение Атрибута Наследия по ходу игры. Гены Биоконструкта просыпаются в прагматичном жителе убежища, путешественник Изменяется, укрывшись на ночь в таинственной пещере, а великого воина после ранения вживляют в корпус Боевого робота. Фантазируйте вместе - и вы придумаете интересное и правдоподобное объяснение случившемуся.
\end{tcolorbox}
\genAndGet{attributes}{attributes}{Наследие}

\subsection{Атрибуты Могущества}
Атрибуты, дающие герою доступ к мистическому дару, псионическим силам и прочим невероятным способностям, ошеломительными даже на фоне других невероятных способностей. Герой может приобретать несколько Атрибутов Могущества, что отражает знакомство с разными аспектами \textbf{Феноменов}.
\paragraph{Феноменальная характеристика (Фх).} Основная характеристика, с помощью которой герою активирует Феномены. Она указана в описании Атрибута, Трюка или Предмета, который используется для активации. Модификатор Феноменальной характеристики (МФх) используется в большинстве формул, описывающих Феномены. 
\newline Если у героя есть несколько источников Феноменальной характеристики, перед активацией Феномена игрок должен заявить, какой источник и соответствующая характеристика будет использоваться.
\begin{tcolorbox}
    Атрибуты Могущества не обязаны иметь мистическую подоплеку. Впрочем, способности проникать в древние информационные библиотеки, ощущать электромагнитные поля, считывать информацию с голографических иероглифов или подключаться к секретной спутниковой сети, иначе, как волшебством, не назовешь. Да и сами обладатели таких возможностей зачастую уверены в их сверхъестественном происхождении.
\end{tcolorbox}

\genAndGet{attributes}{attributes}{Могущество}

\subsection{Боевые Атрибуты}
Атрибуты, ориентированные на боевые столкновения. Они дают значительные преимущества в боевых ситуациях, однако в других сценах ими воспользоваться будет \textit{сложнее}.
\genAndGet{attributes}{attributes}{Боевой}

\subsection{Социальные Атрибуты}
Атрибуты, ориентированные на социальные взаимодействия. С их помощью можно заручиться поддержкой статистов, произвести хорошее Впечатление или даже избежать конфликта до его начала. Однако, когда присутствующие в сцене похватались за оружие, Социальные Атрибуты уже \textit{скорее всего} не помогут.
\genAndGet{attributes}{attributes}{Социальный}

\subsection{Вспомогательные Атрибуты}
Герои с этими Атрибутами прекрасно дополнят любую команду благодаря способностями, серьезно облегчающими жизнь - как им, так и их товарищам.
\begin{tcolorbox}
    В этой категории обретаются самые экзотические Атрибуты. Чем объяснить их способности, вам подскажут жанр и настроение истории, а еще, конечно же, ваши соигроки. Возможно, Двойник - разумная колония нанороботов, Перевертыш - невероятная мутация, а Паразит - биодрон-разведчик.  Хотя не исключено, что все совсем не так, верно? 
\end{tcolorbox}
\genAndGet{attributes}{attributes}{Вспомогательный}

\printindex[attributes]

\section{Трюки}
Трюки – уловки, умения и качества, помогающие герою. Они не так масштабны, как Атрибуты, но не стоит их недооценивать. В некоторых ситуациях Трюк способен не только облегчить жизнь, но и спасти ее. 
\newline Трюки не могут быть приобретены несколько раз, если в описании не указано обратного.
\newline Трюки, зависящие от Модификаторов Характеристик не могут использоваться, если требуемый Модификатор или сумма с ним нулевая или отрицательная. Исключения указаны в описаниях Трюков.

\subsection{Боевые Трюки}
Трюки, ориентированные на боевые столкновения. Они дают значительные преимущества в боевых ситуациях, однако в других сценах ими воспользоваться будет \textit{сложнее}.
\genAndGet{tricks}{tricks}{Боевой}

\subsection{Социальные Трюки}
Трюки, ориентированные на социальные взаимодействия. С их помощью можно заручиться поддержкой статистов, произвести хорошее Впечатление или даже избежать конфликта до его начала. Однако, когда присутствующие в сцене похватались за оружие, Социальные Трюки уже \textit{скорее всего} не помогут.
\genAndGet{tricks}{tricks}{Социальный}

\subsection{Трюки Могущества}
Трюки, влияющие на Феномены и Энергию героя. Лучше всего подходят для магов, псиоников или супергероев.
\genAndGet{tricks}{tricks}{Могущество}

\subsection{Вспомогательные Трюки}
Эти Трюки не имеют сильной специализации на тех или иных Сценах или же несут вспомогательную функцию, которая делает жизнь героя проще.
\genAndGet{tricks}{tricks}{Вспомогательный}

\printindex[tricks]
\section{Недостатки}
Недостатки не обязательно являются отрицательным чертами (с точки зрения героя так уж точно). И все же они способны осложнить жизнь героя, а то и повернуть ее под совершенно непредсказуемым углом. Герой начинает игру с 0-1 Недостатком.
\newline Недостатки могут вводиться в игру по-разному - не стесняйтесь импровизировать. Неряха может выдать местоположение засады характерным запахом, Привязанность невовремя захочет поиграть в дамочку в беде, а Чужак случайно использует оскорбительный жест, заказывая выпивку. Способы испортить жизнь герою ограничены только контекстом и вашей фантазией.
\paragraph{Недостатки, придуманные игроками - отличная идея!} Главное, помните - Недостаток должен осложнять герою жизнь, и быть достаточно широко применимым, иначе в нем нет смысла.
\paragraph{Замена и отказ от Недостатков:} один Недостаток можно сменить на другой, если игрок того желает и это обусловлено развитием характера героя. Например, Вспыльчивый герой, нагрубив не тому человеку, может стать Осторожным, а Любвеобильный, отыскавший ту самую, обзаведется Привязанностью. Игрок вправе и просто вычеркнуть Недостаток, если контекст располагает к этому.
\begin{tcolorbox}
    Помимо прочего, Атрибуты, Трюки и Недостатки призваны создавать образ героя широкими мазками. "Гордая, но слегка Застенчивая Красавица-Прогрессор с Честным лицом и Чистым генофондом" - вполне завершенный образ. Или "Офицер-Ветеран, Мастер защиты, Знаток оружия, Пьяница и Грубиян" - звучит неплохо, верно? А главное, каждое слово имеет под собой игромеханическую основу и будет так или иначе работать на игру.
\end{tcolorbox}
\subsection{Перечень Недостатков}
\genAndGet{tricks}{tricks-flaws}{Недостаток}
\section{Грани и амплуа}
Грани и Амплуа дадут вам множество сюжетных зацепок, расскажут о прошлом героя и о том, на что он надеется в будущем. Многие вопросы, которые неизбежно возникнут после определения Граней, намеренно оставлены без ответов — их предстоит найти игрокам и мастеру!
\begin{tcolorbox}
Фактически, Грани представляют собой глубоко нишевые Уникальные ходы и Недостатки. Они рассчитаны в первую очередь на долгую игру, в которой характер героя меняется и развивается — как и мир вокруг него. Конечно, вы можете использовать их в играх на одну встречу, чтобы добавить образу героя колорита или даже построить вокруг них завязку. При этом стоит держать в голове, что Грани найдут применение далеко не в каждой истории.
\end{tcolorbox}
\paragraph{Амплуа} описывает героя одной емкой фразой, и служит хорошим дополнением к образу, созданному Атрибутами, Трюками и Недостатками. Амплуа совсем не обязано соответствовать перечню Атрибутов. Например, Дверг может выбрать Амплуа Воина и Ремесленника, игнорируя Амплуа Нелюдя, а полуэльф может использовать Амплуа Нелюдя, даже если не приобрел Атрибут Эльфа. Амплуа служат ориентиром для выбора Граней героя и не выполняют никакой игромеханической функции.
\paragraph{Грани} — значимые факты биографии, точки напряжения истории, конфликты героя, внутренние или внешние, а вместе с тем — превосходный источник идей для портрета героя. Грани состоят из \textbf{Орла} — условно положительной стороны Грани (до «но» в описании), и \textbf{Решки} — потенциально негативной стороны Грани (после «но» в описании). Игроки могут придумывать Грани самостоятельно — это поможет обогатить портрет героя деталями, которые подчеркивают жанр и настроение вашей истории.
В отличие от Недостатков и Темной стороны Атрибутов, Грани
рассчитаны на развитие ситуации со временем. Также они могут
прямо подсказать игроку и мастеру, какие Неприятности преследуют героя, каковы его социальный статус, интересы и круг общения, кто ему друг, а кто — враг.
\paragraph{Число Граней:} игрок может определить случайным образом или выбрать число Граней, равное числу Атрибутов героя, из наиболее подходящих ему Амплуа. Для героя без Атрибутов игрок может выбрать одну Грань из любого Амплуа. Орел может быть введен в игру при помощи общедоступного Хода
«Повезло» и обрыва Нити (одной или нескольких на усмотрение мастера) в ситуациях, когда герой может извлечь из этого выгоду.
\paragraph{Решка} вводится в игру как Каприз Судьбы.
Например, если герой обладает Гранью из перечня Бродяги (у героя множество родственников, но никто из них не добился успеха в жизни), игрок может оборвать Нить, объявить, что герой встретил в незнакомом городе троюродного брата и получил кров, стол или информацию. В то же время Грань ясно указывает, что герой вряд ли получит слишком много. При этом игрок может осложнить жизнь герою — протянуть к нему Нить и столкнуть лицом к лицу с восторженным племянником-недорослем, который желает путешествовать вместе с ним!
\paragraph{Смена Амплуа:} мастер может позволить заменить одно Амплуа на другое, если игрок того желает и это обусловлено логикой развития истории. Например, разочаровавшийся в идеалах Паладин может стать Бродягой, а Воин, открывший свое дело, превратится в Дельца. Смена Амплуа не вынуждает героя менять и Грани. Смену Граней, так же, как и смену Недостатков, должно определять развитие характера и образа героя. Разумеется, игрок может и вовсе отказаться от Граней героя, если сочтет нужным.
\newline
\genAndGet{roles}{roles}
\section{Узы}
\paragraph{Узы:} Узы представляют собой утверждения, которыми должен руководствоваться игрок, выбирая линию поведения героя. Это внешние и внутренние обстоятельства, мировоззрение и привычки, ощутимо влияющие на решения героя и определяющие выбор при прочих равных.
\paragraph{Число Уз:} игрок может выбрать для своего героя 0—2 Уз самостоятельно или определить их случайным образом. Когда герой связан Узами, протяните к герою 1 дополнительную Нить Судьбы в начале игровой встречи или сюжетной вехи за каждые Узы героя.
\paragraph{Темные Нити:} если герой совершает действия, противоречащие выбранным Узам, игрок должен оборвать 1 Нить или передать в руки мастера 1 Темную Нить.
\newline
Мастер может использовать Темные Нити в отношении статистов и персон так же, как используются Нити Судьбы в отношении героев, но только в тех случаях, когда статист или персона противостоят героям. Это единственный случай, когда статист может применить Ход Судьбы без проверок и последствий для себя!
\newline
Мастер может обрывать Темные Нити для ввода в игру Капризов Судьбы более 1 раза за сцену. Тем не менее, мастер все еще не может вводить один и тот же Недостаток, Темную сторону и Решку героя больше 1 раза за сцену. Например, если в распоряжении мастера есть 2 Темных Нити, он может 1 раз столкнуть героя с последствиями Недостатка бесплатно, 1 раз ввести в игру Темную сторону его Атрибута и 1 раз ввести в игру Решку героя, но не может ввести в игру один и тот же Недостаток героя дважды. В распоряжении мастера одновременно может находиться число Темных Нитей, равное \textbf{|1 + число игроков|}.
\paragraph{Смена Уз и отказ от них:} игрок может заменить одни Узы на другие или освободить героя от Уз, если это обусловлено логикой развития истории. Не считая специально оговоренных случаев, смена и отказ от Уз происходит в начале игровой встречи.
\paragraph{Выбор Уз и контекст:} Узы очень зависимы от логики происходящего. Если по какой-то причине в течение игровой встречи герой не может даже теоретически столкнуться с затруднениями и ограничениями, вызванными Узами, мастер начинает встречу с 1 Темной Нитью за каждые Узы, которые не будут задействованы.
\newline
Выбор Уз — это не только дополнительная Нить, но и способ сделать характер героя более выпуклым. Если вы задумаетесь над тем, почему герой связан именно этими Узами, то узнаете множество любопытных фактов о его личности. Например, если герой связан Узами единства, он начинает игру с 1 дополнительной Нитью, но должен оборвать 1 Нить или передать мастеру 1 Темную Нить, если его действия могут явно навредить героям остальных игроков. Почему командная работа так важна для него? Быть может, герой — ветеран, знающий, к чему может привести разлад на поле битвы, или в его большой семье все привыкли помогать друг другу. Не исключено даже, что герой — участник преступного синдиката, могущество которого держится на круговой поруке.
\newline
Чтобы определить Узы случайным образом, выберите столбец и бросьте К20.
\begin{center}
\begin{tabular}{ |c|p{7cm}|c|p{7cm}| }
\hline
K20 & \textbf{Узы} & K20 & \textbf{Узы} \\ \hline
1 & Узы единства & 1 & Лишний шум - лишние проблемы \\ \hline
2 & Любовь выдумали поэты и менестрели & 2 & Первый удар - решающий \\ \hline
3 & Честь превыше всего & 3 & Семь раз отмерь, один раз отрежь \\ \hline
4 & Сумел нынче убежать - завтра будешь воевать & 4 & Насилием нельзя изменить мир - только изуродовать \\ \hline
5 & Принимай чужеземцев с радушием & 5 & Слово - лучшее оружие \\ \hline
6 & После нас - хоть потоп & 6 & Цель оправдывает средства \\ \hline
7 & Простолюдины хуже зверей & 7 & Один раз живем \\ \hline
8 & Умеренность и аккуратность & 8 & Закон не ошибается \\ \hline
9 & Вещи не предадут и не обманут & 9 & Знание - сила \\ \hline
10 & Счастья за деньги не купишь & 10 & Старшим виднее \\ \hline
11 & Сделал дело - гуляй смело & 11 & Женщина священна \\ \hline
12 & От стариков никакой пользы & 12 & От чужаков добра не жди \\ \hline
13 & Каждая жизнь бесценна & 13 & Высокий род не оправдывает высокомерия \\ \hline
14 & Сомнения - удел слабаков & 14 & Жестокость внушает уважение \\ \hline
15 & Терпение украшает & 15 & Любовь - величайшее из чудес \\ \hline
16 & Ты - мне, я - тебе & 16 & Предательство постыдно \\ \hline
17 & Мужчина во всем главный & 17 & Никогда не сдавайся \\ \hline
18 & Око за око & 18 & Деньги могут все \\ \hline
19 & Приметы не лгут & 19 & Вовремя предать - значит предвидеть \\ \hline
20 & Я заслуживаю самого лучшего & 20 & Вещи обременяют \\ \hline
\end{tabular}
\end{center}
\paragraph{Узы единства} объясняют, почему герои, порой очень поверхностно знакомые друг с другом, работают в команде и не выступают друг против друга открыто или тайно, даже когда есть возможность извлечь из предательства серьезную выгоду. Узы единства не обязаны связывать всех героев в команде. Иногда один герой искренне радеет за общее дело, а другой ждет удобного момента, чтобы перерезать ему глотку! С другой стороны, в не нацеленной на внутренние конфликты команде Узы единства — хороший способ начать игру с дополнительной Нитью. Узы единства также распространяются и на статистов, которые важны для успеха дела.
\begin{center}
\begin{tabular}{ |c|p{4cm}|p{10cm}| }
\hline
\textbf{К20} & \textbf{Узы единства} & \textbf{Мотив} \\ \hline
1 & Один в поле не воин & Боязнь остаться один на один с некоей угрозой удерживает героя от необдуманных действий \\ \hline
2 & С приказами не спорят & Герой получил приказ, запрещающий ему вредить остальнму, по крайней мере, пока дело не завершено \\ \hline
3 & Лучше делить на всех, чем лишиться всего & Герой не уверен в своих силах и предпочитает получить меньше, но наверняка \\ \hline
4 & Дружба нерушима & Герой испытывает искреннюю симпатию к своим спутникам и считает их друзьями \\ \hline
5 & Без меня они пропадут & Герой чувствует ответственность за своих спутников, хоть и относится к ним слегка снисходительно \\ \hline
6 & Сначала дело, потом - разборки & Герой привык разделять деловые интересы и личную неприязнь. Это не мешает ему считать своих спутников ничтожествами, хотя говорить об этом вслух он вряд ли сочтет разумным \\ \hline
7 & Свары - для любителей & Герой считает себя профессионалом и не позволяет эмоциям взять верх над здравым смыслом. Впрочем, он не будет хвататься за оружие, если кто-то из его спутников действительно \textit{заслужил} хорошую взбучку \\ \hline
8 & Боги ненавидят предателей & Герой убежден, что боги покарают его за предательство \\ \hline
9 & Риск слишком велик & Герой рад бы обогатиться за чужой счет, но боится огласки и преследования властей \\ \hline
10 & Я выше этого & Герой не видет смысла в усобицах. Возможно, он в тайне гордится этим \\ \hline
11 & Для дела важен каждый & Герой уверен, что натянутые отношения с кем-то из спутников (не говоря уж о гибели), серьезно понизят общие шансы на успех \\ \hline
12 & За смирение мне воздастся & Герой верит, что Судьба воздаст ему за терпение по отношению к спутникам, как и за нежелание идти против них \\ \hline
13 & Ценные связи на будущее & Герой стремится сохранить со спутниками хорошие отношения из соображений будущей выгоды \\ \hline
14 & Я здесь не ради выгоды & Успех предприятия на первом месте для героя. Он скорее пожертвует собственной выгодой, чем пойдет на конфликт \\ \hline
15 & Доверие - основа успеха & Герой не представляет командной работы без взаимопомощи. Он полностью доверяет спутникам и ждет от них того же \\ \hline
16 & Честь дороже выгоды & Герой считает предательство бесчестным делом \\ \hline
17 & Я зла не делаю и не помню & Герой редко раздражается и быстро отходит. Причинить спутнику зло, пусть и ради выгоды, кажется ему чудовищным \\ \hline
18 & Просто немыслимо & Герой никогда не задумывался о мести и предательстве в принципе \\ \hline
19 & Мы - команда & Герой считает себя частью единого целого, команды, внутри которой каждый занимается своим делом и получает то, что должно \\ \hline
20 & Я - пример для окружающих & Герой считает себя примером добродетели и печется о своей репутации \\ \hline
\end{tabular}
\end{center}

\section{Языки}
В некоторых сюжетах герои встречают иноземцев или попадают в далекие страны. В таких случаях у героев может возникнуть желание (или даже насущная необходимость) выучить иностранный язык.
\newline В начале игры герой знает родной язык достаточно хорошо, чтобы говорить и писать на нем (если только у игрока нет других идей на этот счет, а у героя - соответствующих Недостатков). В дополнение герой может овладеть числом языков, равным своему |МИн|. Каждый язык сверх этого стоит 1 Очко опыта. Обычно герой говорит на иноземном языке с акцентом, очевидном для аборигенов. 
\begin{tcolorbox}
    Время изучения героями иностранных языков зависит от темпа и жанра истории. Как правило, достаточно четырех-пяти месяцев практики, чтобы овладеть языком на разговорном уровне. С другой стороны, герои вполне могут учиться во время долгих Путешествий, или выделять для занятий пару Интерлюдий в день.
\end{tcolorbox}

\section{Начисление очков опыта}
Герой немедленно получает 1 Очко опыта, если он принял влияние другого героя или статиста (подробнее об этом читайте в части "Социальные взаимодействия").
% Герой немедленно получает 1 Очко опыта, если он:
% \begin{itemize}
% \item[--] Принял Влияние другого героя, если на вашей игре вы не используете Каприз "Чуждое влияние" при разрешении конфликтов героев.
% \item[--] Достиг Общей цели.
% \item[--] Достиг Личной Цели.
% \end{itemize}
%Выбор новой Цели, возможен, если к этому располагает контекст.

Герой получает 1 Очко опыта в конце игровой встречи за каждое
выполненное условие. Многократное выполнение одного и того же
условия не приносит герою дополнительных Очков опыта.
\begin{itemize}
\item[--] Герой столкнулся с проблемой, вызванной Капризом Судьбы.
\item[--] Герой использовал Ход Судьбы.
\item[--] Герой использовал Уникальный ход своего Атрибута.
\item[--] Герой использовал достоинства своего Атрибута.
\item[--] Герой использовал свой Трюк.
\end{itemize}
%\subsection{Цели}
\paragraph{}
При помощи таблиц вы можете определить Цели, встающие перед героями в начале игровой встречи. Выбранная Цель может стать как сюжетным стержнем вашей истории, так и побочной (но от этого не менее интересной) линией.
\paragraph{}
Игрок может выбрать или определить случайным образом 1 Цель своего героя в начале игровой встречи. Если герою удается успешно выполнить ее, он получает 1 дополнительное Очко опыта и может тут же взять новую Цель. Выполнение Цели может занять какое-то время, не исключено, что для этого понадобится несколько игровых встреч. Обратите внимание, что Цель предлагает лишь абстрактную идею. Конкретное наполнение приключения зависит от контекста событий, происходящих в вашей истории.
\paragraph{}
Цель может быть не только личной, но и общей. В этом случае она служит сюжетным стержнем, и Очки опыта за ее выполнение начисляются всем героям, участвующим в истории. В некоторых случаях потребуется выяснить, в каких отношениях Цель находится с героями и с кем именно — например, если выпал вариант «Враг» или «Возлюбленный».
\paragraph{}
Число Целей, которых герой пытается достичь одновременно (и за выполнение которых получает Очки опыта), не может превышать его \textbf{|МОб|} (минимум 1).
\paragraph{Выбор Цели:} в начале киньте К20 и Выберите Цель. Затем киньте К20 и Конкретизируйте Цель. После этого киньте К20 и узнайте, что герой должен сделать с Целью. В заключение киньте К20 и определите, из-за чего достижение Цели под угрозой.
\begin{center}
\begin{tabular}{ |c|c|c|c|c|c| }
\hline
\textbf{К20} & 1-4(Информация) & 5-8(Предмет) & 9-12(Разумное существо) & 13-16(Животное) & 17-20(Место) \\ \hline
1-2 & Опасная & Деньги & Друг & Злобное & Руины \\ \hline
3-4 & Бессмысленная & Зелье & Враг & Упрямое & Холм \\ \hline
5-6 & Удивительная & Ценные бумаги & Возлюбленный & Тупое & Дом \\ \hline
7-8 & Не выглядит важной & Механизм & Родственник & Смирное & Озеро \\ \hline
9-10 & Зашифрованная & Оружие & Соперник & Любопытное & Роща \\ \hline
11-12 & Полезная & Драгоценность & Незнакомец & Егозливое & Подземелье \\ \hline
13-14 & Пугает & Доспех & Важная персона & Опасное & Лес \\ \hline
15-16 & Несерьезная & Картина & Чужеземец & Отвратительное & Лаборатория \\ \hline
17-18 & Ценная & Изваяние & Нелюдь & Ядовитое & Болото \\ \hline
19-20 & {Объемная} & {Артефакт} & Кумир & Очень странное & Замок \\ \hline
\end{tabular}
\end{center}


\begin{center}
\begin{tabular}{ |c|c|c|c|c|c| }
\hline
\multicolumn{6}{|c|}{\textbf{Что герой должен сделать с Целью, если она...}} \\ \hline
\textbf{К20}& \textbf{Информация} & \textbf{Предмет} & \textbf{Разумное существо} & \textbf{Животное} & \textbf{Место} \\ \hline

1 & Уточнить & Расколдовать & Расколдовать & Расколдовать & Расколдовать \\ \hline
2 & Восстановить & Восстановить & Исцелить & Исцелить & Восстановить \\ \hline
3 & Отыскать & Отыскать & Отыскать & Отыскать & Отыскать \\ \hline
4 & Скопировать & Скопировать & Наказать & Перевезти & Очистить \\ \hline
5 & Исказить & Вернуть & Вернуть & Вернуть & Осквернить \\ \hline
6 & Передать & Передать & Изобличить & Передать & Укрепить \\ \hline
7 & Скрыть & Укрыть & Укрыть & Укрыть & Подготовить \\ \hline
8 & Выкупить & Выкупить & Выкупить & Выкупить & Выкупить \\ \hline
9 & Продать & Продать & Продать & Продать & Продать \\ \hline
10 & Выкрасть & Выкрасть & Выкрасть & Выкрасть & Освятить \\ \hline
11 & Дополнить & Спрятать & Спрятать & Спрятать & Занять \\ \hline
12 & Проверить & Подменить & Обокрасть & Подменить & Скомпрометировать \\ \hline
13 & Дискредитировать & Изготовить & Уговорить & Спарить & Спасти \\ \hline
14 & Распространить & Применить & Освободить & Освободить & Обследовать \\ \hline
15 & Получить & Получить & Отблагодарить & Передержать & Отдать \\ \hline
16 & Изучить & Изучить & Соблазнить & Изучить & Изучить \\ \hline
17 & Уничтожить & Уничтожить & Убить & Убить & Уничтожить \\ \hline
18 & Захватить & Захватить & Захватить & Захватить & Захватить \\ \hline
19 & Опровергнуть & Подбросить & Опорочить & Подбросить & Отбить \\ \hline
20 & Защитить & Защитить & Защитить & Защитить & Защитить \\ \hline
\end{tabular}
\end{center}

\begin{center}
\begin{tabular}{ |c|c| }
\hline
\textbf{К20} & \textbf{Достижение Цели под угрозой из-за…} \\ \hline
1 & Преступного синдиката \\ \hline
2 & Старых врагов \\ \hline
3 & Бандитов \\ \hline
4 & Капризов природы \\ \hline
5 & Чиновников \\ \hline
6 & Сил правопорядка \\ \hline
7 & Соперников \\ \hline
8 & Нелюдей \\ \hline
9 & Иноземцев \\ \hline
10 & Неведомых врагов \\ \hline
11 & Родственников \\ \hline
12 & Друзей \\ \hline
13 & Любви \\ \hline
14 & Конкурентов \\ \hline
15 & Важной персоны \\ \hline
16 & Высокородного \\ \hline
17 & Безумца \\ \hline
18 & Животных \\ \hline
19 & Чудовища \\ \hline
20 & Сверхъестественных сил \\ \hline
\end{tabular}
\end{center}

\section{Развитие героя}
Герой может потратить Очки опыта следующим образом:
\begin{itemize}
\item[--] Повысить значение Навыка на 1, потратив 1 Очко опыта.
\item[--] Повысить Богатство на 1, потратив 1 Очко опыта.
\item[--] Изучить новый язык, потратив 1 Очко опыта.
\item[--] Повысить на 1 максимальные Единицы Здоровья, потратив 2 Очка опыта.
\item[--] Повысить на 1 максимальную Энергию, потратив 3 Очка опыта.
\item[--] Изучить Трюк, потратив 5 Очков опыта.
\item[--] Повысить на 1 Основную характеристику, потратив 5 Очков опыта.
\item[--] Получить Атрибут, потратив 10 Очков опыта.
\item[--] Изучить Трюк, потратив 5 Очков опыта.
%\item[--] Повысить на 1 Вторичную характеристику, потратив 5 Очков опыта.
%\item[--] Изучить новый Феномен, потратив 2 Очка опыта.
Повышение Основных характеристик приводит к повышению Вторичных. Например, если герой приобрел 1 единицу Выносливости, то его ЕЗ также вырастут.
\end{itemize}
\paragraph{} Новые способности не могут взяться из ниоткуда. Чтобы отобразить развитие героя, есть два способа:
\begin{enumerate}
    \item Вы можете исходить из уже существующих внутриигровых фактов. Например, прежде чем приобрести Атрибут "Аристократ", герой проявил себя перед государем и заслужил титул, получил его путем вероломных интриг... или же просто купил за внушительную сумму денег. Если герой увеличивает Богатство или Владение оружием, то перед этим он разумно распоряжался своими средствами и принимал участие в битвах.
    \item Вы можете создать факт, договорившись с мастером и соигроками о логичном внутриигровом обосновании приобретения вашего героя. Например, в случае Атрибута "Аристократ" герой может оказаться потерянным наследником древнего рода, повышение Богатства произошло за счет выплаты банковских процентов или выигрыша на скачках, а Владение оружием улучшилось благодаря тренировкам, на которые герой тратил свободное время между игровыми встречами.
\end{enumerate}