\chapter{Социальные взаимодействия}
\paragraph{}
Здесь вы узнаете, как герои и статисты общаются друг с другом, и чего они могут достичь, не прибегая к бескомпромиссному насилию.
\section{Впечатления}
\paragraph{}
Оружием герою может послужить не только верный меч или могучие чары, но и вовремя сказанное слово или лучезарная улыбка. Навык Общения предполагает множество способов получить желаемое, в том числе без участия Обаяния. Однако именно Обаяние отвечает за первое впечатление, которое герой производит на окружающих. Когда герои попадают туда, где о них еще не успели составить определенного мнения, игроки кидают К20 и прибавляют к нему МОб (а также прочие бонусы и штрафы, которые мастер сочтет уместными). Сверьте полученный результат со списком:
\paragraph{0 и меньше — Отвращение:} статисты напрочь отказываются иметь с героем дело. Вышибалы гонят героя из кабака (по крайней мере, пытаются), обыватели бранятся и швыряют в него камнями и мусором. Нападение со стороны вооруженных статистов — вопрос пары фраз, иным хватит и недоброго взгляда. Ополченцы могут не пустить героя в город!
\newline
\textbf{Торговля} невозможна. Лавочники отказываются торговать и угрожают вызвать стражу.
\newline
\textbf{Просьбы о помощи} отвергаются с негодованием, \textbf{расспросы} приводят статистов в ярость.
\newline
\textbf{В потенциально боевой ситуации} статисты атакуют героя и бьются до конца. Если сражение по каким-то причинам невозможно или бессмысленно (герой очевидно сильнее, сражение приведет к неблагоприятным для статистов последствиям), статисты действуют против героя всеми доступными им средствами.
\paragraph{1—3 — Враждебность:} статисты с трудом выносят присутствие героя. В каьаке герой, скорее всего, станет причиной (и основной жертвой) драки. Обыватели и не думают скрывать своего отношения к герою, но воздержатся от неприкрытой травли, если он не даст для того повода. Впрочем, вооруженные статисты или городское ополчение легко найдут повод для стычки!
\newline
\textbf{Торговля} разорительна. Лавочники заламывают драконовские цены, а покупают за сущие гроши (умножьте цену покупки у статиста на 3, а цену, за которую статисты готовы что-то купить у героя, разделите на 3).
\newline
\textbf{Просьбы о помощи} отвергаются, \textbf{расспросы} встречаются в штыки.
\newline
\textbf{В потенциально боевой ситуации} статисты атакуют героя и отступают лишь в случае очевидного перевеса на его стороне. Если сражение по каким-то причинам невозможно или бессмысленно, статисты действуют против героя иными доступными методами.
\paragraph{4—6 — Остракизм:} статисты делают вид, что героя не существует. В каьаке герою откровенно не рады. Столик героя будет пустовать и ему придется самому забирать свой заказ у стойки. Если где-то неподалеку случится преступление, герой будет первым подозреваемым!
\newline
\textbf{Торговля} убыточна. Лавочники требуют внушительную наценку (умножьте цену покупки у статиста на 2, а цену, за которую статисты готовы что-то купить у героя, разделите на 2).
\newline
\textbf{Просьбы о помощи} и \textbf{расспросы} игнорируются. Обыватели смотрят угрюмо и, если герой окликает их, ускоряют шаг или не обращают на него внимания. Положение может изменить лишь внушительный (для статиста) подкуп.
\newline
\textbf{В потенциально боевой ситуации} статисты атакуют героя, если на их стороне перевес. В противном случае они отступят и вернутся с подмогой. Если сражение по каким-то причинам невозможно или бессмысленно, статисты действуют против героя иными доступными методами.
\paragraph{7—9 — Настороженность:} окружающие ведут себя с героем весьма высокомерно и подозрительно. Лавочники, кабатчики и ополченцы обязательно постараются обобрать героя до нитки. Что ж, таков незавидный удел чужаков!
\textbf{Торговля} невыгодна (умножьте цену покупки у статиста на 1,5, а цену, за которую статисты готовы что-то купить у героя, разделите на 1,5).
\textbf{Просьбы о помощи} и \textbf{расспросы} игнорируются, хотя подкуп или униженные мольбы могут сработать.
\textbf{В потенциально боевой ситуации} статисты атакуют героя, если его убийство не кажется затруднительным или может принести выгоду. В противном случае статисты попытаются ограничиться насмешками, грязной бранью и требованиями немедленно уйти.
\textbf{10—12 — Нейтралитет:} героя воспринимают в целом спокойно. Торговцы не будут пытаться обжулить его слишком явно, а обыватели вряд ли вообще обратят на него внимание.
\textbf{Торговля} идет без осложнений. Лавочники продают и покупают по более или менее честной цене.
\textbf{Просьбы о помощи} удовлетворяются, если они не слишком
сложны и затруднительны. Статисты отвечают на \textbf{расспросы}, однако не стремятся дать полную и исчерпывающую информацию. 
\textbf{В потенциально боевой ситуации} статисты атакуют героя, только если очевидно спровоцированы.
\paragraph{13—15 — Доброжелательность:} герой вызывает сдержанное одобрение окружающих. Торговцы готовы продавать и покупать у него по относительно честным ценам, а обыватели охотно идут с ним на контакт. В кабаке герой безусловно станет центром внимания.
\textbf{Торговля} идет неплохо. Лавочники продают и покупают по более или менее честной цене, могут снабдить героя полезной информацией или предоставить небольшую скидку.
\textbf{Просьбы о помощи} удовлетворяются, если они не слишком сложны и затруднительны. Статисты отвечают на \textbf{расспросы} по возможности полно. Очевидно глупые вопросы и просьбы статисты встречают с юмором и даже могут дать полезный совет. В потенциально боевой ситуации статисты атакуют героя, только если спровоцированы. Они попытаются избегнуть боя и успокоить героя, если это возможно.
\paragraph{16—19 — Восторг:} герой кажется окружающим отличным парнем. Он без труда провернет выгодную торговую сделку и найдет союзников (или подельников). Не исключено, что его даже пригласят на ведущую роль в каком-нибудь местном празднике!
\textbf{Торговля} выгодна (разделите цену покупки у статиста на 1,5, умножьте на 1,5 цену, по которой статисты готовы покупать). Лавочники охотно снабдят героя полезной информацией.
\textbf{Просьбы о помощи} удовлетворяются (за исключением опасных или абсурдных). Статисты отвечают на \textbf{расспросы} максимально полно. Очевидно глупые вопросы и просьбы статисты встречают с юмором и даже могут дать полезный совет.
\textbf{В потенциально боевой ситуации} статисты сдадутся на милость героя, если только он не демонстративно кровожаден. В противном случае статисты отступят.
\paragraph{20 и больше — Очарованы:} появление героя вызывает больше радости и интереса, чем прибытие разъездного борделя или бродячего цирка! Герой получит внушительные скидки при торговле, а местные шишки (не говоря уж об обывателях) охотно примет его в своих жилищах.
\textbf{Торговля} крайне прибыльна (разделите цену покупки у статиста на 2, умножьте на 2 цену, по которой статисты готовы покупать). Лавочники охотно снабдят героя полезной информацией и разложат перед ним лучшие товары.
\textbf{Просьбы о помощи} статисты воспринимают как свой личный долг и делают все, что в их силах, чтобы помочь герою и отвечают на \textbf{расспросы} максимально полно. Очевидно глупые вопросы и просьбы удовлетворяются тоже (хотя впоследствии это может выйти герою боком)
\textbf{Бой невозможен} (разве что, бойня). Статисты готовы сдаться на милость героя или даже временно перейти на его сторону.
\paragraph{}
Появившись в незнакомом месте, герой вряд ли получит Впечатление лучше Доброжелательности без дополнительных усилий (хотя это, согласитесь, не так уж мало!). Мастер может сделать исключение в располагающих к этому ситуациях — например, для Красивых, Стильных или Соблазнительных героев на карнавале, Солдат и Офицеров в осажденном городе или Проповедников на религиозном празднике.
\newline
В дальнейшем герой имеет возможность изменить мнение окружающих о себе с помощью проверок Общения (для этого у него должен быть хотя бы шанс поговорить!). В случае раскрытых попыток обмана и прочих манипуляций со стороны героя, мнение может измениться и в худшую сторону!
\newline
Если статисты встречают группу героев, то сделайте одну проверку, применяя наихудшие из возможных штрафов. Если герои разделятся, к каждому из них применяется отдельное Впечатление (требующие новой проверки), хотя некоторые статисты наверняка запомнят, что героя видели в компании грубияна или урода!
\paragraph{}
Список Впечатлений не является жестким руководством к действию. Мастер волен интерпретировать варианты, сообразуясь с логикой происходящего, или использовать заранее подготовленную линию поведения статистов. Однако результаты из списка применяются при совершении проверок Общения.
\newline
Также этот список может использоваться, чтобы случайным образом определить Впечатления статистов от идей, предложенных героем. Разумеется, на 0 и меньше друзья и знакомые не будут бросаться на героя с оружием или кидать в него грязью (хотя всякое случается), но не постесняются сказать все, что думают о герое и его нелепой выдумке!
\section{Добиться своего…}
Пытаясь расположить статиста к себе (или возмутительно обмануть его), герой совершает проверку Общения против \textbf{|10 + Вл статиста + любые бонусы/штрафы, которые мастер сочтет уместными|}. В некоторых ситуациях Воля может быть заменена Наблюдательностью (Мд), например, когда герой пытается отвлечь статиста досужей болтовней или подсунуть ему испорченный товар. Чтобы заподозрить ложь, герои совершают проверку Общения (Мд) против \textbf{|10 + Общение лжеца|}. Договориться с разъяренными врагами может быть очень сложно или даже невозможно (особенно в пылу битвы), в то время как близкие друзья, родственники и поклонники благосклонно примут самые безумные идеи героев. Среднее значение 10 в целевой сложности проверки Общения может быть заменено мастером соответственно ситуации.

\begin{center}
\begin{tabular}{|p{6cm}|p{3cm}|p{2.5cm}|p{2.5cm}|}
\hline
Впечатление цели от собеседника на момент начала разговора & Мирные переговоры & Переговоры с позиции силы & Боевая сцена \\ \hline
Отвращение & 30 + Вл цели & 25 + Вл цели & Время слов прошло, к оружию! \\ \hline
Враждебность & 25 + Вл цели & 20 + Вл цели & 30 + Вл цели \\ \hline
Остракизм & 20 + Вл цели & 15 + Вл цели & 25 + Вл цели \\ \hline
Настороженность & 15 + Вл цели & 10 + Вл цели & 20 + Вл цели \\ \hline
Нейтралитет & 10 + Вл цели & 10 + Вл цели & 15 + Вл цели \\ \hline
Доброжелательность & 10 + Вл цели & 10 + Вл цели & 10 + Вл цели \\ \hline
Восторг & 5 + Вл цели & 5 + Вл цели & 10 + Вл цели \\ \hline
Очарование & 0 + Вл цели & 0 + Вл цели & 5 + Вл цели \\ \hline
\end{tabular}
\end{center}
Переговоры с позиции силы предполагают ситуации, в которых герои имеют (или убедительно делают вид, что имеют) средства заставить статиста сотрудничать и открыто их демонстрируют. Это вовсе не значит, что они на самом деле могут или будут их применять, но у статистов точно останется пренеприятное ощущение, что их принудили к компромиссу. В зависимости от статуса, благосостояния и влияния статиста, после окончания переговоров это может вылиться как в жалкую бессильную истерику, так и в наем лучших охотников за головами
\paragraph{Переговоры с группами:} как правило, группы существ менее восприимчивы к угрозам, лести, лжи и доводам разума. Каждый в группе боится, что если он поддастся на уговоры, то другие посмеются над ним или пустят арбалетный болт в затылок. Помимо этого, удерживать внимание и отслеживать настроение множества существ сразу не так-то просто. При переговорах с группами используйте Выступление и наиболее распространенное значение Воли в группе, которую пытается уговорить герой. Если выступление (и аудитория) не подготовлены заранее, герой совершает проверку с Помехой, когда ведет переговоры с группой из 10 или более существ. Герой совершает проверку с 2 Помехами, когда ведет переговоры с группой из 100 или более существ. В некоторых ситуациях герою точно не обойтись без Луженой глотки!

\paragraph{Стиль Общения} героя зависит от того, модификатор какой Характеристики используется при проверке.
\newline \textbf{Сила:} в случае прямолинейного запугивания посредством Силы успех проверки приносит Восторг (безусловно, показной). Статисты сделают все, что требует герой… пока он смотрит в их сторону. Если герой имеет возможность разыскать статиста и наказать за неповиновение (или статист думает так), статист будет выполнять указания и без непосредственного присутствия героя. Провал проверки немедленно провоцирует яростную атаку, если статист считает, что у него есть шанс на победу или бегство. Статист обязательно вернется с подмогой!
\newline \textbf{Выносливость:} на пирушках важно не только умение героя подать себя, но и способность много съесть и выпить (и при этом продолжать общаться). Успех проверки приносит вполне искреннюю Доброжелательность окружающих. Провал означает, что герой неким образом осрамился, и его репутация серьезно пострадала. Окружающие подвергают героя Остракизму, хотя в будущем у него будут возможности обелиться. Наверное.
\newline \textbf{Интеллект} — король торговли и дипломатии. Доводы разума при успешной проверке принесут герою Доброжелательность статистов. Провал обычно приводит к Нейтралитету (если только герой или его друзья уже не позаботились о худшем варианте!).
\newline \textbf{Обаяние:} для успеха непрямых угроз и заговаривания зубов герою пригодится Обаяние — сочетание эффектной подачи и личностной притягательности. Успех проверки приносит герою Восторг и все ему сопутствующее. Статисты уверены, что им выгоднее согласиться на предложения героя, чем поступить по-своему… до тех пор, пока кто-то не откроет им глаза! В этом случае, как и при провале проверки, герой сполна отведает Враждебности окружающих. Никто не любит оставаться в дураках!
\section{…Любыми средствами}
Подчас не самыми этичными, но, безусловно, эффективными.
\newline \textbf{Соблазнение} использует Обаяние героя в сочетании с сексуально агрессивным поведением различной степени очевидности. Успех проверки Очаровывает жертву… до тех пор пока она уверена, что герой вскоре очутится в ее постели. Если эта уверенность не подкрепляется дальнейшим поведением героя, он становится Отвратителен статисту (хотя статист все еще может быть не прочь затащить героя в постель!). То же происходит и при провале проверки — статист понимает, что им пытаются манипулировать. Отвращение может вылиться как в знатный синяк на смазливом личике героя, так и в значительно более серьезные последствия, если статист облечен властью и богат.
\newline \textbf{Публичные выступления}, как правило, бывают хорошо подготовленными, но даже тогда главную роль играет Обаяние выступающего. Проваленный бросок для странствующего певца редко выливается во что-то большее, чем предложение убраться со сцены. Политики, военачальники и революционеры рискуют гораздо больше, особенно в случаях экспромта!
\newline \textbf{Пытки} являются Экспертным Навыком. Герой может использовать Общение (Ин, Об), чтобы убедить статиста рассказать все до начала пытки. Обычно это сопровождается демонстрацией пыточных принадлежностей. Если герой не преуспел, они идут в дело. За каждую единицу, на которую палач прошел проверку Навыка против \textbf{|10 + Вл жертвы + модификатор Вн жертвы|}, жертва отвечает на один вопрос, исчерпывающе и полно. Мастер решает, может ли палач увеличить свой Навык с помощью пыточных приспособлений и на сколько именно. Так или иначе, к концу сеанса пыток жертва теряет столько Единиц Здоровья, на сколько палач прошел проверку (здесь палачу очень пригодится Трюк Хирургическая точность!). Каждый дополнительный день пыток понижает Волю жертвы на 1 (до минимума в 0). Если жертва не знает ответов, она соврет или постарается сказать палачу то, что тот хочет услышать!
\section{Социальные взаимодействия в бою}
Болтать в пылу битвы — довольно рискованное занятие. Если у героя все же возникло такое желание, он может применить Навык Общения в бою по описанным выше правилам, использовав Действие. Исключением из этого является маневр Провокация.
\section{Влияние на героев}
Статисты и герои могут влиять на других героев с помощью проверок Общения и иными доступными способами. Однако когда герой подвергается влияниям, игрок получает выбор. Он либо полностью принимает последствия, добавляет герою 1 Очко опыта и в дальнейшем играет роль соответственно случившемуся, либо объявляет, что герой не поддался манипуляциям. Если герой упорствует в осажденной крепости или пыточной камере, ему может не поздоровиться…