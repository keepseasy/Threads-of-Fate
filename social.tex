\chapter{Социальные взаимодействия}
Здесь вы узнаете, как герои и статисты общаются друг с другом, и чего они могут достичь, не прибегая к физическому насилию.

\section{Эмоциональный фон Сцены}
Статисты реагируют на героев, опираясь не только на их внешность и красноречие. Куда важнее обстоятельства встречи. 
\newline Эмоциональный фон Сцены подскажет, будет ли у статистов желание оценить героев по одежке, и получат ли они шанс блеснуть красотой, обаянием и прочими талантами. Максимальное/минимальное Впечатление, которое получают герои в Сцене, указано в скобках рядом с названием Эмоционального фона. 
\newline Эмоциональный фон может различаться для разных статистов в одной группе – например, если герои хорошо знакомы кому-то из них.
\begin{tcolorbox}
  Чтобы завоевать доверие статистов, потребуется правильно выбранная линия поведения (и, возможно, больше одной Сцены). Договориться с разъяренными врагами очень сложно или даже невозможно, в то время как близкие друзья, родственники и поклонники благосклонно примут самые безумные идеи героев.
\end{tcolorbox}

\subsection{Явная опасность}
\paragraph{Возможные первые Впечатления:} Настороженность или хуже.
\newline Герои не являются личными врагами статистов, но принадлежат к угрожающей им силе. Такой, как вражеская армия, бандитский отряд или толпа вооруженных до зубов чужаков.

\subsection{Потенциальная угроза}
\paragraph{Возможные первые Впечатления:} Нейтралитет или хуже.
\newline Опасность, исходящая от героев, вполне осязаема. Как правило, ее источник – боевое снаряжение, которое можно быстро пустить в ход. Пистолеты в кобурах, незаряженные арбалеты, холодное оружие в ножнах демонстрируют, что герой готов к конфликту, но не планирует его развязывать (наверное). В этом случае статисты видят в герое не только угрозу, и надеются, что все обойдется.

\subsection{Отсутствие угрозы}
\paragraph{Возможные первые Впечатления:} от Остракизма до Доброжелательности.
\newline Статисты не ожидают агрессии и чувствуют себя в безопасности. Они не прочь поболтать и отпустить шутку-другую. В эту категорию попадают встречи героев со знакомыми, публикой и обслугой в увеселительных заведениях.

\subsection{Неформальная обстановка}
\paragraph{Возможные первые Впечатления:} от Настороженности до Восторга.
\newline Статисты расслаблены, готовы к пространным беседам, тесному взаимодействию и в целом настроены дружелюбно. Скорее всего, герой находится среди друзей, родственников или сослуживцев.

\subsection{Интимная обстановка}
\paragraph{Возможные первые Впечатления:} от Нейтралитета до Очарования.
\newline Так обсуждаются выгодные сделки, хитрые планы, ужасные тайны, грязные секреты и прочая информация, не предназначенная для посторонних ушей. Герой пользуется абсолютным доверием статистов.

\section{Первое Впечатление}
\tbd [литературная вставка] Встречают по одежке и что-то про это.
Для того, чтобы определить Первое Впечатление, в зависимости от эмоционального фона, герой совершает проверку
\begin{itemize}
  \item Обаяния, Выступления(Об), Искусства(Об) или Знатока(Об), если Эмоциональный фон – «Неформальная обстановка» или лучше;
  \item Обаяния, Выступления(Об) или Искусства(Об), если Эмоциональный фон – «Отсутствие угрозы» или лучше;
  \item Обаяния или Выступления(Об), если Эмоциональный фон – «Потенциальная угроза» или лучше;
  \item Обаяния, если Эмоциональный фон - «Явная опасность».
\end{itemize}
Герой добавляет Бонусы свойств и способностей, которые влияют на Впечатления.
В зависимости от полученного значения Впечатление будет следующим:
\paragraph{0 и меньше – Отвращение:} статисты отказываются иметь с героем дело. Вышибалы гонят его из кабака, обыватели бранятся и швыряют камни и мусор. Нападение вооруженных статистов – вопрос пары фраз, иным хватит и недоброго взгляда.
\begin{itemize}
  \item Торговля невозможна. Лавочники отказываются торговать и угрожают вызвать охрану.
  \item Просьбы о помощи отвергаются с негодованием, расспросы приводят статистов в ярость.
  \item В потенциально боевой ситуации статисты атакуют героя и бьются насмерть. Если сражение по каким-то причинам невозможно или бессмысленно (герой очевидно сильнее, сражение приведет к неблагоприятным для статистов последствиям), статисты вредят герою иными доступными методами. 
\end{itemize}
\paragraph{1-3 – Враждебность:} статисты с трудом выносят присутствие героя. В кабаке он, скорее всего, станет причиной драки. Обыватели и не думают скрывать своего отношения, но воздержатся от неприкрытой травли, если герой не даст повода. Впрочем, вооруженные статисты или городское ополчение легко найдут повод для стычки!
\begin{itemize}
  \item Торговля разорительна. Лавочники заламывают драконовские цены, а покупают за сущие гроши. СП покупки у статиста возрастает втрое, а СП, за которую статисты готовы что-то купить у героя, сокращается втрое.
  \item Просьбы о помощи отвергаются, расспросы встречаются в штыки.
  \item В потенциально боевой ситуации статисты атакуют героя и отступают лишь в случае очевидного перевеса на его стороне. Если сражение по каким-то причинам невозможно или бессмысленно, статисты вредят герою иными доступными методами.
\end{itemize}
\paragraph{4-6 – Остракизм:} статисты делают вид, что героя не существует. В кабаке ему откровенно не рады. Столик героя будет пустовать и ему придется самому забирать заказ у стойки. Если где-то неподалеку случится преступление, герой станет первым подозреваемым.
\begin{itemize}
  \item Торговля убыточна. Лавочники требуют внушительную наценку СП покупки у статиста возрастает вдвое, а СП, за которую статисты готовы покупать, сокращается вдвое.
  \item Просьбы о помощи и расспросы игнорируются. Обыватели смотрят угрюмо и, если герой окликает их, ускоряют шаг или не обращают на него внимания. Положение может изменить лишь внушительный для статиста подкуп. 
  \item В потенциально боевой ситуации статисты атакуют героя, если на их стороне перевес. В противном случае они отступят и вернутся с подмогой. Если сражение по каким-то причинам невозможно или бессмысленно, статисты действуют против героя иными доступными методами.
\end{itemize}
\paragraph{7-9 – Настороженность:} окружающие ведут себя высокомерно и подозрительно. Лавочники, кабатчики и ополченцы постараются обобрать героя до нитки. Что ж, таков незавидный удел чужаков.
\begin{itemize}
  \item Торговля невыгодна. СП покупки у статиста возрастает в полтора раза, а СП, за которую статисты готовы покупать, сокращается в полтора раза.
  \item Просьбы о помощи и расспросы игнорируются, хотя подкуп или униженные мольбы могут сработать.
  \item В потенциально боевой ситуации статисты атакуют героя, если его убийство не кажется затруднительным или может принести выгоду. В противном случае статисты ограничатся насмешками, грязной бранью и требованиями свалить в ужасе.
\end{itemize}
\paragraph{10-12 – Нейтралитет:} героя воспринимают в целом спокойно. Торговцы не будут обжуливать его слишком явно, а обыватели вряд ли обратят на него внимание.
\begin{itemize}
  \item Торговля идет без осложнений. Лавочники продают и покупают по СП из таблиц.
  \item Просьбы о помощи удовлетворяются, если они не слишком затруднительны. Статисты отвечают на расспросы, однако не стремятся дать полную и исчерпывающую информацию.
  \item В потенциально боевой ситуации статисты атакуют героя, только если спровоцированы.
\end{itemize}
\paragraph{13-15 – Доброжелательность:} герой вызывает сдержанное одобрение окружающих. Торговцы готовы продавать и покупать у него по честным ценам, а обыватели охотно идут с ним на контакт. В кабаке герой безусловно станет центром внимания.
\begin{itemize}
  \item Торговля идет неплохо. Лавочники продают и покупают по СП из таблиц, могут снабдить героя информацией или предоставить небольшую скидку.
  \item Просьбы о помощи удовлетворяются, если они не слишком затруднительны. Статисты отвечают на расспросы по возможности полно. Очевидно глупые вопросы и просьбы статисты встречают с юмором и даже могут дать полезный совет.
  \item В потенциально боевой ситуации статисты атакуют героя, только если их последовательно провоцируют. Они пытаются избегнуть боя и успокоить героя, если это возможно.
\end{itemize}
\paragraph{16-19 – Восторг:} герой кажется окружающим офигительным. Он без труда провернет выгодную торговую сделку и найдет союзников или подельников. Не исключено, что его даже пригласят на ведущую роль в местном празднике.
\begin{itemize}
  \item Торговля выгодна. СП покупки у статиста сокращается в полтора раза, а СП, за которую статисты готовы покупать, возрастает в полтора раза. Лавочники охотно снабдят героя информацией.
  \item Просьбы о помощи удовлетворяются, за исключением опасных или абсурдных. Статисты отвечают на расспросы максимально полно. Очевидно глупые вопросы и просьбы статисты встречают с юмором и могут дать полезный совет.
  \item В потенциально боевой ситуации статисты сдадутся на милость героя, если только он не демонстративно кровожаден. Тогда статисты отступят.
\end{itemize}
\paragraph{20 и больше – Очарованы:} появление героя вызывает больше радости и интереса, чем прибытие разъездного борделя или бродячего цирка. Герой получит внушительные скидки при торговле, а местные шишки, не говоря об обывателях, охотно примут его в своих жилищах.
\begin{itemize}
  \item Торговля крайне прибыльна. СП покупки у статиста сокращается вдвое, а СП, за которую статисты готовы покупать, возрастает вдвое. Лавочники снабжают героя ценной информацией и раскладывают перед ним лучшие товары.
  \item Статисты делают все, чтобы помочь герою и отвечают на расспросы максимально полно. Очевидно глупые вопросы и просьбы удовлетворяются тоже, хотя позже это может выйти герою боком.
  \item Бой невозможен - разве что, бойня. Статисты готовы сдаться на милость героя или даже временно перейти на его сторону.
\end{itemize}
\begin{tcolorbox}
  Этот список может использоваться, чтобы случайным образом определить Впечатления статистов от идей, предложенных героем. Разумеется, на 0 и меньше друзья и знакомые не будут бросаться на героя с оружием или кидать в него грязью, но не постесняются высказать все, что думают о герое и его нелепой выдумке.
\end{tcolorbox}

\paragraph{Автоматический успех/провал Проверки Впечатлений} возможен при помощи Хода «Повезло». В этом случае герой получает максимально/ минимально возможный результат согласно Эмоциональному фону.
\paragraph{Критический успех/провал Проверки Впечатлений} достигается при помощи Хода «Повезло», либо при выпадении 20 и 1 на К20 соответственно. При этом герой достигает результата, следующего за максимально/ минимально возможным согласно Эмоциональному фону.

\paragraph{Общее впечатление} от группы героев определяется по самому непривлекательному индивиду: проверку совершает герой с наихудшим бонусом к проверке. Если герои разделятся, к каждому из них применяется отдельное Впечатление (требующее новой проверки), хотя некоторые статисты наверняка запомнят, что героя видели в компании грубияна или урода!

\paragraph{Вооруженные люди в броне:} статисты нервничают, когда видят вооруженных людей, облаченных в броню. Это не относится к военным объектам и анархическим вольницам, но обыватели не ждут ничего хорошего от вооруженного человека в бронежилете, пока он не служит в городском ополчении или армии.
\newline Если герои не выглядят носящими оружие и броню по праву службы или необходимости (городское ополчение, наемники в военное время, дровосек с топором, охотник с луком), они не могут получить Впечатление лучше Нейтралитета. В случае любого Впечатления хуже Нейтралитета статисты сообщат о героях властям, хотя в остальном вряд ли будут вести себя вызывающе.

\section{Изменение Впечатления и переговоры}
В дальнейшем герой имеет возможность изменить мнение окружающих о себе с помощью действий и проверок Общения (для этого у него должен быть хотя бы шанс поговорить). В случае раскрытых попыток обмана и манипуляций со стороны героя, мнение о нем может измениться и в худшую сторону.
\newline Пытаясь расположить статиста к себе (или возмутительно обмануть его), герой проверяет Общение против \textbf{|10 + Вл|} статиста. Результат и внешняя форма проверки зависит от Стиля общения. В некоторых ситуациях Воля может быть заменена Наблюдательностью (Мд), или даже подходящим НавыкомЭ, например, когда герой пытается отвлечь статиста досужей болтовней или подсунуть испорченный товар. 

\begin{center} \begin{tabular}{|p{6cm}|p{3cm}|p{2.5cm}|p{2.5cm}|} \hline
Впечатление цели от собеседника на момент начала разговора & Мирные переговоры & Переговоры с позиции силы & Боевая сцена \\ \hline
Отвращение & 30 + Вл цели & 25 + Вл цели & Время слов прошло, к оружию! \\ \hline
Враждебность & 25 + Вл цели & 20 + Вл цели & 30 + Вл цели \\ \hline
Остракизм & 20 + Вл цели & 15 + Вл цели & 25 + Вл цели \\ \hline
Настороженность & 15 + Вл цели & 10 + Вл цели & 20 + Вл цели \\ \hline
Нейтралитет & 10 + Вл цели & 10 + Вл цели & 15 + Вл цели \\ \hline
Доброжелательность & 10 + Вл цели & 10 + Вл цели & 10 + Вл цели \\ \hline
Восторг & 5 + Вл цели & 5 + Вл цели & 10 + Вл цели \\ \hline
Очарование & 0 + Вл цели & 0 + Вл цели & 5 + Вл цели \\ \hline
\end{tabular} \end{center}

Переговоры с позиции силы предполагают ситуации, в которых герои имеют (или убедительно делают вид, что имеют) средства заставить статиста сотрудничать и демонстрируют их. Это вовсе не значит, что они на самом деле могут или будут их применять, но у статистов останется пренеприятное ощущение, что их принудили к компромиссу. В зависимости от статуса, благосостояния и влияния статиста, после окончания переговоров это может откликнуться как бессильной истерикой, так и наймом лучших охотников за головами.
\newline При переговорах с позиции силы герой может заменить проверку Общения проверкой Доблести или Меткости. Если проверка успешна, а статист уцелел, герой легко добьется своего.
\paragraph{Переговоры с группами:} группы существ менее восприимчивы к угрозам, лести, лжи и доводам разума. Каждый боится, что если он поддастся на уговоры, то другие посмеются над ним или пустят пулю в затылок. Помимо этого, удерживать внимание и отслеживать настроение множества существ сразу не так просто. 
\newline При переговорах с группами использется Выступление и самое распространенное значение Воли статистов в группе. Если выступление (и аудитория) не подготовлены заранее, герой совершает проверку с Помехой, когда ведет переговоры с группой из 10 или более существ. Герой совершает проверку с 2 Помехами, когда ведет переговоры с группой из 100 или более существ. Тут герою точно не обойтись без мегафона или Луженой глотки!
\paragraph{Социальные взаимодействия в бою:} болтать в пылу битвы – рискованное занятие. Если у героя возникло такое желание, он применяет Навык Общения или Выступления, использовав Действие.

\subsection{Стиль Общения}
Он зависит от того, модификатор какой Характеристики используется при проверке, и какой линии поведения придерживается герой.
\paragraph{Сила:} прямолинейное запугивание и успех проверки приносит Восторг (безусловно, показной). Статисты сделают все, что требует герой… пока он смотрит в их сторону. Если герой имеет возможность разыскать статиста и наказать за неповиновение (или статист верит в это), статист выполнит указания и без непосредственного присутствия героя.
\newline Провал провоцирует яростную атаку, если статист считает, что у него есть шанс на победу, или бегство. Статист обязательно вернется с подмогой!
\paragraph{Ловкость:} порой секс является эффективным подспорьем в достижении самых разных целей.
\newline Для того чтобы проверить Общение(Лв), герой должен уйти в Интерлюдию. Успех проверки приносит ему ничем не замутненный Восторг статиста. В случае неудачи Впечатление от героя снижается на величину провала.
\paragraph{Выносливость:} на пирушках важно не только умение героя подать себя, но и способность много съесть и выпить, продолжая общаться. Успех приносит Доброжелательность окружающих. Провал означает, что герой осрамился, а его репутация пострадала. Окружающие подвергают героя Остракизму, хотя в будущем у него будут возможности обелиться. Наверное.
\paragraph{Интеллект:} король торговли и дипломатии. Доводы разума при успехе принесут герою Доброжелательность статистов. Провал приводит к Нейтралитету, если только Эмоциональный фон не подразумевает худший вариант.
\paragraph{Мудрость:} поможет подловить лжеца на на мелких деталях или интуитивно не поверить в самые правдоподобные посулы. Чтобы заподозрить обман или притворство, герой должен успешно проверить Общение(Мд) против \textbf{|10 + [Общение лжеца]|}. 
\paragraph{Обаяние:} для непрямых угроз и заговаривания зубов герою пригодится Обаяние – сочетание эффектной подачи и личностной притягательности. 
\newline Успех проверки приносит герою Восторг и все ему сопутствующее. Статисты уверены, что им выгоднее согласиться на предложения героя, чем поступить по-своему, пока кто-то не откроет им глаза. В этом случае, как и при провале проверки, герой сполна отведает Враждебности окружающих. Никто не любит оставаться в дураках!
\paragraph{Соблазнение:} использует Обаяние героя в сочетании с сексуально агрессивным поведением различной степени очевидности. 
\newline Успех проверки Очаровывает жертву, пока она уверена, что герой вскоре очутится в ее постели. Если эта уверенность не подкрепляется поведением героя, он становится Отвратителен статисту (хотя статист все еще может быть не прочь затащить героя в постель). То же происходит и при провале проверки – статист понимает, что им пытаются манипулировать. Отвращение может вылиться как в знатный синяк на смазливом личике, так и в более серьезные последствия, если статист облечен властью и богат. 
\paragraph{Публичные выступления:} как правило, бывают хорошо подготовленными, но даже тогда главную роль играют ВыступлениеЭ и Обаяние. Проваленный бросок для странствующего певца редко выливается во что-то большее, чем предложение убраться с помостков. Политики, военачальники и революционеры рискуют гораздо больше, особенно при экспромтах.
\paragraph{Пытки:} палач может использовать Общение(Ин, Об),  чтобы убедить статиста рассказать все до начала пытки. Обычно это сопровождается демонстрацией пыточных принадлежностей. Если жертва упорствует, они идут в дело. Пытки занимают Интерлюдию. Обычно палачи используют Медицину(Ин, Мд), Ловкость рук(Сл, Лв, Мд) или Дб/ Мт с применением оружия, наносящего Несмертельные Пв. 
\newline За каждую единицу успеха проверки против \textbf{|10 + [Вл жертвы] + [МВн жертвы]|}, она отвечает на один вопрос, исчерпывающе и полно. Если жертва не знает ответов, она соврет или постарается сказать палачу то, что тот хочет услышать.

\section{Влияние на героев}
Герои и статисты могут Влиять друг на друга с помощью проверок Общения и иными доступными способами, например, феноменами или Ходами. 
\newline Когда герой подвергается Влиянию статиста, разрешите ситуацию согласно Капризу Судьбы «Чуждое влияние».

\section{Конфликты героев}
Когда герой подвергается Влиянию или нападению \textit{другого героя}, игрок получает выбор. Он либо полностью принимает последствия, добавляет герою 1 Очко опыта и действует соответственно случившемуся, либо объявляет, что герой не поддался манипуляциям или не может быть атакован. 
\newline Если конфликты героев – важная часть вашей игры, разрешайте их по общим правилам.


