\chapter{МОГУЩЕСТВА}

\paragraph{Получение Могуществ.} Герой начинает игру без известных ему Могуществ. Но есть несколько способов их получить:
\begin{itemize}
\item[--] Выбрав Атрибут или Трюк Могущества.
\item[--] Потратив 2 очка опыта, чтобы получить возможность выучить дополнительное Могущество.
\newline
Однако для того чтобы получить в свое распоряжение новое Могущество, герой должен где-то увидеть формулу, рецепт, ноты или па танца и запомнить их. Где-то герою достаточно зайти в журнальный киоск или библиотеку и приобрести подборку Лучших Заклинаний Десятилетия или пособие «Чары Огня для начинающих», а где-то придется потратить долгие годы, продвигаясь в иерархии тайного культа или рисковать жизнью, пытаясь выкрасть записную книгу другого чародея! Поиск и изучение редких и могущественных заклинаний может стать приключением само по себе!
\item[--] Приобретя предмет, обладающий Могуществом.
\end{itemize}
\begin{tcolorbox}
Помните, что без навыка Концентрация герою будет сложно эффективно применять известные ему Могущества.
\end{tcolorbox}
\paragraph{Характеристика Могущества (ХМ)} героя является характеристикой, с помощью которой он активирует свои Могущества. Она указана в описании Атрибута, Трюка или Предмета, который он использует для активации. Модификатор Характеристики Могущества используется в большинстве формул, описывающих Могущества.
\paragraph{Усиление Могущества.} При активации Могущества, герой может потратить дополнительные еденицы энергии(обычно 1 Эн за каждое усиление), чтобы сделать его более эффективным. Все возможные для каждого Могущества Усиления перечислены в описании Могуществ.
\paragraph{Стоимость} является общим свойством всех Могуществ и определяет количество Энергии, которое герой тратит на их активацию.
\paragraph{Время сотворения} является общим свойством всех Могуществ и определяет время, необходимое для активации Могущества. Если в описании Могущества не указано Время сотворения, то оно равно Действию.
\paragraph{Совершение маневров} во время активации Могущества - если время активации Могущества равно Действию или Быстрому действию, герой может выполнить его активацию, как часть маневра Атака, Атака с разбега, Дистанционная Атака или Беглый огонь с применением этого Могущества.
\paragraph{Размер имеет значение(РИЗ).} Некотрые Могущества сложнее активировать на крупных существах. Увеличте Стоимость и Поддержание Могущества на 1 за каждую категорию размера цели выше Среднего.
\paragraph{Длительность} Могущества определяет время, в течение которого действуют эффект Могущества. Если в описании Могущества не указана Длительность, то эффекты Могущества применяются сразу, как только достигают цели, а Могущество завершает свое действие.
\paragraph{Стоимость Поддержания} Могущества указывает, что по оканчанию Длительности герой может потратить количество энергии, равное Стоимости Поддержания, продлив действие Могущества на его Длительность.
\paragraph{Прерывание:} если в процессе активации или поддержания Могущества герой получает Пв, превышающие его Вл, уже активированная способность немедленно прекращает свое действие, а активация Могущества автоматически считается проваленной, однако герой не возвращает Энергию, затраченную на попытку активации Могущества. В случае с Могуществами, сотворяемыми Мгновенно, это возможно, если противник предварительно выбрал маневр «Выжидание».
\newline
Если герой стремится помешать активирующему Могущество существу, не нанося тому Повреждений, он должен пройти проверку Навыка, логически применимого к ситуации (Общение — для едких шпилек и отвлекающих восклицаний, Атлетика — для отрезвляющих оплеух и т. д.) против \textbf{|20 + Вл существа|}.
\newline
Если вы хотите добавить в подобные ситуации остроты, заявивший Выжидание и сотворяющий заклинание должны совершить Состязание в Рц. Победитель действует первым.
\paragraph{Сопротивление} является проверкой во время Состязания против Концентрации(МХМ) героя, которую должна совершить цель Могущества, чтобы игнорировать эффекты способности. Если в описании способности не указано Сопротивление, то эффекты накладываются без возможности противостоять им. Для упрощения Состязания воспользуйтесь правилом Быстрой проверки Сопротивления. В сопротивлении вместо проверки может быть указано конкретное число. в этом случае вместо Состязания, совершается обычная проверка Концентрации(МХМ), а Сопротивление определяет Сложность Проверки.

\genAndGet{powerForms}

\paragraph{Проверка Наведения} требуется, если герой, активирующий Могущество не видит цель активации. Наведение является проверкой Концентрации(МХМ) если \textbf{Сопротивление Наведению} не указано в описании способности, сложность Наведения равна 15. При проверки герой может получить штрафы и бонусы, указанные ниже.
\newline
\textbf{Бонусы:}
\begin{itemize}
\item[--]Герой до этого касался цели = +1.
\item[--]Герой знает дополнительную информацию о цели, кроме ее описания = +1.
\item[--]Герой хорошо разглядел цель и запомнил, как она выглядит = +1.
\item[--]Герой хорошо представляет место, где цель скрывается от Наведения = +1.
\item[--]Цель не подозревает о том, что на нее производится Наведение = +1.
\item[--]Герой знает приблизительное направление на цель = +1.
\item[--]Герой знает приблизительное расстояние до цели = +1.
\item[--]Герой знает точное положение цели(в этот бонус включены бонусы от знания приблизительного направления и расстояния до цели) = +3.
\item[--]На цель наложена следящая Метка = +5(в этот бонус включен бонус от знания точного положения цели).

\end{itemize}
\textbf{Штрафы:}
\begin{itemize}
\item[--]Герой никогда не видел цель, и вынужден довольствоваться абстрактным описанием = -5.
\item[--]Герой никогда не видел цель, но видел точное изображение цели = -1.
\item[--]Герой не знает, ни расстояния до цели, ни направления на цель = -1.
\item[--]Герой плохо запомнил цель = -1.
\item[--]Герой не имеет представления об окружении вокруг цели = -1.
\item[--]Цель подозревает о том, что на нее могут Навести Могущество = -1.
\item[--]Цель находится в зоне действия Незримой вуали = -1.
\item[--]В предыдущий Круг цель совершала Перемещение или изменила свое местоположение другими существами или силами = -1.
\end{itemize}

\section{Описание Могуществ}
\genAndGet{powers}
