\chapter{Феномены и Исказители}
\textbf{Феномены} имеют разные проявления. Заклинания, паронормальные способности, суперсилы... Но общее у них одно -- за доступ к могуществу герой платит своей внутренней Энергией, которую в некоторых мирах не так уж и просто восполнить. Те же, кто умеет воплощать эти Феномены зовутся \textbf{Исказителями}.
\paragraph{Получение Феноменов.} Герой начинает игру без известных ему Феноменов. Но есть несколько способов их получить:
\begin{itemize}
\item[--] Выбрав Феноменальный Атрибут или Трюк.
\item[--] Потратив 2 очка опыта, чтобы получить возможность выучить дополнительный Феномен.
\newline
Однако для того чтобы получить в свое распоряжение новый Феномен, герой должен где-то увидеть формулу, рецепт, ноты или па танца и запомнить их. Где-то герою достаточно зайти в журнальный киоск или библиотеку и приобрести подборку Лучших Заклинаний Десятилетия или пособие "Чары Огня для начинающих", а где-то придется потратить долгие годы, продвигаясь в иерархии тайного культа или рисковать жизнью, пытаясь выкрасть записную книгу другого чародея! Поиск и изучение редких и могущественных Феноменов может стать приключением само по себе!
\item[--] Приобретя предмет, обладающий Функцией.
\end{itemize}
\begin{tcolorbox}
Без навыка Концентрация герою будет сложно эффективно применять многие Феномены, но большинство Снарядов не требуют этого навыка для применения.
\end{tcolorbox}
\paragraph{Феноменальная характеристика (Фх)} героя является характеристикой, с помощью которой он активирует свои Феномены. Она указана в описании Атрибута, Трюка или Предмета, который он использует для активации. Модификатор Феноменальной характеристики (МФх) используется в большинстве формул, описывающих Феномены.
\paragraph{Усиление Феномена.} При активации Феномена, герой может потратить дополнительные еденицы энергии(обычно 1 Эн за каждое усиление), чтобы сделать его более эффективным. Все возможные для каждого Феномена Усиления перечислены в его описании.
\paragraph{Стоимость} является общим свойством всех Феноменов и определяет количество Энергии, которое герой тратит на их активацию.
\paragraph{Время сотворения} является общим свойством всех Феноменов и определяет время, необходимое для активации Феномена. Если в описании Феномена не указано Время сотворения, то оно равно Действию.
\paragraph{Совершение маневров} во время активации Феномена - если время активации Феномена равно Действию или Быстрому действию, герой может выполнить его активацию, как часть маневра Атака, Атака с разбега, Дистанционная Атака или Беглый огонь с применением этого Феномена.
\paragraph{Длительность} Феномена определяет время, в течение которого действуют эффект Феномена. Если в описании Феномена не указана Длительность, то эффекты Феномена применяются сразу, как только достигают цели, а Феномен завершает свое действие.
\paragraph{Стоимость Поддержания} Феномена указывает, что по оканчанию Длительности герой может потратить количество энергии, равное Стоимости Поддержания, продлив действие Феномена на его Длительность.
\paragraph{Прерывание:} если в процессе активации или поддержания Феномена герой получает Пв, превышающие его Вл, уже активированная способность немедленно прекращает свое действие, а активация Феномена автоматически считается проваленной, однако герой не возвращает Энергию, затраченную на попытку активации Феномена. В случае с Феноменами, сотворяемыми Мгновенно, это возможно, если противник предварительно выбрал маневр <<Выжидание>>.
\newline
Если герой стремится помешать активирующему Феномен существу, не нанося тому Повреждений, он должен пройти проверку Навыка, логически применимого к ситуации (Общение — для едких шпилек и отвлекающих восклицаний, Атлетика — для отрезвляющих оплеух и т. д.) против \textbf{|20 + Вл существа|}.
\newline
Если вы хотите добавить в подобные ситуации остроты, заявивший Выжидание и сотворяющий заклинание должны совершить Состязание в Рц. Победитель действует первым.
\paragraph{Сопротивление} является проверкой во время Состязания против Концентрации(МФх) героя, которую должна совершить цель Феномена, чтобы игнорировать эффекты способности. Если в описании способности не указано Сопротивление, то эффекты накладываются без возможности противостоять им. Для упрощения Состязания воспользуйтесь правилом Быстрой проверки Сопротивления. В сопротивлении вместо проверки может быть указано конкретное число. в этом случае вместо Состязания, совершается обычная проверка Концентрации(МФх), а Сопротивление определяет Сложность Проверки.


\paragraph{Проверка Наведения} требуется, если герой, активирующий Феномен не видит цель активации. Наведение является проверкой Концентрации(МФх) если \textbf{Сопротивление Наведению} не указано в описании способности, сложность Наведения равна 15. При проверки герой может получить штрафы и бонусы, указанные ниже.
\newline
\textbf{Бонусы:}
\begin{itemize}
\item[--]Герой до этого касался цели = +1.
\item[--]Герой знает дополнительную информацию о цели, кроме ее описания = +1.
\item[--]Герой хорошо разглядел цель и запомнил, как она выглядит = +1.
\item[--]Герой хорошо представляет место, где цель скрывается от Наведения = +1.
\item[--]Цель не подозревает о том, что на нее производится Наведение = +1.
\item[--]Герой знает приблизительное направление на цель = +1.
\item[--]Герой знает приблизительное расстояние до цели = +1.
\item[--]Герой знает точное положение цели(в этот бонус включены бонусы от знания приблизительного направления и расстояния до цели) = +3.
\item[--]На цель наложена следящая Метка = +5(в этот бонус включен бонус от знания точного положения цели).

\end{itemize}
\textbf{Штрафы:}
\begin{itemize}
\item[--]Герой никогда не видел цель, и вынужден довольствоваться абстрактным описанием = -5.
\item[--]Герой никогда не видел цель, но видел точное изображение цели = -1.
\item[--]Герой не знает, ни расстояния до цели, ни направления на цель = -1.
\item[--]Герой плохо запомнил цель = -1.
\item[--]Герой не имеет представления об окружении вокруг цели = -1.
\item[--]Цель подозревает о том, что на нее могут Навести Феномен = -1.
\item[--]Цель находится в зоне действия Незримой вуали = -1.
\item[--]В предыдущий Круг цель совершала Перемещение или изменила свое местоположение другими существами или силами = -1.
\end{itemize}

\paragraph{Форма Феномена} является первоначальным их описанием и заранее определяет многие свойства Феномена.
\newline Полное описание феноменов сгруппировано по их формам.

\genAndGet{powerForms}{powerForms}{}

\section{Алфавитный перечень Феноменов}
\printindex[powers]
