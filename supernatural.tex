\chapter{Феномены и Исказители}
\textbf{Феномены} - собирательный термин для всевозможных сверхсил, пугающих и необъяснимых. Они имеют самую разную природу. Паронормальные способности, мутации, телепатия, психокинезия и даже всамделишная волшба - или технологии, от нее неотличимые. Общее у них одно - они поглощают Энергию, которую не так просто восполнить. Те, кто умеет воплощать эти феномены, зовутся \textbf{Исказителями}. Своей удивительной мощью они искажают и разрушают привычные законы мира - зачастую, в своих собственных интересах.
\paragraph{Получение феноменов.} Герой начинает игру без известных ему феноменов. Есть несколько способов их использовать:
\begin{itemize}
  \item Выбрав Феноменальный Атрибут или Трюк, герой сразу получает доступ к нескольким Феноменам.
  \item В описании Атрибутов Могущества указано, как обладатель этого Атрибута может получить дополнительные Феномены.
    \newline Каждый Атрибут имеет свой способ получения феноменов, что отражает выбранный исказителем путь постижения непостижимого.
  \item Некоторые предметы обладают Функциями, позволяющими творить Феномены.
\end{itemize}
\begin{tcolorbox}
  Без навыка Концентрация герою будет сложно эффективно применять многие Феномены, но большинство Снарядов не требуют этого навыка для применения.
\end{tcolorbox}
\paragraph{Феноменальная характеристика (Фх).} Основная Характеристика, с помощью которой исказитель активирует феномены. Она указана в описании Атрибута, Трюка или Предмета, который он использует для активации. Модификатор Феноменальной характеристики (МФх) используется в большинстве формул, описывающих феномены.
\paragraph{Усиление Феномена.} При активации феномена исказитель часто может потратить дополнительную Эн и усилить эффект. Возможные для феноменов Усиления перечислены в их описании. Герой вправе применять одно и то же Усиление несколько раз.
\paragraph{Стоимость} Определяет количество Энергии, которое исказитель тратит на активацию.
\paragraph{Время активации} Определяет время, необходимое для активации. Если в описании не указано Время активации, то она расходует Действие.
\paragraph{Маневры и феномены.} Если активация феномена расходует Действие или Быстрое действие, исказитель может активировать его, как маневр Атака, Дистанционная Атака, Комбинированная атака, Беглый огонь или Концентрированный огонь с применением этого феномена. 
\paragraph{Длительность} Феномена определяет время, в течение которого действуют эффекты феномена. Если в описании не указана Длительность, то эффекты применяются мгновенно.
\paragraph{Поддержание.} По окончании Длительности герой может потратить количество Эн, равное Стоимости Поддержания, продлив действие феномена на указанный срок.
\newline Одновременно исказитель способен поддерживать \textbf{|МФх|} феноменов.
\paragraph{Размер имеет значение (РИЗ).} Стоимость феномена возрастает на \textbf{|2*МРз|} цели, если присутствует пометка (РИЗ). Поддержание феномена возрастает на \textbf{|МРз|} цели. Исполинские цели не могут подвергнуться эффектам такого феномена. 
\paragraph{Прерывание активации.} Если в процессе активации или поддержания феномена исказитель одномоментно теряет ЕЗ, числом превышающие его Вл, активированные феномены немедленно прерываются, а активация феномена автоматически считается проваленной, однако исказитель не возвращает Энергию, затраченную на попытку активации феномена. В случае с феноменами, активируемыми мгновенно, это возможно, если противник предварительно выбрал маневр «Выжидание». 
\newline Когда герой стремится помешать активатору феномена, не нанося тому Пв, он должен пройти проверку Навыка, логически применимого к ситуации (Общение — для едких шпилек и отвлекающих восклицаний, Ловкость рук — для отрезвляющих оплеух и т. д.) против \textbf{|20 + Вл|} исказителя.
\begin{tcolorbox}
  Если вы хотите добавить в подобные ситуации остроты, Выжидающий и активатор феномена должны Состязаться в Рц. Победитель действует первым. 
\end{tcolorbox}
\paragraph{Сопротивление.} Сложность проверки Концентрации(МФх) исказителя для успешной актвивации Феномена.
\newline Если в описании Сопротивления указана Характеристика или Навык, то вместо статичного значения можно совершить проверку по правилам Состязания, где бонус к Проверке равен \textbf{|Сопротивление - 10|}.
\newline Если в описании не указано Сопротивление, эффекты не дают возможности противостоять им.
\begin{tcolorbox}
  Чтобы Сцены с использованием феноменов не затягивались, авторский коллектив рекомендует использовать Эффективные показатели Навыков и Характеристик для Сопротивления – по крайней мере, когда речь не идет о Персонах и героях.
\end{tcolorbox}

\paragraph{Проверка Наведения.} Требуется, если активатор феномена не видит цель. Наведение является проверкой Концентрации(Фх). Если Сопротивление Наведению не указано в описании способности, сложность Наведения равна \textbf{|15|}. При проверке герой может получить штрафы и бонусы, указанные ниже.
\newline \textbf{Бонусы:}
\begin{itemize}
  \item[--] Герой касался цели = +1.
  \item[--] Герой знает дополнительную информацию о цели, кроме ее описания = +1.
  \item[--] Герой хорошо разглядел цель и запомнил, как она выглядит = +1.
  \item[--] Герой хорошо представляет место, где цель скрывается от Наведения = +1.
  \item[--] Цель не подозревает о том, что на нее производится Наведение = +1.
  \item[--] Герой знает приблизительное направление на цель = +1.
  \item[--] Герой знает приблизительное расстояние до цели = +1.
  \item[--] Герой знает точное положение цели (включает бонусы от знания приблизительного направления и расстояния до цели) = +3.
  \item[--] На цель наложен следящий Феномен = +5(включает бонус от знания точного положения цели) и +2 за каждый дополнительный следящий феномен.
\end{itemize}
\textbf{Штрафы:}
\begin{itemize}
  \item[--] Герой никогда не видел цель, и вынужден довольствоваться абстрактным описанием = -5.
  \item[--] Герой никогда не видел цель, но видел ее точное изображение = -1.
  \item[--] Герой не знает ни расстояния до цели, ни направления на цель = -1.
  \item[--] Герой плохо запомнил цель визуально = -1.
  \item[--] Герой не имеет представления об окружении цели = -1.
  \item[--] Цель подозревает о том, что на нее Наводят феномен = -1.
  \item[--] Цель скрывается с помощью Феноменов = -1 за каждый скрывающий Феномен.
  \item[--] В предыдущий Круг цель совершала Перемещение или изменила свое местоположение благодаря другими существам или силам = -1.
\end{itemize}

\paragraph{Форма Феномена.} Определяет свойства и внешние проявления феномена. Полное описание феноменов сгруппировано по их формам.

\genAndGet{powerForms}{powerForms}{}

\section{Алфавитный перечень Феноменов}
\printindex[powers]
